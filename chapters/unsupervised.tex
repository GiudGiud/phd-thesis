\chapter{Unsupervised Clustering for Spatial Homogenization}
\label{chap:unsupervised}

first paragraph: motivation
-recall from previous chapter:
  -found clusters in \ac{MGXS}
  -must cluster \ac{MGXS} to reduce U-238 capture rate errors to level of fission
-motivate chapter two-fold:
  -simultaneously achieve degenerate's accuracy and null's convergence
  -need to be more adapative than \ac{LNS} for arbitrary geometries

second paragraph: over-arching goal
-achieve accuracy of \ac{MC} with efficient deterministic multi-group methods
-recall motivation in chapter 1??
-present flow chart figure Fig.~\ref{fig:chap10-flow-chart}
-following chapter will evaluate methodology wrt eigenvalues, fission rates, U-238 capture reats

third paragraph: outline chapter
-

-need to say that chapter presents a general methodology, only part of which will be evaluated in the following chapter

\begin{figure}
  \centering
  \includegraphics[width=0.9\linewidth]{figures/pipeline/flow-chart}
  \caption{}
\caption[flowchar]{flowchart}
\label{fig:chap10-flow-chart}
\end{figure}
  

%%%%%%%%%%%%%%%%%%%%%%%%%%%%%%%%%%%%%%%%%%%%%%%%%%%%%%%%%%%%%%%%%%%%%%%%%%%%%%%
\section{Latent Spectral Index Model}
\label{sec:chap10-latent-model}

-\ac{LSIM} - latent variable model
-three goals:
  -infer number of points/pins in each set
  -infer conditional probabilities?
  -assign each region to the most likely set?


%%%%%%%%%%%%%%%%%%%%%%%%%%%%%%%%%%%%%%%%%%%%%%%%%%%%%%%%%%%%%%%%%%%%%%%%%%%%%%%
\section{Overview of Approach}
\label{sec:chap10-overview}

-competing paragidms:
  -clustering
  -input -> target (e.g., trees)
  -some combination of the two

-break apart diagram for machine learning piece into parts
  -clustering
    -flow chart: litmus tests, feature eng./selectiontransformation, clustering
  -input -> target (e.g., regression)
    -flow chart: litmus tests, target selection, feature eng./selection/transformation, model fitting, prediction
     -can take outputs as inputs for clustering scheme


%%%%%%%%%%%%%%%%%%%%%%%%%%%%%%%%%%%%%%%%%%%%%%%%%%%%%%%%%%%%%%%%%%%%%%%%%%%%%%%
\section{Feature Engineering}
\label{sec:chap10-feature-engineer}

first paragraph: motivation
-need to define ``feature''
-recall plots from preceding chapter with clusters
-mention conflict:
  -need 70+ groups for accurate solutions
  -uncertainties / noise in few-group \ac{MGXS} is less, making it easier to identify clusters
-cluster \ac{MGXS} directly
  -simply draw 1D boundaries b/w \ac{MGXS} in histograms
  -could cluster multiple microscopic \ac{MGXS} simultaneously - nuclides, reactions, groups
  -multiple \ac{MGXS} may eliminate ``de-noise'' data, eliminate outliers
    -good segue to next paragraph!!

second paragraph: more generally
-include non-\ac{MGXS} variables to cluster
  -separation distance clusters may grow with more variables, making them easier to identify
-goal: identify variables which reflect spatial self-shielding effects
-can simply cluster multiple micro \ac{MGXS} simultaneously

third paragraph: outline
-

-mention that some features may be 1) fuel pin specific (at the least), 2) nuclide specific, 3) reaction specific, 4) energy group specific

%%%%%%%%%%%%%%%%%%%%%%%%%%%%%%%%%%%%
\subsection{Statistical Uncertainty}
\label{subsec:chap10-stat-uncertainty}

first paragraph: explain what it is
-point to a few figures
-explain why std. dev. may be higher for different pins
-indicate that it may reflect the difference in track density in differen pins
  -is this why we see rings in the \ac{MGXS} in the full core??
-recall equation for std. dev.
-recall uncertainty propagation
-also relative error - normalize out the mean \ac{MGXS}

\begin{equation}
\label{eqn:chap10-rel-err}
\eta_{x,i,k,g} = \frac{\sigma_{\hat{\sigma}_{x,i,k,g}}}{\hat{\sigma}_{x,i,k,g}}
\end{equation}

second paragraph: visualizations
-look at single fuel assembly with GUI
-three plots: full, two different clusters selected

\begin{figure}[h!]
\centering
\begin{subfigure}{0.42\textwidth}
  \centering
  \includegraphics[width=0.9\linewidth]{figures/unsupervised/features/assm-16/geometry}
  \caption{}
  \label{fig:chap10-fiss-mean-std-geom}
\end{subfigure}%
\begin{subfigure}{0.42\textwidth}
  \centering
  \includegraphics[width=0.9\linewidth]{figures/unsupervised/features/assm-16/u235-fiss/mean-std/mgxs}
  \caption{}
  \label{fig:chap10-fiss-mean-std-mgxs}
\end{subfigure}
\begin{subfigure}{0.42\textwidth}
  \centering
  \includegraphics[width=0.9\linewidth]{figures/unsupervised/features/assm-16/u235-fiss/mean-std/geometry-2}
  \caption{}
  \label{fig:chap10-fiss-mean-std-geom-2}
\end{subfigure}%
\begin{subfigure}{0.42\textwidth}
  \centering
  \includegraphics[width=0.9\linewidth]{figures/unsupervised/features/assm-16/u235-fiss/mean-std/mgxs-2}
  \caption{}
  \label{fig:chap10-fiss-mean-std-mgxs-2}
\end{subfigure}
\begin{subfigure}{0.42\textwidth}
  \centering
  \includegraphics[width=0.9\linewidth]{figures/unsupervised/features/assm-16/u235-fiss/mean-std/geometry-3}
  \caption{}
  \label{fig:chap10-fiss-mean-std-geom-3}
\end{subfigure}%
\begin{subfigure}{0.42\textwidth}
  \centering
  \includegraphics[width=0.9\linewidth]{figures/unsupervised/features/assm-16/u235-fiss/mean-std/mgxs-3}
  \caption{}
  \label{fig:chap10-fiss-mean-std-mgxs-3}
\end{subfigure}
\caption[Clustering of U-235 fission MGXS standard deviations]{Scatter plots of the pin-wise U-235 fission (group 2 of 2) \ac{MGXS} means ($x$) and standard deviations ($y$) for the 1.6\% enriched fuel assembly.}
\label{fig:chap10-mean-std}
\end{figure}

\clearpage

\begin{figure}[h!]
\centering
\begin{subfigure}{0.42\textwidth}
  \centering
  \includegraphics[width=0.9\linewidth]{figures/unsupervised/features/assm-16/geometry}
  \caption{}
  \label{fig:chap10-capt-mean-std-geom}
\end{subfigure}%
\begin{subfigure}{0.42\textwidth}
  \centering
  \includegraphics[width=0.9\linewidth]{figures/unsupervised/features/assm-16/u238-capt/mean-std/mgxs}
  \caption{}
  \label{fig:chap10-capt-mean-std-mgxs}
\end{subfigure}
\begin{subfigure}{0.42\textwidth}
  \centering
  \includegraphics[width=0.9\linewidth]{figures/unsupervised/features/assm-16/u238-capt/mean-std/geometry-2}
  \caption{}
  \label{fig:chap10-capt-mean-std-geom-2}
\end{subfigure}%
\begin{subfigure}{0.42\textwidth}
  \centering
  \includegraphics[width=0.9\linewidth]{figures/unsupervised/features/assm-16/u238-capt/mean-std/mgxs-2}
  \caption{}
  \label{fig:chap10-capt-mean-std-mgxs-2}
\end{subfigure}
\begin{subfigure}{0.42\textwidth}
  \centering
  \includegraphics[width=0.9\linewidth]{figures/unsupervised/features/assm-16/u238-capt/mean-std/geometry-3}
  \caption{}
  \label{fig:chap10-capt-mean-std-geom-3}
\end{subfigure}%
\begin{subfigure}{0.42\textwidth}
  \centering
  \includegraphics[width=0.9\linewidth]{figures/unsupervised/features/assm-16/u238-capt/mean-std/mgxs-3}
  \caption{}
  \label{fig:chap10-capt-mean-std-mgxs-3}
\end{subfigure}
\caption[Clustering of U-238 capture MGXS standard deviations]{Scatter plots of the pin-wise U-238 capture (group 1 of 2) \ac{MGXS} means ($x$) and standard deviations ($y$) for the 1.6\% enriched fuel assembly.}
\label{fig:chap10-mean-std}
\end{figure}

\clearpage

%%%%%%%%%%%%%%%%%%%%%%%%%%%%%%%%%%
\subsection{Fractional Reactivity}
\label{subsec:chap10-frac-reactivity}

first paragraph: explain what it is
-divide each pin's energy-integrated nu-fission rxn rate by the total energy- and volume-integrated absorption rates across all fuel pins
-Eqn.~\ref{eqn:chap10-frac-reactivity}
  -multiply by 1E5 to get units of \ac{pcm}
-explain what it indicates (differential moderation, adjacency to guide tubes, spectral hardening)
-note that $k$ is ONLY for fuel regions!!!

\begin{equation}
\label{eqn:chap10-frac-reactivity}
\alpha_{k} = \frac{\displaystyle\sum\limits_{g=1}^{G}\nu\hat{\sigma}_{f,i,k,g}\hat{\phi}_{k,g}}{\displaystyle\sum\limits_{k=1}^{K}\displaystyle\sum\limits_{g=1}^{G}\hat{\sigma}_{a,i,k,g}\hat{\phi}_{k,g}}
\end{equation}

second paragraph: visualizations
-look at single fuel assembly with GUI
-three plots: full, two different clusters selected

\begin{figure}[h!]
\centering
\begin{subfigure}{0.45\textwidth}
  \centering
  \includegraphics[width=0.9\linewidth]{figures/unsupervised/features/assm-16/geometry}
  \caption{}
  \label{fig:chap10-fiss-mean-pcm-geom}
\end{subfigure}%
\begin{subfigure}{0.45\textwidth}
  \centering
\includegraphics[width=0.9\linewidth]{figures/unsupervised/features/assm-16/u235-fiss/mean-pcm/mgxs}
  \caption{}
  \label{fig:chap10-fiss-mean-pcm-mgxs}
\end{subfigure}
\begin{subfigure}{0.45\textwidth}
  \centering
  \includegraphics[width=0.9\linewidth]{figures/unsupervised/features/assm-16/u235-fiss/mean-pcm/geometry-2}
  \caption{}
  \label{fig:chap10-fiss-mean-pcm-geom-2}
\end{subfigure}%
\begin{subfigure}{0.45\textwidth}
  \centering
  \includegraphics[width=0.9\linewidth]{figures/unsupervised/features/assm-16/u235-fiss/mean-pcm/mgxs-2}
  \caption{}
  \label{fig:chap10-fiss-mean-pcm-mgxs-2}
\end{subfigure}
\begin{subfigure}{0.45\textwidth}
  \centering
  \includegraphics[width=0.9\linewidth]{figures/unsupervised/features/assm-16/u235-fiss/mean-pcm/geometry-3}
  \caption{}
  \label{fig:chap10-fiss-mean-pcm-geom-3}
\end{subfigure}%
\begin{subfigure}{0.45\textwidth}
  \centering
  \includegraphics[width=0.9\linewidth]{figures/unsupervised/features/assm-16/u235-fiss/mean-pcm/mgxs-3}
  \caption{}
  \label{fig:chap10-fiss-mean-pcm-mgxs-3}
\end{subfigure}
\caption[Clustering of U-235 fission MGXS fractional reactivities]{Scatter plots of the pin-wise U-235 fission (group 2 of 2) \ac{MGXS} means ($x$) and fractional reactivities ($y$) for the 1.6\% enriched fuel assembly.}
\label{fig:chap10-mean-pcm}
\end{figure}

\clearpage

\begin{figure}[h!]
\centering
\begin{subfigure}{0.45\textwidth}
  \centering
  \includegraphics[width=0.9\linewidth]{figures/unsupervised/features/assm-16/geometry}
  \caption{}
  \label{fig:chap10-capt-mean-pcm-geom}
\end{subfigure}%
\begin{subfigure}{0.45\textwidth}
  \centering
  \includegraphics[width=0.9\linewidth]{figures/unsupervised/features/assm-16/u238-capt/mean-pcm/mgxs}
  \caption{}
  \label{fig:chap10-capt-mean-pcm-mgxs}
\end{subfigure}
\begin{subfigure}{0.45\textwidth}
  \centering
  \includegraphics[width=0.9\linewidth]{figures/unsupervised/features/assm-16/u238-capt/mean-pcm/geometry-2}
  \caption{}
  \label{fig:chap10-capt-mean-pcm-geom-2}
\end{subfigure}%
\begin{subfigure}{0.45\textwidth}
  \centering
  \includegraphics[width=0.9\linewidth]{figures/unsupervised/features/assm-16/u238-capt/mean-pcm/mgxs-2}
  \caption{}
  \label{fig:chap10-capt-mean-pcm-mgxs-2}
\end{subfigure}
\begin{subfigure}{0.45\textwidth}
  \centering
  \includegraphics[width=0.9\linewidth]{figures/unsupervised/features/assm-16/u238-capt/mean-pcm/geometry-3}
  \caption{}
  \label{fig:chap10-capt-mean-pcm-geom-3}
\end{subfigure}%
\begin{subfigure}{0.45\textwidth}
  \centering
  \includegraphics[width=0.9\linewidth]{figures/unsupervised/features/assm-16/u238-capt/mean-pcm/mgxs-3}
  \caption{}
  \label{fig:chap10-capt-mean-pcm-mgxs-3}
\end{subfigure}
\caption[Clustering of U-238 capture MGXS fractional reactivities]{Scatter plots of the pin-wise U-238 capture (group 1 of 2) \ac{MGXS} means ($x$) and fractional reactivities ($y$) for the 1.6\% enriched fuel assembly.}
\label{fig:chap10-mean-pcm}
\end{figure}

\clearpage


%%%%%%%%%%%%%%%%%%%%%%%%%%%
\subsection{Spectral Index}
\label{subsec:chap10-spec-index}

first paragraph: explain what it is
-divide each pin's nu-fission and absorption \ac{MGXS} 
-Eqn.~\ref{eqn:chap10-spec-index}
-explain what it indicates (differential moderation, adjacency to guide tubes, spectral hardening)
-note $i$ and $j$ may designate two difference nuclides
-note $\gamma$ is radiative capture - need to change this in theory section???

\begin{equation}
\label{eqn:chap10-spec-index-general}
\beta_{x,i,k,g} = \frac{\hat{\sigma}_{x,i,k,g}}{\hat{\sigma}_{f,k,g}^{235}}
\end{equation}

\begin{equation}
\label{eqn:chap10-spec-index-u238-capt}
\beta_{\gamma,k,g}^{238} = \frac{\hat{\sigma}_{\gamma,k,g}^{238}}{\hat{\sigma}_{f,k,g}^{235}}
\end{equation}

second paragraph: visualizations
-look at single fuel assembly with GUI
-three plots: full, two different clusters selected
-indicate which energy group and nuclide

\begin{figure}[h!]
\centering
\begin{subfigure}{0.42\textwidth}
  \centering
  \includegraphics[width=0.9\linewidth]{figures/unsupervised/features/assm-16/geometry}
  \caption{}
  \label{fig:chap10-fiss-mean-spect-ind-geom}
\end{subfigure}%
\begin{subfigure}{0.42\textwidth}
  \centering
  \includegraphics[width=0.9\linewidth]{figures/unsupervised/features/assm-16/u235-fiss/mean-spect-ind/mgxs}
  \caption{}
  \label{fig:chap10-fiss-mean-spect-ind-mgxs}
\end{subfigure}
\begin{subfigure}{0.42\textwidth}
  \centering
  \includegraphics[width=0.9\linewidth]{figures/unsupervised/features/assm-16/u235-fiss/mean-spect-ind/geometry-2}
  \caption{}
  \label{fig:chap10-fiss-mean-spect-ind-geom-2}
\end{subfigure}%
\begin{subfigure}{0.42\textwidth}
  \centering
  \includegraphics[width=0.9\linewidth]{figures/unsupervised/features/assm-16/u235-fiss/mean-spect-ind/mgxs-2}
  \caption{}
  \label{fig:chap10-fiss-mean-spect-ind-mgxs-2}
\end{subfigure}
\begin{subfigure}{0.42\textwidth}
  \centering
  \includegraphics[width=0.9\linewidth]{figures/unsupervised/features/assm-16/u235-fiss/mean-spect-ind/geometry-3}
  \caption{}
  \label{fig:chap10-fiss-mean-spect-ind-geom-3}
\end{subfigure}%
\begin{subfigure}{0.42\textwidth}
  \centering
  \includegraphics[width=0.9\linewidth]{figures/unsupervised/features/assm-16/u235-fiss/mean-spect-ind/mgxs-3}
  \caption{}
  \label{fig:chap10-fiss-mean-spect-ind-mgxs-3}
\end{subfigure}
\caption[Clustering of U-235 fission MGXS spectral indices]{Scatter plots of the pin-wise U-235 fission (group 2 of 2) \ac{MGXS} means ($x$) and spectral indices ($y$) for the 1.6\% enriched fuel assembly.}
\label{fig:chap10-mean-spect-ind}
\end{figure}

\clearpage

\begin{figure}[h!]
\centering
\begin{subfigure}{0.42\textwidth}
  \centering
  \includegraphics[width=0.9\linewidth]{figures/unsupervised/features/assm-16/geometry}
  \caption{}
  \label{fig:chap10-capt-mean-spect-ind-geom}
\end{subfigure}%
\begin{subfigure}{0.42\textwidth}
  \centering
  \includegraphics[width=0.9\linewidth]{figures/unsupervised/features/assm-16/u238-capt/mean-spect-ind/mgxs}
  \caption{}
  \label{fig:chap10-capt-mean-spect-ind-mgxs}
\end{subfigure}
\begin{subfigure}{0.42\textwidth}
  \centering
  \includegraphics[width=0.9\linewidth]{figures/unsupervised/features/assm-16/u238-capt/mean-spect-ind/geometry-2}
  \caption{}
  \label{fig:chap10-capt-mean-spect-ind-geom-2}
\end{subfigure}%
\begin{subfigure}{0.42\textwidth}
  \centering
  \includegraphics[width=0.9\linewidth]{figures/unsupervised/features/assm-16/u238-capt/mean-spect-ind/mgxs-2}
  \caption{}
  \label{fig:chap10-capt-mean-spect-ind-mgxs-2}
\end{subfigure}
\begin{subfigure}{0.42\textwidth}
  \centering
  \includegraphics[width=0.9\linewidth]{figures/unsupervised/features/assm-16/u238-capt/mean-spect-ind/geometry-3}
  \caption{}
  \label{fig:chap10-capt-mean-spect-ind-geom-3}
\end{subfigure}%
\begin{subfigure}{0.42\textwidth}
  \centering
  \includegraphics[width=0.9\linewidth]{figures/unsupervised/features/assm-16/u238-capt/mean-spect-ind/mgxs-3}
  \caption{}
  \label{fig:chap10-capt-mean-spect-ind-mgxs-3}
\end{subfigure}
\caption[Clustering of U-238 capture MGXS spectral indices]{Scatter plots of the pin-wise U-238 capture (group 1 of 2) \ac{MGXS} means ($x$) and spectral indices ($y$) for the 1.6\% enriched fuel assembly.}
\label{fig:chap10-mean-spect-ind}
\end{figure}

\clearpage

%%%%%%%%%%%%%%%%%%%%%%%%%%%%%
\subsection{Nuclide Fraction}
\label{subsec:chap10-nuclide-frac}

first paragraph: explain what it is
-divide pin-wise micro \ac{MGXS} for each nuclide/reaction/group by the corresponding total \ac{MGXS} for that nuclide/reaction/group trio
-Eqn.~\ref{eqn:chap10- nuclide-frac}
-explain what it indicates...???
-note $t$ designates the total collision reaction type

\begin{equation}
\label{eqn:chap10-nuclide-frac}
\tau_{x,i,k,g} = \frac{\hat{\sigma}_{x,i,k,g}}{\hat{\sigma}_{t,i,k,g}}
\end{equation}

second paragraph: visualizations
-look at single fuel assembly with GUI
-three plots: full, two different clusters selected
-indicate which energy group, nuclide, and reaction type

\begin{figure}[h!]
\centering
\begin{subfigure}{0.42\textwidth}
  \centering
  \includegraphics[width=0.9\linewidth]{figures/unsupervised/features/assm-16/geometry}
  \caption{}
  \label{fig:chap10-fiss-mean-nuc-frac-geom}
\end{subfigure}%
\begin{subfigure}{0.42\textwidth}
  \centering
  \includegraphics[width=0.9\linewidth]{figures/unsupervised/features/assm-16/u235-fiss/mean-nuc-frac/mgxs}
  \caption{}
  \label{fig:chap10-fiss-mean-nuc-frac-mgxs}
\end{subfigure}
\begin{subfigure}{0.42\textwidth}
  \centering
  \includegraphics[width=0.9\linewidth]{figures/unsupervised/features/assm-16/u235-fiss/mean-nuc-frac/geometry-2}
  \caption{}
  \label{fig:chap10-fiss-mean-nuc-frac-geom-2}
\end{subfigure}%
\begin{subfigure}{0.42\textwidth}
  \centering
  \includegraphics[width=0.9\linewidth]{figures/unsupervised/features/assm-16/u235-fiss/mean-nuc-frac/mgxs-2}
  \caption{}
  \label{fig:chap10-fiss-mean-nuc-frac-mgxs-2}
\end{subfigure}
\begin{subfigure}{0.42\textwidth}
  \centering
  \includegraphics[width=0.9\linewidth]{figures/unsupervised/features/assm-16/u235-fiss/mean-nuc-frac/geometry-3}
  \caption{}
  \label{fig:chap10-fiss-mean-nuc-frac-geom-3}
\end{subfigure}%
\begin{subfigure}{0.42\textwidth}
  \centering
  \includegraphics[width=0.9\linewidth]{figures/unsupervised/features/assm-16/u235-fiss/mean-nuc-frac/mgxs-3}
  \caption{}
  \label{fig:chap10-fiss-mean-nuc-frac-mgxs-3}
\end{subfigure}
\caption[Clustering of U-235 fission MGXS nuclide fractions]{Scatter plots of the pin-wise U-235 fission (group 2 of 2) \ac{MGXS} means ($x$) and nuclide fractions ($y$) for the 1.6\% enriched fuel assembly.}
\label{fig:chap10-mean-nuc-frac}
\end{figure}

\clearpage

\begin{figure}[h!]
\centering
\begin{subfigure}{0.42\textwidth}
  \centering
  \includegraphics[width=0.9\linewidth]{figures/unsupervised/features/assm-16/geometry}
  \caption{}
  \label{fig:chap10-capt-mean-nuc-frac-geom}
\end{subfigure}%
\begin{subfigure}{0.42\textwidth}
  \centering
  \includegraphics[width=0.9\linewidth]{figures/unsupervised/features/assm-16/u238-capt/mean-nuc-frac/mgxs}
  \caption{}
  \label{fig:chap10-capt-mean-nuc-frac-mgxs}
\end{subfigure}
\begin{subfigure}{0.42\textwidth}
  \centering
  \includegraphics[width=0.9\linewidth]{figures/unsupervised/features/assm-16/u238-capt/mean-nuc-frac/geometry-2}
  \caption{}
  \label{fig:chap10-capt-mean-nuc-frac-geom-2}
\end{subfigure}%
\begin{subfigure}{0.42\textwidth}
  \centering
  \includegraphics[width=0.9\linewidth]{figures/unsupervised/features/assm-16/u238-capt/mean-nuc-frac/mgxs-2}
  \caption{}
  \label{fig:chap10-capt-mean-nuc-frac-mgxs-2}
\end{subfigure}
\begin{subfigure}{0.42\textwidth}
  \centering
  \includegraphics[width=0.9\linewidth]{figures/unsupervised/features/assm-16/u238-capt/mean-nuc-frac/geometry-3}
  \caption{}
  \label{fig:chap10-capt-mean-nuc-frac-geom-3}
\end{subfigure}%
\begin{subfigure}{0.42\textwidth}
  \centering
  \includegraphics[width=0.9\linewidth]{figures/unsupervised/features/assm-16/u238-capt/mean-nuc-frac/mgxs-3}
  \caption{}
  \label{fig:chap10-capt-mean-nuc-frac-mgxs-3}
\end{subfigure}
\caption[Clustering of U-238 capture MGXS nuclide fractions]{Scatter plots of the pin-wise U-238 capture (group 1 of 2) \ac{MGXS} means ($x$) and nuclide fractions ($y$) for the 1.6\% enriched fuel assembly.}
\label{fig:chap10-mean-nuc-frac}
\end{figure}

\clearpage

%%%%%%%%%%%%%%%%%%%%%%%%%%%
\subsection{Total Fraction}
\label{subsec:chap10-tot-frac}

first paragraph: explain what it is
-divide pin-wise micro \ac{MGXS} for each nuclide/reaction/group by the corresponding total \ac{MGXS} for that nuclide/reaction/group trio
-Eqn.~\ref{eqn:chap10-tot-frac}
-explain what it indicates...???
-note that $I$ designates the total number of nuclides in a fuel pin

\begin{equation}
\label{eqn:chap10-tot-frac}
\omega_{x,i,k,g} = \frac{\hat{\sigma}_{x,i,k,g}}{\displaystyle\sum\limits_{i=1}^{I}\hat{\sigma}_{t,i,k,g}}
\end{equation}

second paragraph: visualizations
-look at single fuel assembly with GUI
-three plots: full, two different clusters selected
-indicate which energy group, nuclide, and reaction type


%%%%%%%%%%%%%%%%%%%%%%%%%%%%%%
\section{Feature Selection}
\label{sec:chap10-litmus}

%%%%%%%%%%%%%%%%%%%%%%%%%%%%%%
\section{Target Selection}
\label{sec:chap10-feature-select}

%%%%%%%%%%%%%%%%%%%%%%%%%%%%%%%%%%%
\section{Feature Transformation}
\label{sec:chap10-feature-transform}



%%%%%%%%%%%%%%%%%%%%%%%%%%%%
\section{Litmus Tests}
\label{sec:chap10-litmus}

-which \ac{MGXS} need to be clustered


%%%%%%%%%%%%%%%%%%%%%%%%%%%%%%%
\section{Clustering Algorithms}
\label{subsec:chap10-clustering}

-very brief overview
-refer to literature for more info
-used canned algorithms in scikit-learn \texttt{sklearn.cluster}
-while there are many, many algorithms out there, only a few selected for discussion here
  -one concern is scalability to large number of samples

%%%%%%%%%%%%%%%%%%%%%%%%%%%%%%%%%
\subsection{$k$-Means Clustering}
\label{subsec:chap10-kmeans}

-introduce scheme
-algorithmic
-initialization
-model selection

%%%%%%%%%%%%%%%%%%%%%%%%%%%%%%%%%%%%
\subsection{Hierarchical Clustering}
\label{subsec:chap10-agglomerative}

-introduce scheme
-algorithmic
-initialization
-model selection

%%%%%%%%%%%%%%%%%%%%%%%%%%%%
\subsection{DBSCAN}
\label{subsec:chap10-dbscan}

-introduce scheme
-algorithmic
-initialization
-model selection

%%%%%%%%%%%%%%%%%%%%%%%%%%%%%%%%%
\section{Gaussian Mixture Models}
\label{sec:chap10-gmms}

-introduce scheme
-algorithmic
-initialization
-intro fact that it is not scalable - not evaluated here
-mention dirichlet process mixture models
-model selection criteria are nice
-model selection


%%%%%%%%%%%%%%%%%%%%%%%%%%%%%%%%%%%%%%%%%%%%%%%%%%%%%%%%%%%%%%%%%%%%%%%%%%%%%%%
\section{Approaches to Clustering}
\label{sec:chap10-cluster}

\begin{itemize}[noitemsep]
  \item dataset restrictions
  \begin{itemize}[noitemsep]
    \item \textbf{``pinch'' clustering} - a \textit{single} nuclide/group/reaction triplet
    \item \textbf{``combined'' clustering} - one or more nuclide/group pairs \textit{together}
    \item \textbf{``local'' clustering} - one or more nuclide/group pairs \textit{separately}
  \end{itemize}
  \item clustering algorithms
  \begin{itemize}[noitemsep]
    \item discriminative clustering ($k$-Means++, Agglomerative, etc.)
    \item generative clustering (Gaussian Mixtures, Dirichlet Processes)
  \end{itemize}
  \item ``regression-informed'' clustering
  \begin{itemize}[noitemsep]
    \item use a regressor to predict ``smoothed'' of target variable(s)
    \begin{itemize}[noitemsep]
      \item decision tree regression
      \item ensemble (boosting, random forest) regression
      \item Gaussian Process regression
    \end{itemize}
    \item cluster predictions produced by regressor
    \item advantage is that objective function \textit{may} be more appropriate
  \end{itemize}
\end{itemize}