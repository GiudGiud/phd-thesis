\chapter{Conclusions}
\label{chap:conclusions-future-work}

This research was motivated by a desire to obtain Monte Carlo-quality solutions with computationally efficient deterministic neutron transport methods. This thesis approached this objective by employing continuous energy \ac{MC} neutron transport simulations to generate accurate \ac{MGXS} for fine-mesh deterministic transport methods. The new methods developed in this thesis were designed to accelerate the convergence of \ac{MGXS} tallied on full-core, fine-spatial meshes by leveraging the phenomenon of pin-wise \ac{MGXS} clustering. These methods reduce the computational burden of \ac{MC}-based \ac{MGXS} generation techniques, positioning it as a reactor agnostic alternative to today's deterministic methods which rely on engineering approximations.

This chapter concludes by evaluating the results presented in this thesis with respect to this over-arching objective, and by defining a roadmap of the milestones which must be addressed in the future to design a production-ready methodology for \ac{MC}-based reactor agnostic fine-mesh \ac{MGXS} generation. The key results demonstrated by this thesis are discussed in Sec.~\ref{sec:chap12-conclusions}, the author's contributions to the field of reactor physics are outlined in Sec.~\ref{sec:chap12-contributions}, and opportunities for future research are summarized in Sec.~\ref{sec:chap12-future-work}.

%-notes from bryan's thesis conclusions
%  -conclusions
%    -first sentence: something that i did
%    -remaining sentences: summarize / quantify conclusions
%    -final sentence: what does it all mean / impact statement
%  -future work
%    -break up into multiple sections
%    -first sentence: problem / approximation
%    -remaining sentences: how this impacted my results
%    -final sentence: possible solutions  

%-look at my executive summary - perhaps include a one-page summary of bullets???

%%%%%%%%%%%%%%%%%%%%%%%%%%%%%%%%%%%%%%%%%%%%%%%%%%%%%%%%%%%%%%%%%%%%%%%%%%%%%%%%
\section{Summary of Work}
\label{sec:chap12-conclusions}

This thesis replaces the multi-level framework used by traditional \ac{MGXS} generation methods with a single whole-core MC calculation as summarized in Sec.~\ref{subsec:chap12-single-step}. Sec.~\ref{subsec:chap12-approx-error} highlights the results which quantify different sources of approximation error in multi-group methods with \ac{MGXS} generated by \ac{MC}. The clustering of pin-wise \ac{MGXS} due to spatial self-shielding effects is recapped in Sec.~\ref{subsec:chap12-mgxs-clustering}. Finally, Sec.~\ref{subsec:chap12-homogenization-schemes} reviews the performance of the new pin-wise spatial homogenization schemes to model \ac{MGXS} clustering.

%%%%%%%%%%%%%%%%%%%%%%%%%%%%%%%%%%%%%%%%%%%%%%%%%%%%%%%
\subsection{A Single-Step Approach for MGXS Generation}
\label{subsec:chap12-single-step}

Today's state-of-the-art methods for \ac{MGXS} generation use a multi-level approach in space, energy and angle to account for self-shielding effects while approximating the flux used to collapse cross sections (see Sec.~\ref{subsec:chap2-mgxs-lib-std-approach}). The flux approximations used in multi-level schemes are based on engineering prescriptions for specific reactor configurations and spectra and are not generalizable to new core designs. In addition, \ac{MC}-based \ac{MGXS} generation methods to date have retained the multi-level geometric framework to tabulate MGXS for individual reactor components for subsequent use in full-core multi-group calculations. Although the use of MC within a multi-level scheme eliminates the need to approximate the flux in energy, it does not account for spatial self-shielding effects throughout a reactor core. This thesis replaces the multi-level framework in place of a full-core \ac{MC} calculation which simultaneously accounts for all energy and spatial effects with a single simulation of the complete heterogeneous geometry.

This work required the development of a ``simulation triad'' encompassing three primary simulation codes. The OpenMC Monte Carlo code was utilized to generate multi-group cross sections on high-spatial-fidelity tally meshes. Second, the \ac{MGXS} were used by the OpenMOC code for deterministic multi-group transport calculations. Finally, the OpenCG library enabled the processing and transfer of tally data on combinatorial geometry meshes between OpenMC and OpenMOC. The simulation triad directly modeled all energy and spatial self-shielding effects with a single full-core OpenMC calculation of the complete heterogeneous geometry to generate \ac{MGXS} for use in OpenMOC.

%%%%%%%%%%%%%%%%%%%%%%%%%%%%%%%%%%%%%%%%%%%%%%%%%%%%%%%
\subsection{Approximation Error in Multi-Group Methods}
\label{subsec:chap12-approx-error}

This thesis investigated approximation error in multi-group deterministic methods even when the ``true'' flux from \ac{MC} is used to collapse cross sections for \ac{MGXS} generation. The case studies presented in Chap.~\ref{chap:biases} quantifed the eigenvalue bias between OpenMC and OpenMOC by separately varying the angular discretization, flat source region discretization and energy group structure while holding all other variables constant. A systematic bias on the order of -300 \ac{pcm} was demonstrated for a representative 2D \ac{PWR} fuel pin cell with converged angular and spatial discretization schemes and 70-group cross sections. The iso-in-lab scattering feature was employed in OpenMC to enable direct comparisons with OpenMOC which assumes an isotropic scattering source, but this only mitigated the bias by approximately a third, or 100 \ac{pcm}. The remaining -200 \ac{pcm} was shown to be caused by over-predictions of U-238 capture rates in resonance energy groups. Furthermore, as the energy group structures were refined, the errors were magnified for those energy groups encompassing the three lowest lying U-238 capture resonances.

%-Surprisingly, the bias grew in magnitude (but decreased in value) with more energy groups.
%-the eigenvalue bias varied substantially with the \ac{FSR} discretization, but was invariant to the spatial tally mesh used to generate \ac{MGXS}. 
%-largest for interior zones of fuel pin, smallest for outer rim of fuel pin nearest to the clad/moderator

In collaboration with Gibson~\cite{gibson2016thesis}, the U-238 capture rate errors and resulting eigenvalue bias were demonstrated to be the result of the flux separability approximation (Sec.~\ref{subsec:chap2-angle}) which permits the use of constant-in-angle \ac{MGXS}. In particular, it was shown that the flux separability approximation is not generally valid for U-238 capture since the angular neutron flux leaving a fuel pin is much more self-shielded in resonance groups than it is for the neutron flux entering the fuel pin from the moderator. Since the angular dependence of \ac{MGXS} is not typically modeled in deterministic multi-group transport codes, an equivalence scheme is needed to correct for the loss of angular information when cross sections are collapsed with the scalar rather than the angular flux. 

Chap.~\ref{chap:sph} investigated the use of SuPerHomog\'{e}n\'{e}isation (SPH) factors as one possible equivalence scheme to enforce reaction rate preservation between OpenMC and OpenMOC. The \ac{SPH} factor approach uses a reference fixed source to correct the total \ac{MGXS} to preserve reaction rates between fine and coarse mesh methods. The \ac{SPH} factors systematically eliminated the few percent reaction rate errors in U-238 resonance groups, and correspondingly reduced the eigenvalue bias from -200 \ac{pcm} to approximately 10 \ac{pcm}. In particular, the \ac{SPH} factors reduced the total \ac{MGXS} in resonance groups by 1 -- 3\% to resolve errors of the same magnitude in each of the lowest lying U-238 capture resonance groups. Notwithstanding these results, it is unclear if a generalizable scheme based upon \ac{SPH} factors may be used to correct for the flux separability approximation. Future work should further investigate equivalence methods which adequately preserve reaction rates in fine-mesh transport methods with \ac{MC}-generated \ac{MGXS}.

%-depending on energy group structure and \ac{FSR} discretization

%-segue into pin-wise spatial homogenization error

%%%%%%%%%%%%%%%%%%%%%%%%%%%%%%%%%%%%%%%%
\subsection{Clustering of Pin-Wise MGXS}
\label{subsec:chap12-mgxs-clustering}

%This improvement in accuracy comes at the computational expense of converging reaction rate and flux tallies to acceptably low uncertainties.

%The single-step approach harnesses the accuracy of \ac{MC} to generate reactor agnostic \ac{MGXS} for more computationally efficient, deterministic multi-group transport codes.

%As discussed in Secs.~\ref{subsec:chap12-mgxs-clustering} and~\ref{subsec:chap12-homogenization-schemes}, new pin-wise spatial homogenization schemes were developed to accelerate the convergence of the \ac{MGXS} tallies in each fine-mesh region in order for MC to be practical for reactor agnostic fine-mesh MGXS generation.

first paragraph: clustering exists?
-exploration of clustering motivated by idea that pins with similar neighboring heterogeneities have similar microscopic \ac{MGXS}
-population variance of pin-wise \ac{MGXS} increases due to presence of spatial heterogeneities
  -depends on sensitivity to spatial self-shielding effects for each nuclide, reaction type, and energy group
  -on the order of 1 -- 2\% for group 1/2 U-238 capture and group 2/2 U-235 fission
-distributions of pin-wise \ac{MGXS} plotted as histograms and quantile-quantile plots show distinct clustering due to groups of fuel pins with similar neighboring heterogeneities, and hence similar spatially self-shielded flux spectra
  -distributions look very different for each benchmark due to different heterogeneities
  -clustering is more clearly defined for thermal U-235 fission \ac{MGXS}
  -pop. var. of pin-wise \ac{MGXS} about to mean mean is 2.5\% for U-238 capt, only 1\% for U-235 fiss
  -clustering effects are more important to model for U-238 capture than U-235 fission
    -though they may not be as easily identified from ``noisy'' \ac{MC} tally data

second paragraph: define null, degenerate schemes
-devised two pin-wise spatial homogenization schemes to quantify impact of modeling or neglecting \ac{MGXS} clustering
-define null, degenerate homogenization
  -null homogenization
    -all fuel pins form a single cluster
    -a single \ac{MGXS} is assigned to all pins of a given enrichment
    -does not distinguish spatial self-shielding effects experienced by different fuel pins
    -least accurate but MC tallies converge quickly (track density is large within each spatial tally zone)
  -degenerate homogenization
    -each fuel pin is a unique cluster
    -a unique \ac{MGXS} is assigned to each individual fuel pin instance
    -distinguishes spatial self-shielding effects experienced by each fuel pins
    -most accurate but MC tallies converge slowly (track density is small within each spatial tally zone)

third paragraph: quantified gap b/w null and degenerate schemes
-quantified clustering of \ac{MGXS} in each fuel pin due to spatial self-shielding effects
  -i.e., moderation from neighboring guide tubes, water reflectors, etc.
-six \ac{PWR} benchmarks with varying spatial self-shielding effects
-mention energy group structures
  -2, 8 and 70-group structures were considered
  -energy group structure has a large impact on the results
  -really need 70-group structure for accurate results (i.e., <10\% mean error in capture rates) 
-three metrics:
  -eigenvalues: eigenvalues are not impacted by clustering
  -fission rates: fission rates are only marginally impacted, max error reduced by 5 -- 20\%
  -capture rates: capture rates are much more sensitive
    -degenerate reduces both max and mean error 2--4$\times$
    -largest errors for fuel pins near control rod guide tubes and along assembly-assembly and assembly-reflector interfaces
    -need to use degenerate scheme to reduce errors to level of fission rate errors
    -degenerate scheme ``smooths'' U-238 capture rates (i.e., reduces largest errors)
    -i.e., dominant factor in U-238 capture rate error is due to spatial self-shielding approx.

fourth paragraph: strong conclusion
-MGXS clustering is very important for modeling U-238 capture rates
  -and therefore to predict Pu-239 production over time
  -as U-235 is burned more of the overall fissions occur in Pu-239 that is bred
  -a large fraction of fission in LWRs comes from Pu-239 at EOL  
  -hence accurate fission rate predictions in depletion calcs depend on accurate local models of Pu-239 buildup over time
  -many adv. reactors are designed to convert U-238 conversion to Pu-239, including SFRs and some MSRs
    -many designs have fast spectra which are less sensitive to local spatial self-shielding spectral effects
    -nevertheless, MGXS clustering must be modeled for highly accurate U-238 capture preductions in reactors with lots of heterogeneities
      -including absorbing materials for power shaping and reflectors to reduce neutron leakage
-however, modeling MGXS clustering with degenerate scheme is impractical
  -uses very fine spatial tally mesh with low particle track density per spatial tally zone

%%%%%%%%%%%%%%%%%%%%%%%%%%%%%%%%%%%%%%%%%%%%%%%%%%%%
\subsection{Pin-Wise Spatial Homogenization Schemes}
\label{subsec:chap12-homogenization-schemes}

first paragraph: motivate new schemes
-developed new spatial homogenization schemes
  -approach accuracy of degenerate scheme by accounting for spatial self-shielding effects
  -approach MC convergence of null scheme by homogenizing over large spatial tally zones
-need to homogenize across many fuel pins to reduce the MGXS statistical uncertainties and batchwise deviations
-minimize number of spatial tally zones to minimize statistical uncertainties
-modeling MGXS clustering to enable highly accurate U-238 capture predictions

second paragraph: introduce schemes
-Local Neighbor Symmetry (LNS) scheme:
  -engineering-based approach to cluster fuel pins
  -nearest neighbor-like analysis of combinatorial geometry
  -define ``geometric template'' of fuel pins to be homogenized together
-Inferential \ac{MGXS} (\textit{i}\ac{MGXS}) scheme
  -uses unsupervised machine learning algorithms to identify clusters in ``noisy'' \ac{MC} tally data
  -no knowledge of the reactor geometry
  -uses data processing pipeline to analyze tally data
  -designed to model arbitrary spatial self-shielding effects from neighboring heterogneities
  -use machine learning methods designed to find clusters in data
    -feature extraction, feature selection, dimensionality reduction, predictor training, model selection and spatial homogenization stages

third paragraph: evaluated new schemes  
-evaluated for each of the six heterogeneous \ac{PWR} benchmarks
-clustered geometries:
  -LNS distinguishes fuel pins with neighboring \acp{CRGT} and/or \acp{BP}, but not those at interfaces
  -\textit{i}\ac{MGXS} distinguishes pins irregardless of neighbors
    -hierarchical clustering
    -first few clusters distinguish pins as if in an repeating lattice (i.e., close neighbors)
    -later clusters distinguish pins along interfaces    
-eigenvalues not affected by either \ac{LNS} or \textit{i}\ac{MGXS} (preserved to within 10 \ac{pcm})
-fission rates approach that of degenerate scheme, but not impacted much to beging with
-capture rates approach that of degenerate scheme
  -\ac{LNS} performed as well or slightly better than degenerate scheme for single assm benchmarks
  -\ac{LNS} fails to distinguish inter-assembly and assemlby-reflector interfaces
    -errors systematically / particularly large for fuel pins near baffle/reflector
    -but should note that these pins have the smallest reaction rate magnitudes due to neutron leakage
  -\textit{i}\ac{MGXS} was evaluated for different numbers of clusters
    -accuracy greatly improved with just a few (order 10) clusters, with diminishing returns for more clusters
    -for the same number of materials, accuracy was less than \ac{LNS} for single assembly benchmarks
    -accuracy improved above \ac{LNS} for benchmarks with assm-assm and assembly-reflector interfaces
      -distinguished fuel pins along interfaces which \ac{LNS} was unable to do
      
fourth paragraph: convergence
-evaluated for the five assembly and 2$\times$2 colorset benchmarks
-unable to evaluate convergence for quarter core \ac{BEAVRS} due to computational restraints
-eigenvalues:
  -all schemes converge to same eigenvalue bias with 10$^8$ particle histories
  -eigenvalues fluctuate on the order of several hundred \ac{pcm} with less than 10$^{9}$ histories
    -need at least 100,000,000 histories for reliable / stable eigenvalue calcs for such geometries
      -more histories needed for larger geometries
-capture rates:
  -null homogenization converges with 10$^6$ histories
  -degenerate homogenization isn't even fully converged with 10$^9$ histories
    -was unable to fully converge, couldn't run enough particle histories
  -\ac{LNS} and \textit{i}\ac{MGXS} converge / stabilize faster than degenerate or a reference \ac{MC} calculation
    -mean errors converge faster (5 -- 20+$\times$ depending on the number of clusters)
    -max errors may converge faster but if few clusters are used
      -max error convergence is limited by the smallest cluster
      -i.e., a cluster may be needed to model the few fuel pins at the corners of a geometry adjacent to a water/baffle
        -these are the pins where the errors are the largest
        -but with so few pins assigned to the cluster, the track density and therefore acceleration is minimal

fifth paragraph: hope for future
-related back to original goals: reactor agnostic, single-step, convergence rate, accuracy of \ac{MC}, ...
-look at writeup in exec summary
-focus on model selection criteria in future work section
-need to talk about differences b/w \ac{LNS} and \textit{i}\ac{MGXS}
-talk about degree of customization required for engineering-based schemes like \ac{LNS}
-number of materials:
  -\ac{LNS} requires 400+ of materials to model 12,000+ fuel pin instances in \ac{BEAVRS}
    -analysis of geometry is based on cell/universe/lattice IDs
-MGXS uncertainties shown to converge to a prescribed level faster for null than degenerate
  -convergence rate is the same irregardless of the number of clusters
  -uncertainty reduced according to the number of fuel pins assigned to each spatial homogenization set
  -factor proportional to inverse square root of number of spatial homogenization zones
  -faster for \ac{LNS}, \textit{i}\ac{MGXS}
-The results in this thesis point to the potential for iMGXS as a means to efficiently generate MGXS with reactor agnostic MC calculations of the complete heterogeneous geometry in a single step.
-iMGXS accounts for spatial self-shielding spectral effects without engineering heuristics based on the reactor core configuration.
-This thesis demonstrates a path forward for MGXS generation with reactor agnostic MC for computationally efficient deterministic transport codes with the iMGXS scheme

The iMGXS scheme enables deterministic reactor physics simulations to produce accurate results from MGXS generated by MC faster than would be possible with a direct calculation with MC. Furthermore, the iMGXS scheme was shown to be advantageous over geometric heuristic approaches such as LNS which must be highly customized for specific types of core geometries. This thesis demonstrated the promise for iMGXS as a means to efficiently generate MGXS with reactor agnostic MC calculations of the complete heterogeneous geometry in a single step.

%%%%%%%%%%%%%%%%%%%%%%%%%%%%%%%%%%%%%%%%%%%%%%%%%%%%%%%%%%%%%%%%%%%%%%%%%%%%%%%%
\section{Contributions}
\label{sec:chap12-contributions}

-include use of \ac{SPH} factors
-discuss replacing multi-level scheme
-itemize which approximation errors were investigated
  -MOC: angular discretization, \ac{FSR} discretization, energy discretization
  -MGXS: isotropic in lab approximation, energy discretization, spatial homogenization zones (by FSR within each pin, across many ``clustered pins'')
-have bullet specifically for pin-wise spatial homogenization schemes
 -include individual sub-bullets for null, degenerate, LNS and iMGXS

\begin{itemize}
\item Generated MGXS with a single MC simulation of a complete heterogeneous reactor geometry for use fine-mesh deterministic transport calculations.
\item Implemented a simulation triad that generated MGXS with OpenMC, used MGXS in deterministic OpenMOC calculations, and transferred MGXS between codes and built clustered geometries with OpenCG.
\item Developed LNS and iMGXS pin-wise spatial homogenization schemes to account for spatial self-shielding effects on the MGXS in each fuel pin while simultaneously accelerating MC tally convergence.
\item Quantified approximation error between reference OpenMC and deterministic OpenMOC calculations of the eigenvalues, pin-wise fission rates and U-238 capture rates for each homogenization scheme.
\item Demonstrated that LNS and iMGXS spatial homogenization schemes require fewer MC histories to generate MGXS and converge deterministic calculations to a given accuracy than an equivalent MC calculation.
\end{itemize}

%%%%%%%%%%%%%%%%%%%%%%%%%%%%%%%%%%%%%%%%%%%%%%%%%%%%%%%%%%%%%%%%%%%%%%%%%%%%%%%%
\section{Future Work}
\label{sec:chap12-future-work}

first paragraph: motivation and outline
-this is an exploratory thesis
  -generate MGXS with MC for fine-mesh transport
  -single-step process
  -diagnose approx. error
-thesis identified many challenges and opportunities
-this section itemizes the author's assessment of what should be done and in what order
  -Sec.~\ref{subsec:further-imgxs} - improve \textit{i}\ac{MGXS}
  -Sec.~\ref{subsec:improve-mc-methods} - improve \ac{MC} methods to generate \ac{MGXS}

This thesis identified several issues which must be investigated in the future in order for \textit{i}MGXS to be useful in a production code setting. First, a systematic evaluation of the types of features, as well as algorithms for dimensionality reduction and clustering, which may be used in \textit{i}MGXS should be performed to better understand their impact on the resulting clustered geometry. A future study should also score each configuration of the \textit{i}MGXS data processing pipeline based on how many MC particle histories each scheme requires to accurately identify MGXS clusters from ``noisy'' MC tally data. 


%%%%%%%%%%%%%%%%%%%%%%%%%%%%%%%%%%%%%%%%%%%%%%%%%%%%%%%%%%%%
\subsection{Further Evaluation of the \textit{i}MGXS Scheme}
\label{subsec:chap12-further-imgxs}

first paragraph: motivate and outline
-this work only scratched the surface of what is possible
-Sec.~\ref{subsubsec:imgxs-noisy-mc-data} should consider using 
-Sec.~\ref{subsubsec:optimize-imgxs} investigate various configurations of the \textit{i}MGXS scheme
-Sec.~\ref{subsubsec:optimize-simulation-triad} speeding up the various components of the simulation triad

%%%%%%%%%%%%%%%%%%%%%%%%%%%%%%%%%%%%%%%%%%%%%%%%%%%%%%%%%%%%%%%%%%%%
\subsubsection{Evaluate \textit{i}MGXS Scheme with Noisy Tally Data}
\label{subsubsec:chap12-imgxs-noisy-mc-data}

first paragraph: 
-recall that convergence results in Sec.~\ref{sec:chap11-converge} used ``fully converged'' MC tally data
  -gave an upper bound for how quickly results converge in best case scenario
-what: repeat convergence studies with noisy ``MC'' tally data from statepoints at each batch
-why:  need to determine how quickly OpenMOC results stabilize 
-what would we expect??
  -errors will likely be larger for fewer batches
    -fuel pins more likely to be assigned to the wrong clusters
  -eventually errors will asympotically approach those shown for the upper bound
  -studies needed to determine how quickly the clustering predictions converge
-this would revise our estimates for the expected runtime in Sec.~\ref{subsec:chap11-runtimes}
  -wouldn't reduce but could potentially increase the runtime estimates
  -also possible that clustering predictions ``converge'' or ``stabilize'' before the OpenMOC solutions
-should also perform convergence studies with BEAVRS full core
-Score iMGXS configurations by number of MC histories needed to accurately identify MGXS clusters from ``noisy'' MC tally data

%%%%%%%%%%%%%%%%%%%%%%%%%%%%%%%%%%%%%%%%%%%%%%%%%%%%%%%%%%%%%%%%%%%%
\subsubsection{Optimize the \textit{i}MGXS Data Processing Pipeline}
\label{subsubsec:chap12-optimize-imgxs}

first paragraph: 
-Systematically evaluate various configurations of the iMGXS data processing pipeline:
-Feature extraction – Engineer new, better features?
-Feature selection – Develop techniques to select “best” features?
-Dimensionality reduction – Use fewer features for clustering?
-Clustering models – Which use the fewest clusters to reduce errors?
-Model selection – Which method best correlates with error reduction? none of them robustly works for my case studies!
-evaluate each with respect to score with noisy data Sec.~\ref{subsubsec:imgxs-noisy-mc-data}

%%%%%%%%%%%%%%%%%%%%%%%%%%%%%%%%%%%%%%%%%%%%%%%%%%%%%%%%%%%%%%%%%%%%%%
\subsubsection{Optimize Computational Performance of Simulation Triad}
\label{subsubsec:chap12-optimize-simulation-triad}

first paragraph:
-speed up OpenMOC full core: \\
  -find a way use quarter assembly CMFD mesh \\
    -use mesh only over fissile zones??
  -linear source to reduce number of spatial zones \\
  -vectorize transport solver over energy groups \\
  -3D \ac{MOC}
-speed up OpenMC:
  -particle tracking times in Tab.~\ref{table:chap11-openmc-rates} were 3$\times$ slower w/ tallies
  -numerous lingering issues, including: 

%%%%%%%%%%%%%%%%%%%%%%%%%%%%%%%%%%%%%%%%%%%%%%%%%%%%%%
\subsection{Improved Methods to Generate MGXS with MC}
\label{subsec:chap12-improve-mc-methods}

first paragraph: motivation
-recall that relatively little work to date to use \ac{MC} to generate \ac{MGXS} for fine-mesh transport codes as compared to coarse mesh diffusion codes
-methods lie along two axes:
  -improve tally efficiency (first one)
  -improve equivalence b/w \ac{MC} and multi-group deterministic codes (last three)
-Sec.~\ref{subsubsec:tally-estimators} to improve tally estimators
-Sec.~\ref{subsubsec:transport-mgxs} to model anisotropic scattering
-Sec.~\ref{subsubsec:angular-dependent-mgxs} equivalence method for scalar flux-weighted \ac{MGXS}
-Sec.~\ref{subsubsec:multi-physics-mgxs} multi-physics feedback in \ac{MGXS}

Future work should also consider employing \textit{i}MGXS in calculations with thermal-hydraulic feedback and nuclide depletion where the moderator density, fuel temperature and burnup will be needed as features to predict MGXS clustering. In addition, new methods must be developed to compute transport-corrected MGXS with MC which appropriately account for anisotropic scattering, thereby eliminating the isotropic in lab scattering approximations used throughout this work. Finally, this thesis may serve as inspiration to employ machine learning algorithms to automate the time-consuming process of selecting reduced energy group structures and their optimal energy group boundaries for MGXS.

%%%%%%%%%%%%%%%%%%%%%%%%%%%%%%%%%%%%%%%%%%%%%%%%%%%%%%%
\subsubsection{Improved Reaction Rate Tally Estimators}
\label{subsubsec:chap12-tally-estimators}

first paragraph: motivate and outline problem
-recall mixture of analog and track-length tally estimators employed to generate \ac{MGXS} with OpenMC in Sec.~\ref{subsubsec:chap3-tally-types-summary}
  -analog estimators used for group constants which depend on the outgoing energy
    -i.e., scattering matrix and fission spectrum
  -all other constants use more efficient track-length tally estimators
-recall separation between eigenvalues in Sec.~\ref{subsec:chap11-eigenvalue-converge} for few particle histories
  -all clustering algorithms eventually converge with enough particle histories
  -hypothesize that gap due to rxn rate imbalance (not preserved) due to mixture of tally estimators  
  -improved estimator may improve convergence to identical eigenvlaue

second paragraph: possible solutions
-cite Nelson NDPP~\cite{nelson2014improved} as one option
-simple heuristics may be employed for hydrogeneous systems to greatly improve tallying efficiency

%%%%%%%%%%%%%%%%%%%%%%%%%%%%%%%%%%%%%%%%%%%%%%%%%%
\subsubsection{Account for Anisotropic Scattering}
\label{subsubsec:chap12-transport-mgxs}

first paragraph: motivate and outline problem
-recall use of OpenMC's iso-in-lab scattering feature in Sec.~\ref{subsec:chap4-iso-in-lab}
  -enable ``apples-to-apples'' comparisons between OpenMC and OpenMOC
-recall impact on pin-wise reaction rates
  -compare pin-wise fission and U-238 capture rate distributions:
    -with iso-in-lab scattering: Figs.~\ref{fig:chap7-fiss-rates-full-core} and~\ref{fig:chap7-capt-rates-full-core}
    -with anisotropic scattering: Fig.~\ref{fig:benchmarks-beavrs-aniso}
  -iso-in-lab approx. causes fission rates to peak at approx. 2.3 -- 2.9 as compared to 1.5 -- 1.6
    -due to artificial scattering of neutrons from reflector back into core
    -reaction rates peak in assemblies near corners of the core nearest to the reflector
-need to model anisotropic scattering in multi-group deterministic code to solve ``correct'' or physical problem

second paragraph: possible solutions
-use full anisotropic scattering matrices in MOC
  -unclear what expansion order is necessary
  -this is the most theoretically rigorous approach
  -downside is that this is computationally expensive
    -more expensive to 1) compute scattering source and 2) tally flux moments
  -cite Nelson's work???
-improved transport cross sections
  -recall Sec.~\ref{subsec:chap2-transport-corr}
  -modify the total cross section in an attempt to produce an equivalent solution 
  -advantage is that the computational burden is no larger than it is currently
  -cite Liu's work???

%%%%%%%%%%%%%%%%%%%%%%%%%%%%%%%%%%%%%%%%%%%%%%%%%%%%%%%%%%%%%
\subsubsection{Equivalence Method for Angular-Dependent MGXS}
\label{subsubsec:chap12-angular-dependent-mgxs}

first paragraph: motivate and outline problem
-recall eigenvalue bias in simple 1D slab and 2D fuel pin benchmarks in Chap.~\ref{chap:biases}
-shown to be the result of errors in U-238 capture resonance groups
  -over-predicts capture by up to 2\% in group 27/70 which contains the 6.67 eV resonance 
  -trend was exacerbated for smaller energy groups
-cite nate's thesis~\cite{gibson2016thesis}
  -effect is the result of using the scalar rather than the angular flux to collapse \ac{MGXS}
-recall use of \ac{SPH} factors to correct this in Chap.~\ref{chap:sph}

second paragraph: possible solutions
-recall \ac{SPH} factors
  -equivalence factor use to preserve reaction rates in fissile zones given a fixed source in each region
  -\ac{SPH} factors are intrinsically coupled to spatial discretization
  -\ac{SPH} factors does not distinguish between sources of approximation error, and simply tries to preserve a reference reaction rate solution
    -all errors are indiscriminately treated with \ac{SPH}
      -angular, spatial, energy discretization, scattering kernel, etc. 
  -why it is challenging: 
    -requires knowledge of the fixed sources
    -is iterative and hence computationally expensive
  -possible changes: may be able to tabulate a set of \ac{SPH} factors for a specific energy group structure and spatial discretization
  -unclear how much the factors depend on enrichment, spatial discretization, etc.
-Consistent-P or BHS~\cite{bell1967transport}
  -expand collision term with Legendre moments, transfer to scattering kernel
-angular-dependent \ac{MGXS}
  -separate \ac{MGXS} for different angles in deterministic code
  -angular dependence is intrinsically coupled to spatial discretization
  -why it is challenging:
    -burdensome to track extra data in deterministic code
    -burdensome to tally extra data in \ac{MC}
    -must \textit{a priori} know or approximate angular dependence and its relationship with spatial mesh
  -possible changes: ``jump'' conditions, i.e. coarse angular discretization
    -still requires some knowledge of spatial discretization mesh

%%%%%%%%%%%%%%%%%%%%%%%%%%%%%%%%%%%%%%%%%%%%%%%%%%%%%%%%%%
\subsubsection{Account for Multi-Physics Feedback in MGXS}
\label{subsubsec:chap12-multi-physics-mgxs}

first paragraph: motivate and outline problem
-all calculations here were for steady-state scenarios with fresh \ac{PWR} fuel without T-H feedback
-\ac{MC} calculations used to generate \ac{MGXS} in future may:
  -model temperature-dependent cross sections (i.e., multipole)
  -model moderator density and/or temperature gradients
  -model fuel burnup with unique appropriate isotopic vectors in each fuel pin

second paragraph: possible solutions
-may be possible to use new features to model fuel burnup
  -i.e., fuel temperature, density of nearby moderator (T-H calcs), fuel burnup (depletion calcs)
-may be able to train machine learning regression models to predict \ac{MGXS} with these features
  -i.e., decision tree or support vector regression
  -challenging since this will not necessarily preserve global reactivity
    -must use integrated flux and reaction rates within fuel pin of interest rather than predicted \ac{MGXS} itself

%-multi-physics applications: moderator density, fuel temperature, burnup, etc. as features \\

%%%%%%%%%%%%%%%%%%%%%%%%%%%%%%%%%%%%%%%%%%%%%%%%%%%%
\subsection{Inspiration for New Research Directions}
\label{subsec:chap12-inspiration}

first paragraph: research may inspire new areas of research
-ML might be useful in other simulation application areas
  -large scale parameter selection for simulations
  -replace engineering / human judgement with data-informed decision-making
    -more flexible/extensible and accurate for parameter regimes (i.e., reactor designs) for which there may be little prior experience / heuristics
-could be closely related topics, such as choosing energy group boundaries
  -machine learning to optimize energy group structures \\
    -can \textit{i}MGXS reduce the number of necessary energy groups?? \\
-or could be in very different disciplines where a similar tradeoffs are made between simulation accuracy and speed