	\chapter{Unsupervised Clustering for Spatial Homogenization}
\label{chap:unsupervised}

The preceding chapter illustrated the clustering of pin-wise \ac{MGXS} as a result of spatial self-shielding effects. It was shown that \ac{MGXS} clustering must be appropriately modeled to accurately resolve pin-wise U-238 capture rates. The \ac{LNS} spatial homogenization scheme was introduced to predict \ac{MGXS} clustering with a geometric template-like approach. The \ac{LNS} scheme was shown to achieve the same level of accuracy as degenerate homogenization while simultaneously accelerating the \ac{MC} tally convergence rate for simple benchmark problems. However, the \ac{LNS} scheme suffered from its inability to adapt to predict \ac{MGXS} clustering in geometries with water reflectors and steel baffles. In addition, the \ac{LNS} scheme did not scale well with the complexity of the core geometry, resulting in a large number of materials and thus an under-accelerated convergence rate. This chapter introduces an adaptable and scalable alternative to \ac{LNS} which uses unsupervised statistical learning methods to predict \ac{MGXS} clustering.

The novel homogenization methodology presented here -- referred to as \textit{intelligent \ac{MGXS}} (\textit{i}\ac{MGXS}) spatial homogenization -- seeks to achieve the overarching goal of this thesis: to obtain Monte Carlo quality solutions with computationally efficient deterministic transport methods. The goal of \textit{i}\ac{MGXS} is to use algorithms developed by the machine learning community to infer \ac{MGXS} clusters directly from \ac{MC} tally data rather than predict clustering from an analysis of the core geometry. The \textit{i}\ac{MGXS} scheme is more generalizable and can flexibly accommodate arbitrary core heterogeneities better than \ac{LNS} or other geometric-based approaches which must be extensively customized for particular core geometries. In addition, the \textit{i}\ac{MGXS} scheme aims to greatly accelerate the convergence rate of \ac{MGXS} tallied with \ac{MC} with respect to the the degenerate and \ac{LNS} schemes as shown in Fig.~\ref{fig:chap10-flow-chart}.

\begin{figure}[h!]
\centering
\includegraphics[width=\linewidth]{figures/unsupervised/flow-chart}
\vspace{2mm}
\caption[Expected relative runtime for different homogenization schemes]{The expected relative runtime for the degenerate, \ac{LNS} and \textit{i}\ac{MGXS} spatial homogenization schemes with respect to a reference \ac{MC} calculation.}
\label{fig:chap10-flow-chart}
\end{figure}

This chapter begins by introducing a general latent variable model for \ac{MGXS} clustering which motivates \textit{i}\ac{MGXS} spatial homogenization in Sec.~\ref{sec:chap10-latent-model}. An overview of the \textit{i}\ac{MGXS} scheme is given in Sec.~\ref{sec:chap10-overview}, with in-depth presentations of the each stage of tally data pre-processing, including feature engineering (Sec.~\ref{sec:chap10-feature-engineer}), target selection (Sec.~\ref{sec:chap10-target-select}), feature transformation (Sec.~\ref{sec:chap10-feature-transform}), dimensionality reduction (Sec.~\ref{sec:chap10-dimensions-reduce}) and litmus tests (Sec.~\ref{sec:chap10-litmus}). Sec.~\ref{subsec:chap10-clustering} highlights a few statistical clustering algorithms which may be interchangeably employed within the code framework for i\ac{MGXS} implemented for this thesis. A few heuristics for unsupervised cluster model selection are discussed in Sec.~\ref{sec:chap10-model-select}. Sec.~\ref{sec:chap10-geometries} illustrates the material configurations produced by the \textit{i}\ac{MGXS} scheme for the heterogeneous \ac{PWR} benchmarks studied in this thesis. Finally, Sec.~\ref{sec:chap10-cluster} details two approaches used to evaluate the ideal and realized convergence rates of the \textit{i}\ac{MGXS} scheme. The eigenvalues and pin-wise fission and U-238 capture rates produced with the \textit{i}\ac{MGXS} scheme are evaluated in the following chapter.
  

%%%%%%%%%%%%%%%%%%%%%%%%%%%%%%%%%%%%%%%%%%%%%%%%%%%%%%%%%%%%%%%%%%%%%%%%%%%%%%%%
%\section{Latent Spectral Variable Model}
%\label{sec:chap10-lsvm}
%
%This section postulates the existence of a probability distribution from which pin-wise \ac{MGXS} are drawn when generated from Monte Carlo tallies. In particular, this section introduces a \textit{latent variable model} -- termed the \ac{LSVM} -- which encapsulates the spatial self-shielding effects in heterogeneous geometries that leads to clustering of pin-wise \ac{MGXS}. \ac{LSVM} inspires the development of the unsupervised statistical clustering methodology for spatial homogenization that is the topic of this and the following chapter. Latent variable models are a widely used theoretical construct for representing \textit{hidden variables} which parameterize the probability distributions that are thought to have produced some observed dataset. This section loosely follows Bishop's presentation of probabilistic mixture models~\cite{bishop2006pattern}, including the mathematical notation frequently employed by discussions of latent variable models in the statistics and machine learning literature. Sec.~\ref{subsec:chap10-lsvm-math} introduces the concept of latent spectral variables which dictate a parametrization of a mixture distribution which generated the pin-wise \ac{MGXS}. Sec.~\ref{subsec:chap10-lsvm-graph} illustrates \ac{LSVM} with a few graphical models.
%
%%Sec.~\ref{subsec:chap10-lsvm-additive-noise} begins by decompose \ac{MC} estimates for \ac{MGXS} into a multi-component additive noise model. 
%
%%%%%%%%%%%%%%%%%%%%%%%%%%%%%%%%%%%
%\subsection{Latent Spectral Variables}
%\label{subsec:chap10-lsvm-math}
%
%
%
%As shown in Chap.~\ref{chap:spatial}, spatial self-shielding effects from geometric heteoregneities leads to clustering of pin-wise \ac{MGXS}. This may 
%
%The \ac{MGXS} clustering effects induced by spatial self-shielding from various core heterogeneities, 
%
%\begin{align}
%\label{eqn:chap10-mgxs-offsets}
%\sigma_{k} &= \pi_{k,1}\Delta_{1} + \pi_{k,2}\Delta_{2} + \cdots + \pi_{k,M}\Delta_{M} \\
%&= \displaystyle\sum\limits_{m=1}^{M}\pi_{k,m}\Delta_{m}
%\end{align}
%
%
%
%As discussed in Sec.~\ref{sec:chap3-mgxs-gen}, the \ac{MGXS} tallied by \ac{MC} are random variables each with an estimated sample mean and variance. \ac{LSVM} describes the probability distribution that produced the population of \ac{MGXS} random variables for a particular type of material zone. In this thesis, \ac{LSVM} is used to describe the population of pin-wise microscopic \ac{MGXS} for all instances of fuel pins of a particular enrichment in a core geometry. For example, the tallied pin-wise microscopic \ac{MGXS} $\hat{\sigma}_{x,i,k,g}$ for some reaction $x$, nuclide $i$, fuel pin instance $k$ and energy group $g$ constitute the dataset produced by some probability distribution. \ac{LSVM} hypothesizes that the statistical process which generated the \ac{MGXS} dataset is a \textit{mixture model distribution} with $M$ mixture components. Each component $m$ is a normalized distribution $p(\hat{\sigma}_{x,i,k,g}; \boldsymbol{\theta}_{m})$ parameterized by vector $\boldsymbol{\theta}_{m}$, where for simplicity is assumed that all components are from the same parametric family of distributions\footnote{This restriction may not accurately characterize the mixtures components which generates pin-wise \ac{MGXS}, but is useful to simplify the number of symbols needed to introduces \ac{LSVM}.}\textsuperscript{,}\footnote{If the components are normal then $p(\hat{\sigma}_{x,i,k,g}; \boldsymbol{\theta}_{m}) = \mathcal{N}(\boldsymbol{\theta}_{m})$ with parameter vector $\boldsymbol{\theta}_{m} = \left[\mu_{m} \;\; \sigma_{m}\right]^{T}$.}. The components are combined in a linear superposition with \textit{mixing coefficients} $\pi_{m}$ to produce the mixture model distribution $f(\hat{\sigma}_{x,i,k,g})$:
%
%\begin{equation}
%\label{eqn:chap10-mix-model}
%f(\hat{\sigma}_{x,i,k,g}) = \displaystyle\sum\limits_{m=1}^{M} \pi_{m} p(\Delta_{m}; \boldsymbol{\theta}_{m})
%\end{equation}
%
%%\begin{equation}
%%\label{eqn:chap10-mix-model}
%%f(\hat{\sigma}_{x,i,k,g}) = \displaystyle\sum\limits_{m=1}^{M} \pi_{m} p(\hat{\sigma}_{x,i,k,g}; \boldsymbol{\theta}_{m})
%%\end{equation}
%
%\noindent where the coefficients ${\pi_{m}}$ are defined to sum to unity such that the model is normalized:
%
%%If each mixture has $N$ parameters (\textit{e.g}, $|\boldsymbol{\theta}| = N$), then $\boldsymbol{\Theta}$ is the $K \times N$ matrix of mixture model parameters, and $\boldsymbol{\Lambda}$ is the $K$-dimensional vector $\boldsymbol{\lambda} = \left[\lambda_{1} \;\; \lambda_{2} \;\; \dots \;\; \lambda_{m}\right]^{T}$ of mixing weights. The weights are defined to sum to unity such that the mixture model is normalized:
%
%\begin{equation}
%\label{eqn:chap10-mix-weights-norm}
%0 \le \pi_{m} \le 1 \;\;\; , \;\;\; \displaystyle\sum\limits_{m=1}^{M} \pi_{m} = 1
%\end{equation}
%
%The statistical model in Eqn.~\ref{eqn:chap10-mix-model} is a complete prescription for how the dataset of $\hat{\sigma}_{x,i,k,g}$ are generated for the population of fuel pins. In particular, \ac{LSVM} hypothesizes the existence of a \textit{latent spectral variable} $z_{k}$ for each fuel pin instance. Furthermore, \ac{LSVM} posits that the latent spectral variable may assume $Z$ integral values such that $z_{k} \in \{1, 2, \dots, Z\}$. From a practical standpoint, the latent spectral variable corresonds to the shape of the flux across energy group $g$ in fuel pin instance $k$\footnote{The shape of the flux across energy group $g$ cannot be directly observed from energy-integrated \ac{MC} tally data, and hence the spectral variable is a latent or hidden variable.}. \ac{LSVM} prescribes a one-to-one mapping from each possible value of the latent variable $z_{k}$ to a corresponding vector of mixing coefficients $\boldsymbol{\pi}(z_{k})$:
%
%\begin{equation}
%\label{eqn:chap10-mix-coeffs}
%\boldsymbol{\pi}(z_{k}) = \boldsymbol{\pi}_{k} = \left[\lambda_{1,k} \;\; \lambda_{2,k} \;\; \cdots \;\; \lambda_{m,k}\right]^{T}
%\end{equation}
%
%\noindent used to generate the dataset of $\hat{\sigma}_{x,i,k,g}$ according to the following stochastic process:
%
%\begin{equation}
%\label{eqn:chap10-lsvm}
%\hat{\sigma}_{x,i,k,g} \;\; \sim \;\; f(\hat{\sigma}_{x,i,k,g})
%\end{equation}
%
%%\ell \;\; &\sim \;\; \text{Multinomial}(p_{1}, p_{2}, \dots, p_{M}) \\
%% \boldsymbol{\lambda} \;\; &\sim \;\; \text{Categorical}(p_{1}, p_{2}, \dots, p_{M}) \\
%
%%\footnote{The symbol $\phi$ is adopted from the widely used notation for latent variables in the machine learning literature and should not be confused for the neutron flux.}
%
%
%%%%%%%%%%%%%%%%%%%%%%%%%%%%%%%%%%%%%%%%%%%%
%\subsection{An Additive Noise Model for MGXS}
%\label{subsec:chap10-lsvm-additive-noise}
%
%Additive noise models are useful theoretical construct widely considered in information theory to model many stochastic processes found in nature. This section introduces an additive noise model for \ac{MC} estimates for \ac{MGXS}. Consider an arbitrary propbability distribution $f_{x,i,k,g}(\cdot)$ from which the estimated pin-wise \ac{MGXS} $\hat{\sigma}_{x,i,k,g}$ (Eqn.~\ref{eqn:chap3-general-micro}) for reaction $x$, nuclide $i$, spatial zone $k$ and energy group $g$ are drawn:
%
%\begin{equation}
%\label{eqn:chap10-mgxs-draw}
%\hat{\sigma}_{x,i,k,g} \;\; \sim \;\; f_{x,i,k,g}(\hat{\sigma})
%\end{equation}
%
%\noindent The reaction $x$, nuclide $i$ and energy group $g$ subscripts may dropped for brevity:
%
%\begin{equation}
%\label{eqn:chap10-mgxs-draw-brevity}
%\hat{\sigma}_{k} \;\; \sim \;\; f_{k}(\hat{\sigma})
%\end{equation}
%
%An additive noise model may treat the \ac{MC} estimate for the \ac{MGXS} $\hat{\sigma}_{k}$ as the composition of the ``true'' \ac{MGXS} $\sigma_{k}$ (\textit{i.e.}, the expectation of $f(\hat{\sigma}_{k})$, or $\mathbb{E}[\hat{\sigma}_{k}]$) and a random ``noise'' term $\epsilon_{k}$. The noise $\epsilon_{k}$ is a random variable drawn from an arbitrary distribution $p(\epsilon; \boldsymbol{\theta}_{k})$ parameterized by $\boldsymbol{\theta}_{k}$ which represents the statistical uncertainty of the \ac{MC} estimate\footnote{If the \ac{MC} realizations are i.i.d. then the noise variables are normally distributed.}. The additive noise model for $\hat{\sigma}_{k}$ is then simply the sum of the true \ac{MGXS} and the random noise terms:
%
%\begin{align}
%\label{eqn:chap10-add-noise}
%\epsilon_{k} \;\; &\sim \;\; p(\epsilon; \boldsymbol{\theta}_{k}) \\
%\hat{\sigma}_{k} &= \sigma_{k} + \epsilon_{k}
%\end{align}
%
%As shown in Chap.~\ref{}, spatial self-shielding effects from geometric heteoregneities leads to clustering of pin-wise \ac{MGXS}. This may 
%
%The \ac{MGXS} clustering effects induced by spatial self-shielding from various core heterogeneities, 
%
%\begin{align}
%\label{eqn:chap10-mgxs-offsets}
%\sigma_{k} &= \pi_{k,1}\Delta_{1} + \pi_{k,2}\Delta_{2} + \cdots + \pi_{k,M}\Delta_{M} \\
%&= \displaystyle\sum\limits_{m=1}^{M}\pi_{k,m}\Delta_{m}
%\end{align}
%
%
%\begin{align}
%\label{eqn:chap10-mgxs-noise}
%\epsilon_{m} \;\; &\sim \;\; p(\epsilon; \boldsymbol{\theta}_{m}) \\
%\hat{\Delta}_{m} &= \Delta_{m} + \epsilon_{m}
%\end{align}
%
%compose the noise into different spectral effects
%
%\begin{align}
%\label{eqn:chap10-sum-noise}
%\hat{\sigma}_{k} &= \pi_{k,1}\hat{\Delta}_{1} + \pi_{k,2}\hat{\Delta}_{2} + \cdots + \pi_{k,M}\hat{\Delta}_{M} \\
%&= \pi_{k,1}\left(\Delta_{1} + \epsilon_{1}\right) + \pi_{k,2}\left(\Delta_{2} + \epsilon_{m}\right) + \cdots + \pi_{k,M}\left(\Delta_{M} + \epsilon_{m}\right) \\
%&= \displaystyle\sum\limits_{m=1}^{M}\pi_{k,m}\left(\Delta_{m} + \epsilon_{m}\right) \\
%&= \displaystyle\sum\limits_{m=1}^{M}\pi_{k,m}\Delta_{m} + \displaystyle\sum\limits_{m=1}^{M}\pi_{k,m}\epsilon_{m} \\
%% &= \sigma_{k} + \displaystyle\sum\limits_{m=1}^{M}\pi_{k,m}\epsilon_{m}
%\end{align}
%
%
%%%%%%%%%%%%%%%%%%%%%%%%%%%%%
%\subsection{Graphical Plate Notation}
%\label{subsec:chap10-lsvm-graph}
%
%\begin{figure}
%\centering
%\begin{tikzpicture}
%\tikzstyle{main}=[circle, minimum size = 8mm, thick, draw =black!80, node distance = 16mm]
%\tikzstyle{param}=[regular polygon, regular polygon sides=4, minimum size = 8mm, thick, draw =black!80, node distance = 16mm]
%  \node[param] (zk) [label=above:$z_k$] { };
%  \node[param] (theta) [left=of zk,label=left:$\boldsymbol{\Theta}$] { };
%  \node[main, fill = black!20] (sigma) [below=of zk, label=below:$\hat{\sigma}_{x,i,k,g}$] { };
%  \node[param] (pi) [left=of sigma,label=left:$\boldsymbol{\pi}$] { };
%   \draw[dagconn] (zk) to (sigma);
%   \draw[dagconn] (theta) to (sigma);
%   \draw[dagconn] (pi) to (sigma);
%   \node[plate=K, inner sep=20pt, fit=(zk) (sigma)] (plate1) {};
%\end{tikzpicture}
%\vspace{4mm}
%\caption[Graphical model for LSVM for a specific reactor configuration]{A graphical model to describe \ac{LSVM} for a specific reactor configuration.}
%\label{fig:chap10-lsvm-plate}
%\end{figure}
%
%\begin{figure}
%\centering
%\begin{tikzpicture}
%\tikzstyle{main}=[circle, minimum size = 8mm, thick, draw =black!80, node distance = 16mm]
%\tikzstyle{param}=[regular polygon, regular polygon sides=4, minimum size = 8mm, thick, draw =black!80, node distance = 16mm]
%  \node[param] (alpha) [label=above:$\boldsymbol{\alpha}$] { };
%  \node[main] (theta) [below left=of alpha,label=above:$\theta_{m}$] { };
%  \node[main] (pi) [below right=of alpha,label=above:$\pi_{m}$] { };
%  \node[param] (zk) [below=of theta,label=below:$z_k$] { };
%  \node[main, fill = black!20] (sigma) [below=of pi, label=below:$\hat{\sigma}_{x,i,k,g}$] { };
%   \draw[dagconn] (alpha) to (theta);
%   \draw[dagconn] (alpha) to (pi);
%   \draw[dagconn] (zk) to (sigma);
%   \draw[dagconn] (theta) to (sigma);
%   \draw[dagconn] (pi) to (sigma);
%   \node[plate=M, inner sep=20pt, fit=(theta) (pi)] (plate1) {};
%   \node[plate=K, inner sep=20pt, fit=(zk) (sigma)] (plate2) {};
%\end{tikzpicture}
%\vspace{4mm}
%\caption[Graphical model for LSVM for an arbitrary reactor configuration]{A graphical model to describe \ac{LSVM} for an arbitrary reactor configuration.}
%\label{fig:chap10-lsvm-plate-general}
%\end{figure}
%
%
%-we need to infer the latent variables $z_{k}$ for each fuel pin instance!!!
%-mention covariance between samples - the could be correlated
%-three goals:
%  -infer number of points/pins in each set
%  -infer conditional probabilities?
%  -assign each region to the most likely set?
%-is this section needed???
%-mention possibility of a multi-level latent variable model
%-mention bayesian approach to solving this using maximum likelihood


%%%%%%%%%%%%%%%%%%%%%%%%%%%%%%%%%%%%%%%%%%%%%%%%%%%%%%%%%%%%%%%%%%%%%%%%%%%%%%%
\section{Overview of \textit{i}MGXS}
\label{sec:chap10-overview}

The \textit{i}\ac{MGXS} spatial homogenization scheme is a multi-stage data processing \textit{pipeline}. The objective of the scheme is to infer the optimal assignment of cluster labels to fuel pin instances directly from \ac{MC} tally data. The cluster labels output by \textit{i}\ac{MGXS} are then used to generate track density-weighted \ac{MGXS} (see Sec.~\ref{subsec:chap9-lns-math}) for each cluster of fuel pin instances using the same process as that employed by \ac{LNS} homogenization (Eqn.~\ref{eqn:chap9-lns-micro}). The \textit{i}\ac{MGXS} methodology differs from the \ac{LNS} scheme in that it makes no consideration of the geometry and materials configuration and only examines tallied \ac{MC} tally data when assigning cluster labels to fuel pin instances.

A high-level overview of the various stages of the \textit{i}\ac{MGXS} data processing pipeline is illustrated in Fig.~\ref{fig:chap10-pipeline}. The \textit{i}\ac{MGXS} pipeline may be configured in various ways and this thesis makes no presumption that the incarnation presented in Fig.~\ref{fig:chap10-pipeline} is the best or most reliable version. Future work may develop schemes which improve upon the particular formulation of \textit{i}\ac{MGXS} presented in this thesis. Irregardless of the particular configuration, the overarching concept is that the \textit{i}\ac{MGXS} pipeline provides a \textit{mapping} between \ac{MC} tally data and cluster labels for each fuel pin\footnote{Similarly, the \ac{LNS} scheme defines a mapping between the combinatorial geometry and the cluster labels.}. Each of the six stages in Fig.~\ref{fig:chap10-pipeline} is detailed in the following sections of this thesis.

\begin{figure}[h!]
\centering
\includegraphics[width=0.35\linewidth]{figures/unsupervised/pipeline}
\vspace{2mm}
\caption[MGXS pipeline]{The multi-stage \textit{i}\ac{MGXS} data processing pipeline.}
\label{fig:chap10-pipeline}
\end{figure}

%-competing paradigms:
%  -clustering
%  -input -> target (e.g., trees)
%  -some combination of the two

%-break apart diagram for machine learning piece into parts
%  -clustering
%    -flow chart: litmus tests, feature eng./selection/transformation, clustering
%  -input -> target (e.g., regression)
%    -flow chart: litmus tests, target selection, feature eng./selection/transformation, model fitting, prediction
%     -can take outputs as inputs for clustering scheme


%%%%%%%%%%%%%%%%%%%%%%%%%%%%%%%%%%%%%%%%%%%%%%%%%%%%%%%%%%%%%%%%%%%%%%%%%%%%%%%
\section{Feature Engineering}
\label{sec:chap10-feature-engineer}

The \textit{feature engineering} stage in the data processing pipeline in Fig.~\ref{fig:chap10-pipeline} builds \textit{features} from \ac{MC} tally data. In machine learning, features are simply variables which are used as inputs to a predictive model. Features may be engineered based upon prior domain knowledge or inferred from automated feature learning algorithms such as neural networks. In the context of \textit{i}\ac{MGXS}, features are restricted to tallies, or combinations of tallies, from \ac{MC} simulations which provide information about which fuel pin instances experience similar spatial self-shielding effects. For example, the pin-wise \ac{MGXS} $\hat{\sigma}_{x,i,k,g}$ themselves may be used as features since they exhibit the very clustering effect which \textit{i}\ac{MGXS} attempts to predict\footnote{Using the pin-wise \ac{MGXS} as the only feature(s) for unsupervised clustering would be equivalent to specifying boundaries between the samples illustrated in the rug plots in Sec.~\ref{subsec:chap9-histograms}.}. Other tallied quantities may also be used as features for predicting which fuel pin instances have similarly self-shielded \ac{MGXS}. The goal of engineering \textit{i}\ac{MGXS} features is to enable machine learning algorithms to identify \ac{MGXS} clusters as quickly as possible from ``noisy'' or unconverged tally data.

The \textit{i}\ac{MGXS} scheme splits the \ac{MC} tally dataset up into \textit{samples} for each particular instance $k$ of a fuel pin as illustrated in Fig.~\ref{fig:chap10-feature-eng}. A \textit{sample} is a random vector with $J$ entries for each feature $\hat{f}_{j,k}$ corresponding to a particular instance $k$ of a fuel pin. A sample may be comprised of features derived from \ac{MC} tally data for one or multiple nuclides, energy groups and/or reaction types. Ideally, the features should maximize the \textit{separation distance} in feature space $\{f_{1}, f_{2}, \dots, f_{J}\}$ between samples from the same cluster. Since features in \textit{i}\ac{MGXS} are tallied quantities from \ac{MC} simulations, some samples may include ``outlier'' feature realizations which take on values far removed from the ``true'' value of the feature\footnote{The ``true'' values are the expected values for each feature.}. Outliers may result in the incorrect assignment of cluster labels to fuel pin instances. The predictive models used in \textit{i}\ac{MGXS} are trained with the complete vector of features in order to mitigate the sensitivity of the models to outlier features and minimize the frequency of mis-labeled cluster assignments.

\begin{figure}[h!]
\centering
\includegraphics[width=0.4\linewidth]{figures/unsupervised/feature-eng}
\vspace{2mm}
\caption[\textit{i}MGXS sample feature vector construction]{\textit{i}\ac{MGXS} builds feature vectors for each sample (fuel pin instance).}
\label{fig:chap10-feature-eng}
\end{figure}

%\begin{equation}
%\hat{f}_{1,1}
%\end{equation}
%
%\begin{equation}
%\hat{f}_{2,1}
%\end{equation}
%
%\begin{equation}
%\hat{f}_{J,1}
%\end{equation}
%
%\begin{equation}
%\hat{f}_{1,2}
%\end{equation}
%
%\begin{equation}
%\hat{f}_{2,2}
%\end{equation}
%
%\begin{equation}
%\hat{f}_{J,2}
%\end{equation}
%
%\begin{equation}
%\hat{f}_{1,K}
%\end{equation}
%
%\begin{equation}
%\hat{f}_{2,K}
%\end{equation}
%
%\begin{equation}
%\hat{f}_{J,K}
%\end{equation}

In addition, it should be noted that features may not necessarily be defined for the same energy group structure as the \ac{MGXS} one wishes to cluster. For example, some or all features may be tallied on a relatively coarse energy group structure in order to minimize their \ac{MC} statistical uncertainties. It is in fact be beneficial to tally features in few groups in order to identify clusters with fewer \ac{MC} particle histories than would needed to distinguish structure from ``noisy'' fine group features.
These coarse group features may be input to a clustering algorithm for spatial homogenization of pin-wise \ac{MGXS} defined on a fine(r) energy group structure. For example, the following sections and chapters cluster features defined for a coarse 2-group structure in order to assign each fuel pin to a cluster for spatial homogenization of the ``fine'' 70-group pin-wise \ac{MGXS} data needed to minimize reaction rate errors to 1 -- 2\% or less.

The following sections introduce the features employed by \textit{i}\ac{MGXS} in this and the following chapter. These include the pin-wise \ac{MGXS} and their statistical uncertainties (Sec.~\ref{subsec:chap10-stat-uncertainty}), as well as features referred to as fractional reactivites (Sec.~\ref{subsec:chap10-frac-reactivity}), spectral indices (Sec.~\ref{subsec:chap10-spec-index}), nuclide fractions (Sec.~\ref{subsec:chap10-nuclide-frac}) and total fractions (Sec.~\ref{subsec:chap10-tot-frac}). Each feature is specific to a fuel pin instance, and may further be distinguished for one or more nuclides, energy groups and/or reaction types.

%The clustering algoritms used in \textit{i}\ac{MGXS} may more easily identify clusters from the less ``noisy'' feature data, and the cluster labels used to spatially homogenize fine(r) group pin-wise \ac{MGXS}. 


%first paragraph: motivationa 
%  -could cluster multiple microscopic \ac{MGXS} simultaneously - nuclides, reactions, groups

%%%%%%%%%%%%%%%%%%%%%%%%%%%%%%%%%%%%
\subsection{Statistical Uncertainty}
\label{subsec:chap10-stat-uncertainty}

first paragraph: explain what it is
-point to a few figures
-explain why std. dev. may be higher for different pins
-indicate that it may reflect the difference in track density in differen pins
  -is this why we see rings in the \ac{MGXS} in the full core??
-recall equation for std. dev.
-recall uncertainty propagation
-also relative error - normalize out the mean \ac{MGXS}

-MATCH COLOR SCHEMES

\begin{equation}
\label{eqn:chap10-rel-err}
\eta_{x,i,k,g} = \frac{\sigma_{\hat{\sigma}_{x,i,k,g}}}{\hat{\sigma}_{x,i,k,g}}
\end{equation}

second paragraph: visualizations
-look at single fuel assembly with GUI
-three plots: full, two different clusters selected

\begin{figure}[h!]
\centering
\begin{subfigure}{0.42\textwidth}
  \centering
  \includegraphics[width=0.9\linewidth]{figures/unsupervised/features/assm-16/geometry}
  \caption{}
  \label{fig:chap10-fiss-mean-std-geom}
\end{subfigure}%
\begin{subfigure}{0.42\textwidth}
  \centering
  \includegraphics[width=0.9\linewidth]{figures/unsupervised/features/assm-16/u235-fiss/mean-std/mgxs}
  \caption{}
  \label{fig:chap10-fiss-mean-std-mgxs}
\end{subfigure}
\begin{subfigure}{0.42\textwidth}
  \centering
  \includegraphics[width=0.9\linewidth]{figures/unsupervised/features/assm-16/u235-fiss/mean-std/geometry-2}
  \caption{}
  \label{fig:chap10-fiss-mean-std-geom-2}
\end{subfigure}%
\begin{subfigure}{0.42\textwidth}
  \centering
  \includegraphics[width=0.9\linewidth]{figures/unsupervised/features/assm-16/u235-fiss/mean-std/mgxs-2}
  \caption{}
  \label{fig:chap10-fiss-mean-std-mgxs-2}
\end{subfigure}
\begin{subfigure}{0.42\textwidth}
  \centering
  \includegraphics[width=0.9\linewidth]{figures/unsupervised/features/assm-16/u235-fiss/mean-std/geometry-3}
  \caption{}
  \label{fig:chap10-fiss-mean-std-geom-3}
\end{subfigure}%
\begin{subfigure}{0.42\textwidth}
  \centering
  \includegraphics[width=0.9\linewidth]{figures/unsupervised/features/assm-16/u235-fiss/mean-std/mgxs-3}
  \caption{}
  \label{fig:chap10-fiss-mean-std-mgxs-3}
\end{subfigure}
\caption[Clustering of U-235 fission MGXS standard deviations]{Scatter plots of the pin-wise U-235 fission (group 2 of 2) \ac{MGXS} means ($x$) and standard deviations ($y$) for the 1.6\% enriched fuel assembly.}
\label{fig:chap10-mean-std}
\end{figure}

\clearpage

\begin{figure}[h!]
\centering
\begin{subfigure}{0.42\textwidth}
  \centering
  \includegraphics[width=0.9\linewidth]{figures/unsupervised/features/assm-16/geometry}
  \caption{}
  \label{fig:chap10-capt-mean-std-geom}
\end{subfigure}%
\begin{subfigure}{0.42\textwidth}
  \centering
  \includegraphics[width=0.9\linewidth]{figures/unsupervised/features/assm-16/u238-capt/mean-std/mgxs}
  \caption{}
  \label{fig:chap10-capt-mean-std-mgxs}
\end{subfigure}
\begin{subfigure}{0.42\textwidth}
  \centering
  \includegraphics[width=0.9\linewidth]{figures/unsupervised/features/assm-16/u238-capt/mean-std/geometry-2}
  \caption{}
  \label{fig:chap10-capt-mean-std-geom-2}
\end{subfigure}%
\begin{subfigure}{0.42\textwidth}
  \centering
  \includegraphics[width=0.9\linewidth]{figures/unsupervised/features/assm-16/u238-capt/mean-std/mgxs-2}
  \caption{}
  \label{fig:chap10-capt-mean-std-mgxs-2}
\end{subfigure}
\begin{subfigure}{0.42\textwidth}
  \centering
  \includegraphics[width=0.9\linewidth]{figures/unsupervised/features/assm-16/u238-capt/mean-std/geometry-3}
  \caption{}
  \label{fig:chap10-capt-mean-std-geom-3}
\end{subfigure}%
\begin{subfigure}{0.42\textwidth}
  \centering
  \includegraphics[width=0.9\linewidth]{figures/unsupervised/features/assm-16/u238-capt/mean-std/mgxs-3}
  \caption{}
  \label{fig:chap10-capt-mean-std-mgxs-3}
\end{subfigure}
\caption[Clustering of U-238 capture MGXS standard deviations]{Scatter plots of the pin-wise U-238 capture (group 1 of 2) \ac{MGXS} means ($x$) and standard deviations ($y$) for the 1.6\% enriched fuel assembly.}
\label{fig:chap10-mean-std}
\end{figure}

\clearpage

%%%%%%%%%%%%%%%%%%%%%%%%%%%%%%%%%%
\subsection{Fractional Reactivity}
\label{subsec:chap10-frac-reactivity}

first paragraph: explain what it is
-divide each pin's energy-integrated nu-fission rxn rate by the total energy- and volume-integrated absorption rates across all fuel pins
-Eqn.~\ref{eqn:chap10-frac-reactivity}
  -multiply by 1E5 to get units of \ac{pcm}
-explain what it indicates (differential moderation, adjacency to guide tubes, spectral hardening)
-note that $k$ is ONLY for fuel regions!!!

\begin{equation}
\label{eqn:chap10-frac-reactivity}
\alpha_{k} = \frac{\displaystyle\sum\limits_{g=1}^{G}\nu\hat{\sigma}_{f,i,k,g}\hat{\phi}_{k,g}}{\displaystyle\sum\limits_{k=1}^{K}\displaystyle\sum\limits_{g=1}^{G}\hat{\sigma}_{a,i,k,g}\hat{\phi}_{k,g}}
\end{equation}

second paragraph: visualizations
-look at single fuel assembly with GUI
-three plots: full, two different clusters selected

\begin{figure}[h!]
\centering
\begin{subfigure}{0.45\textwidth}
  \centering
  \includegraphics[width=0.9\linewidth]{figures/unsupervised/features/assm-16/geometry}
  \caption{}
  \label{fig:chap10-fiss-mean-pcm-geom}
\end{subfigure}%
\begin{subfigure}{0.45\textwidth}
  \centering
\includegraphics[width=0.9\linewidth]{figures/unsupervised/features/assm-16/u235-fiss/mean-pcm/mgxs}
  \caption{}
  \label{fig:chap10-fiss-mean-pcm-mgxs}
\end{subfigure}
\begin{subfigure}{0.45\textwidth}
  \centering
  \includegraphics[width=0.9\linewidth]{figures/unsupervised/features/assm-16/u235-fiss/mean-pcm/geometry-2}
  \caption{}
  \label{fig:chap10-fiss-mean-pcm-geom-2}
\end{subfigure}%
\begin{subfigure}{0.45\textwidth}
  \centering
  \includegraphics[width=0.9\linewidth]{figures/unsupervised/features/assm-16/u235-fiss/mean-pcm/mgxs-2}
  \caption{}
  \label{fig:chap10-fiss-mean-pcm-mgxs-2}
\end{subfigure}
\begin{subfigure}{0.45\textwidth}
  \centering
  \includegraphics[width=0.9\linewidth]{figures/unsupervised/features/assm-16/u235-fiss/mean-pcm/geometry-3}
  \caption{}
  \label{fig:chap10-fiss-mean-pcm-geom-3}
\end{subfigure}%
\begin{subfigure}{0.45\textwidth}
  \centering
  \includegraphics[width=0.9\linewidth]{figures/unsupervised/features/assm-16/u235-fiss/mean-pcm/mgxs-3}
  \caption{}
  \label{fig:chap10-fiss-mean-pcm-mgxs-3}
\end{subfigure}
\caption[Clustering of U-235 fission MGXS fractional reactivities]{Scatter plots of the pin-wise U-235 fission (group 2 of 2) \ac{MGXS} means ($x$) and fractional reactivities ($y$) for the 1.6\% enriched fuel assembly.}
\label{fig:chap10-mean-pcm}
\end{figure}

\clearpage

\begin{figure}[h!]
\centering
\begin{subfigure}{0.45\textwidth}
  \centering
  \includegraphics[width=0.9\linewidth]{figures/unsupervised/features/assm-16/geometry}
  \caption{}
  \label{fig:chap10-capt-mean-pcm-geom}
\end{subfigure}%
\begin{subfigure}{0.45\textwidth}
  \centering
  \includegraphics[width=0.9\linewidth]{figures/unsupervised/features/assm-16/u238-capt/mean-pcm/mgxs}
  \caption{}
  \label{fig:chap10-capt-mean-pcm-mgxs}
\end{subfigure}
\begin{subfigure}{0.45\textwidth}
  \centering
  \includegraphics[width=0.9\linewidth]{figures/unsupervised/features/assm-16/u238-capt/mean-pcm/geometry-2}
  \caption{}
  \label{fig:chap10-capt-mean-pcm-geom-2}
\end{subfigure}%
\begin{subfigure}{0.45\textwidth}
  \centering
  \includegraphics[width=0.9\linewidth]{figures/unsupervised/features/assm-16/u238-capt/mean-pcm/mgxs-2}
  \caption{}
  \label{fig:chap10-capt-mean-pcm-mgxs-2}
\end{subfigure}
\begin{subfigure}{0.45\textwidth}
  \centering
  \includegraphics[width=0.9\linewidth]{figures/unsupervised/features/assm-16/u238-capt/mean-pcm/geometry-3}
  \caption{}
  \label{fig:chap10-capt-mean-pcm-geom-3}
\end{subfigure}%
\begin{subfigure}{0.45\textwidth}
  \centering
  \includegraphics[width=0.9\linewidth]{figures/unsupervised/features/assm-16/u238-capt/mean-pcm/mgxs-3}
  \caption{}
  \label{fig:chap10-capt-mean-pcm-mgxs-3}
\end{subfigure}
\caption[Clustering of U-238 capture MGXS fractional reactivities]{Scatter plots of the pin-wise U-238 capture (group 1 of 2) \ac{MGXS} means ($x$) and fractional reactivities ($y$) for the 1.6\% enriched fuel assembly.}
\label{fig:chap10-mean-pcm}
\end{figure}

\clearpage


%%%%%%%%%%%%%%%%%%%%%%%%%%%
\subsection{Spectral Index}
\label{subsec:chap10-spec-index}

first paragraph: explain what it is
-divide each pin's nu-fission and absorption \ac{MGXS} 
-Eqn.~\ref{eqn:chap10-spec-index}
-explain what it indicates (differential moderation, adjacency to guide tubes, spectral hardening)
-note $i$ and $j$ may designate two difference nuclides
-note $\gamma$ is radiative capture - need to change this in theory section???

\begin{equation}
\label{eqn:chap10-spec-index-general}
\beta_{x,i,k,g} = \frac{\hat{\sigma}_{x,i,k,g}}{\hat{\sigma}_{f,k,g}^{235}}
\end{equation}

\begin{equation}
\label{eqn:chap10-spec-index-u238-capt}
\beta_{\gamma,k,g}^{238} = \frac{\hat{\sigma}_{\gamma,k,g}^{238}}{\hat{\sigma}_{f,k,g}^{235}}
\end{equation}

second paragraph: visualizations
-look at single fuel assembly with GUI
-three plots: full, two different clusters selected
-indicate which energy group and nuclide

\begin{figure}[h!]
\centering
\begin{subfigure}{0.42\textwidth}
  \centering
  \includegraphics[width=0.9\linewidth]{figures/unsupervised/features/assm-16/geometry}
  \caption{}
  \label{fig:chap10-fiss-mean-spect-ind-geom}
\end{subfigure}%
\begin{subfigure}{0.42\textwidth}
  \centering
  \includegraphics[width=0.9\linewidth]{figures/unsupervised/features/assm-16/u235-fiss/mean-spect-ind/mgxs}
  \caption{}
  \label{fig:chap10-fiss-mean-spect-ind-mgxs}
\end{subfigure}
\begin{subfigure}{0.42\textwidth}
  \centering
  \includegraphics[width=0.9\linewidth]{figures/unsupervised/features/assm-16/u235-fiss/mean-spect-ind/geometry-2}
  \caption{}
  \label{fig:chap10-fiss-mean-spect-ind-geom-2}
\end{subfigure}%
\begin{subfigure}{0.42\textwidth}
  \centering
  \includegraphics[width=0.9\linewidth]{figures/unsupervised/features/assm-16/u235-fiss/mean-spect-ind/mgxs-2}
  \caption{}
  \label{fig:chap10-fiss-mean-spect-ind-mgxs-2}
\end{subfigure}
\begin{subfigure}{0.42\textwidth}
  \centering
  \includegraphics[width=0.9\linewidth]{figures/unsupervised/features/assm-16/u235-fiss/mean-spect-ind/geometry-3}
  \caption{}
  \label{fig:chap10-fiss-mean-spect-ind-geom-3}
\end{subfigure}%
\begin{subfigure}{0.42\textwidth}
  \centering
  \includegraphics[width=0.9\linewidth]{figures/unsupervised/features/assm-16/u235-fiss/mean-spect-ind/mgxs-3}
  \caption{}
  \label{fig:chap10-fiss-mean-spect-ind-mgxs-3}
\end{subfigure}
\caption[Clustering of U-235 fission MGXS spectral indices]{Scatter plots of the pin-wise U-235 fission (group 2 of 2) \ac{MGXS} means ($x$) and spectral indices ($y$) for the 1.6\% enriched fuel assembly.}
\label{fig:chap10-mean-spect-ind}
\end{figure}

\clearpage

\begin{figure}[h!]
\centering
\begin{subfigure}{0.42\textwidth}
  \centering
  \includegraphics[width=0.9\linewidth]{figures/unsupervised/features/assm-16/geometry}
  \caption{}
  \label{fig:chap10-capt-mean-spect-ind-geom}
\end{subfigure}%
\begin{subfigure}{0.42\textwidth}
  \centering
  \includegraphics[width=0.9\linewidth]{figures/unsupervised/features/assm-16/u238-capt/mean-spect-ind/mgxs}
  \caption{}
  \label{fig:chap10-capt-mean-spect-ind-mgxs}
\end{subfigure}
\begin{subfigure}{0.42\textwidth}
  \centering
  \includegraphics[width=0.9\linewidth]{figures/unsupervised/features/assm-16/u238-capt/mean-spect-ind/geometry-2}
  \caption{}
  \label{fig:chap10-capt-mean-spect-ind-geom-2}
\end{subfigure}%
\begin{subfigure}{0.42\textwidth}
  \centering
  \includegraphics[width=0.9\linewidth]{figures/unsupervised/features/assm-16/u238-capt/mean-spect-ind/mgxs-2}
  \caption{}
  \label{fig:chap10-capt-mean-spect-ind-mgxs-2}
\end{subfigure}
\begin{subfigure}{0.42\textwidth}
  \centering
  \includegraphics[width=0.9\linewidth]{figures/unsupervised/features/assm-16/u238-capt/mean-spect-ind/geometry-3}
  \caption{}
  \label{fig:chap10-capt-mean-spect-ind-geom-3}
\end{subfigure}%
\begin{subfigure}{0.42\textwidth}
  \centering
  \includegraphics[width=0.9\linewidth]{figures/unsupervised/features/assm-16/u238-capt/mean-spect-ind/mgxs-3}
  \caption{}
  \label{fig:chap10-capt-mean-spect-ind-mgxs-3}
\end{subfigure}
\caption[Clustering of U-238 capture MGXS spectral indices]{Scatter plots of the pin-wise U-238 capture (group 1 of 2) \ac{MGXS} means ($x$) and spectral indices ($y$) for the 1.6\% enriched fuel assembly.}
\label{fig:chap10-mean-spect-ind}
\end{figure}

\clearpage

%%%%%%%%%%%%%%%%%%%%%%%%%%%%%
\subsection{Nuclide Fraction}
\label{subsec:chap10-nuclide-frac}

first paragraph: explain what it is
-divide pin-wise micro \ac{MGXS} for each nuclide/reaction/group by the corresponding total \ac{MGXS} for that nuclide/reaction/group trio
-Eqn.~\ref{eqn:chap10- nuclide-frac}
-explain what it indicates...???
-note $t$ designates the total collision reaction type

\begin{equation}
\label{eqn:chap10-nuclide-frac}
\tau_{x,i,k,g} = \frac{\hat{\sigma}_{x,i,k,g}}{\hat{\sigma}_{t,i,k,g}}
\end{equation}

second paragraph: visualizations
-look at single fuel assembly with GUI
-three plots: full, two different clusters selected
-indicate which energy group, nuclide, and reaction type

\begin{figure}[h!]
\centering
\begin{subfigure}{0.42\textwidth}
  \centering
  \includegraphics[width=0.9\linewidth]{figures/unsupervised/features/assm-16/geometry}
  \caption{}
  \label{fig:chap10-fiss-mean-nuc-frac-geom}
\end{subfigure}%
\begin{subfigure}{0.42\textwidth}
  \centering
  \includegraphics[width=0.9\linewidth]{figures/unsupervised/features/assm-16/u235-fiss/mean-nuc-frac/mgxs}
  \caption{}
  \label{fig:chap10-fiss-mean-nuc-frac-mgxs}
\end{subfigure}
\begin{subfigure}{0.42\textwidth}
  \centering
  \includegraphics[width=0.9\linewidth]{figures/unsupervised/features/assm-16/u235-fiss/mean-nuc-frac/geometry-2}
  \caption{}
  \label{fig:chap10-fiss-mean-nuc-frac-geom-2}
\end{subfigure}%
\begin{subfigure}{0.42\textwidth}
  \centering
  \includegraphics[width=0.9\linewidth]{figures/unsupervised/features/assm-16/u235-fiss/mean-nuc-frac/mgxs-2}
  \caption{}
  \label{fig:chap10-fiss-mean-nuc-frac-mgxs-2}
\end{subfigure}
\begin{subfigure}{0.42\textwidth}
  \centering
  \includegraphics[width=0.9\linewidth]{figures/unsupervised/features/assm-16/u235-fiss/mean-nuc-frac/geometry-3}
  \caption{}
  \label{fig:chap10-fiss-mean-nuc-frac-geom-3}
\end{subfigure}%
\begin{subfigure}{0.42\textwidth}
  \centering
  \includegraphics[width=0.9\linewidth]{figures/unsupervised/features/assm-16/u235-fiss/mean-nuc-frac/mgxs-3}
  \caption{}
  \label{fig:chap10-fiss-mean-nuc-frac-mgxs-3}
\end{subfigure}
\caption[Clustering of U-235 fission MGXS nuclide fractions]{Scatter plots of the pin-wise U-235 fission (group 2 of 2) \ac{MGXS} means ($x$) and nuclide fractions ($y$) for the 1.6\% enriched fuel assembly.}
\label{fig:chap10-mean-nuc-frac}
\end{figure}

\clearpage

\begin{figure}[h!]
\centering
\begin{subfigure}{0.42\textwidth}
  \centering
  \includegraphics[width=0.9\linewidth]{figures/unsupervised/features/assm-16/geometry}
  \caption{}
  \label{fig:chap10-capt-mean-nuc-frac-geom}
\end{subfigure}%
\begin{subfigure}{0.42\textwidth}
  \centering
  \includegraphics[width=0.9\linewidth]{figures/unsupervised/features/assm-16/u238-capt/mean-nuc-frac/mgxs}
  \caption{}
  \label{fig:chap10-capt-mean-nuc-frac-mgxs}
\end{subfigure}
\begin{subfigure}{0.42\textwidth}
  \centering
  \includegraphics[width=0.9\linewidth]{figures/unsupervised/features/assm-16/u238-capt/mean-nuc-frac/geometry-2}
  \caption{}
  \label{fig:chap10-capt-mean-nuc-frac-geom-2}
\end{subfigure}%
\begin{subfigure}{0.42\textwidth}
  \centering
  \includegraphics[width=0.9\linewidth]{figures/unsupervised/features/assm-16/u238-capt/mean-nuc-frac/mgxs-2}
  \caption{}
  \label{fig:chap10-capt-mean-nuc-frac-mgxs-2}
\end{subfigure}
\begin{subfigure}{0.42\textwidth}
  \centering
  \includegraphics[width=0.9\linewidth]{figures/unsupervised/features/assm-16/u238-capt/mean-nuc-frac/geometry-3}
  \caption{}
  \label{fig:chap10-capt-mean-nuc-frac-geom-3}
\end{subfigure}%
\begin{subfigure}{0.42\textwidth}
  \centering
  \includegraphics[width=0.9\linewidth]{figures/unsupervised/features/assm-16/u238-capt/mean-nuc-frac/mgxs-3}
  \caption{}
  \label{fig:chap10-capt-mean-nuc-frac-mgxs-3}
\end{subfigure}
\caption[Clustering of U-238 capture MGXS nuclide fractions]{Scatter plots of the pin-wise U-238 capture (group 1 of 2) \ac{MGXS} means ($x$) and nuclide fractions ($y$) for the 1.6\% enriched fuel assembly.}
\label{fig:chap10-mean-nuc-frac}
\end{figure}

\clearpage

%%%%%%%%%%%%%%%%%%%%%%%%%%%
\subsection{Total Fraction}
\label{subsec:chap10-tot-frac}

first paragraph: explain what it is
-divide pin-wise micro \ac{MGXS} for each nuclide/reaction/group by the corresponding total \ac{MGXS} for that nuclide/reaction/group trio
-Eqn.~\ref{eqn:chap10-tot-frac}
-explain what it indicates...???
-note that $I$ designates the total number of nuclides in a fuel pin

\begin{equation}
\label{eqn:chap10-tot-frac}
\omega_{x,i,k,g} = \frac{\hat{\sigma}_{x,i,k,g}}{\displaystyle\sum\limits_{i=1}^{I}\hat{\sigma}_{t,i,k,g}}
\end{equation}

second paragraph: visualizations
-look at single fuel assembly with GUI
-three plots: full, two different clusters selected
-indicate which energy group, nuclide, and reaction type

%%%%%%%%%%%%%%%%%%%%%%%%%%%%%%%%%%%%%%%%%%%%%%%%%%%%%%%%%%%%%%%%%%%%%%%%%%%%%%%
\section{Target Selection}
\label{sec:chap10-target-select}

%%%%%%%%%%%%%%%%%%%%%%%%%%%%%%%%%%%%%%%%%%%%%%%%%%%%%%%%%%%%%%%%%%%%%%%%%%%%%%%
\section{Feature Transformation}
\label{sec:chap10-feature-transform}

first paragraph: what is goal???
-curse of dimensionality - dimensionality reduction
-transform number of features to find new ones with more descriptive power than original set
  -new features may be more easily ordered than original ones
    -allowing for easier feature selection (Sec.~\ref{sec:chap10-litmus})
-can account for correlations between features
  -rather than treating each features independent

%%%%%%%%%%%%%%%%%%%%%%%%%%%%%%%%%%
\subsection{Feature Decomposition}
\label{subsec:chap10-feature-transform-decomp}

%%%%%%%%%%%%%%%%%%%%%%%%%%%%%%%%%%%%%%%%%
\subsection{Principal Component Analysis}
\label{subsec:chap10-feature-transform-pca}

%%%%%%%%%%%%%%%%%%%%%%%%%%%%%%%%%%%%%%%%%%%
\subsection{Independent Component Analysis}
\label{subsec:chap10-feature-transform-ica}

%%%%%%%%%%%%%%%%%%%%%%%%%%%%%%%%%%%%%%%%%
\subsection{Feature -- Target Regression}
\label{subsec:chap10-feature-transform-regression}


%%%%%%%%%%%%%%%%%%%%%%%%%%%%%%%%%%%%%%%%%%%%%%%%%%%%%%%%%%%%%%%%%%%%%%%%%%%%%%%
\section{Dimensionality Reduction}
\label{sec:chap10-dimensions-reduce}

-dimensionality reduction

%%%%%%%%%%%%%%%%%%%%%%%%%%%%%%%%%%%%%%%%%%%%%%%%%%%%%%%%%%%%%%%%%%%%%%%%%%%%%%%
\section{Litmus Tests}
\label{sec:chap10-litmus}

first paragraph: what are ``litmust tests''?
-which \ac{MGXS} need to be clustered
  -which nuclides
  -which reaction type
  -which energy group

-first choose a reaction type for a nuclide, energy group
-then run litmus tests to determine whether to cluster it

-note that results in following chapter may not make full use of litmus tests
  -instead, user will indicate wish to cluster individual MGXS like U-238 capture

%%%%%%%%%%%%%%%%%%%%%%%%%%%%%
\subsection{Nuclide Fraction}
\label{subsec:chap10-litmus-nuc-frac}

-used to choose reaction type
-threshold on $\tau_{x,i,k,g}$
-only choose reaction's which contribute at least some amount towards the total cross section
  -assumes that reaction with little contribution are not important to cluster anyway
  -but may mistakenly make approx that MGXS with small magnitudes don't disproportionately reflect self-shielding effects
    -but this is a good assumption since larger MGXS will have smaller uncertainties anyway

%%%%%%%%%%%%%%%%%%%%%%%%%%%%%%%%%%%%%%%%%%
\subsection{Nuclide Fraction Thresholding}
\label{subsec:chap10-litmus-nuc-frac}

-used to choose reaction type
-threshold on the population variance divide by the population mean

%%%%%%%%%%%%%%%%%%%%%%%%%%%%%%%%%%%%%%%%
\subsection{Total Fraction Thresholding}
\label{subsec:chap10-litmus-tot-frac}

-used to determine whether to cluster MGXS for a particular nuclide, group, rxn type
-threshold on $\omega_{x,i,k,g}$
  -only cluster if the MGXS makes up a substantial fraction of the MGXS for all nuclides
  -otherwise, self-shielding of the nuclide won't matter anyway
  -this must be of the macro not the micro MGXS btw!!!

%%%%%%%%%%%%%%%%%%%%%%%%%%%%
\subsection{Normality Tests}
\label{subsec:chap10-litmus-normality}

-used to determine whether to cluster MGXS for a particular nuclide, group, rxn type
-look at $p$-value for Shapiro-Wilks tests of normality
-don't cluster data that may have come from a normal distribution!
-if the $p$-value is below a threshold, then data may be rejected as non-normal and may be good candidate for clustering



%%%%%%%%%%%%%%%%%%%%%%%%%%%%%%%%%%%%%%%%%%%%%%%%%%%%%%%%%%%%%%%%%%%%%%%%%%%%%%%
\section{Clustering Algorithms}
\label{subsec:chap10-clustering}

-very brief overview
-refer to literature for more info
-used canned algorithms in scikit-learn \texttt{sklearn.cluster}
-while there are many, many algorithms out there, only a few selected for discussion here
  -one concern is scalability to large number of samples

%%%%%%%%%%%%%%%%%%%%%%%%%%%%%%%%%
\subsection{$k$-Means Clustering}
\label{subsec:chap10-kmeans}

-introduce scheme
-algorithmic
-initialization
-model selection

%%%%%%%%%%%%%%%%%%%%%%%%%%%%%%%%%%%%
\subsection{Hierarchical Clustering}
\label{subsec:chap10-agglomerative}

-introduce scheme
-algorithmic
-initialization
-model selection

%%%%%%%%%%%%%%%%%%%%%%%%%%%%
\subsection{DBSCAN}
\label{subsec:chap10-dbscan}

-introduce scheme
-algorithmic
-initialization
-model selection

%%%%%%%%%%%%%%%%%%%%%%%%%%%%%%%%%
\subsection{Gaussian Mixture Models}
\label{subsec:chap10-gmms}

-introduce scheme
-algorithmic
-initialization
-intro fact that it is not scalable - not evaluated here
-mention dirichlet process mixture models
-model selection criteria are nice
-model selection

%%%%%%%%%%%%%%%%%%%%%%%%%%%%%%%%%%%%%%%%%%%%%%%%%%%%%%%%%%%%%%%%%%%%%%%%%%%%%%%
\section{Model Selection}
\label{sec:chap10-model-select}


%%%%%%%%%%%%%%%%%%%%%%%%%%%%%%%%%%%%%%%%%%%%%%%%%%%%%%%%%%%%%%%%%%%%%%%%%%%%%%%
\section{Clustered Geometries}
\label{sec:chap10-geometries}

-agglomerative clustering was used to generate the figures
-which nuclide, group, reaction was used for pinch clustering???

\begin{figure}[h!]
\centering
\begin{subfigure}{0.48\textwidth}
  \centering
  \includegraphics[width=0.9\linewidth]{figures/unsupervised/geometries/with-features/2-clusters/pinch/assm-16}
  \caption{}
  \label{fig:chap10-assm-16-pinch-2}
\end{subfigure}%
\begin{subfigure}{0.48\textwidth}
  \centering
  \includegraphics[width=0.9\linewidth]{figures/unsupervised/geometries/with-features/2-clusters/combined/assm-16}
  \caption{}
  \label{fig:chap10-assm-16-combined-2}
\end{subfigure}
\begin{subfigure}{0.48\textwidth}
  \centering
  \includegraphics[width=0.9\linewidth]{figures/unsupervised/geometries/with-features/4-clusters/pinch/assm-16}
  \caption{}
  \label{fig:chap10-assm-16-pinch-4}
\end{subfigure}%
\begin{subfigure}{0.48\textwidth}
  \centering
  \includegraphics[width=0.9\linewidth]{figures/unsupervised/geometries/with-features/4-clusters/combined/assm-16}
  \caption{}
  \label{fig:chap10-assm-16-combined-4}
\end{subfigure}
\begin{subfigure}{0.48\textwidth}
  \centering
  \includegraphics[width=0.9\linewidth]{figures/unsupervised/geometries/with-features/8-clusters/pinch/assm-16}
  \caption{}
  \label{fig:chap10-assm-16-pinch-8}
\end{subfigure}%
\begin{subfigure}{0.48\textwidth}
  \centering
  \includegraphics[width=0.9\linewidth]{figures/unsupervised/geometries/with-features/8-clusters/combined/assm-16}
  \caption{}
  \label{fig:chap10-assm-16-combined-8}
\end{subfigure}
\caption[Materials for the 1.6\% fuel assembly with clustering homogenization]{Materials for the 1.6\% enriched fuel assembly with clustering homogenization. The materials for 2, 4, and 8 clusters are illustrated in (a), (c) and (e) for pinch clustering, and in (b), (d) and (f) for combined clustering, respectively.}
\label{fig:chap10-assm-16-geometries}
\end{figure}

\clearpage

\begin{figure}[h!]
\centering
\begin{subfigure}{0.48\textwidth}
  \centering
  \includegraphics[width=0.9\linewidth]{figures/unsupervised/geometries/with-features/2-clusters/pinch/assm-31-20BPs}
  \caption{}
  \label{fig:chap10-assm-31-20BPs-pinch-2}
\end{subfigure}%
\begin{subfigure}{0.48\textwidth}
  \centering
  \includegraphics[width=0.9\linewidth]{figures/unsupervised/geometries/with-features/2-clusters/combined/assm-31-20BPs}
  \caption{}
  \label{fig:chap10-assm-31-20BPs-combined-2}
\end{subfigure}
\begin{subfigure}{0.48\textwidth}
  \centering
  \includegraphics[width=0.9\linewidth]{figures/unsupervised/geometries/with-features/4-clusters/pinch/assm-31-20BPs}
  \caption{}
  \label{fig:chap10-assm-31-20BPs-pinch-4}
\end{subfigure}%
\begin{subfigure}{0.48\textwidth}
  \centering
  \includegraphics[width=0.9\linewidth]{figures/unsupervised/geometries/with-features/4-clusters/combined/assm-31-20BPs}
  \caption{}
  \label{fig:chap10-assm-31-20BPs-combined-4}
\end{subfigure}
\begin{subfigure}{0.48\textwidth}
  \centering
  \includegraphics[width=0.9\linewidth]{figures/unsupervised/geometries/with-features/8-clusters/pinch/assm-31-20BPs}
  \caption{}
  \label{fig:chap10-assm-31-20BPs-pinch-8}
\end{subfigure}%
\begin{subfigure}{0.48\textwidth}
  \centering
  \includegraphics[width=0.9\linewidth]{figures/unsupervised/geometries/with-features/8-clusters/combined/assm-31-20BPs}
  \caption{}
  \label{fig:chap10-assm-31-20BPs-combined-8}
\end{subfigure}
\caption[Materials for the 3.1\% fuel assembly with clustering homogenization]{Materials for the 3.1\% enriched fuel assembly with 20 \acp{BP} with clustering homogenization. The materials for 2, 4, and 8 clusters are illustrated in (a), (c) and (e) for pinch clustering, and in (b), (d) and (f) for combined clustering, respectively.}
\label{fig:chap10-assm-31-20BPs-geometries}
\end{figure}

\clearpage

\begin{figure}[h!]
\centering
\begin{subfigure}{0.48\textwidth}
  \centering
  \includegraphics[width=0.9\linewidth]{figures/unsupervised/geometries/with-features/2-clusters/pinch/2x2}
  \caption{}
  \label{fig:chap10-2x2-pinch-2}
\end{subfigure}%
\begin{subfigure}{0.48\textwidth}
  \centering
  \includegraphics[width=0.9\linewidth]{figures/unsupervised/geometries/with-features/2-clusters/combined/2x2}
  \caption{}
  \label{fig:chap10-2x2-combined-2}
\end{subfigure}
\begin{subfigure}{0.48\textwidth}
  \centering
  \includegraphics[width=0.9\linewidth]{figures/unsupervised/geometries/with-features/4-clusters/pinch/2x2}
  \caption{}
  \label{fig:chap10-2x2-pinch-4}
\end{subfigure}%
\begin{subfigure}{0.48\textwidth}
  \centering
  \includegraphics[width=0.9\linewidth]{figures/unsupervised/geometries/with-features/4-clusters/combined/2x2}
  \caption{}
  \label{fig:chap10-assm-2x2-combined-4}
\end{subfigure}
\begin{subfigure}{0.48\textwidth}
  \centering
  \includegraphics[width=0.9\linewidth]{figures/unsupervised/geometries/with-features/8-clusters/pinch/2x2}
  \caption{}
  \label{fig:chap10-assm-2x2-pinch-8}
\end{subfigure}%
\begin{subfigure}{0.48\textwidth}
  \centering
  \includegraphics[width=0.9\linewidth]{figures/unsupervised/geometries/with-features/8-clusters/combined/2x2}
  \caption{}
  \label{fig:chap10-assm-2x2-combined-8}
\end{subfigure}
\caption[Materials for the 2$\times$2 periodic colorset with clustering homogenization]{Materials for the 2$\times$2 periodic colorset with clustering homogenization. The materials for 2, 4, and 8 clusters are illustrated in (a), (c) and (e) for pinch clustering, and in (b), (d) and (f) for combined clustering, respectively.}
\label{fig:chap10-2x2-geometries}
\end{figure}

\clearpage

\begin{figure}[h!]
\centering
\begin{subfigure}{0.48\textwidth}
  \centering
  \includegraphics[width=0.9\linewidth]{figures/unsupervised/geometries/with-features/2-clusters/pinch/reflector}
  \caption{}
  \label{fig:chap10-reflector-pinch-2}
\end{subfigure}%
\begin{subfigure}{0.48\textwidth}
  \centering
  \includegraphics[width=0.9\linewidth]{figures/unsupervised/geometries/with-features/2-clusters/combined/reflector}
  \caption{}
  \label{fig:chap10-reflector-combined-2}
\end{subfigure}
\begin{subfigure}{0.48\textwidth}
  \centering
  \includegraphics[width=0.9\linewidth]{figures/unsupervised/geometries/with-features/4-clusters/pinch/reflector}
  \caption{}
  \label{fig:chap10-reflector-pinch-4}
\end{subfigure}%
\begin{subfigure}{0.48\textwidth}
  \centering
  \includegraphics[width=0.9\linewidth]{figures/unsupervised/geometries/with-features/4-clusters/combined/reflector}
  \caption{}
  \label{fig:chap10-reflector-combined-4}
\end{subfigure}
\begin{subfigure}{0.48\textwidth}
  \centering
  \includegraphics[width=0.9\linewidth]{figures/unsupervised/geometries/with-features/8-clusters/pinch/reflector}
  \caption{}
  \label{fig:chap10-reflector-pinch-8}
\end{subfigure}%
\begin{subfigure}{0.48\textwidth}
  \centering
  \includegraphics[width=0.9\linewidth]{figures/unsupervised/geometries/with-features/8-clusters/combined/reflector}
  \caption{}
  \label{fig:chap10-reflector-combined-8}
\end{subfigure}
\caption[Materials for the 2$\times$2 colorset with reflector with clustering homogenization]{Materials for the 2$\times$ colorset with water reflector with clustering homogenization. The materials for 2, 4, and 8 clusters are illustrated in (a), (c) and (e) for pinch clustering, and in (b), (d) and (f) for combined clustering, respectively.}
\label{fig:chap10-reflector-geometries}
\end{figure}

\clearpage

\begin{figure}[h!]
\centering
\begin{subfigure}{0.48\textwidth}
  \centering
  \includegraphics[width=0.9\linewidth]{figures/unsupervised/geometries/with-features/2-clusters/pinch/full-core}
  \caption{}
  \label{fig:chap10-full-core-pinch-2}
\end{subfigure}%
\begin{subfigure}{0.48\textwidth}
  \centering
  \includegraphics[width=0.9\linewidth]{figures/unsupervised/geometries/with-features/2-clusters/combined/full-core}
  \caption{}
  \label{fig:chap10-full-core-combined-2}
\end{subfigure}
\begin{subfigure}{0.48\textwidth}
  \centering
  \includegraphics[width=0.9\linewidth]{figures/unsupervised/geometries/with-features/4-clusters/pinch/full-core}
  \caption{}
  \label{fig:chap10-full-core-pinch-4}
\end{subfigure}%
\begin{subfigure}{0.48\textwidth}
  \centering
  \includegraphics[width=0.9\linewidth]{figures/unsupervised/geometries/with-features/4-clusters/combined/full-core}
  \caption{}
  \label{fig:chap10-full-core-combined-4}
\end{subfigure}
\begin{subfigure}{0.48\textwidth}
  \centering
  \includegraphics[width=0.9\linewidth]{figures/unsupervised/geometries/with-features/8-clusters/pinch/full-core}
  \caption{}
  \label{fig:chap10-full-core-pinch-8}
\end{subfigure}%
\begin{subfigure}{0.48\textwidth}
  \centering
  \includegraphics[width=0.9\linewidth]{figures/unsupervised/geometries/with-features/8-clusters/combined/full-core}
  \caption{}
  \label{fig:chap10-full-core-combined-8}
\end{subfigure}
\caption[Materials for BEAVRS with clustering homogenization (2 -- 8 clusters)]{Materials for the quarter core \ac{BEAVRS} model with clustering homogenization. The materials for 2, 4, and 8 clusters are illustrated in (a), (c) and (e) for pinch clustering, and in (b), (d) and (f) for combined clustering, respectively.}
\label{fig:chap10-full-core-geometries-8}
\end{figure}

\clearpage

\begin{figure}[h!]
\centering
\begin{subfigure}{0.48\textwidth}
  \centering
  \includegraphics[width=0.9\linewidth]{figures/unsupervised/geometries/with-features/16-clusters/pinch/full-core}
  \caption{}
  \label{fig:chap10-full-core-pinch-16}
\end{subfigure}%
\begin{subfigure}{0.48\textwidth}
  \centering
  \includegraphics[width=0.9\linewidth]{figures/unsupervised/geometries/with-features/16-clusters/combined/full-core}
  \caption{}
  \label{fig:chap10-full-core-combined-16}
\end{subfigure}
\begin{subfigure}{0.48\textwidth}
  \centering
  \includegraphics[width=0.9\linewidth]{figures/unsupervised/geometries/with-features/32-clusters/pinch/full-core}
  \caption{}
  \label{fig:chap10-full-core-pinch-32}
\end{subfigure}%
\begin{subfigure}{0.48\textwidth}
  \centering
  \includegraphics[width=0.9\linewidth]{figures/unsupervised/geometries/with-features/32-clusters/combined/full-core}
  \caption{}
  \label{fig:chap10-full-core-combined-32}
\end{subfigure}
\begin{subfigure}{0.48\textwidth}
  \centering
  \includegraphics[width=0.9\linewidth]{figures/unsupervised/geometries/with-features/64-clusters/pinch/full-core}
  \caption{}
  \label{fig:chap10-full-core-pinch-64}
\end{subfigure}%
\begin{subfigure}{0.48\textwidth}
  \centering
  \includegraphics[width=0.9\linewidth]{figures/unsupervised/geometries/with-features/64-clusters/combined/full-core}
  \caption{}
  \label{fig:chap10-full-core-combined-64}
\end{subfigure}
\caption[Materials for BEAVRS with clustering homogenization (16 -- 64 clusters)]{Materials for the quarter core \ac{BEAVRS} model with clustering homogenization. The materials for 16, 32, and 64 clusters are illustrated in (a), (c) and (e) for pinch clustering, and in (b), (d) and (f) for combined clustering, respectively.}
\label{fig:chap10-full-core-geometries-64}
\end{figure}

\clearpage

%%%%%%%%%%%%%%%%%%%%%%%%%%%%%%%%%%%%%%%%%%%%%%%%%%%%%%%%%%%%%%%%%%%%%%%%%%%%%%%
\section{Evaluating the Methodology}
\label{sec:chap10-cluster}

%%%%%%%%%%%%%%%%%%%%%%%%%%%%%%%
\subsection{Perfect Clustering}
\label{subsec:chap10-perfect-cluster}

%%%%%%%%%%%%%%%%%%%%%%%%%%%%%%%%
\subsection{Adaptive Clustering}
\label{subsec:chap10-adaptive-cluster}

\begin{itemize}[noitemsep]
  \item dataset restrictions
  \begin{itemize}[noitemsep]
    \item \textbf{``pinch'' clustering} - a \textit{single} nuclide/group/reaction triplet
    \item \textbf{``combined'' clustering} - one or more nuclide/group pairs \textit{together}
    \item \textbf{``local'' clustering} - one or more nuclide/group pairs \textit{separately}
  \end{itemize}
  \item clustering algorithms
  \begin{itemize}[noitemsep]
    \item discriminative clustering ($k$-Means++, Agglomerative, etc.)
    \item generative clustering (Gaussian Mixtures, Dirichlet Processes)
  \end{itemize}
  \item ``regression-informed'' clustering
  \begin{itemize}[noitemsep]
    \item use a regressor to predict ``smoothed'' of target variable(s)
    \begin{itemize}[noitemsep]
      \item decision tree regression
      \item ensemble (boosting, random forest) regression
      \item Gaussian Process regression
    \end{itemize}
    \item cluster predictions produced by regressor
    \item advantage is that objective function \textit{may} be more appropriate
  \end{itemize}
\end{itemize}