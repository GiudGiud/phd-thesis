\chapter{Convergence of MOC Source Iteration}
\label{chap:moc-convergence}

In the previous chapters, the implementation of methods in OpenMOC has been discussed in great detail. All of these methods rely on the \ac{MOC} form of the transport equation presented in Chapter~\ref{chap:moc}. More specifically, these methods rely on the source iteration form the \ac{MOC} equations, possibly used in conjunction with \ac{CMFD} acceleration discussed in Chapter~\ref{chap:cmfd}. In this chapter, the convergence of source iteration with and without \ac{CMFD} acceleration is addressed. While convergence is straightforward for physical cross-sections, the transport correction discussed in Section~\ref{sec:transport-correction} can cause convergence issues. 

In Section~\ref{sec:source-iteration-intro}, the multi-group transport problem is formalized in matrix form which forms the basis of discussing convergence of source iteration. Section~\ref{sec:equiv-cpm} relates multi-group transport forms to collision probabilities, allowing for an analytical description of iteration matrix elements. Section~\ref{sec:moc-iteration-schemes} discusses methods to solve the resulting system of equations, focusing on source iteration. Theoretical issues with source iteration are discussed for transport-corrected cross-sections. Then, in Section~\ref{sec:diagonal-stabilization} a solution is proposed, termed diagonal stabilization. Convergence issues of source iteration are demonstrated on realistic problems in Section~\ref{sec:convergence-results} and diagonal stabilization is shown to fix the convergence issues. Finally, Section~\ref{sec:convergence-conclusion} concludes the chapter by highlighting the important aspects of the convergence behavior.

\section{Introduction}
\label{sec:source-iteration-intro}

As discussed in Chapter~\ref{chap:moc}, \ac{MOC} discretizes the transport equation and solves a linear system of the form
\begin{equation}
	\boldsymbol{\phi} = J \left(\frac{1}{k} F + S \right) \boldsymbol{\phi}
	\label{eq:si-transport}
\end{equation}
where $\boldsymbol{\phi}$ represents the vector of all scalar fluxes, $J$ is the transport sweep matrix, $F$ is the fission matrix, $S$ is the scattering matrix, and $k$ is the eigenvalue. This form is not unique to \ac{MOC}, but rather many methods solve the multi-group transport equation in the same form, differing only in the way matrix-vector products with $J$ are computed.

In many efficient transport solvers, these matrices are not explicitly formed but rather they implicitly solve these equations with sweeps taking the place of explicit formulation of the matrix elements. Therefore, these matrices are often presented as operators rather than matrices, but the underlying matrix elements could be computed if desired. 

For instance, recall from Chapter~\ref{chap:moc} that OpenMOC forms components of these matrices on-the-fly. Components of the matrix $J$ are also never explicitly computed. Instead, the internal structure of $J$ is used to efficiently compute matrix-vector products with $J$. Other transport methods take similar approaches for computational efficiency.

For physical cross-sections, the matrices $J$, $F$, and $S$ are all positive. However, when transport correction is introduced, as outlined in Chapter~\ref{chap:mgxs}, components of the scattering matrix can become negative. Specifically, recall that for a region $i$ and energy group $g$, the transport correction $\Delta \Sigma_{\textit{tr}, i, g}$ is applied to both the total cross-section $\Sigma_{t, i, g}$ and within-group scattering cross-section $\Sigma_{s, i}^{g \rightarrow g}$ resulting in a transport cross-section $\Sigma_{\textit{tr}, i, g}$ and transport-corrected within-group scattering cross-section $\tilde{\Sigma}_{s, i}^{g \rightarrow g}$ as
\begin{equation}
\begin{split}
\Sigma_{\textit{tr}, i, g} = \Sigma_{t, i, g} -  \Delta \Sigma_{\textit{tr}, i, g} \\
\tilde{\Sigma}_{s, i}^{g \rightarrow g} = \Sigma_{s, i}^{g \rightarrow g} -  \Delta \Sigma_{\textit{tr}, i, g}
\end{split}
\label{eq:total-xs}
\end{equation}

When using large numbers of energy groups, the within-group scattering cross-sections can become small, and the modified within-group scattering cross-section can become negative with transport correction. Tabuchi discovered negative within-group scattering cross-sections to be an issue when converging inner within-group scattering iterations of \ac{MOC}~\cite{ty-problem}. Specifically, it was identified that the inner iteration matrix could have a spectral radius greater than unity, causing the system not to converge. Tabuchi later proposed a stabilization technique for inner iterations~\cite{ty-solution}.

In this chapter, the ideas presented by Tabuchi are expanded to explain the observed convergence behavior of source iteration without inner iterations. Convergence is analyzed with transport corrected cross-sections. Realistic full core cases are presented that observe the failed convergence behavior and a new stabilizing method is proposed that stabilizes source iteration.

\section{Equivalence of Collision Probability Methods}
\label{sec:equiv-cpm}

Deterministic methods such as flat source \ac{MOC} or step characteristics are equivalent to the collision probability form, aside from discretization errors~\cite{ty-problem}. Although collision probabilities do not explicitly enter the transport method, the neutron balance equation takes the form in Eq.~\ref{eq:cpm-form}:
\begin{equation}
	\Sigma_{\textit{tr}, i, g} \phi_{i,g} V_i = \sum_j P_{ji, g} q_{j,g} V_j
	\label{eq:cpm-form}
\end{equation}
where $P_{ji,g}$ represents the collision probability of a neutron of energy group $g$ from region $j$ to region $i$, $V_i$ represents the volume of region $i$, $q_{i,g}$ represents the neutron source and $\phi_{i,g}$ represents the average scalar flux in region $i$ and group $g$. In general, the neutron source can be computed by summing contributions from all $G$ energy groups as
\begin{equation}
	q_{i,g} = \sum_{g'=1}^G \left( \Sigma_{s,i}^{g'\rightarrow g} \phi_{i,g'} + \chi_{i,g} \nu \Sigma_{f,i,g'} \phi_{i,g'} \right)
\end{equation}
where $\Sigma_{s,i}^{g'\rightarrow g}$ is the scattering cross-section in region $i$ from group $g'$ to group $g$, $\chi_{i,g}$ is the fission emission probability for group $g$ in region $i$, and $\nu \Sigma_{f,i,g'}$ is the fission production in region $i$ from group $g'$. Combining this definition with Eq.~\ref{eq:cpm-form}, as well as the reciprocity relationship, 
\begin{equation}
	P_{ij, g} \Sigma_{\textit{tr}, i, g} V_i = P_{ji, g} \Sigma_{\textit{tr}, j, g} V_j,
	\label{eq:reciprocity}
\end{equation}
neutron balance can be presented in the form of Eq.~\ref{eq:cpm-balance}.
\begin{equation}
	\phi_{i,g} = \sum_j \frac{P_{ij, g} \sum_{g'=1}^G \left( \Sigma_{s,j}^{g'\rightarrow g} \phi_{j,g'} + \chi_{j,g} \nu \Sigma_{f,j,g'} \phi_{j,g'} \right)}{\Sigma_{\textit{tr}, j, g}} 
	\label{eq:cpm-balance}
\end{equation}
Notice that this is in the form of Eq.~\ref{eq:si-transport}. Therefore, the matrix $A = J\left(\frac{1}{k}F + S\right)$, with rows and columns indexed by $(i, g)$  where $i$ is the region and $g$ is the energy group, can be expressed as
\begin{equation}
	A_{(i,g), (j, g')} = P_{ij}^g \left(\frac{\chi_{j,g} \nu\Sigma_{f,j,g'} / k + \Sigma_{s,j}^{g' \rightarrow g}}{\Sigma_{\textit{tr}, j, g}}\right).
	\label{eq:a-matrix}
\end{equation}
Matrix $A$ is square and dense with length equal to the number of scalar fluxes. 

\section{Iteration Schemes}
\label{sec:moc-iteration-schemes}

Once a form for the matrix elements is defined, many iterative solvers could be used to solve this system. First, a naive power method approach is presented which does not account for structure of the \ac{MOC} equations but ensures convergence. Then, the common source iteration technique is introduced which takes advantage of \ac{MOC} structure but is not necessarily guaranteed to converge.

\subsection{Power Method}

The power method is one common method to solve an eigenvalue problem, such as the form given in Eq.~\ref{eq:si-transport}, which yields the dominant eigenvector corresponding to the steady-state flux distribution. To invoke the power method, Eq.~\ref{eq:si-transport} can be re-arranged to the form in Eq.~\ref{eq:eigenvalue} where $I$ represents the identity matrix.
\begin{equation}
\begin{split}
Z \equiv \left(I-JS\right)^{-1} JF \\
Z \boldsymbol{\phi} =  k \boldsymbol{\phi}
\end{split}
\label{eq:eigenvalue}
\end{equation}
With the power method scheme, repeated multiplication by $Z$ yields the dominant eigenvector~\cite{numerical-analysis}. Rather than performing strict power method iterations, the estimate of the eigenvalue $k_n$ at iteration $n$ is often included in the source. This scheme is presented in Eq.~\ref{eq:power-method}. These iterations are often termed \textit{outer iterations}.
\begin{equation}
\boldsymbol{\phi}_{n+1} =  \left(I-JS\right)^{-1} J\frac{F}{k_n} \boldsymbol{\phi}_n
\label{eq:power-method}
\end{equation}
Since explicitly taking the matrix inverse of $I-JS$ is unwise, an iterative scheme is necessary to solve the linear system. In many transport methods, such as \ac{MOC}, computing each element of $J$ can be just as expensive as computing a matrix-vector product. In addition, each matrix-vector product with matrix $J$ is usually quite expensive (often termed \textit{transport sweeps}). Therefore, few methods are available to efficiently solve the linear system in practice. 

\subsection{Source Iteration}

An alternative way to solve the transport equation is to directly apply the transport sweep to the full neutron source. Note that the neutron source $\mathbf{q}$ can be computed as 
\begin{equation*}
	\mathbf{q} = \left(\frac{1}{k} F + S \right) \boldsymbol{\phi}. 
\end{equation*}
The transport sweep matrix $J$ therefore yields the scalar flux distribution associated with the computed source distribution. In this form, a new iterative process could be introduced in which a new source distribution is computed at each iteration, yielding a new estimate of the associated scalar flux distribution. Therefore, solving the transport equation in this form where source terms are lagged is termed \textit{source iteration}.  Specifically, the eigenvalue problem in Eq.~\ref{eq:si-transport} is iteratively solved with the left hand side updated and the right hand side lagged as
\begin{equation}
	\boldsymbol{\phi}_{n+1} =  J\left(\frac{F\boldsymbol{\phi}_n}{k_n} + S\boldsymbol{\phi}_n\right).
	\label{eq:power-method-full-source}
\end{equation}
It is important to note that this process is non-linear due to the iteration matrix $J\left(\frac{1}{k_n} F + S\right)$ being dependent on the iteration number $n$. To simplify this relationship, assume that the eigenvalue $k$ is perfectly known to be $k_{\textit{crit}}$, the eigenvalue associated with the dominant mode of the system. In reality, the exact value of $k_{\textit{crit}}$ is not known a priori but observations and intuition suggests that it does not strongly impact convergence. With the eigenvalue fixed, the system becomes 
\begin{equation}
	\boldsymbol{\phi}_{n+1} =  A \boldsymbol{\phi}_n
\end{equation}
where matrix $A$ is defined in Eq.~\ref{eq:a-matrix} with $k = k_{\textit{crit}}$. This process is equivalent to power method iterations with the matrix $A$, which converges to the eigenvector associated with the dominant eigenvalue of $A$. Since $k_{\textit{crit}}$ is the dominant eigenvalue of the original system, 1.0 must be an eigenvalue of $A$. In addition, if an everywhere positive solution exists, it must be associated with the physical solution.

Recall from Eq.~\ref{eq:a-matrix} that the iteration matrix $A$ is everywhere positive and real as long as the within-group scattering cross-section is positive. According to the Perron-Frobenius Theorem~\cite{perron-frobenius}, square matrices of all positive real entries have a unique largest real eigenvalue which is dominant and associated with an everywhere positive eigenvector. In addition, all other eigenvectors must have a negative component. Without transport correction, the iteration matrix $A$ is an all positive and real matrix, implying that the largest eigenvalue is 1.0 and is associated with the physical solution. Therefore, the process detailed in Eq.~\ref{eq:power-method-full-source} will converge to the physical solution.

However, with the transport correction, it is possible to have negative within-group scattering causing this criteria not to hold. If this criteria does not hold, the system might still converge, but convergence cannot be guaranteed under the Perron-Frobenius Theorem. Therefore, the iteration scheme should be updated to ensure convergence. 

\section{Stabilization of Source Iteration}
\label{sec:diagonal-stabilization}

Note that Eq.~\ref{eq:damped-transport} is a mathematically valid rewriting of Eq.~\ref{eq:si-transport} for \textit{any matrix} $D$ where $I+D$ is invertible. 
\begin{equation}
	\boldsymbol{\phi} = (I+D)^{-1} \left[J \left(\frac{1}{k} F + S \right) + D \right]\boldsymbol{\phi}
	\label{eq:damped-transport}
\end{equation}
Next, the same source iteration scheme is applied where the left hand side is updated with the right hand side constant. Since $I+D$ needs to be easily invertible for this new scheme to be efficient, $D$ is chosen to be diagonal. The convergence discussion then follows the same discussion as before except the new iteration matrix  now has the form $\tilde{A} = (I+D)^{-1} \left[ J \left(\frac{1}{k_{\textit{crit}}} F + S \right) + D \right]$  with the matrix $\tilde{A}$ is described by
\begin{equation}
	\tilde{A}_{(i,g), (j, g')} = \frac{1}{1 + D_{(i,g), (i,g)}}\left[P_{ij}^g \left(\frac{\chi_{j,g} \nu\Sigma_{f,j,g'} / k_{\textit{crit}} + \Sigma_{s,j}^{g' \rightarrow g}}{\Sigma_{\textit{tr}, j, g}}\right) + D_{(i,g), (j, g')}\right]
	\label{eq:a-tilde}
\end{equation}
where the rows and columns are again indexed by region, energy group pairs. The diagonal elements of $D$ are chosen to be:
\begin{equation}
	D_{(i,g), (i,g)} = \left\{\begin{array}{lr}
		\frac{-\rho \Sigma_{s,i}^{g \rightarrow g}}{\Sigma_{\textit{tr}, i, g}} , & \text{for } \Sigma_{s,i}^{g \rightarrow g} < 0\\
		0, & \text{otherwise}
	\end{array}\right.
	\label{eq:d-matrix}
\end{equation}
where the damping coefficient $\rho$ is a positive value chosen by the user. For $\rho = 1$, the diagonal update ensures that there are no negative diagonal elements in the iteration matrix. 

%If $A$ is symmetric, this approach can be proven to ensure convergence for $\rho=1$. A detailed proof is given in Appendix~\ref{app:convergence}. However, the matrix $A$ is rarely symmetric for common reactor physics problems. Still, this method could have utility even for non-symmetric $A$.

The diagonal stabilization scheme has the effect of shifting the iteration matrix eigenvalues to be more positive. The intuition for this shift is due to the Gershgorin Disk Theorem where eigenvalues exist in disks around diagonal elements. Since the diagonal elements are positively increased and all matrix elements are contracted, the Gershgorin Disks are likewise shifted positively with their radii contracted. While this improves stability, it also tightens the positive-eigenvalue modes, causing larger dominance ratios in the iteration matrix, and slower convergence. Therefore, $\rho$ should be chosen to be small while still ensuring convergence. 

\section{Convergence Results}
\label{sec:convergence-results}

In order to test the convergence behavior of the stabilization methods, the stabilization methods were implemented in OpenMOC. This section focuses on results for the realistic reactor physics test problems presented in Chapter~\ref{chap:beavrs}, all using the same cross-section set. Results in this chapter focus on the flat source 3D \ac{MOC} solver in OpenMOC for simplicity. All results use the \ac{MOC} parameters given in Table~\ref{tab:convergence-tests-params} unless otherwise specified. While finer \ac{MOC} parameters would be necessary to accurately resolve the solution, the convergence has not been observed to change substantially from refining \ac{MOC} parameters beyond these values.

\begin{table}[ht]
	\centering
	\caption{MOC parameters for convergence studies}
	\medskip
	\begin{tabular}{lc}
		\hline
		Source Approximation & Flat \\
		Number of Sectors in Moderator & 8 \\
		Number of Sectors in Fuel & 4 \\
		Height of Flat Source Regions & 2.0 cm \\
		Radial Ray Spacing & 0.1 cm \\
		Axial Ray Spacing & 1.5 cm \\
		Number of Azimuthal Angles & 32 \\
		Number of Polar Angles & 6 \\
		\hline
	\end{tabular}
	\label{tab:convergence-tests-params}
\end{table}

To analyze the convergence behavior, an error metric is needed. Usually, error is analyzed in terms of RMS error in the fission distribution over all fissile regions. During simulations, this can only be compared in terms of results from previous iterations. However, for these convergence studies, a reference solution is used which represents the final fission distribution of a tightly converged case. Therefore, error can reliably be assessed rather than the difference with previous iterations.

Often \ac{CMFD} acceleration is necessary to achieve reasonable convergence on practical reactor physics problems. \ac{CMFD} acceleration, as detailed in Chapter~\ref{chap:cmfd}, is implemented in OpenMOC with damping on the current correction term~\cite{smith2002casmo}, often termed the \ac{CMFD} relaxation factor. When \ac{CMFD} acceleration is applied, the behavior is much more difficult to analyze since \ac{CMFD} is a non-linear process. Therefore, discussion of \ac{CMFD} results will focus on hypotheses to explain the observed results rather than rigorous analysis. Results are presented with and without \ac{CMFD} acceleration.

In this chapter, the \ac{CMFD} mesh is always chosen to be a uniform mesh with cells of pin-cell pitch in the radial dimensions and 2.0 cm in the axial dimension. In addition, a relaxation factor of 0.7 is applied to the \ac{CMFD} acceleration scheme for computing corrected diffusion coefficients. Many \ac{CMFD} implementations choose to collapse to a fewer number of energy groups in order to improve the speed of the CMFD solver. While the \ac{CMFD} equations are computationally less expensive in collapsed few-group structures, it is possible that collapsing group structures could cause slower convergence by not fully resolving spectral effects. Results with a variety of \ac{CMFD} group structures are presented. While the \ac{MOC} calculation always uses a 70 group structure, \ac{CMFD} group structures are formed with $G_C$ \ac{CMFD} groups, with group ranges taken from the CASMO-4 Manual~\cite{edenius1995casmo}.

\subsection{Single Assembly with Axial Water Reflectors}
\label{sec:sa-axial-ref}

The single assembly model detailed in Section~\ref{sec:beavrs-single-assembly} is first analyzed. It is important to remember that this problem includes axial water reflectors. These large regions of water where the transport-correction is large could potentially have a strong impact on convergence.

\subsubsection{Source Iteration Convergence without CMFD Acceleration}

The single assembly model is first analyzed using pure source iteration, without any \ac{CMFD} acceleration. In order to focus on the asymptotic convergence behavior, rather than the behavior far from convergence, the \ac{MOC} simulations start with a \textit{good guess} of the scalar flux distribution. Specifically, a converging simulation where the error drops below $3\times 10^{-5}$ on RMS fission rate error is chosen as the starting point.

The convergence behavior is presented in Figure~\ref{fig:no-cmfd} in which the relative error is plotted as a function of iteration number for a variety of stabilizing schemes.
\begin{figure}[ht!]
	\centering
	\includegraphics[width=1.0\linewidth]{figures/convergence/no_cmfd.png}
	\caption{Convergence behavior of different stabilization schemes. TY Stabilization refers to the approach presented by Tabuchi whereas Diagonal Stabilization refers to the approach presented in this paper with damping coefficient $\rho$.}
	\label{fig:no-cmfd}
\end{figure}
Notice that the case without stabilization diverges while all the stabilization cases converge except for the new diagonal stabilization scheme with damping factor $\rho = 1/128$. This is expected since as $\rho$ approaches zero as the stabilization method becomes equivalent to source iteration without stabilization. This clearly shows that this problem requires stabilization to converge with source iteration. In addition, the more conservative the stabilization scheme, the more convergence is slowed. The scheme proposed by Tabuchi slows convergence the most. The diagonal damping scheme hinders convergence significantly less, especially for $\rho < 1$. 

\subsubsection{Convergence with CMFD Acceleration}

The results with \ac{CMFD} applied and \textit{without} any \ac{MOC} source iteration stabilization are shown in Figure~\ref{fig:sa-cmfd-no-stab} for a variety of \ac{CMFD} group structures with the error once again relative to a tightly converged reference.
\begin{figure}[ht!]
	\centering
	\includegraphics[width=0.65\linewidth]{figures/convergence/sa_no_stab_cmfd.png}
	\caption{Convergence behavior for a variety of \ac{CMFD} group structures with $G_C$ \ac{CMFD} groups and \textit{without} \ac{MOC} source iteration stabilization.}
	\label{fig:sa-cmfd-no-stab}
\end{figure}
As expected, most of the convergence histories have trouble converging. This is likely due to the convergence issue observed with pure source iteration. However, the 70 group \ac{CMFD} scheme appears to be stable. It is important to remember that the \ac{CMFD} equations do not suffer from the issue of a negative dominant eigenvalue in the source iteration scheme since since they solve an alternate form of the original eigenvalue system with standard eigenvalue solver techniques. 

If 70 group \ac{CMFD} is driving the convergence process, then the instability in the underlying source iteration technique might be overcome by the \ac{CMFD} acceleration prolongation update in which \ac{MOC} scalar fluxes are updated with the \ac{CMFD} solution. One way of interpreting this behavior is to think of the convergence as being dominated by the \ac{CMFD} acceleration where \ac{MOC} transport sweeps are used purely to resolve local behavior, integrate reaction rates to form \ac{CMFD} cross-sections, and tally currents crossing \ac{CMFD} mesh surfaces. 

However, with a reduced \ac{CMFD} group structure, the flux distribution in energy within a single \ac{CMFD} group \textit{must} be resolved by the \ac{MOC} source iteration process unless an effective energy interpolation can be formed during the \ac{CMFD} prolongation stage. Since the underlying \ac{MOC} source iteration process is unstable, any reliance on the process for convergence using a reduced group structure should lead to instability.

\subsubsection{Convergence with CMFD Acceleration and Source Iteration Stabilization}

To verify that the convergence issue can be remedied with \ac{MOC} source iteration stabilization, the same test is conducted but using diagonal stabilization with $\rho = 1/4$. For pure source iteration, this scheme was able to stabilize source iteration, as previously shown in Figure~\ref{fig:no-cmfd}. The results using this stabilization with \ac{CMFD} acceleration are shown in Figure~\ref{fig:sa-cmfd-stab}. As expected, the \ac{MOC} source iteration stabilization fixes the convergence issues for all \ac{CMFD} group structures, indicating that the convergence issue was caused by the underlying \ac{MOC} source iteration.
\begin{figure}[ht!]
	\centering
	\includegraphics[width=0.65\linewidth]{figures/convergence/sa_stab_cmfd.png}
	\caption{Convergence behavior for a variety of \ac{CMFD} group structures with $G_C$ \ac{CMFD} groups and \textit{with} Diagonal \ac{MOC} source iteration stabilization ($\rho = 1/4$).}
	\label{fig:sa-cmfd-stab}
\end{figure}

\subsection{Single Assembly without Axial Water Reflectors}
\label{sec:sa-no-axial-ref}

Next, the truncated single assembly model is studied, as detailed in Section~\ref{sec:trunc-single-assembly}. This model is the same as the single assembly model but without the axial water reflectors. Without reflectors, this model is more similar to standard lattice physics models. The convergence behavior is analyzed \textit{without} source iteration stabilization. The results with \ac{CMFD} acceleration for a variety of \ac{CMFD} group structures are presented in Figure~\ref{fig:truncated-convergence}.
\begin{figure}[ht!]
	\centering
	\includegraphics[width=0.65\linewidth]{figures/convergence/sa_trunc_no_stab.png}
	\caption{Convergence behavior for the single assembly \textit{without} axial reflectors with a variety of \ac{CMFD} group structures with $G_C$ \ac{CMFD} groups and \textit{without} \ac{MOC} source iteration stabilization.}
	\label{fig:truncated-convergence}
\end{figure}
Notice that the convergence issues are not present in this model. This indicates the issue is caused by the axial water reflectors. More generally, it appears to be a problem with deep water reflectors that have negative cross-sections. Recall from the theory discussion that iteration matrix components in Eq.~\ref{eq:a-matrix} were formed with the product of collision probability and self-scattering cross-sections. In deep water reflectors, the probability of transport between water regions containing negative cross-sections is significantly higher than in the active fuel where many neutrons collide with fuel. This motivates the idea of neutron behavior in deep reflectors causing the instability. 

\subsection{Full Core Behavior}

The previous results were focused on a single assembly with and without axial reflectors which proved the existence of source iteration instability. Since this issue seems to only appear at very low errors, the reader might judge this issue to be purely academic. However, for problems where deeper reflectors exist, such as a typical full core \ac{PWR} problem, this issue is exacerbated. To test this effect, full core models are simulated using the same cross-section set applied in the single assembly studies. The geometry outside the core is radially discretized into square regions of width $1/3$ of a pin-cell pitch width.

\subsubsection{2D Extruded Model}

Before simulating the explicit 3D model, the 2D extruded model detailed in Section~\ref{sec:beavrs-2D} is studied. This model lacks the axial water reflectors but still contains significant water reflectors in the radial direction. The results are shown in Figure~\ref{fig:fc-2D}. 
\begin{figure}[ht!]
	\centering
	\includegraphics[width=0.65\linewidth]{figures/convergence/fc_2D.png}
	\caption{Convergence behavior of \ac{CMFD} group structures with $G_C$ \ac{CMFD} groups and Diagonal \ac{MOC} source iteration stabilization with stabilization coefficient $\rho$.}
	\label{fig:fc-2D}
\end{figure}
The simulation is run with 8 group and 70 group \ac{CMFD} with Diagonal Stabilization. The damping coefficient $\rho$ is chosen for both cases to be zero (equivalent to no stabilization) and $1/4$. For all reduced \ac{CMFD} group structures, similar results are observed as seen with the 8 group \ac{CMFD} structure in Figure~\ref{fig:fc-2D} in which reasonable convergence can only be obtained by applying the \ac{MOC} diagonal stabilization.

\subsubsection{Explicit 3D Model}

Now the full core 3D BEAVRS model is simulated, as detailed in Section~\ref{sec:beavrs-3D}. Since the problem is large, the \ac{MOC} angular quadrature is significantly coarsened to 4 azimuthal angles and 2 polar angles in order to run a significant number of iterations on the Cetus partition of the Argonne BlueGene/Q supercomputer. These parameters are very coarse, but still exhibit the convergence issue. Since the run time with coarse angles can become dominated by the \ac{CMFD} solution time, it is infeasible to use 70 \ac{CMFD} groups. Therefore, results are presented only for a reduced 8 group \ac{CMFD} structure. The results are shown in Figure~\ref{fig:fc-3D} both with and without using the Diagonal Stabilization technique with $\rho = 1/4$. 

\begin{figure}[ht!]
	\centering
	\includegraphics[width=0.65\linewidth]{figures/convergence/full-core-3D.png}
	\caption{Convergence behavior of OpenMOC on the full core \ac{BEAVRS} benchmark with and without Diagonal Stabilization ($\rho = 1/4$).}
	\label{fig:fc-3D}
\end{figure}

These results show that for the 3D full core \ac{PWR} problem, a stabilization technique is necessary in order to reach any reasonable convergence, especially when it is computationally infeasible to use a many-group \ac{CMFD} solver.

\subsubsection{Explicit 3D Model with Linear Source}

The preceding tests have all been conducted using a flat source approximation. However, the goal of this thesis is to solve full core \ac{PWR} problems with a linear source approximation. While the theory of the diagonal stabilization was developed for flat source, it can also be applied to simulations with a linear source approximation. In the OpenMOC implementation, the same damping of each flat source flux is applied to the associated linear source moments. The results using a linear source approximation are presented in Figure~\ref{fig:fc-3D-ls} for the full core 3D BEAVRS benchmark, showing similar results to those with a flat source approximation.

\begin{figure}[ht!]
	\centering
	\includegraphics[width=0.65\linewidth]{figures/convergence/full-core-3D-ls.png}
	\caption{Convergence behavior of OpenMOC's linear source solver on the full core \ac{BEAVRS} benchmark with and without Diagonal Stabilization ($\rho = 1/4$).}
	\label{fig:fc-3D-ls}
\end{figure}

\section{Conclusion}
\label{sec:convergence-conclusion}

In this chapter, a theoretical analysis of source iteration convergence was presented which showed the possibility for source iteration instabilities. The theory was broadened to include source iterations without inner iterations. A new diagonal stabilization technique was presented which is substantially less conservative than previously proposed techniques. Realistic reactor physics problems were presented that showed instability with the source iteration process. The diagonal stabilization technique was shown to successfully stabilize the source iteration process without having much impact on convergence rate.

Results were also presented for convergence with \ac{CMFD} acceleration which showed that \ac{CMFD} acceleration without group collapse was able to stabilize the source iteration process. For \ac{CMFD} acceleration with a collapsed group structure, the previously discussed stabilization techniques were necessary to ensure convergence. In the case of full core \ac{PWR} problems, a collapsed group structure is often attractive to reduce run time, necessitating a stabilization method in order to converge.


\vfill
\begin{highlightsbox}[frametitle=Highlights]
	\begin{itemize}
		
		\item The \ac{MOC} eigenvalue problem can be cast a generalized eigenvalue problem
		
		\item For computational efficiency, source iteration is preferable to more conventional methods for solving eigenvalue problems
		
		\item Without transport correction, the Perror-Frobenius theorem guarantees convergence of source iteration
		
		\item With transport correction, source iteration is not guaranteed to converge
		
		\item A stabilization scheme, termed diagonal stabilization, is introduced which stabilizes source iteration allowing for convergence
		
		\item While diagonal stabilization was formed based on the flat source \ac{MOC} equations, it has also been shown to stabilize source iteration for a linear source approximation
		
	\end{itemize}
\end{highlightsbox}
\vfill