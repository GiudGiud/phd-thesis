\begin{abstractpage}

A key challenge for whole-core deterministic neutron transport methods is accurate reactor agnostic multi-group cross section (MGXS) generation. Monte Carlo (MC) presents the most accurate method for reactor agnostic multi-group cross section generation since it does not require the use of any local approximations to the flux. This improvement in accuracy comes at the computational expense of converging group constant tallies to acceptably low uncertainties. MC methods have increasingly been used to generate few group constants for coarse mesh diffusion, most notably by the Serpent MC code. This work develops clustering techniques which employ engineering heuristics or machine learning algorithms to accelerate the convergence of MGXS tallied on fine, heterogeneous spatial meshes in Monte Carlo.

%This work investigates the use of MC methods to generate MGXS for high-fidelity whole-core transport codes.

%No systematic methodology of approximations has been developed which is broadly applicable to all reactor types.

%The nuclear reactor physics community has long strived for deterministic neutron transport-based tools for full-core reactor analysis. 

This thesis is organized along two main themes. First, the efficacy of MGXS generation with MC for fine mesh transport calculations is rigorously assessed. Some of the approximations made by MC-based MGXS generation are quantified, including the angular, energy- and spatial-dependence of condensed MGXS. The second theme develops a novel methodology called inferential MGXS (\textit{i}MGXS) which directly models all energy and spatial self-shielding effects to generate accurate MGXS with a single whole-core MC calculation. The \textit{i}MGXS scheme is a data processing pipeline that uses unsupervised machine learning algorithms to infer the clustering of MGXS from ``noisy'' MC tally data with limited human supervision.

The \textit{i}MGXS scheme is evaluated for six PWR benchmarks with heterogeneities with varying spatial self-shielding effects, including control rod guide tubes and water reflectors. The \textit{i}MGXS scheme was shown to enable highly accurate predictions of U-238 capture rate predictions, reducing the error by 3 -- 4$\times$ with respect to schemes which neglect to model MGXS clustering. In addition, the scheme requires at least 10$\times$ fewer MC particle histories to converge MGXS for accurate multi-group deterministic calculations than a reference MC calculation. These results demonstrate the promise for \textit{i}MGXS as a means to efficiently generate MGXS with reactor agnostic MC calculations of the complete heterogeneous geometry in a single step with little human oversight. The \textit{i}MGXS scheme may be valuable for reactor physics analysis of advanced LWR core designs as well as next generation reactors with spatial self-shielding effects that are poorly modeled by the engineering approximations in today's methods for MGXS generation.

%for which engineering approximations for MGXS generation poorly model spatial self-shielding effects of aribtrary core heterogeneities.

%which do not easily lend themselves to engineering approximations for generating MGXS for neutron transport calculations.

%Nevertheless, it points to the potential for \textit{i}MGXS to harness \ac{MC} to efficiently generate accurate \ac{MGXS} for deterministic transport codes.

%The inference of MGXS clustering with the \textit{i}MGXS scheme enables deterministic reactor physics simulations to produce accurate results from MGXS generated by MC faster than would be possible with a direct calculation with MC.

\end{abstractpage}
