\chapter{Domain Decomposition}
\label{chap:dd}

%%%%%%%%%%%%%%%%%%%%%%%%%%%%%%%%%%%%%%%%%%%%%%%%%%%%%%%%%%%%%%%%%%%%%%%%%%%%%%%%
%%%%%%%%%%%%%%%%%%%%%%%%%%%%%%%%%%%%%%%%%%%%%%%%%%%%%%%%%%%%%%%%%%%%%%%%%%%%%%%%
\section{Background}

The memory challenges discussed in Chapter~\ref{chap:intro} can be addressed
with \emph{domain decomposition}. We specifically focus on the spatial domain
because that is where the largest memory burdens are incurred: spatially
heterogeneous tallies and materials needed for full-core depletion. While many
nuances to the concept can be conceived, the core idea of domain
decomposition is easily conceptualized: distributed compute resources load
materials into memory for different subsections of the problem domain, and score
tally results for regions only within that domain.



