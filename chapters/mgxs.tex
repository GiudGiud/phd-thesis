\chapter{Multi-Group Cross Sections}
\label{chap:mgxs}

%%%%%%%%%%%%%%%%%%%%%%%%%%%%%%%%%%%%%%%%%%%%%%%%%%%%%%%%%%%%%%%%%%%%%%%%%%%%%%%
\section{Background}
\label{sec:chap2-background}

-discuss each approximation in sequence, building approximations upon one another similar to \ac{MOC} derivation in M.S. thesis??

\begin{itemize}[noitemsep]
  \item continuous energy neutron transport equation
  \item \ac{MGXS} equations
  \begin{itemize}[noitemsep]
    \item generic reaction
    \item scattering matrix
    \item chi fission spectrum
  \end{itemize}
  \item equation and figures mapped to variables
\end{itemize}


%%%%%%%%%%%%%%%%%%%%%%%%%%%
\subsection{Approximations}
\label{subsec:chap2-approx}

\begin{itemize}
  \item motivate bias studies in chapter 4
\end{itemize}

Equation~\ref{eqn:chap1-transport-eqn-src} is defined with $\Psi$, $Q$ and $\Sigma^T$ as continuous functions of energy. The first approximation to numerically solve this equation is to discretize the energy domain into distinct \textit{energy groups} $g \in G = \{1, 2, ..., G\}$ where group $g$ spans the continuous range of energies from $E_{g}$ to $E_{g-1}$. This is otherwise known as the \textit{multi-group approximation}. The multi-group form of the Boltzmann equation is presented below:

\begin{equation}
\label{eqn:chap2-transport-eqn-mg}
\mathbf{\Omega} \cdot \nabla \Psi_g(\mathbf{r},\mathbf{\mathbf{\Omega}}) + \Sigma^T_{g}(\mathbf{r})\Psi_g(s,\mathbf{\Omega}) = Q_g(\mathbf{r},\mathbf{\Omega})
\end{equation}

Likewise, the multi-group form of the neutron source (Equation~\ref{eqn:chap1-source}) is given by:

\begin{dmath}
\label{eqn:chap2-source-mg}
Q_g(\mathbf{r},\mathbf{\Omega}) = \displaystyle\sum\limits_{g'=1}^G \int_{4\pi} \mathrm{d}\mathbf{\Omega'}\Sigma_{g'\rightarrow g}^S(\mathbf{r},{\mathbf{\Omega'}\rightarrow\mathbf{\Omega}}) \Psi_{g'}(\mathbf{r},\mathbf{\Omega'}) + \\ \frac{\chi_{g}(\mathbf{r})}{4\pi k_{eff}} \displaystyle\sum\limits_{g'=1}^G \nu\Sigma_{g'}^F(\mathbf{r}) \int_{4\pi} \mathrm{d}\mathbf{\Omega'} \Psi_{g'}(\mathbf{r},\mathbf{\Omega'})
\end{dmath}

\begin{equation}
\label{eqn:moc-theory-condensed-total-xs}
\Sigma_{g}^T(\mathbf{r}) = \frac{\int_{E_{g}}^{E_{g-1}}\mathrm{d}E'\Sigma^T(\mathbf{r},E')\Psi(\mathbf{r},\mathbf{\Omega},E')}{\int_{E_{g}}^{E_{g-1}}\mathrm{d}E'\Psi(\mathbf{r},\mathbf{\Omega},E')}
\end{equation}

\begin{equation}
\label{eqn:moc-theory-condensed-fission-xs}
\Sigma_{g}^F(\mathbf{r}) = \frac{\int_{E_{g}}^{E_{g-1}}\mathrm{d}E'\Sigma^F(\mathbf{r},E')\Psi(\mathbf{r},\mathbf{\Omega},E')}{\int_{E_{g}}^{E_{g-}1}\mathrm{d}E'\Psi(\mathbf{r},\mathbf{\Omega},E')}
\end{equation}

\begin{equation}
\label{eqn:moc-theory-condensed-scatter-xs}
\Sigma_{g'\rightarrow g}^S(\mathbf{r},\mathbf{\Omega'}\rightarrow \mathbf{\Omega}) = \frac{\int_{E_{g'}}^{E_{g'-1}}\mathrm{d}E'\int_{E_{g}}^{E_{g-1}}\mathrm{d}E''\Sigma^S(\mathbf{r},\mathbf{\Omega'}\rightarrow \mathbf{\Omega},E'\rightarrow E'')\Psi(\mathbf{r},\mathbf{\Omega'},E')}{\int_{E_{g'}}^{E_{g'-1}}\mathrm{d}E'\Psi(\mathbf{r},\mathbf{\Omega'},E')}
\end{equation}

\begin{equation}
\label{eqn:moc-theory-condensed-chi}
\chi_{g'\rightarrow g}(\mathbf{r}) = \frac{\int_{E_{g'}}^{E_{g'-1}}\mathrm{d}E'\int_{E_{g}}^{E_{g-1}}\mathrm{d}E''\chi(\mathbf{r},E'\rightarrow E'')\nu\Sigma^F(\mathbf{r},\mathbf{\Omega},E')\Psi(\mathbf{r},\mathbf{\Omega'},E')}{\int_{E_{g'}}^{E_{g'-1}}\mathrm{d}E'\nu\Sigma^F(\mathbf{r},\mathbf{\Omega},E')\Psi(\mathbf{r},\mathbf{\Omega'},E')}
\end{equation}

Although \autoref{eqn:moc-theory-condensed-chi} assumes a dependence of $\chi$ on both the energy of the neutron causing fission $g'$ and the fission emission energy group $g$, the former is typically summed over to simplify the multi-group $\chi$ to the following approximation:

\begin{equation}
\label{eqn:moc-theory-condensed-chi-group}
\chi_{g}(\mathbf{r}) = \displaystyle\sum\limits_{g=1}^{G}\chi_{g'\rightarrow g}(\mathbf{r})
\end{equation}


%%%%%%%%%%%%%%%%%%%%%%%%%%%%%%%%%%%%%
\subsubsection{Energy Discretization}
\label{subsubsec:chap2-energy-discrete}

%%%%%%%%%%%%%%%%%%%%%%%%%%%%%%%%%%%%%%
\subsubsection{Spatial Discretization}
\label{subsubsec:chap2-space-discrete}

In addition to the flat source approximation, it is assumed that the material properties are constant across each FSR. The area-averaged cross sections for FSR $i \in \{1, 2, ..., I\}$ with area $A_{i}$ are defined as:

\begin{equation}
\label{eqn:chap2-area-avg-total-xs}
\Sigma_{i,g}^{T} = \frac{\int_{\mathbf{r}\in A_{i}}\mathrm{d}\mathbf{r}\Sigma_{g}^T(\mathbf{r})\Phi_{g}(\mathbf{r})}{\int_{\mathbf{r}\in A_{i}}\mathrm{d}\mathbf{r}\Phi_{g}(\mathbf{r})}
\end{equation}

\begin{equation}
\label{eqn:chap2-area-avg-fission-xs}
\Sigma_{i,g}^{F} = \frac{\int_{\mathbf{r}\in A_{i}}\mathrm{d}\mathbf{r}\Sigma_{g}^F(\mathbf{r})\Phi_{g}(\mathbf{r})}{\int_{\mathbf{r}\in A_{i}}\mathrm{d}\mathbf{r}\Phi_{g}(\mathbf{r})}
\end{equation}

\begin{equation}
\label{eqn:chap2-area-avg-scatter-xs}
\Sigma_{i,g'\rightarrow g}^{S} = \frac{\int_{\mathbf{r}\in A_{i}}\mathrm{d}\mathbf{r}\Sigma_{g'\rightarrow g}^S(\mathbf{r})\Phi_{g'}(\mathbf{r})}{\int_{\mathbf{r}\in A_{i}}\mathrm{d}\mathbf{r}\Phi_{g'}(\mathbf{r})}
\end{equation}

\begin{equation}
\label{eqn:chap2-area-avg-chi}
\chi_{i,g} = \frac{\int_{\mathbf{r}\in A_{i}}\mathrm{d}\mathbf{r}\chi_{g}(\mathbf{r})}{\int_{\mathbf{r}\in A_{i}}\mathrm{d}\mathbf{r}}
\end{equation}


%%%%%%%%%%%%%%%%%%%%%%%%%%%%%%%%%%%%
\subsubsection{Isotropic Scattering}
\label{subsubsec:chap2-iso-scatter}

\begin{itemize}
  \item mention ``iso-in-lab'' feature in OpenMC
\end{itemize}

%%%%%%%%%%%%%%%%%%%%%%%%%%%%%%%%%
\subsubsection{Scalar Flux Weighting}
\label{subsubsec:chap2-const-in-angle}

\begin{itemize}[noitemsep]
  \item connect to Nate's ``batman'' plot
\end{itemize}


%%%%%%%%%%%%%%%%%%%%%%%%%%%%%%%%%%%%%%%%%%%%%%%%%%%%%%%%%%%%%%%%%%%%%%%%%%%%%%%
\section{MGXS Generation with Monte Carlo}
\label{sec:chap2-mgxs-mc}

\begin{itemize}[noitemsep]
  \item \ac{MC} is \emph{reactor agnostic}
  \item stochastic approx. to integrals in Section~\ref{sec:chap2-background}
\end{itemize}

%%%%%%%%%%%%%%%%%%%%%%
\subsection{Past Work}
\label{subsec:chap2-past-work}

\begin{itemize}[noitemsep]
  \item \ac{MGXS} for coarse mesh diffusion
  \begin{itemize}[noitemsep]
    \item Redmond
    \item Pounders
    \item Serpent
  \end{itemize}
  \item \ac{MGXS} for fine mesh transport
  \begin{itemize}[noitemsep]
    \item Nelson with NDPP
  \end{itemize}
\end{itemize}

%%%%%%%%%%%%%%%%%%%%%%%%%%%%%%%%%%%%%%%%%%%%%
\subsection{Spatial Self-Shielding Treatment}
\label{subsec:chap2-spatial-shield}

\begin{itemize}[noitemsep]
  \item spatial self-shielding treatment is natural in \ac{MC}
  \item challenge is slow convergence rate
  \item motivate clustering
\end{itemize}