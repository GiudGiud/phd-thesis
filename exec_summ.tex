\documentclass[12pt,twoside]{mitthesis-exec}

%%%%%%%%%%%%%%%%%%%%%%%%%%%%%%%%%%%%%%%%%%%%%%%%%%%%%%%%%%%%%%%%%%%%%%%%%%%%%%%%
% PREAMBLE

\usepackage[bitstream-charter]{mathdesign} % Use BT Charter font
\usepackage[T1]{fontenc}                   % Use T1 encoding instead of OT1
\usepackage[utf8]{inputenc}                % Use UTF8 input encoding
\usepackage{microtype}                     % Improve typography
\usepackage{amsmath}                       % AMS Math extensions
\usepackage{booktabs}                      % Improve table spacing
\usepackage{graphicx}                      % Extended graphics capabilities
\usepackage{tocbibind}                     % Include listings in TOC
\usepackage[printonlyused]{acronym} % withpage: for showing page of use
\usepackage{listings}                      % Source code listings
\usepackage{caption}
\usepackage{subcaption}
\usepackage[rgb,table]{xcolor}
\usepackage{url}
\usepackage{soul}
\usepackage{array}
\usepackage{pdfpages}
\usepackage{mathtools}
\usepackage{setspace}
\usepackage{pbox}
\usepackage{tikz}
\usetikzlibrary{calc,shapes,decorations.pathreplacing,positioning}
\usepackage{pgfplots}
\usepackage{siunitx}
\usepackage{multirow}
\usepackage{breqn}

% Specialties for tables
\usepackage{array}
\newcolumntype{L}[1]{>{\raggedright\let\newline\\\arraybackslash\hspace{0pt}}m{#1}}
\newcolumntype{C}[1]{>{\centering\let\newline\\\arraybackslash\hspace{0pt}}m{#1}}
\newcolumntype{R}[1]{>{\raggedleft\let\newline\\\arraybackslash\hspace{0pt}}m{#1}}

\usepackage[breaklinks=true]{hyperref}
\hypersetup{colorlinks=true, linkcolor=black, citecolor=black, urlcolor=black,
  pdftitle={Reactor Agnostic Multi-Group Cross Section Generation for Fine Mesh Deterministic Neutron Transport Simulations},
  pdfauthor={William Robert Dawson Boyd}
}
\pagestyle{plain}

%\usepackage{floatrow}
%\floatsetup[table]{style=plaintop}
%\floatsetup[widefigure]{margins=hangleft}

% Highlights and emphasis boxes from Bryans thesis
\usepackage[framemethod=tikz]{mdframed}
\definecolor{mitred}{rgb}{0.698,0.0314,0.216}
\definecolor{mitgray}{rgb}{0.690,0.694,0.710}
\definecolor{canyellow}{rgb}{0.933, 0.965, 0.424}
\newmdenv[nobreak=false, skipabove=2ex, skipbelow=2ex, innerlinewidth=3pt, innerlinecolor=black, backgroundcolor=mitgray!75, roundcorner=10pt, frametitlerule=true, frametitlerulewidth=2.5pt, frametitlefont=\color{white}\Large\bfseries, frametitlealignment=\centering, frametitlebackgroundcolor=mitred, frametitleaboveskip=2ex, frametitlebelowskip=2ex, innertopmargin=3ex, innerbottommargin=2ex]{highlightsbox}
\newmdenv[nobreak=false, skipabove=2ex, skipbelow=2ex, innerlinewidth=2pt, innerlinecolor=black, backgroundcolor=white, roundcorner=10pt]{emphbox}

% Appendix
\usepackage[toc,page]{appendix}

% Don't reset footnote counter between chapters
\usepackage{chngcntr}
\counterwithout{footnote}{chapter}

% Algorithm constructs
\usepackage[chapter]{algorithm} % Provides algorithm environment
\usepackage{algorithmicx}       % Provides algorithmic block
\usepackage{algpseudocode}      % Option of algorithmicx package
\renewcommand{\thealgorithm}{\thechapter-\arabic{algorithm}}
\newcommand\Algphase[1]{%
\vspace*{-.7\baselineskip}\Statex\hspace*{\dimexpr-\algorithmicindent-2pt\relax}\rule{\columnwidth}{0.4pt}%
\Statex\hspace*{-\algorithmicindent}{#1}%
\vspace*{-.7\baselineskip}\Statex\hspace*{\dimexpr-\algorithmicindent-2pt\relax}\rule{\columnwidth}{0.4pt}%
}
\newcommand{\algrule}[1][.4pt]{\par\vskip.5\baselineskip\hrule height #1\par\vskip.5\baselineskip}

% Configure captions
\captionsetup{labelfont=bf, labelsep=colon}
\captionsetup[algorithm]{labelfont=bf, labelsep=colon}

% Use Latin Modern for typewriter fonts
\renewcommand{\ttdefault}{lmtt}

% Add \unit macro
\newcommand{\unit}[1]{\ensuremath{\, \mathrm{#1}}}

\definecolor{gray}{rgb}{0.4,0.4,0.4}
\definecolor{darkblue}{rgb}{0.0,0.0,0.6}
\definecolor{cyan}{rgb}{0.0,0.6,0.6}
\lstset{
  basicstyle=\footnotesize\ttfamily,
  columns=fullflexible,
  showstringspaces=false,
  commentstyle=\color{gray}\upshape,
  frame=single,
  xleftmargin=0.55in
}

\lstdefinelanguage{XML}
{
  morestring=[b]",
  morestring=[s]{>}{<},
  morecomment=[s]{<?}{?>},
  morecomment=[s]{<!--}{-->},
  stringstyle=\color{black},
  identifierstyle=\color{darkblue},
  keywordstyle=\color{cyan},
  morekeywords={}
}

\setcounter{secnumdepth}{4}
\setcounter{tocdepth}{3}

\renewcommand{\contentsname}{Table of Contents}
\renewcommand{\bibname}{References}

\makeatletter \renewcommand\thealgorithm{\arabic{algorithm}} \@addtoreset{algorithm}{chapter} \makeatother

\begin{document}

%%%%%%%%%%%%%%%%%%%%%%%%%%%%%%%%%%%%%%%%%%%%%%%%%%%%%%%%%%%%%%%%%%%%%%%%%%%%%%%%
% TITLE PAGE

\title{EXECUTIVE SUMMARY \\~\\ Reactor Agnostic Multi-Group Cross Section Generation for Fine Mesh Deterministic Neutron Transport Simulations}

\author{William Robert Dawson Boyd III}
\prevdegrees{B.S., Georgia Institute of Technology (2010) \\
             M.S., Massachusetts Institute of Technology (2014)}
\department{Department of Nuclear Science and Engineering}
\degree{Doctor of Philosophy in Nuclear Science and Engineering}

\degreemonth{February}
\degreeyear{2017}
\thesisdate{November 4, 2016}

\supervisor{Benoit Forget}{Associate Professor of Nuclear Science and Engineering}
\reader{Kord S. Smith}{KEPCO Professor of the Practice of Nuclear Science and Engineering}
\chairman{Emilio Bagglietto}{Associate Professor of Nuclear Science and Engineering}


%%%%%%%%%%%%%%%%%%%%%%%%%%%%%%%%%%%%%%%%%%%%%%%%%%%%%%%%%%%%%%%%%%%%%%%%%%%%%%%%
%%%%%%%%%%%%%%%%%%%%%%%%%%%%%%%%%%%%%%%%%%%%%%%%%%%%%%%%%%%%%%%%%%%%%%%%%%%%%%%%
% ABSTRACT

\setcounter{savepage}{\thepage}

\begin{abstractpage}

A key challenge for whole-core transport methods is accurate reactor agnostic multi-group cross section (MGXS) generation. Monte Carlo (MC) presents the most accurate method for reactor agnostic multi-group cross section generation since it does not require the use of any local approximations to the flux. This accuracy comes at the computational expense of converging group constant tallies to acceptably low uncertainties. This work develops methods that use MC to generate the fine-spatial mesh MGXS that are needed by high-fidelity whole-core transport codes. The novel techniques developed by this thesis employ engineering-based and statistical clustering algorithms to accelerate the convergence rate and reduce the data storage requirements for MGXS tallied on fine, heterogeneous spatial meshes in Monte Carlo.

-need to discuss angular-dependent MGXS, SPH
  -simple 1D slab and 2D fuel pin cell geometries
-need stronger conclusions
-need sentence summarizing future work
-mention the flux separability approximation, angular-dependent MGXS
-define all acronyms
-simplify discussion of the benchmarks
  -just mention increasing heterogeities, metrics for verification
-mention simulation triad - openmc, openmoc, opencg

%MC methods have increasingly been used to generate few group constants for coarse mesh diffusion, most notably by the Serpent MC code.

%The nuclear reactor physics community has long strived for deterministic neutron transport-based tools for whole-core reactor analysis. A key challenge for whole-core transport methods is accurate reactor agnostic multi-group cross section (MGXS) generation. Monte Carlo (MC) presents the most accurate method for reactor agnostic multi-group cross section generation since it does not require the use of any local approximations to the flux. This accuracy comes at the computational expense of converging group constant tallies to acceptably low uncertainties. MC methods have increasingly been used to generate few group constants for coarse mesh diffusion, most notably by the Serpent MC code. This work develops methods that use MC to generate the fine-spatial mesh MGXS that are needed by high-fidelity whole-core transport codes. The novel techniques developed by this thesis employ engineering-based and statistical clustering algorithms to accelerate the convergence rate and reduce the data storage requirements for MGXS tallied on fine, heterogeneous spatial meshes in Monte Carlo.

%No systematic methodology of approximations has been developed which is broadly applicable to all reactor types.

-need to mention replacement of multi-level scheme
-two themes:
1) quantify and diagnose approximation error
2) replace multi-level scheme with single full-core MC calculations
    -engineering-based and statistical clustering-based methods
      -simultaneously model spatial self-shielding effects while accelerating MC tally convergence to a prescribed accuracy 
    -model pin-wise spatial self-shielding effects in MGXS generation
      -mention clustering of pin-wise MGXS from spatial self-shielding
      -Local Neighbor Symmetry (LNS) algorithm to model spatial self-shielding effects with ``geometric templates''
      -
    -iMGXS data processing pipeline, uses unsupervised machine learning to identify clustering trends
  -must mention that global reactivity is shown to be very nearly preserved for all clustering schemes 
  -pin-wise fission rates are only marginally impacted by clustering of U-235 fission
  -pin-wise U-238 capture rates are most impacted by clustering predictions
    -this is important to predict Pu-239 buildup in burnup calculations, etc.

This thesis is organized along two main themes. First, the efficacy of MGXS generation with MC for fine-mesh transport calculations is rigorously assessed. Some of the approximations made by MC-based MGXS generation are quantified, including the angular, energy- and spatial-dependence of condensed MGXS. The second theme develops a novel methodology called inferential MGXS (\textit{i}MGXS) which directly models all energy and spatial self-shielding effects to generate accurate MGXS with a single whole-core MC calculation. The \textit{i}MGXS scheme is a data processing pipeline that uses unsupervised machine learning algorithms to infer the clustering of MGXS from ``noisy'' MC tally data with limited human supervision.

The \textit{i}MGXS scheme is evaluated for six PWR benchmarks with heterogeneities with varying spatial self-shielding effects, including control rod guide tubes and water reflectors. The \textit{i}MGXS scheme was shown to enable highly accurate predictions of U-238 capture rates, reducing the error by a factor of four with respect to schemes which neglect to model MGXS clustering. In addition, the scheme requires at least an order of magnitude fewer MC particle histories to converge MGXS for accurate multi-group deterministic calculations than a reference MC calculation. These results demonstrate the promise for \textit{i}MGXS as a means to efficiently generate MGXS with reactor agnostic MC calculations of the complete heterogeneous geometry with little human oversight. The \textit{i}MGXS scheme may be valuable for future reactor physics analyses of advanced LWR core designs and next generation reactors with spatial heterogeneities that are poorly modeled by the engineering approximations in today's methods for MGXS generation.

%for which engineering approximations for MGXS generation poorly model spatial self-shielding effects of aribtrary core heterogeneities.

%which do not easily lend themselves to engineering approximations for generating MGXS for neutron transport calculations.

%Nevertheless, it points to the potential for \textit{i}MGXS to harness \ac{MC} to efficiently generate accurate \ac{MGXS} for deterministic transport codes.

%The inference of MGXS clustering with the \textit{i}MGXS scheme enables deterministic reactor physics simulations to produce accurate results from MGXS generated by MC faster than would be possible with a direct calculation with MC.

\end{abstractpage}


\singlespacing 

%%%%%%%%%%%%%%%%%%%%%%%%%%%%%%%%%%%%%%%%%%%%%%%%%%%%%%%%%%%%%%%%%%%%%%%%%%%%%%%%
\section*{Introduction}

%%%%%%%%%%%%%%%%%%%%%%%%%%%%%%%%%%%%%%%
\subsection*{Background and Motivation}

The development and deployment of neutron physics simulations is governed by tradeoffs between accuracy and speed. High-fidelity simulations are accurate and flexible but require significant computational resources, while the approximations made by low-fidelity methods reduce the number of variables which greatly improves the time-to-solution. As a result, it is common to employ a mix of high- and low-fidelity tools for reactor analysis -- for example, high-fidelity tools are frequently used to inform and benchmark low-fidelity models for use within a narrow envelope of design parameters. This thesis develops a new approach within the same vein by employing continuous energy Monte Carlo (MC) neutron transport simulations to generate accurate multi-group cross sections (MGXS) for computationally efficient deterministic transport methods.

%\textbf{This thesis is motivated by the desire to obtain Monte Carlo-quality solutions with computationally efficient deterministic neutron transport methods.}

This thesis investigates the use of Monte Carlo methods to generate MGXS, such as the U-235 fission MGXS illustrated in Fig.~\ref{fig:u235-sigf}, for whole-core deterministic reactor analysis. Monte Carlo presents a natural approach to replace engineering prescriptions to approximate the flux with a stochastic approximation of the exact flux. The advantage of a MC-based approach is that all of the relevant physics modeled in MC may be directly embedded into MGXS. This improvement in accuracy comes at the computational expense of converging group constant tallies to acceptably low uncertainties. MC methods have increasingly been used to generate few group constants for coarse mesh diffusion, most notably by the Serpent MC code~\cite{serpent2013manual}. However, there exist few rigorous and comprehensive analyses of MGXS generation for heterogeneous fine mesh deterministic transport methods~\cite{redmond1997multigroup,cai2014condensation,nelson2014improved}. \textbf{This thesis develops and evaluates MC-based methods to generate MGXS for fine mesh deterministic neutron transport codes.}

\begin{figure}[h!]
\centering
\includegraphics[width=0.75\linewidth]{figures/intro/u235-ce-mg-xs}
\caption[U-235 continuous energy and multi-group fission cross section]{U-235 continuous energy and 16-group fission cross section.}
\label{fig:u235-sigf}
\end{figure}

In addition, MC-based MGXS generation methods to date have retained the multi-level geometric framework illustrated in Fig.~\ref{fig:multi-level-flow-chart} to tabulate MGXS for individual reactor components -- such as infinite fuel pins and/or assemblies -- for subsequent use in whole-core multi-group calculations. The multi-level approach is inspired by legacy MGXS generation techniques which apply high-fidelity models of the energy self-shielding physics to low-fidelity geometric models of unique core components. The complexity of the energy treatment is then reduced at each level as larger and more complex geometric models are considered. \textbf{This thesis abandons the multi-level framework in place of a whole-core MC calculation which simultaneously accounts for all energy and spatial self-shielding effects in a single step.}

\begin{figure}[h!]
\centering
\includegraphics[width=0.9\linewidth]{figures/intro/multi-step-flow-chart}
\caption[Multi-level approach to reactor analysis]{Current multi-level framework for reactor analysis.}
\label{fig:multi-level-flow-chart}
\end{figure}

In theory, whole-core MC calculations can be used to tally MGXS in each spatial zone (\textit{e.g.}, 100 axial depletion zones within each of 50,000+ fuel pins in a PWR core) to account for the spatial variation of the flux. However, such simulations have not been employed for practical reasons -- in particular, the computational expense of performing such calculations has been prohibitive for MC codes until recent years. Furthermore, roughly the same number of particle histories would be required to converge the MGXS tallies in each spatial zone as would be required for a direct whole-core calculation by MC. \textbf{Therefore, a new method is required to accelerate the convergence of the MGXS tallies in each fine mesh region in order for MC to be practical for reactor agnostic fine mesh MGXS generation.}

This thesis proposes to use statistical clustering methods to accelerate the convergence of whole-core MC calculations for MGXS generation. This novel approach relies on the fact that many distinct spatial zones across a reactor core experience similar if not identical spatial self-shielding effects, and therefore have similar if not identical MGXS. The stochastic nature of MC simulations will contribute statistical ``noise'' to the tally estimates for the MGXS. As a result, the MGXS estimates for similarly self-shielded spatial zones form clusters which converge as more particle histories are simulated. \textbf{The goal of this thesis is to develop and apply machine learning algorithms to identify MGXS clusters from ``noisy'' Monte Carlo tally data. This methodology aims to generate MGXS for deterministic neutron transport codes in a reactor agnostic and computationally efficient manner.}

%%%%%%%%%%%%%%%%%%%%%%%%
\subsection*{Objectives}

The subject matter of this thesis is organized along two main themes:

\begin{itemize}
\item \textbf{\textit{Approximation Error}} -- Quantify and diagnose approximation error in MGXS generated from MC methods for simple heterogeneous benchmark problems.
\item \textbf{\textit{Statistical Clustering}} -- Develop statistical clustering methods to accelerate the convergence of MGXS on heterogeneous MC tally meshes.
\end{itemize}

The first theme of this thesis rigorously assesses the efficacy of MGXS generation with MC for fine mesh transport calculations. Some of the approximations made by MC-based MGXS generation are quantified, including an in-depth analysis of systematic bias resulting from constant-in-angle total MGXS. The second theme develops a new methodology called \textit{inferential multi-group cross sections} (\textit{i}MGXS) to simultaneously capture local and global spatial self-shielding effects in MGXS for whole-core calculations. The \textit{i}MGXS scheme applies statistical clustering methods to accelerate the convergence of MGXS tallied on fine, heterogeneous spatial meshes in Monte Carlo. The \textit{i}MGXS scheme is evaluated for a six heterogeneous PWR benchmarks to quantify its accuracy and convergence relative to alternative schemes for MC-based MGXS generation.

%%%%%%%%%%%%%%%%%%%%%%%%%%%%%%%%%%%%%%%%%%%%%%%%%%%%%%%%%%%%%%%%%%%%%%%%%%%%%%%%
\section*{Software Infrastructure: A Simulation Triad}

This thesis investigates Monte Carlo as a means to generate multi-group cross sections for fine mesh transport codes. This work required the development of a ``simulation triad'' encompassing three primary codes as illustrated in Fig.~\ref{fig:simulation-triad}. First, the OpenMC Monte Carlo code~\cite{romano2013openmc} was utilized to generate multi-group cross sections. Second, the MGXS were used by the OpenMOC code~\cite{boyd2014openmoc} for deterministic multi-group transport calculations. Finally, the OpenCG library~\cite{boyd2015opencg} enabled the processing and transfer of tally data on combinatorial geometry (CG) meshes between OpenMC and OpenMOC. The following sections summarize the author's contributions to each component code in the simulation triad to support the objectives of this thesis.

\begin{figure}[h!]
  \centering
  \includegraphics[width=0.95\linewidth]{figures/workflow/triad/simulation-triad}
\caption[A simulation triad of OpenMC, OpenMOC and OpenCG]{A simulation triad consisting of the OpenMC, OpenMOC and OpenCG codes ``glued'' together with Python formed the foundation for this thesis research.}
\label{fig:simulation-triad}
\end{figure}

%%%%%%%%%%%%%%%%%%%%
\subsection*{OpenMC}

The OpenMC code is a continuous energy Monte Carlo neutron transport code~\cite{romano2013openmc} with support for general constructive solid geometry models. This thesis designed and implemented a fully-featured OpenMC Python API to enable the processing of large tally datasets to generate MGXS. In addition, a distributed cell tally algorithm~\cite{lax2014distribcell} was implemented in OpenMC to compute MGXS across repeated fuel pin cells in reactor core geometries. Finally, an option for isotropic in lab scattering was implemented and used to generate MGXS. The isotropic scattering feature enabled ``apples-to-apples'' comparisons between the reference eigenvalues and reaction rates produced by OpenMC and those computed from isotropic multi-group calculations with OpenMOC. In total, the author contributed over 20,000 lines of Python and 2,000 lines of Fortran code to the open source release version of OpenMC to support this work.

%%%%%%%%%%%%%%%%%%%%%
\subsection*{OpenMOC}

The OpenMOC code is a multi-group neutron transport code implementing the deterministic Method of Characteristics (MOC)~\cite{boyd2014openmoc}. The OpenMOC code is capable of performing 2D MOC neutron transport calculations for LWR core configurations. The \texttt{openmoc.materialize} Python module was implemented to automate the loading MGXS data into OpenMOC \texttt{Material} objects. The module is designed to support the large MGXS libraries generated by OpenMC. In addition, a new module was implemented  to interface with the OpenCG region differentiation algorithm, which constructed ``clustered geometries'' for OpenMOC. Finally, a scheme to compute SuPerHomog\'{e}n\'{e}isation (SPH) factors to ensure reaction rate consistency with OpenMC was implemented in OpenMOC. In total, the author contributed over 15,000 lines of C++ and nearly 20,000 lines of Python code to the open source release version of OpenMOC to support this work.

%%%%%%%%%%%%%%%%%%%%
\subsection*{OpenCG}

A new combinatorial goemetry (CG) Python library called OpenCG~\cite{boyd2015opencg} was developed to construct complicated reactor geometries for OpenMOC. Two novel algorithms known as Local Neighbor Symmetry (LNS) and region differentiation were developed in OpenCG to support the \textit{i}MGXS methodology. The LNS algorithm analyzes a combinatorial geometry to identify neighbor cells, or pairs of cells which are adjacent to one another. The LNS algorithm is analogous to the geometric templates used in lattice physics codes such as CASMO~\cite{edenius1995casmo} to identify fuel pins which have similar MGXS. The region differentiation algorithm was developed to efficiently construct geometries which represent the assignment of MGXS to arbitrary ``clusters'' of fuel pins. The differentiated geometries are exported to OpenMOC for deterministic neutron transport calculations. The author contributed over 10,000 lines of Python code in OpenCG to support this work.


%%%%%%%%%%%%%%%%%%%%%%%%%%%%%%%%%%%%%%%%%%%%%%%%%%%%%%%%%%%%%%%%%%%%%%%%%%%%%%%%%
%\section*{Angular-Dependent MGXS and SPH Factors}
%
%%%%%%%%%%%%%%%%%%%%%%%%%%%%%%%%%%%%%%%%%%%%%%
%\subsection*{Flux Separability Approximation}
%
%\begin{dmath}
%\label{eqn:sigt-flux-separable}
%\Sigma_{t,g}(\mathbf{r}) = \frac{\int\displaylimits_{E_{g}}^{E_{g-1}} \Sigma_{t}(\mathbf{r},E)\psi(\mathbf{r},\mathbf{\Omega},\mathrm{d}E)}{\psi_{g}(\mathbf{r},\mathbf{\Omega})} \approx \frac{\int\displaylimits_{E_{g}}^{E_{g-1}} \Sigma_{t}(\mathbf{r},E)\phi(\mathbf{r},E)}{\phi_{g}(\mathbf{r})}
%\end{dmath}
%
%\clearpage
%
%%%%%%%%%%%%%%%%%%%%%%%%%%%%%%%%%%%%%%%%%%%
%\subsection*{Energy-Dependent Flux Errors}
%
%\begin{figure}[h!]
%\centering
%\includegraphics[width=\linewidth]{figures/sph/pin-cell/rel-err-inner-outer}
%\caption[Flux relative error by energy group with SPH]{The energy-dependent relative error of the 70-group OpenMOC scalar flux with respect to the OpenMC flux in a fuel pin for the innermost, outermost and all FSRs.}
%\label{fig:rel-err-energy}
%\end{figure}
%
%\clearpage

%%%%%%%%%%%%%%%%%%%%%%%%%%%%%%%%%%%%%
%\subsection*{Angular-Dependent MGXS}
%
%\begin{figure}[h]
%\begin{subfigure}{.5\textwidth}
%  \centering
%  \includegraphics[width=\linewidth]{figures/sph/batman-1}
%  \caption{}
%  \label{fig:batman-plots-a}
%\end{subfigure}
%\begin{subfigure}{.5\textwidth}
%  \centering
%  \includegraphics[width=\linewidth]{figures/sph/batman-2}
%  \caption{}
%  \label{fig:batman-plots-b}
%\end{subfigure}
%\caption[Angular-dependent capture MGXS]{Angular-dependent capture MGXS for the 6.67 eV resonance group as a function of azimuthal angle for two different FSRs. The radial axis is given in units of barns and the azimuthal axis in units of degrees. \textit{Image courtesy of N. Gibson~\cite{gibson2016thesis}.}}
%\label{fig:batman-plots}
%\end{figure}
%
%\clearpage
%
%%%%%%%%%%%%%%%%%%%%%%%%%%%%%%%%%%%%%%%%%%%%%%%%%%%
%\subsection*{SuPerHomog\'{e}n\'{e}isation Factors}
%
%-SPH factors were first proposed by H\'{e}bert~\cite{hebert1993consistent} to preserve reaction rates during energy condensation and spatial homogenization \\
%-SPH factor algorithm requires knowledge of a reference source that is used in a multi-group fixed source solver to derive multiplicative factors that adjust the total MGXS to force neutron balance \\
%
%\begin{dmath}
%\label{eqn:sph-transport-eqn-iterate}
%\mathbf{\Omega} \cdot \nabla \psi_{g}^{(n)}(\mathbf{r},\mathbf{\Omega}) + \mu_{k,g}^{(n-1)}\Sigma_{t,k,g}\psi_{g}^{(n)}(\mathbf{r},\mathbf{\Omega}) = Q_{k,g}(\mathbf{\Omega})
%\end{dmath}
%
%\begin{dmath}
%\label{eqn:sph-update-sigt}
%\Sigma_{t,k,g}^{(n)} = \mu_{k,g}^{(n-1)}\Sigma_{t,k,g}^{(0)}
%\end{dmath}
%
%\begin{algorithm}[h]
%\caption{SPH Factor Algorithm}
%\label{alg:sph}
%\begin{algorithmic}[1]
%%  \State Initialize MGXS from MC tallies
%%  \State Compute neutron source from MC flux and MGXS
%  \State Initialize SPH factors to unity
%  \While{SPH factors are not converged}
%    \State Update MGXS with SPH factors \Comment{Eqn.~\ref{eqn:sph-update-sigt}}
%    \State Solve fixed source transport problem \Comment{Eqn.~\ref{eqn:sph-transport-eqn-iterate}}
%    \State Compute new SPH factors
%  \EndWhile
%  \State Compute final MGXS with SPH factors \Comment{Eqn.~\ref{eqn:sph-update-sigt}}
%\end{algorithmic}
%\end{algorithm}
%
%\begin{table}[h!]
%  \centering
%  \caption[Eigenvalues with and without SPH factors for a fuel pin]{The impact of SPH factors on the eigenvalue bias $\Delta\rho$ with varying energy group structures for a fuel pin.}  
%  \label{table:sph-keff}
%  \vspace{6pt}
%  \begin{tabular}{c S[table-format=6.1] S[table-format=6.1]}
%  \toprule
%  & \multicolumn{2}{c}{{\bf $\boldsymbol{\Delta\rho}$ [pcm]}} \\
%  \cline{2-3}
%  \multirow{-2}{*}{{\bf \# Groups}} &
%  \multicolumn{1}{c}{{\bf Without SPH}} &
%  \multicolumn{1}{c}{{\bf With SPH}} \\
%  \midrule
%1 & 66 & -14 \\
%2 & 34 & -6\\
%4 & -57 & 1 \\
%8 & -102 & 2 \\
%16 & -111 & 4 \\
%25 & -182 & -1 \\
%40 & -202 & 2 \\
%70 & -211 & -3 \\
%  \bottomrule
%\end{tabular}
%\end{table}
%
%\clearpage

%%%%%%%%%%%%%%%%%%%%%%%%%%%%%%%%%%%%%%%%%%%%%%%%%%%%%%%%%%%%%%%%%%%%%%%%%%%%%%%%
\section*{Tallying Pin-Wise MGXS with Monte Carlo}

This thesis develops a new methodology for MGXS generation which employs a single Monte Carlo calculation is used to model the complete heterogeneous geometry in a single step. The advantage of this approach is that it uses the ``true'' MC flux to collapse the flux. However, it 

In theory, multi-group cross sections are to vary continuously in space. Howver, most deterministic methods used to solve the multi-group transport equation make the simplifying assumption that material properties are constant across each spatial mesh cell. \textit{Spatial homogenization} is used to compute flux-weighted volume-averaged cross sections within each spatial mesh cell. This thesis develops the inferential MGXS spatial homogenization scheme in which a single Monte Carlo calculation is used to model the complete heterogeneous geometry to generate MGXS. In particular, \textit{i}MGXS captures spatial self-shielding effects from core heterogeneities which induce clustering of the MGXS in different fuel pin types. The motivation for MGXS clustering and a brief description of the scheme is presented in the following sections.

%-each scheme models complete heterogeneous geometry in a single step
%-uses the true ``MC'' flux to collapse the MGXS for fissile and non-fissile zones
%-the schemes vary by how they treat the MGXS for the different fuel pins across a reactor geometry
%
%This thesis investigates the effect of spatial homogenization on the various fuel pins in LWR benchmarks.
%
%This thesis is based on the observation that \textbf{pins with similar neighboring heterogeneities have similar microscopic MGXS}. If pins with similar microscopic MGXS can be identified, the MGXS tallied in these pin instances may be \textit{spatially homogenized} to
%
% compute an estimate which is nearly as accurate as the \ac{MGXS} from degenerate homogenization, and nearly as converged as the \ac{MGXS} from null homogenization. 
%
%%-need to define spatial self-shielding
%
%Spatially-homogenized \ac{MGXS} preserve reaction rates in discrete spatial zones.
%
%This chapter investigates this hypothesis by analyzing the pin-wise \ac{MGXS} tallied with OpenMC to identify patterns -- namely, clustering -- for pins with similar neighbors. 

%%%%%%%%%%%%%%%%%%%%%%%%%%%%%%%%%%%%%%%%%%%%%%%%%%%%%%%%%%%%%%%%%%%%%
\subsection*{Clustering of Pin-Wise MGXS}

Core heterogeneities such as control rod guide tubes, burnable poisons and reflectors induce clustering of pin-wise MGXS due to spatial self-shielding effects. An illustration of the clustering of U-238 radiative capture MGXS is given in Fig.~\ref{fig:capt-mean-std} for a single PWR fuel assembly with control rod guide tubes. The scatter plots illustrate the clustering of MGXS tallied for each of the 264 fuel pins in the assembly. The MGXS separate into four clusters due to the softening of the flux due to different levels of moderation from neighboring CRGTs in increasing order for the following four groupings of fuel pins: (a) pins not adjacent to a CRGT, (b) pins corner adjacent to a CRGT, (c) pins facially adjacent to a CRGT and (d) pins facially and corner adjacent to separate CRGTs. The spatial homogenization schemes introduced here employ machine learning to infer the clustering of pin-wise MGXS from ``noisy'' MC tally data.

\begin{figure}[h!]
\centering
\begin{subfigure}{0.45\textwidth}
  \centering
  \includegraphics[width=0.9\linewidth]{figures/unsupervised/features/assm-16/u238-capt/mean-std/geometry-2}
  \caption{}
  \label{fig:chap10-capt-mean-std-geom-2}
\end{subfigure}%
\begin{subfigure}{0.45\textwidth}
  \centering
  \includegraphics[width=0.9\linewidth]{figures/unsupervised/features/assm-16/u238-capt/mean-std/mgxs-2}
  \caption{}
  \label{fig:chap10-capt-mean-std-mgxs-2}
\end{subfigure}
\begin{subfigure}{0.45\textwidth}
  \centering
  \includegraphics[width=0.9\linewidth]{figures/unsupervised/features/assm-16/u238-capt/mean-std/geometry-3}
  \caption{}
  \label{fig:chap10-capt-mean-std-geom-3}
\end{subfigure}%
\begin{subfigure}{0.45\textwidth}
  \centering
  \includegraphics[width=0.9\linewidth]{figures/unsupervised/features/assm-16/u238-capt/mean-std/mgxs-3}
  \caption{}
  \label{fig:chap10-capt-mean-std-mgxs-3}
\end{subfigure}
\caption[Clustering of U-238 capture MGXS]{Scatter plots of the pin-wise U-238 capture MGXS means ($x$) and standard deviations ($y$) in barns for a 1.6\% enriched fuel assembly.}
\label{fig:capt-mean-std}
\end{figure}

%%%%%%%%%%%%%%%%%%%%%%%%%%%%%%%%%%%%%%%%%%%%%%%%%%%%%%%%%%%%%
\subsection*{Null, Degenerate and LNS Spatial Homogenization}

This work developed several different spatial homogenization schemes to model spatial self-shielding effects in MGXS. Although all spatial zones may experience spatial self-shielding, this thesis only models the impact of spatial self-shielding on MGXS in fissile regions. Each scheme uses a single MC calculation of the complete heterogeneous geometry to collapse MGXS with the ``true'' flux. The null, LNS and degenerate schemes model spatial self-shielding for each fuel pin with increasing granularity and complexity.

\textbf{\textit{Null spatial homogenization} is the simplest method and averages all spatial self-shielding effects across the entire geometry.} Null homogenization does not account for spatial self-shielding effects experienced by different fuel pins and is equivalent to averaging the data points in Fig.~\ref{fig:capt-mean-std} to compute a single MGXS to be used in all fuel pins. \textbf{\textit{Degenerate spatial homogenization} takes the opposite approach and assigns each fuel pin its own MGXS.} The degenerate scheme accounts for the different spatial self-shielding effects experienced by each instance of each fuel pin throughout a heterogeneous geometry, and is equivalent to modeling each of the data points in Fig.~\ref{fig:capt-mean-std} as its own unique cluster. Although degenerate homogenization is more accurate than null homogenization, it requires many more MC particle histories to converge the uncertainties on the MGXS tallied separately for each fuel pin.

\textbf{\textit{LNS spatial homogenization} uses OpenCG's LNS algorithm to predict the clustering of pin-wise MGXS based on an analysis of the combinatorial geometry.} The LNS scheme is akin to \textit{geometric templates} employed by some lattice physics codes, such as CASMO~\cite{rhodes2006casmo}, to predict which groupings of pins are likely to experience similar spatial self-shielding effects and hence have similar MGXS. The LNS algorithm groups fuel pins together based on an analysis of each pin's neighbors. The MGXS are homogenized across all pin within the same LNS grouping. LNS homogenization is an attempt to model MGXS clustering with fewer materials than degenerate homgenization in order accelerate the MC tally convergence rate by homogenizing MGXS across many pins. However, the LNS algorithm scales poorly for large geometries and fails to predict spatial self-shielding effects in arbitrary core geometries, such as those that occur at assembly-assembly and assembly-reflector interfaces. These shortcomings motivate the need for an unsupervised approach to accurately and scalably predict MGXS clustering.

%Pins with like neighboring pins, within assemblies with like neighboring assemblies, will receive the same LNS identifier.

%LNS homogenization computes a single set of MGXS for the fuel pin in each set classified by the LNS algorithm using a MC track density-weighted averages of the reaction rates and flux tallies in each pin.

%\begin{equation}
%\label{eqn:imgxs-micro}
%\hat{\sigma}_{x,i,m,g} = \frac{\displaystyle\sum\limits_{k=1}^{K}\mathbb{1}_{\mathbb{S}_{m}}(k) \langle \sigma_{x,i}, \psi \rangle_{k,g}^{t\ell}}{\displaystyle\sum\limits_{k=1}^{K}\mathbb{1}_{\mathbb{S}_{m}}(k) \langle \psi \rangle_{k,g}^{t\ell}}
%\end{equation}

%%%%%%%%%%%%%%%%%%%%%%%%%%%%%%%%%%%%%%%%%%%%%%%%%%%
\subsection*{\textit{i}MGXS Spatial Homogenization}

\textbf{\textit{Inferential MGXS spatial homogenization} uses algorithms developed by the machine learning community to infer MGXS clusters directly from MC tally data.} The \textit{i}MGXS scheme can flexibly accommodate arbitrary core heterogeneities better than heuristic approaches like LNS which must be customized for particular core geometries. In addition, the \textit{i}MGXS scheme accelerates the convergence rate of MGXS tallied with MC with respect to the degenerate and LNS schemes since it homogenizes MC tallies across many more fuel pins.

% rather than predict clustering from an analysis of the core geometry.

The \textit{i}MGXS spatial homogenization scheme is a multi-stage data processing pipeline as illustrated in Fig.~\ref{fig:imgxs-flow-chart}. The \textit{feature extraction} stage of the data processing pipeline in builds features (\textit{i.e.}, random variables) derived from ``noisy'' MC tally data which provide information about MGXS clustering for each fuel pin. The \textit{feature selection} stage removes redundant and/or highly correlated features to minimize the number of features used to train a clustering model. The \textit{dimensionality reduction} stage projects samples in a lower-dimensional vector space with the same descriptive power as the original features to reduce training time and and improve clustering model performance. The \textit{predictor training} stage builds a series of clustering models for different parameters (\textit{e.g.}, varying numbers of clusters), each of which predicts clusters of fuel pins that are grouped together with the same MGXS. The $k$-means~\cite{macqueen1967kmeans, lloyd1982kmeans}, agglomerative~\cite{johnson1967hierarchical}, BIRCH~\cite{zhang1996birch} and Gaussian Mixture Model~\cite{mclachlan1988mixture} clustering algorithms were each employed within the \textit{i}MGXS scheme for this thesis. The \textit{model selection} stage evaluates heuristics to choose clustering model parameters, including the number of clusters. Various methods which score clustering models based upon their intra- and inter-cluster similarities, or penalize model complexity using regularization, were evaluated within the \textit{i}MGXS scheme.

\begin{figure}[h!]
\centering
\includegraphics[width=0.88\linewidth]{figures/unsupervised/features/engineering/flow-chart}
\vspace{2mm}
\caption[\textit{i}MGXS flow chart]{The \textit{i}MGXS data processing pipeline.}
\label{fig:imgxs-flow-chart}
\end{figure}


%%%%%%%%%%%%%%%%%%%%%%%%%%%%%%%%%%%%%%%%%%%%%%%%%%%%%%%%%%%%%%%%%%%%%%%%%%%%%%%%
\section*{Model Validation}

%%%%%%%%%%%%%%%%%%%%%%%%%%%%%%%%%%%%%%%%%%
\subsection*{Heterogeneous PWR Benchmarks}

A series of six heterogeneous benchmarks were derived from the Benchmark for Evaluation And Validation of Reactor Simulations (BEAVRS) Pressurized Water Reactor (PWR) model~\cite{horelik2013beavrs} to evaluate the \textit{i}MGXS scheme. BEAVRS is a highly-detailed PWR specification which was created to validate high-fidelity core analysis methods. Each of the benchmarks was fabricated in 2D due to the geometric constraints in OpenMOC. One of the benchmarks was constructed as a 2$\times$2 colorset of 1.6\% and 3.1\% enriched fuel assemblies surrounded by a water reflector as depicted in Fig.~\ref{fig:reflector}. \textbf{The colorset is designed to quantify the impact of the moderation provided by the reflector, as well as the leakage of neutrons through the reflector, on spatially self-shielded MGXS.} The most complex benchmark is a 2D planar slice of the top right quadrant of the BEAVRS model as depicted in Fig.~\ref{fig:full-core}. The quarter core model is the all rods out configuration at the core axial mid-plane with a critical boron concentration, and includes assemblies with 1.6\%, 2.4\% and 3.1\% enriched fuel depicted in Fig.~\ref{fig:full-core}, respectively. \textbf{The BEAVRS model is designed to quantify the effects of radial heterogneities including the inter-assembly water gaps, stainless steel baffle surrounding the fuel assemblies, core barrel, neutron shield panels on spatially self-shielded MGXS.} 

%\textbf{The benchmarks were designed with increasingly complex heterogeneous features -- and corresponding spatial self-shielding effects -- in order to understand their implications for accurate pin-wise MGXS generation.} In particular, the impact of fuel enrichment, control rod guide tubes (CRGTs), burnable poisons (BPs), inter-assembly currents, water reflectors, steel baffles and the core barrel and vessel is considered.

%Five of the six benchmark models are based upon sub-components of the full core BEAVRS model, while the sixth benchmark is of the full BEAVRS core model. 

%The space outside of the pressure vessel is filled with air with vacuum boundaries applied to the top and right with reflective booundaries on the bottom and left.
 
%The reflected colorset includes reflective boundaries on the top and left with vacuum boundaries on the bottom and right.

%The colorset benchmark includes the effects of inter-assembly spatial heterogeneities on the spatially self-shielded MGXS of fuel pins of different enrichments placed adjacent to one another. 

\begin{figure}[h!]
\centering
\begin{subfigure}{0.47\textwidth}
  \centering
  \includegraphics[width=0.93\linewidth]{figures/benchmarks/reflector}
  \caption{}
  \label{fig:reflector}
\end{subfigure}%
\begin{subfigure}{0.47\textwidth}
  \centering
  \includegraphics[width=0.93\linewidth]{figures/benchmarks/quarter-core}
  \caption{}
  \label{fig:full-core}
\end{subfigure}
\caption[PWR benchmarks]{The 2$\times$2 colorset with water reflector (a) and the quarter core BEAVRS (b) models used to evaluate the \textit{i}MGXS spatial homogenization scheme.}
\label{fig:benchmarks}
\end{figure}

%%%%%%%%%%%%%%%%%%%%%%%%%%%%%%%%%%%%%%%%%%
\subsection*{Validation Metrics}

Each of the benchmarks is analyzed with OpenMC to generate 70-group MGXS for use in deterministic transport calculations with OpenMOC. In addition, \textbf{a series of OpenMC simulations were used to calculate reference eigenvalues and pin-wise fission rates and U-238 capture rates for each benchmark.} The eigenvalue is a key integral quantity used to assess the reactivity of a reactor. The fission rates are directly related to the relative power density of each fuel pin which is important for fuel depletion as well as thermal hydraulic feedback. The U-238 capture rates result in the production of Pu-239 which contributes up to 40\% of the power produced from fission in PWRs at the end-of-life (EOL). Hence, the spatial distributions of the fission rates and U-238 capture rates must be correctly modeled for accurate high-fidelity depletion calculations. 

\textbf{The eigenvalue, fission rate and capture rate errors between OpenMC and OpenMOC were quantified for each benchmark and spatial homogenization scheme.} The type of spatial homogenization strongly impacts the resultant accuracy of OpenMOC's U-238 capture rate predictions. In contrast, the homogenization schemes has little impact on the OpenMOC fission rate predictions. As expected, the various schemes are inconsequential to the eigenvalue predictions, since each method uses the same MC flux to collapse the MGXS and preserve global reactivity. Hence, only the U-238 capture rate errors are presented and analyzed in the following section.

% This underscores the importance of accounting for spatial heterogeneities -- such as the moderation from CRGTs and reflectors -- when generating MGXS to predict U-238 capture and Pu-239 production in LWRs.

%%%%%%%%%%%%%%%%%%%%%%%%%%%
\section*{Results}

%%%%%%%%%%%%%%%%%%%%%%%%%%%%%%%%%%%%%%%%%%%%%%%%%
\subsection*{\textit{i}MGXS Clustered Geometries}

\begin{figure}[h!]
\centering
\begin{subfigure}{0.47\textwidth}
  \centering
  \includegraphics[width=0.95\linewidth]{figures/benchmarks/2x2}
  \caption{}
  \label{fig:reflector}
\end{subfigure}%
\begin{subfigure}{0.47\textwidth}
  \centering
  \includegraphics[width=0.95\linewidth]{figures/quantification/homogenization/2x2-degenerate-materials}
  \caption{}
  \label{fig:reflector-degenerate}
\end{subfigure}
\begin{subfigure}{0.47\textwidth}
  \centering
  \includegraphics[width=0.95\linewidth]{figures/patterns/lns/reflector/materials}
  \caption{}
  \label{fig:reflector-lns}
\end{subfigure}%
\begin{subfigure}{0.47\textwidth}
  \centering
  \includegraphics[width=0.95\linewidth]{figures/unsupervised/geometries/with-features/8-clusters/combined/reflector}
  \caption{}
  \label{fig:reflector-8-clusters}
\end{subfigure}
\caption[Materials for the 2$\times$2 colorset]{The materials for the 2$\times$2 colorset with null (a), degenerate (b), LNS (c) and \textit{i}MGXS spatial homogenization with 8 GMM clusters (d).}
\label{fig:colorset-geometries}
\end{figure}

\clearpage

\begin{figure}[h!]
\centering
\includegraphics[width=\linewidth]{figures/unsupervised/geometries/with-features/8-clusters/combined/full-core-zoom}
\caption[Materials for the BEAVRS]{The quarter core BEAVRS model with \textit{i}MGXS spatial homogenization.}
\label{fig:full-core-8-clusters}
\end{figure}

%%%%%%%%%%%%%%%%%%%%%%%%%
\subsection*{Eigenvalues}

\begin{table}[ht!]
  \centering
  \caption[OpenMOC eigenvalue bias]{OpenMOC eigenvalue bias $\Delta\rho$ for \textit{i}MGXS spatial homogenization.}
  \small
  \label{table:eigenvalues}
  \vspace{6pt}
  \begin{tabular}{p{4cm} R{0.9cm} R{0.9cm} R{0.9cm} R{0.9cm} R{0.9cm} R{0.9cm}}
  \toprule
  & \multicolumn{6}{S[table-format=6.1]}{\textbf{\# Clusters}} \\
  \cline{2-7}
  \multirow{-2}{*}{\bf Benchmark} &
  \multicolumn{1}{c}{\textbf{1}} & 
  \multicolumn{1}{c}{\textbf{2}} & 
  \multicolumn{1}{c}{\textbf{4}} & 
  \multicolumn{1}{c}{\textbf{8}} & 
  \multicolumn{1}{c}{\textbf{16}} & 
  \multicolumn{1}{c}{\textbf{\# pins}} \\
  \midrule
Colorset w/ Reflector & -141 & -136 & -136 & -134 & -129 & -141 \\
  \midrule
BEAVRS Quarter Core & -122 & -119 & -120 & -119 & -119 & -116 \\
  \bottomrule
\end{tabular}
\end{table}

%%%%%%%%%%%%%%%%%%%%%%%%%%%%%%%%%%%%%%%%%%%%%%%%
\subsection*{Pin-Wise U-238 Capture Rates}

\begin{figure}[h!]
\centering
\includegraphics[width=0.83\linewidth]{figures/results/spatial/reflector/capt-err}
\vspace{2mm}
\caption[U-238 capture rate errors for the 2$\times$2 colorset]{U-238 capture rate percentage relative errors for the 2$\times$2 colorset with null, degenerate, LNS and \textit{i}MGXS spatial homogenization with 2, 8 and 16 clusters.}
\label{fig:refl-capt-err}
\end{figure}

\clearpage

\begin{figure}[h!]
\centering
\begin{subfigure}{0.9\textwidth}
  \centering
  \includegraphics[width=0.65\linewidth]{figures/results/capt-to-fiss/spatial/full-core/capt-to-fiss-err-null}
  \caption{}
  \label{fig:chap11-full-core-capt-err-null}
\end{subfigure}
\begin{subfigure}{0.9\textwidth}
  \centering
\includegraphics[width=0.65\linewidth]{figures/results/capt-to-fiss/spatial/full-core/capt-to-fiss-err-birch-40}
  \caption{}
  \label{fig:chap11-full-core-capt-err-birch-40}
\end{subfigure}
\caption[U-238 capture rate errors for BEAVRS]{U-238 capture percent relative errors for the quarter core BEAVRS model with null homogenization (a) and \textit{i}MGXS homogenization with 20 clusters (b).}
\label{fig:chap11-full-core-capt-err-b}
\end{figure}

\clearpage

\begin{figure}[h!]
\centering
\includegraphics[width=0.85\linewidth]{figures/results/compare/reflector/compare-capt}
\caption[U-238 capture rate comparison for the colorset]{A comparison of U-238 capture rate spatial distributions for \textit{i}MGXS with GMM clustering and null spatial homogenization schemes for the 2$\times$2 colorset.}
\label{fig:refl-capt-rates-comp}
\end{figure}

\clearpage

%%%%%%%%%%%%%%%%%%%%%%%%%%%%%%%%%%%%%%%%%%%%%%%%
\subsection*{Convergence Rates of MOC Solutions}

\begin{figure}[h!]
\centering
\begin{subfigure}{\textwidth}
  \centering
  \includegraphics[width=0.9\linewidth]{figures/results/convergence/reflector/max-capt-err-evo-exec-summary}
  \caption{}
  \label{fig:refl-max-converge}
\end{subfigure}
\begin{subfigure}{\textwidth}
  \centering
  \includegraphics[width=0.9\linewidth]{figures/results/convergence/reflector/mean-capt-err-evo-exec-summary}
  \caption{}
  \label{fig:refl-mean-converge}
\end{subfigure}
\vspace{2mm}
\caption[Fission rate covergence for the 2$\times$2 colorset]{Convergence of the max (a) and mean (b) absolute U-238 capture rate percent relative errors for the 2$\times$2 colorset.}
\label{fig:refl-capture-converge}
\end{figure}

\clearpage

%%%%%%%%%%%%%%%%%%%%%%%%%%%%%%%%%%%%%%%%%%%%%%%%%%%%%%%%%%%%%%%%%%%%%%%%%%%%%%%%
\section*{Conclusions}

%%%%%%%%%%%%%%%%%%%%%%%%%%%%%%%%%%%%%%%%
\subsection*{Contributions to the Field}

-tightly coupled framework for MC and MOC for MGXS \\
-useful for both MGXS generation as well as validation of MOC \\
-identified fundamental issue with flux separability approx. and 

%%%%%%%%%%%%%%%%%%%%%%%%%
\subsection*{Future Work}

-speed up OpenMOC full core: \\
  -find a way use quarter assembly CMFD mesh \\
  -linear source to reduce number of spatial zones \\
  -vectorize transport solver over energy groups \\

evaluate \textit{i}MGXS scheme:
-add anisotropic scattering to OpenMOC to enable solution of the ``correct'' problem
-systematic study of featues -- ones actually matter? \\
-systematic study to understand impact of dimensionality reduction \\
-systematic study to understand impact of clustering algorithms \\
-more research into model selection schemes -- none of them robustly works for my case studies! \\
-evaluate \textit{i}MGXS with clustering ``on-the-fly'' with noisy MC tally data \\
-can \textit{i}MGXS reduce the number of necessary energy groups?? \\

reach goals:\\
-multi-physics applications: moderator density, fuel temperature, burnup, etc. as features \\
-machine learning to optimize energy group structures \\


%%%%%%%%%%%%%%%%%%%%%%%%%%%%%%%%%%%%%%%%%%%%%%%%%%%%%%%%%%%%%%%%%%%%%%%%%%%%%%%%
%%%%%%%%%%%%%%%%%%%%%%%%%%%%%%%%%%%%%%%%%%%%%%%%%%%%%%%%%%%%%%%%%%%%%%%%%%%%%%%%
% BIBLIOGRAPHY

\begin{singlespace}
\bibliographystyle{ans}
\bibliography{references}
\end{singlespace}

\end{document}
