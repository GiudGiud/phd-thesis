% -*-latex-*-
% 
% For questions, comments, concerns or complaints:
% thesis@mit.edu
% 
%
% $Log: cover.tex,v $
% Revision 1.8  2008/05/13 15:02:15  jdreed
% Degree month is June, not May.  Added note about prevdegrees.
% Arthur Smith's title updated
%
% Revision 1.7  2001/02/08 18:53:16  boojum
% changed some \newpages to \cleardoublepages
%
% Revision 1.6  1999/10/21 14:49:31  boojum
% changed comment referring to documentstyle
%
% Revision 1.5  1999/10/21 14:39:04  boojum
% *** empty log message ***
%
% Revision 1.4  1997/04/18  17:54:10  othomas
% added page numbers on abstract and cover, and made 1 abstract
% page the default rather than 2.  (anne hunter tells me this
% is the new institute standard.)
%
% Revision 1.4  1997/04/18  17:54:10  othomas
% added page numbers on abstract and cover, and made 1 abstract
% page the default rather than 2.  (anne hunter tells me this
% is the new institute standard.)
%
% Revision 1.3  93/05/17  17:06:29  starflt
% Added acknowledgements section (suggested by tompalka)
% 
% Revision 1.2  92/04/22  13:13:13  epeisach
% Fixes for 1991 course 6 requirements
% Phrase "and to grant others the right to do so" has been added to 
% permission clause
% Second copy of abstract is not counted as separate pages so numbering works
% out
% 
% Revision 1.1  92/04/22  13:08:20  epeisach

% NOTE:
% These templates make an effort to conform to the MIT Thesis specifications,
% however the specifications can change.  We recommend that you verify the
% layout of your title page with your thesis advisor and/or the MIT 
% Libraries before printing your final copy.
\title{Reactor Agnostic Multi-Group Cross Section Generation for Fine Mesh Deterministic Neutron Transport Simulations}

\author{William Robert Dawson Boyd III}

% If you wish to list your previous degrees on the cover page, use the 
% previous degrees command:
\prevdegrees{B.S., Georgia Institute of Technology (2010) \\
             M.S., Massachusetts Institute of Technology (2014)}

% You can use the \\ command to list multiple previous degrees
%       \prevdegrees{B.S., University of California (1978) \\
%                    S.M., Massachusetts Institute of Technology (1981)}
\department{Department of Nuclear Science and Engineering}

% If the thesis is for two degrees simultaneously, list them both
% separated by \and like this:
% \degree{Doctor of Philosophy \and Master of Science}
\degree{Doctor of Philosophy in Nuclear Science and Engineering}

% As of the 2007-08 academic year, valid degree months are September, 
% February, or June.  The default is June.
\degreemonth{February}
\degreeyear{2016}
\thesisdate{November 4, 2016}

%% By default, the thesis will be copyrighted to MIT.  If you need to copyright
%% the thesis to yourself, just specify the `vi' documentclass option.  If for
%% some reason you want to exactly specify the copyright notice text, you can
%% use the \copyrightnoticetext command.  
%\copyrightnoticetext{\copyright IBM, 1990.  Do not open till Xmas.}

% If there is more than one supervisor, use the \supervisor command
% once for each.
\supervisor{Kord Smith}{KEPCO Professor of the Practice of Nuclear Science and Engineering} 
% Deparment of Nuclear Science and Engineering
\supervisor{Benoit Forget}{Associate Professor of Nuclear Science and Engineering} 
% Department of Nuclear Science & Engineering

% This is the department committee chairman, not the thesis committee
% chairman.  You should replace this with your Department's Committee
% Chairman.
\chairman{Ju Li}{Professor of Nuclear Science and Engineering}{Chair, Committee on Graduate Students}

% Make the titlepage based on the above information.  If you need
% something special and can't use the standard form, you can specify
% the exact text of the titlepage yourself.  Put it in a titlepage
% environment and leave blank lines where you want vertical space.
% The spaces will be adjusted to fill the entire page.  The dotted
% lines for the signatures are made with the \signature command.
\maketitle

% The abstractpage environment sets up everything on the page except
% the text itself.  The title and other header material are put at the
% top of the page, and the supervisors are listed at the bottom.  A
% new page is begun both before and after.  Of course, an abstract may
% be more than one page itself.  If you need more control over the
% format of the page, you can use the abstract environment, which puts
% the word "Abstract" at the beginning and single spaces its text.

%% You can either \input (*not* \include) your abstract file, or you can put
%% the text of the abstract directly between the \begin{abstractpage} and
%% \end{abstractpage} commands.



% First copy: start a new page, and save the page number.
%\cleardoublepage

% Uncomment the next line if you do NOT want a page number on your
% abstract and acknowledgments pages.
%\setcounter{savepage}{\thepage}
%\begin{abstractpage}
%\begin{abstractpage}

The development of high fidelity multi-group neutron transport-based simulation tools for full core Light Water Reactor (LWR) analysis has been a long-standing goal of the reactor physics community. While direct transport simulations have previously been far too computationally expensive, advances in computer hardware have allowed large scale simulations to become feasible. Therefore, many have focused on developing full core neutron transport solvers that do not incorporate the approximations and assumptions of traditional nodal diffusion solvers. 

Due to the computational expense of direct full core 3D deterministic neutron transport methods, many have focused on 2D/1D methods which solve 3D problems as a coupled system of radial and axial transport problems. However, the coupling of radial and axial problems also introduces approximations. Instead, the work in this thesis focuses on explicitly solving the 3D deterministic neutron transport equations with the Method of Characteristics (MOC).

MOC has been widely used for 2D lattice physics calculations due to its ability to accurately and efficiently simulate reactor physics problems with explicit geometric detail. The work in this thesis strives to overcome the significant computational cost of solving the 3D MOC equations by implementing efficient track generation, axially extruded ray tracing, Coarse Mesh Finite Difference (CMFD) acceleration, linear track-based source approximations, and scalable domain decomposition. 

Additionally, significant attention has been be given to complications that arise in full core simulations with transport-corrected cross-sections. The convergence behavior of transport methods is analyzed, leading to a new strategy for stabilizing the source iteration scheme for neutron transport simulations. The methods are incorporated into the OpenMOC reactor physics code and simulation results are presented for the full core BEAVRS LWR benchmark. Parameter refinement studies and comparisons with reference OpenMC Monte Carlo solutions show that converged full core 3D MOC simulations are feasible on modern supercomputers for the first time.

%However, 3D full core LWR simulations present significant challenges due to greatly increased computational cost.

%The Method of Characteristics (MOC) has seen wide interest in reactor physics because of its accuracy and efficiency in computing lattice physics problems. While most of its use has been in solving 2D problems, there has been recent interest in extending MOC to 3D in order to more accurately calculate 3D power distributions in LWRs. While the method is naturally extensible to 3D, it presents significant computational difficulties. Methods will be presented which mitigate the computational difficulties of 3D MOC by using domain decomposition, efficient track generation, axially extruded ray tracing, CMFD acceleration, and a linear source approximation. Significant attention will be given to complications that arise in full core simulations. 3D MOC results will be presented for the full core simulation of the BEAVRS benchmark, showing that 3D MOC can be a viable tool for anal

\end{abstractpage}
%\end{abstractpage}




% Additional copy: start a new page, and reset the page number.  This way,
% the second copy of the abstract is not counted as separate pages.
% Uncomment the next 6 lines if you need two copies of the abstract
% page.
% \setcounter{page}{\thesavepage}
% \begin{abstractpage}
% \begin{abstractpage}

The development of high fidelity multi-group neutron transport-based simulation tools for full core Light Water Reactor (LWR) analysis has been a long-standing goal of the reactor physics community. While direct transport simulations have previously been far too computationally expensive, advances in computer hardware have allowed large scale simulations to become feasible. Therefore, many have focused on developing full core neutron transport solvers that do not incorporate the approximations and assumptions of traditional nodal diffusion solvers. 

Due to the computational expense of direct full core 3D deterministic neutron transport methods, many have focused on 2D/1D methods which solve 3D problems as a coupled system of radial and axial transport problems. However, the coupling of radial and axial problems also introduces approximations. Instead, the work in this thesis focuses on explicitly solving the 3D deterministic neutron transport equations with the Method of Characteristics (MOC).

MOC has been widely used for 2D lattice physics calculations due to its ability to accurately and efficiently simulate reactor physics problems with explicit geometric detail. The work in this thesis strives to overcome the significant computational cost of solving the 3D MOC equations by implementing efficient track generation, axially extruded ray tracing, Coarse Mesh Finite Difference (CMFD) acceleration, linear track-based source approximations, and scalable domain decomposition. 

Additionally, significant attention has been be given to complications that arise in full core simulations with transport-corrected cross-sections. The convergence behavior of transport methods is analyzed, leading to a new strategy for stabilizing the source iteration scheme for neutron transport simulations. The methods are incorporated into the OpenMOC reactor physics code and simulation results are presented for the full core BEAVRS LWR benchmark. Parameter refinement studies and comparisons with reference OpenMC Monte Carlo solutions show that converged full core 3D MOC simulations are feasible on modern supercomputers for the first time.

%However, 3D full core LWR simulations present significant challenges due to greatly increased computational cost.

%The Method of Characteristics (MOC) has seen wide interest in reactor physics because of its accuracy and efficiency in computing lattice physics problems. While most of its use has been in solving 2D problems, there has been recent interest in extending MOC to 3D in order to more accurately calculate 3D power distributions in LWRs. While the method is naturally extensible to 3D, it presents significant computational difficulties. Methods will be presented which mitigate the computational difficulties of 3D MOC by using domain decomposition, efficient track generation, axially extruded ray tracing, CMFD acceleration, and a linear source approximation. Significant attention will be given to complications that arise in full core simulations. 3D MOC results will be presented for the full core simulation of the BEAVRS benchmark, showing that 3D MOC can be a viable tool for anal

\end{abstractpage}
% \end{abstractpage}

\cleardoublepage

%%%%%%%%%%%%%%%%%%%%%%%%%%%%%%%%%%%%%%%%%%%%%%%%%%%%%%%%%%%%%%%%%%%%%%
% -*-latex-*-
