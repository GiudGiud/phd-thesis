\chapter{Introduction}
\label{chap:intro}

%%%%%%%%%%%%%%%%%%%%%%%%%%%%%%%%%%%%%%%%%%%%%%%%%%%%%%%%%%%%%%%%%%%%%%%%%%%%%%%
\section{Motivation}
\label{sec:chap1-motivation}

%Intro paragraph\\
%-talk about the increasingly important role of simulation in nuclear?\\
%-challenges for today's nuclear fleet which simulation is well-poised to tackle\\
%-segue into talk about nuclear reactor / neutron physics paragraph\\
%-based on empirical models\\
%-need to account for angular dependence of neutron flux (e.g. "transport" methods)\\

Numerical simulation of neutron physics inside a nuclear reactor is fundamental to the design and operation of nuclear power plants. Neutron physics simulations are necessary for determining core reactivity, power distributions, isotopic depletion, and transient behavior. Accurate and predictive simulations can improve the operation of current nuclear reactors by reducing safety margins, incorporating accident tolerant fuels, and extending to longer operating cycle lengths. In addition, predictive neutron physics simulations are critical to the evalulation of advanced reactor designs, many of which operate in very different physics regimes than currently operating reactors. 

Current generation nuclear reactor designs, such as the Westinghouse AP 1000\texttrademark \ac{PWR}, employ much greater geometric and material complexity than previous generations. These new complexities such as axial enrichment zoning and partial length \ac{BP}s allow for greater efficiency and longer cycle lengths by reducing power peaking. These additional features introduce much greater axial hetergeneity compared with previous designs.

Nodal diffusion methods are commonly used to simulate neutron phsyics inside modern reactors. These methods are very fast and efficient but have difficulty capturing localized gradients. While higher fidelity methods are capable of resolving these gradients, they are often significantly slower. This can be prohibitive in reactor analysis where many simulations are required in a relatively short time frame. The goal of high fidelity modeling in this realm is to create a tool that can benchmark and inform the development of nodal diffusion solvers. 

Even as a benchmark tool, high fidelity neutron physics simulations can be too computationally intense to be useful. Therefore, there is a need for high fidelity neutron physics simulations that are accurate and reliable but also computationally efficient. Due to the axial hetergeneity of modern reactor designs, these high fidelity models should be capable of resolving gradients not only in the radial plane but also in the axial direction. This thesis develops a high fidelity 3D \ac{MOC} solver capable of forming benchmark solutions in reasonable computational time.

%%%%%%%%%%%%%%%%%%%%%%%%%%%%%%%%%%%%%%%%%%%%%%%%%%%%%%%%%%%%%%%%%%%%%%%%%%%%%%%
\section{Background}
\label{sec:chap1-background}
TO DO