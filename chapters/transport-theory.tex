\chapter{Transport Theory}
\label{chap:transport}

This chapter briefly introduces the fundamentals of neutron interactions in order to understand the formation of the neutron transport equation. First, neutron reactions are discussed in Section~\ref{sec:transport-fundamentals}, leading to a concise formula for their calculation in terms of the neutron population. Then, in Section~\ref{sec:transport-eq}, the sources and sinks of neutrons are identified in order to determine a general balance equation for the neutron population. Assumptions are then introduced and identified, leading finally to the multi-group transport equation, whose solution is the subject of this thesis. The emphasis in this description is identifying approximations and discussing their impact on solution accuracy. Other more traditional discussions of the neutron transport equation can be found throughout the literature~\cite{henry, duderstadt, duderstadt-martin, bell1967transport, hebert2009applied}. This thesis aims to form a cohesive description of steady-state neutron transport from fundamentals, mentioning all non-trivial approximations.

\section{Neutron Reactions}
\label{sec:transport-fundamentals}

The ultimate goal in neutron transport analysis is to determine the rates of neutron-induced reactions throughout the reactor core. Neutrons can induce or undergo various reactions when they strike materials existing in the reactor core, often referred to as \textit{target nuclei}. These interactions include scattering, capture, and fission, though others exist. Since neutrons are the initiators of these events, understanding their behavior is critical to determining the reaction rates.

The neutron population can be categorized by location $\mathbf{r}$, direction of travel $\mathbf{\Omega}$, and energy $E$ at each time $t$. Here, vectorial quantities (such as direction $\mathbf{\Omega}$) are in bold to differentiate them from scalar quantities (such as energy $E$). Since neutrons modeled inside a nuclear reactor core exist far below the relativistic range, their energy $E$ can be directly related to their velocity $v$ as
\begin{equation}
E = \frac{1}{2} m_n v^2
\end{equation}
where $m_n$ is the mass of the neutron. Therefore, in the context of neutron transport, neutron velocity is nearly synonymous with neutron energy.

The reaction rate $R_X(\mathbf{r}, \mathbf{\Omega}, E, t)$ of type $X$ induced by neutrons of density $n(\mathbf{r}, \mathbf{\Omega}, E, t)$ striking a target material composed of $K$ isotopes, indexed by $k$, each with a number density of $\rho_k(\mathbf{r}, t)$ can be calculated as
\begin{equation}
R_X(\mathbf{r}, \mathbf{\Omega}, E, t) = \sum_{k=1}^K \sigma_{k,X}(\mathbf{r}, \mathbf{\Omega}, E, t) \rho_k(\mathbf{r}, t) n(\mathbf{r}, \mathbf{\Omega}, E, t) v(E) 
\label{eqn:rr_fundamental}
\end{equation}
where $\sigma_{k,X}(\mathbf{r}, \mathbf{\Omega}, E, t)$ is the microscopic nuclear cross-section of isotope $k$\cite{duderstadt}. The nuclear cross-section is a fundamentally important quantity in nuclear engineering that is often interpreted as being related to the probability of a neutron interacting with the target nuclei. This particularly interesting material attribute is discussed further in Chapter~\ref{chap:mgxs}. For the purposes of this discussion on neutron transport, it is necessary to calculate reaction rates -- the goal of neutron transport calculations.

Notice that of the four components in Eq.~\ref{eqn:rr_fundamental}, the first two relate to the target material and the last two relate to the impinging neutrons. These terms are therefore grouped into the macroscopic nuclear cross-section $\Sigma_X(\mathbf{r}, \mathbf{\Omega}, E, t)$ defined as
\begin{equation}
\Sigma_X(\mathbf{r}, \mathbf{\Omega}, E, t) \equiv \sum_{k=1}^K \sigma_{k,X}(\mathbf{r}, \mathbf{\Omega}, E, t) \rho_k(\mathbf{r}, t)
\end{equation} 
and the neutron angular flux $\psi(\mathbf{r}, \mathbf{\Omega}, E, t)$ defined as
\begin{equation}
\psi(\mathbf{r}, \mathbf{\Omega}, E, t) \equiv n(\mathbf{r}, \mathbf{\Omega}, E, t) v(E).
\label{eqn:angular_neutron_flux}
\end{equation}
With these definitions, the reaction rates can be calculated simply as the product of macroscopic cross-section and angular neutron flux:
\begin{equation}
R_X(\mathbf{r}, \mathbf{\Omega}, E, t) = \Sigma_X(\mathbf{r}, \mathbf{\Omega}, E, t) \psi(\mathbf{r}, \mathbf{\Omega}, E, t).
\label{eqn:rr_differential}
\end{equation}

Often, integrated reaction rates over all angles for a region in space are desired. For instance, determining the fission reaction rate within a fuel rod dictates the amount of heat produced by the rod, a necessary input for thermal analysis of a reactor core. The time-dependent fission reaction rate $R_F(t)$ within a volume $V$ with macroscopic fission cross-section $\Sigma_f(\mathbf{r}, \mathbf{\Omega}, E, t)$ at time $t$ can be calculated by integrating Eq.~\ref{eqn:rr_differential} over the volume, all directions, and all energies:

\begin{equation}
R_{F}(t) = \int_{V} d\mathbf{r} \,  \int_{4\pi} d\mathbf{\Omega} \, \int_{0}^{\infty} dE \, \Sigma_f(\mathbf{r}, \mathbf{\Omega}, E, t) \psi(\mathbf{r}, \mathbf{\Omega}, E, t)
\label{eqn:rr_psi}
\end{equation}

With known macroscopic cross-sections, all reaction rates can be determined by obtaining the neutron angular flux.

\section{The Neutron Transport Equation}
\label{sec:transport-eq}

Now that the fundamentals of calculating reaction rates have been established, a balance equation must be formed in order to solve for the neutron angular flux distribution. First, consider the rate of change in neutron population in a given volume $V$ over time in terms of the neutron sources and neutron sinks:
\begin{equation}
\frac{\partial}{\partial t} \int_V d\mathbf{r} \, n(\mathbf{r},\mathbf{\Omega},E,t) = \text{sources} - \text{sinks}
\end{equation}
From Eq.~\ref{eqn:angular_neutron_flux}, this neutron density rate of change can be written in terms of the neutron angular flux. After re-arranging terms, this yields Eq.~\ref{eqn:nt_fundamentals_1}.
\begin{equation}
\int_V d\mathbf{r} \, \left( \frac{1}{v(E)} \frac{\partial}{\partial t} \psi(\mathbf{r},\mathbf{\Omega},E,t)\right) + \text{sinks} = \text{sources}.
\label{eqn:nt_fundamentals_1}
\end{equation}
Now, all the relevant neutron sources and neutron sinks must be identified. The neutron sources are identified as
\begin{equation}
\begin{split}
\text{sources} = \text{prompt fission} \, + \, & \text{in-scattering} \, +  \, \text{neutron influx} \, + \\ & \text{delayed neutron emission} \, + \, \text{external sources} \\
\end{split}
\end{equation}
The prompt fission term refers to neutrons that are emitted during fission reactions. During the fission process, the impingent neutron of direction $\mathbf{\Omega}$ and energy $E$ is modeled as briefly being joined with the target nucleus to form a \textit{compound nucleus}. Then, the compound nucleus breaks into several pieces, including fission products but also releasing some additional neutrons. The number of additional neutrons follows a stochastic process. In deterministic transport, only the mean is modeled so only the average number of neutrons released from the fission process is relevant. This is nuclide-dependent but in order to model the transport through materials rather individual isotopes, an average number of neutrons released from fission in the material $\nu_p(\mathbf{r},\mathbf{\Omega}, E, t)$ is assumed for every fission event. Due to the compound nucleus model from quantum mechanics~\cite{compound-nucleus}, it is possible to assume that the direction and energy of released fission neutrons follow a prompt emission spectrum $\chi_p(\mathbf{r},\mathbf{\Omega}, E,t)$ that is independent of the impingent neutron direction and energy. Therefore, the overall fission rate at each location can be calculated by integrating the product of fission cross-section $\Sigma_f(\mathbf{r},\mathbf{\Omega}, E, t)$  and angular flux over all incoming directions and energies. The prompt emission spectrum is then applied to determine the rate of neutrons emitted at direction $\mathbf{\Omega}$ and energy $E$ as
\begin{equation}
\begin{split}
\text{prompt } & \text{fission} =  \\
 \int_V d\mathbf{r} \, & \chi_p(\mathbf{r}, \mathbf{\Omega}, E,t) \int\displaylimits_{0}^{\infty} dE' \, \int\displaylimits_{4\pi} d\mathbf{\Omega'} \, \nu_p(\mathbf{r},\mathbf{\Omega'}, E', t) \Sigma_f(\mathbf{r},\mathbf{\Omega'}, E', t) \psi(\mathbf{r},\mathbf{\Omega'}, E',t)
\end{split}
\end{equation}
In-scattering refers to neutrons that collide with a target nucleus and scatter into direction $\mathbf{\Omega}$ and energy $E$ from another direction and energy pair $\mathbf{\Omega'}$ and $E'$, respectively. Unlike the prompt fission source, the outgoing neutron direction and energy cannot be decoupled from the incoming direction and energy. This is because the scattering deflection ($\mathbf{\Omega} \cdot \mathbf{\Omega'}$) is highly dependent on the incoming neutron energy. For instance, a neutron is likely to be deflected far less from a light target than a heavier target due to conservation of momentum \cite{duderstadt}. Therefore, a scattering macroscopic cross-section governing the probability of this scattering process is defined in terms of both incoming and outgoing neutron directions and energies as $\Sigma_{s}(\mathbf{r}, \mathbf{\Omega'}\rightarrow \mathbf{\Omega},{E'\rightarrow E},t)$. Integrating over all possible incoming neutron directions and energies yields the in-scattering source term:
\begin{equation}
\text{in-scattering} = \int_V d\mathbf{r} \, \int\displaylimits_{0}^{\infty} dE' \, \int\displaylimits_{4\pi} \, d\mathbf{\Omega'} \Sigma_{s}(\mathbf{r}, \mathbf{\Omega'}\rightarrow \mathbf{\Omega},{E'\rightarrow E},t) \psi(\mathbf{r}, \mathbf{\Omega'},E', t)
\end{equation}
The neutron influx term refers to the rate of neutrons entering the volume $V$ from outside its boundary. A surface $S$ is defined that bounds the volume $V$ and has a surface normal vector $\mathbf{n}$. In this way, all neutrons impingent on the surface $S$ with their direction pointed towards the volume ($\mathbf{\Omega} \cdot \mathbf{n} < 0$) contribute to the neutron influx. In addition, since the rate depends on the velocity of neutrons across the surface, but our definition of angular neutron flux encompasses the neutron velocity in the direction of travel $\mathbf{\Omega}$, the dot product must be taken with the surface normal $\mathbf{n}$. This leads to the definition of the neutron influx as:
\begin{equation}
\text{neutron influx} = - \int_{S \cap \left(\mathbf{\Omega} \cdot \mathbf{n} < 0 \right)} dS \, \left(\mathbf{\Omega} \cdot \mathbf{n} \right) \psi(\mathbf{r}, \mathbf{\Omega}, E, t)
\label{eqn:neutron-influx}
\end{equation}
where the negation is due to the dot product $\left(\mathbf{\Omega} \cdot \mathbf{n} \right)$ being negative.

The delayed neutron emission term refers to neutrons emitted from isotopes formed by the fission process. Some fission products are radioactive and decay via neutron emission. These neutrons are often far lower in energy than prompt fission neutrons. For the purpose of this discussion, no assumptions are made about the form of delayed neutron source and is left as a general function:
\begin{equation}
\text{delayed neutron emission} = \int_V d\mathbf{r} \, D(\mathbf{r}, \mathbf{\Omega}, E, t)
\end{equation}
A more detailed discussion of delayed neutrons can be found elsewhere~\cite{duderstadt, hebert2009applied}. Lastly, neutrons can be released from external sources found within the reactor core. For instance, radioactive sources that emit neutrons are often placed in the core during startup. Similar to the delayed neutron source, no assumptions are made about the form of these external sources:
\begin{equation}
\text{external sources} = \int_V d\mathbf{r} \, S(\mathbf{r}, \mathbf{\Omega}, E, t)
\end{equation}
Now that all of the relevant sources have been identified, it is time to identify the neutron sinks. With few exceptions, the sinks are identified as:
\begin{equation}
\text{sinks} = \text{neutron leakage} \, + \, \text{all neutron interactions}
\end{equation}
The neutron leakage term refers to the loss of neutrons leaving the volume $V$. This is calculated very similar to neutron influx in Eq.~\ref{eqn:neutron-influx} but for neutrons pointed away from the volume ($\mathbf{\Omega} \cdot \mathbf{n} \geq 0$).
\begin{equation}
\text{neutron leakage} = \int_{S \cap \left(\mathbf{\Omega} \cdot \mathbf{n} \geq 0 \right)} dS \, \left(\mathbf{\Omega} \cdot \mathbf{n} \right) \psi(\mathbf{r}, \mathbf{\Omega}, E, t)
\label{eqn:neutron-leakage}
\end{equation}
Combining Eq.~\ref{eqn:neutron-influx} and Eq.~\ref{eqn:neutron-leakage}, the net leakage can be calculated as
\begin{equation}
\text{net leakage} = \text{neutron leakage} - \text{neutron influx} = \int_{S} dS \, \left(\mathbf{\Omega} \cdot \mathbf{n} \right) \psi(\mathbf{r}, \mathbf{\Omega}, E, t)
\label{eqn:net-leakage-surf}
\end{equation}
where the two integrals can be combined because together they form a non-overlapping partition of $S$. Using Gauss divergence theorem, the net leakage term can be cast as a volume integral:
\begin{equation}
\text{net leakage} = \int_V d\mathbf{r} \, \mathbf{\Omega} \cdot \nabla \psi(\mathbf{r},\mathbf{\Omega},E,t)
\label{eqn:net-leakage}
\end{equation}
The last term that needs to be defined is for all neutron interactions within the volume $V$. Since the direction $\mathbf{\Omega}$ and energy $E$ are continuous variables, in order to conserve momentum and energy, an observable interaction must change the neutron direction and energy. Therefore any observable reaction should be regarded as a loss of neutron population traveling at the incoming direction and energy. A total cross-section $\Sigma_{t}(\mathbf{r},\mathbf{\Omega},E, t)$ is defined relating to the total probability of any interaction per unit length. Therefore the total number of interactions within the volume can be calculated as
\begin{equation}
\text{all neutron interactions} = \int_V d\mathbf{r} \, \Sigma_{t}(\mathbf{r},\mathbf{\Omega},E, t)\psi(\mathbf{r},\mathbf{\Omega},E,t).
\label{eqn:total-interactions}
\end{equation}
Combining all of these terms, a neutron balance equation is formed in Eq.~\ref{eqn:first-balance}
\begin{equation}
\begin{split}
\int_V d\mathbf{r} \, \frac{1}{v(E)} \frac{\partial \psi(\mathbf{r},\mathbf{\Omega},E,t)}{\partial t} \, + \, \int_V d\mathbf{r} \, \mathbf{\Omega} \cdot \nabla \psi(\mathbf{r},\mathbf{\Omega},E,t) \, + \, \int_V d\mathbf{r} \, \Sigma_{t}(\mathbf{r},\mathbf{\Omega},E, t)\psi(\mathbf{r},\mathbf{\Omega},E,t)\\
  =  \, \int_V d\mathbf{r} \, \chi_p(\mathbf{r}, \mathbf{\Omega}, E,t) \int\displaylimits_{0}^{\infty} dE' \, \int\displaylimits_{4\pi} d\mathbf{\Omega'} \, \nu_p(\mathbf{r},\mathbf{\Omega'}, E', t) \Sigma_f(\mathbf{r},\mathbf{\Omega'}, E', t) \psi(\mathbf{r},\mathbf{\Omega'}, E',t )\\
 + \, \int_V d\mathbf{r} \, \int\displaylimits_{0}^{\infty} dE' \, \int\displaylimits_{4\pi} \, d\mathbf{\Omega'} \Sigma_{s}(\mathbf{r}, \mathbf{\Omega'}\rightarrow \mathbf{\Omega},{E'\rightarrow E},t) \psi(\mathbf{r}, \mathbf{\Omega'},E', t) \\ 
 + \, \int_V d\mathbf{r} \, S(\mathbf{r}, \mathbf{\Omega}, E, t) +  \int_V d\mathbf{r} \, D(\mathbf{r}, \mathbf{\Omega}, E, t)
\end{split}
\label{eqn:first-balance}
\end{equation}

Since the volume $V$ was defined arbitrarily and all terms are integrated over the volume, in order for the equality to hold the function must be identical across all possible volumes. This is only true if the underlying functions are identical. Therefore the integral can be dropped from all terms yielding a new balance equation in Eq.~\ref{eqn:first-diff-balance}.
\begin{equation}
	\begin{split}
		\frac{1}{v(E)} \frac{\partial \psi(\mathbf{r},\mathbf{\Omega},E,t)}{\partial t} \, & + \, \mathbf{\Omega} \cdot \nabla \psi(\mathbf{r},\mathbf{\Omega},E,t) \, + \, \Sigma_{t}(\mathbf{r},\mathbf{\Omega},E, t)\psi(\mathbf{r},\mathbf{\Omega},E,t) = \\
		& \phantom{+} \, \chi_p(\mathbf{r}, \mathbf{\Omega}, E,t) \int\displaylimits_{0}^{\infty} dE' \, \int\displaylimits_{4\pi} d\mathbf{\Omega'} \, \nu_p(\mathbf{r},\mathbf{\Omega'}, E', t) \Sigma_f(\mathbf{r},\mathbf{\Omega'}, E', t) \psi(\mathbf{r},\mathbf{\Omega'}, E',t )\\
		& + \, \int\displaylimits_{0}^{\infty} dE' \, \int\displaylimits_{4\pi} \, d\mathbf{\Omega'} \Sigma_{s}(\mathbf{r}, \mathbf{\Omega'}\rightarrow \mathbf{\Omega},{E'\rightarrow E},t) \psi(\mathbf{r}, \mathbf{\Omega'},E', t) \\ 
		& + \, S(\mathbf{r}, \mathbf{\Omega}, E, t) + D(\mathbf{r}, \mathbf{\Omega}, E, t).
	\end{split}
	\label{eqn:first-diff-balance}
\end{equation}
This is the time-dependent neutron transport equation. In the context of this thesis, only steady-state problems will be analyzed, removing any temporal dependence. Therefore all time-dependence can be eliminated from Eq.~\ref{eqn:first-diff-balance}. In doing so, the delayed neutron emission is encompassed by the fission emission term, yielding a new fission emission spectrum $\chi(\mathbf{r}, \mathbf{\Omega}, E)$ and a new average number of neutrons per fission $\nu(\mathbf{r}, \mathbf{\Omega}, E)$ in Eq~\ref{eqn:first-time-independent-balance}.
\begin{equation}
\begin{split}
\mathbf{\Omega} \cdot \nabla \psi(\mathbf{r},\mathbf{\Omega},E) \, + & \, \Sigma_{t}(\mathbf{r},\mathbf{\Omega},E)\psi(\mathbf{r},\mathbf{\Omega},E) = \\
& \phantom{+} \, \chi(\mathbf{r}, \mathbf{\Omega}, E) \int\displaylimits_{0}^{\infty} dE' \, \int\displaylimits_{4\pi} d\mathbf{\Omega'} \, \nu(\mathbf{r},\mathbf{\Omega'},E') \Sigma_f(\mathbf{r},\mathbf{\Omega'}, E') \psi(\mathbf{r},\mathbf{\Omega'}, E')\\
& + \, \int\displaylimits_{0}^{\infty} dE' \, \int\displaylimits_{4\pi} \, d\mathbf{\Omega'} \Sigma_{s}(\mathbf{r}, \mathbf{\Omega'}\rightarrow \mathbf{\Omega},{E'\rightarrow E}) \psi(\mathbf{r}, \mathbf{\Omega'},E') \\ 
& + \, S(\mathbf{r}, \mathbf{\Omega}, E)
\end{split}
\label{eqn:first-time-independent-balance}
\end{equation}

Next, external sources are often assumed to be trivial. During full power operation of a nuclear power reactor, this is indeed true. The fission source overwhelms any external neutron source. Once the external sources are removed it is clear that the trivial solution $\psi(\mathbf{r},\mathbf{\Omega},E) = 0$ is a solution, and might indeed be the only solution that solves the problem for the provided cross-sections. 

In reality, the cross-sections are far from known. Instead, there are feedback mechanisms that force them to vary based on the neutron population. For instance, an increase in neutron population would likely lead to an increase in temperature that often causes neutron absorption to increase. Since cross-sections at steady-state operation are not precisely known, there is numerical difficulty in demanding equilibrium between sinks and sources. Therefore, a scaling factor $k$ is introduced in Eq.~\ref{eqn:k-indroduction} that modifies the fission source. This additional degree of freedom allows for solutions of the neutron transport equation were the unscaled sources do not exactly match the sinks.
\begin{equation}
	\begin{split}
		\mathbf{\Omega} \cdot \nabla \psi(\mathbf{r},\mathbf{\Omega},E) \, + & \, \Sigma_{t}(\mathbf{r},\mathbf{\Omega},E)\psi(\mathbf{r},\mathbf{\Omega},E) = \\
		& \phantom{+} \, \frac{\chi(\mathbf{r}, \mathbf{\Omega}, E)}{k} \int\displaylimits_{0}^{\infty} dE' \, \int\displaylimits_{4\pi} d\mathbf{\Omega'} \, \nu(\mathbf{r},\mathbf{\Omega'}E') \Sigma_f(\mathbf{r},\mathbf{\Omega'}, E') \psi(\mathbf{r},\mathbf{\Omega'}, E')\\
		& + \, \int\displaylimits_{0}^{\infty} dE' \, \int\displaylimits_{4\pi} \, d\mathbf{\Omega'} \Sigma_{s}(\mathbf{r}, \mathbf{\Omega'}\rightarrow \mathbf{\Omega},{E'\rightarrow E}) \psi(\mathbf{r}, \mathbf{\Omega'},E')
	\end{split}
	\label{eqn:k-indroduction}
\end{equation}
A closer examination of the scaling factor $k$ shows it is an eigenvalue of the system. This will become more obvious when a system of equations is formed to solve the neutron transport equation. In simulating steady state behavior, the dominant mode is often desired. The eigenvalue $k$ relating to that dominant mode is termed the criticality of the reactor. Only an eigenvalue $k=1$ indicates that the reactor is \textit{critical} -- that neutron sources exactly match sinks. However, due to imperfect knowledge of the system and its constituent material cross-sections the eigenvalue is never exactly $k=1$ in practical applications.

%If this mode corresponds to $k>1$, then the reactor is termed to be super-critical, meaning that sources overpower the sinks for the provided reactor configuration and cross-sections. A value of $k = 1$ implies the reactor is operating at perfect steady-state conditions, and the reactor is termed to be critical. For a value of $k < 1$ the reactor is termed to be sub-critical.

In the fission source term, the fission emission spectrum $\chi(\mathbf{r}, \mathbf{\Omega}, E)$ dictates how neutrons are emitted from fission events. Since the energy of emitted neutrons is so large in comparison with typical neutron energies causing neutron emission, as well as compound nucleus creation, there is virtually no dependence on angle. Therefore, the emission is assumed to be isotropic, as presented in Eq.~\ref{eqn:tr-isotropic-emission}.
\begin{equation}
	\begin{split}
		\mathbf{\Omega} \cdot \nabla \psi(\mathbf{r},\mathbf{\Omega},E) \, + & \, \Sigma_{t}(\mathbf{r},\mathbf{\Omega},E)\psi(\mathbf{r},\mathbf{\Omega},E) = \\
		& \phantom{+} \, \frac{\chi(\mathbf{r},E)}{4\pi k} \int\displaylimits_{0}^{\infty} dE' \, \int\displaylimits_{4\pi} d\mathbf{\Omega'} \, \nu(\mathbf{r},\mathbf{\Omega'},E') \Sigma_f(\mathbf{r},\mathbf{\Omega'}, E') \psi(\mathbf{r},\mathbf{\Omega'}, E')\\
		& + \, \int\displaylimits_{0}^{\infty} dE' \, \int\displaylimits_{4\pi} \, d\mathbf{\Omega'} \Sigma_{s}(\mathbf{r}, \mathbf{\Omega'}\rightarrow \mathbf{\Omega},{E'\rightarrow E}) \psi(\mathbf{r}, \mathbf{\Omega'},E')
	\end{split}
	\label{eqn:tr-isotropic-emission}
\end{equation}

Next, angular dependence is removed from non-scattering cross-sections, as well as the mean number of neutrons released by fission $\nu(\mathbf{r},\mathbf{\Omega},E')$ in Eq.~\ref{eqn:xs-angular-dependence-removal}. Since materials with strong orientation structures (eg. a crystalline structure) are uncommon in typical reactor configurations, this assumption introduces virtually no bias.
\begin{equation}
	\begin{split}
		\mathbf{\Omega} \cdot \nabla \psi(\mathbf{r},\mathbf{\Omega},E) \, + & \, \Sigma_{t}(\mathbf{r},E)\psi(\mathbf{r},\mathbf{\Omega},E) = \\
		& \phantom{+} \, \frac{\chi(\mathbf{r},E)}{4\pi k} \int\displaylimits_{0}^{\infty} dE' \, \nu(\mathbf{r},E') \Sigma_f(\mathbf{r}, E') \int\displaylimits_{4\pi} d\mathbf{\Omega'} \,  \psi(\mathbf{r},\mathbf{\Omega'}, E')\\
		& + \, \int\displaylimits_{0}^{\infty} dE' \, \int\displaylimits_{4\pi} \, d\mathbf{\Omega'} \Sigma_{s}(\mathbf{r}, \mathbf{\Omega'}\rightarrow \mathbf{\Omega},{E'\rightarrow E}) \psi(\mathbf{r}, \mathbf{\Omega'},E')
	\end{split}
	\label{eqn:xs-angular-dependence-removal}
\end{equation}

Many neutron transport methods rely on solving Eq.~\ref{eqn:xs-angular-dependence-removal} as it only incorporates very mild assumptions. For instance, many Monte Carlo solvers implicitly solve the neutron transport equation in this form~\cite{mcnpx2003manual, serpent2013manual, openmc2016manual, bucholz1982scale}. However, this form is too general for many deterministic solvers to be feasible. One of the added complexities comes from the angular dependence of the scattering source.

For computational efficiency, the scattering is therefore assumed to be isotropic. This assumption, unlike many of the others in this section, does indeed induce a measurable bias. In typical LWRs, significant hydrogen is present which scatters neutrons preferentially in the forward direction, which is not captured with the isotropic scattering assumption. However, in Section~\ref{eqn:transport-correction}, an adjustment will be introduced to account for the discrepancy between isotropic and anisotropic scattering. With the isotropic scattering assumption, all neutron sources are be simulated as equal in all directions, as presented in Eq.~\ref{eqn:transport-isotropic}.
\begin{equation}
	\begin{split}
		\mathbf{\Omega} \cdot \nabla \psi(\mathbf{r},\mathbf{\Omega},E) \, + & \, \Sigma_{t}(\mathbf{r},E)\psi(\mathbf{r},\mathbf{\Omega},E) = \\
		& \phantom{+} \, \frac{\chi(\mathbf{r},E)}{4\pi k} \int\displaylimits_{0}^{\infty} dE' \, \nu(\mathbf{r},E') \Sigma_f(\mathbf{r}, E') \int\displaylimits_{4\pi} d\mathbf{\Omega'} \,  \psi(\mathbf{r},\mathbf{\Omega'}, E')\\
		& + \, \int\displaylimits_{0}^{\infty} dE' \,  \frac{\Sigma_{s}(\mathbf{r},{E'\rightarrow E})}{4\pi}\int\displaylimits_{4\pi} \, d\mathbf{\Omega'} \psi(\mathbf{r}, \mathbf{\Omega'},E')
	\end{split}
	\label{eqn:transport-isotropic}
\end{equation}

The only angular dependence of source terms in Eq.~\ref{eqn:transport-isotropic} are due to the angular flux. Therefore it is convenient to define the scalar flux as the integral of the angular flux over all directions in Eq.~\ref{eqn:scalar-flux}.
\begin{equation}
\phi(\mathbf{r}, E) = \int\displaylimits_{4\pi} d\mathbf{\Omega} \,  \psi(\mathbf{r},\mathbf{\Omega}, E).
\label{eqn:scalar-flux}
\end{equation}
With this definition, the balance equation can be written more succinctly in Eq.~\ref{eqn:pre-multigroup}.
\begin{equation}
	\begin{split}
		\mathbf{\Omega} \cdot \nabla \psi(\mathbf{r},\mathbf{\Omega},E) \, + & \, \Sigma_{t}(\mathbf{r},E)\psi(\mathbf{r},\mathbf{\Omega},E) = \\
		& \phantom{+} \, \frac{\chi(\mathbf{r},E)}{4\pi k} \int\displaylimits_{0}^{\infty} dE' \, \nu(\mathbf{r},E') \Sigma_f(\mathbf{r}, E') \phi(\mathbf{r}, E')\\
		& + \, \int\displaylimits_{0}^{\infty} dE' \,  \frac{\Sigma_{s}(\mathbf{r},{E'\rightarrow E})}{4\pi} \phi(\mathbf{r}, E')
	\end{split}
	\label{eqn:pre-multigroup}
\end{equation}

\section{Multi-Group Transport}

Similar to how angular dependence of the scattering source causes increased computational complexity for deterministic methods, the continuous energy nature of the neutron transport equation \textit{greatly} increases computational complexity. Therefore, many deterministic methods rely on a multi-group form of the neutron transport equation in which neutron populations are grouped into energy intervals or \textit{groups}.

To determine the neutron population within a given energy group $g$, Eq.~\ref{eqn:pre-multigroup} is integrated over the associated energy interval with minimum energy corresponding to $E_{g-1}$ and maximum energy corresponding to $E_{g}$ which leads to Eq.~\ref{eqn:energy-range-int-transport}.
\begin{equation}
\begin{split}
\mathbf{\Omega} \cdot \nabla \psi_g(\mathbf{r},\mathbf{\Omega}) \, + & \, \int_{E_{g-1}}^{E_{g}} dE \, \Sigma_{t}(\mathbf{r},E)\psi(\mathbf{r},\mathbf{\Omega},E) = \\
& \phantom{+} \, \int_{E_{g-1}}^{E_g} dE \, \frac{\chi(\mathbf{r},E)}{4\pi k} \int\displaylimits_{0}^{\infty} dE' \, \nu(\mathbf{r},E') \Sigma_f(\mathbf{r}, E') \phi(\mathbf{r}, E')\\
& + \, \int_{E_{g-1}}^{E_g} dE \, \int\displaylimits_{0}^{\infty} dE' \,  \frac{\Sigma_{s}(\mathbf{r},{E'\rightarrow E})}{4\pi} \phi(\mathbf{r}, E')
\end{split}
\label{eqn:energy-range-int-transport}
\end{equation}
where the multi-group angular flux $\psi_g(\mathbf{r},\mathbf{\Omega})$ is defined as
\begin{equation}
\psi_g(\mathbf{r},\mathbf{\Omega}) = \int_{E_{g-1}}^{E_g} dE \, \psi(\mathbf{r},\mathbf{\Omega},E).
\end{equation}
If the entire energy range $[0, \infty)$ is partitioned into $G$ non-overlapping energy intervals, yielding $G+1$ energy boundaries, then integrals over the entire range can be cast as the sum of integrals over each energy interval. This is introduced in Eq.~\ref{eqn:transport-group-introduction}.
\begin{equation}
\begin{split}
\mathbf{\Omega} \cdot \nabla \psi_g(\mathbf{r},\mathbf{\Omega}) \, + & \, \int_{E_{g-1}}^{E_g} dE \, \Sigma_{t}(\mathbf{r},E)\psi(\mathbf{r},\mathbf{\Omega},E) = \\
& \phantom{+} \, \int_{E_{g-1}}^{E_g} dE \, \frac{\chi(\mathbf{r},E)}{4\pi k} \sum_{g'=1}^G \left( \int\displaylimits_{E_{g'-1}}^{E_{g'}} dE' \, \nu(\mathbf{r},E') \Sigma_f(\mathbf{r}, E') \phi(\mathbf{r}, E') \right)\\
& + \, \int_{E_{g-1}}^{E_g} dE \, \sum_{g'=1}^G \left( \int\displaylimits_{E_{g'-1}}^{E_{g'}} dE' \,  \frac{\Sigma_{s}(\mathbf{r},{E'\rightarrow E})}{4\pi} \phi(\mathbf{r}, E') \right).
\end{split}
\label{eqn:transport-group-introduction}
\end{equation}
For convenience, the group-wise scalar flux $\phi_g(\mathbf{r})$ is defined in Eq.~\ref{eqn:multi-group-scalar-flux}.
\begin{equation}
\phi_g(\mathbf{r}) = \int_{E_{g-1}}^{E_g} dE \, \phi(\mathbf{r},E)
\label{eqn:multi-group-scalar-flux}
\end{equation}
and \textit{multi-group cross-sections} are defined in Eq.~\ref{eqn:multi-group-xs-definitions-first} -- Eq.~\ref{eqn:multi-group-xs-definitions-last}.
\begin{eqnarray}
\Sigma_{t}^g(\mathbf{r},\mathbf{\Omega}) = \frac{\int_{E_{g-1}}^{E_g} dE \, \Sigma_{t}(\mathbf{r},E)\psi(\mathbf{r},\mathbf{\Omega},E)}{\int_{E_{g-1}}^{E_g} dE \, \psi(\mathbf{r},\mathbf{\Omega},E)} 
\label{eqn:multi-group-xs-definitions-first}\\
\nu \Sigma_f^g\left(\mathbf{r}\right) = \frac{\int\displaylimits_{E_{g'-1}}^{E_{g'}} dE \, \nu(\mathbf{r},E) \Sigma_f(\mathbf{r}, E) \phi(\mathbf{r}, E)}{\int\displaylimits_{E_{g'-1}}^{E_{g'}} dE \, \phi(\mathbf{r}, E)} \\
\chi_g\left(\mathbf{r}\right) = \frac{\int\displaylimits_{E_{g'-1}}^{E_{g'}} dE \, \chi(\mathbf{r},E) \sum_{g'=1}^G \nu \Sigma_f^{g'}(\mathbf{r}) \phi_{g'}(\mathbf{r})}{\sum_{g=1}^G \nu \Sigma_f^{g}(\mathbf{r}) \phi_{g}(\mathbf{r})}
\end{eqnarray}
\begin{equation}
\begin{split}
\Sigma_{s}^{g' \rightarrow g}\left(\mathbf{r}\right) = \frac{\int_{E_{g-1}}^{E_g} dE \, \int\displaylimits_{E_{g'-1}}^{E_{g'}} dE' \,  \Sigma_{s}(\mathbf{r},{E'\rightarrow E}) \phi(\mathbf{r}, E') }{\int_{E_{g-1}}^{E_g} dE \, \phi(\mathbf{r},E)}
\end{split}
\label{eqn:multi-group-xs-definitions-last}
\end{equation}
Incorporating these definitions into the transport equation yields the \textit{multi-group transport equation} given in Eq.~\ref{eqn:full-multi-group-transport}.
\begin{equation}
\begin{split}
\mathbf{\Omega} \cdot \nabla \psi_{g}(\mathbf{r},\mathbf{\Omega}) & \, + \, \Sigma_t^{g}(\mathbf{r},\mathbf{\Omega}) \psi_{g}(\mathbf{r},\mathbf{\Omega}) =  \\
& \frac{1}{4 \pi} \left( \frac{\chi_{g}\left(\mathbf{r}\right)}{k} \sum_{g'=1}^{G} \nu_{g'}\left(\mathbf{r}\right) \Sigma_f^{g'}\left(\mathbf{r}\right) \phi_{g'}\left(\mathbf{r}\right) + \, \sum_{g'=1}^G \,  \Sigma_{s}^{g' \rightarrow g}\left(\mathbf{r}\right) \phi_{g'}(\mathbf{r}) \right)
\end{split}
\label{eqn:full-multi-group-transport}
\end{equation}
It is important to note that no assumptions were needed from the energy independent form in Eq.~\ref{eqn:pre-multigroup} to arrive at the multi-group transport equation in Eq.~\ref{eqn:full-multi-group-transport}. However, the expressions for multi-group cross-sections are dependent on either scalar flux $\phi(\mathbf{r}, E)$ or angular flux $\psi(\mathbf{r},\mathbf{\Omega},E)$ which are unknown. Therefore, approximations need to be made in order to determine appropriate values for multi-group cross-sections. This will be the subject of Chapter~\ref{chap:mgxs}.

\newpage
\vfill
\begin{highlightsbox}[frametitle=Highlights]
	\begin{itemize}
		\item By properly identifying sources and sinks of neutrons, a balance equation can be derived from fundamentals.
		\item After light approximations, sources are limited to scattering and fission. Sinks are limited to neutron interactions and net leakage.
		\item In order to model reactors with imperfect knowledge of cross-sections, the eigenvalue $k$ is introduced which scales the fission source, forcing equilibrium between sources and sinks.
		\item This multi-group transport equation removes dependence on the continuous energy variable, producing a balance equation for each energy interval which are inter-dependent.
		\item The derivation of the multi-group transport equation from a continuous energy form involves no additional approximations, but relies on knowing the final solution in order to obtain accurate multi-group cross-section values.
		\item For the purpose of this thesis, scattering is assumed to be isotropic which introduces significant bias. However, this bias is hoped to be overcome with an appropriate transport correction, discussed in Chapter~\ref{chap:mgxs}.
	\end{itemize}
\end{highlightsbox}
\vfill