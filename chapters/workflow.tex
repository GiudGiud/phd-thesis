\chapter{Simulation Workflow}
\label{chap:workflow}

%%%%%%%%%%%%%%%%%%%%%%%%%%%%%%%%%%%%%%%%%%%%%%%%%%%%%%%%%%%%%%%%%%%%%%%%%%%%%%%%
\section{A Simulation Triad}
\label{chap3:triad}


%%%%%%%%%%%%%%%%%%%%%%%%%%%%%%%%%%%%%%%%%%%%%%%%%%%%%%%%%%%%%%%%%%%%%%%%%%%%%%%%
\section{OpenMC}
\label{sec:chap3-openmc}

%%%%%%%%%%%%%%%%%%%%%%%
\subsection{Python API}
\label{sec:chap3-py-api}

%%%%%%%%%%%%%%%%%%%%%%%%%%%%%%%%
\subsubsection{Tally Arithmetic}
\label{sec:chap3-tally-arithmetic}

%%%%%%%%%%%%%%%%%%%%%%%%%%%%%%%%%
\subsubsection{Pandas DataFrames}
\label{sec:chap3-pandas-df}

%%%%%%%%%%%%%%%%%%%%%%%%%%%%%%%%%%%%%
\subsection{Distributed Cell Tallies}
\label{sec:chap3-distribcells}

\begin{itemize}
  \item Capture \ac{MGXS} spatial variation
\end{itemize}

%%%%%%%%%%%%%%%%%%%%%%%%%%%%
\subsection{MGXS Generation}
\label{sec:chap3-mgxs}

\begin{itemize}[noitemsep]
  \item Python \ac{API} software stacks together in \texttt{openmc.mgxs}
\end{itemize}

%%%%%%%%%%%%%%%%%%%%%%%%%%%%%%%%%%%%%%%%
\subsection{Isotropic-in-Lab Scattering}
\label{sec:chap3-iso-in-lab}


%%%%%%%%%%%%%%%%%%%%%%%%%%%%%%%%%%%%%%%%%%%%%%%%%%%%%%%%%%%%%%%%%%%%%%%%%%%%%%%%
\section{OpenMOC}
\label{sec:chap3-openmoc}

\begin{itemize}[noitemsep]
  \item reference Annals paper
  \item reference parallel perf. papers
  \item reference 3D \ac{MOC} papers
  \item brief description of \ac{MOC} equations
  \begin{itemize}[noitemsep]
    \item flat/linear source approx.
    \item isotropic approx.
    \item constant-in-angle \ac{MGXS}
    \item uses total/transport, nu-fission, (nu-)scattering matrix, chi
  \end{itemize}
\end{itemize}


%%%%%%%%%%%%%%%%%%%%%%%%%%%%%%%%%%%%%%%%%%%%%%%%%%%%%%%%%%%%%%%%%%%%%%%%%%%%%%%%
\section{OpenCG}
\label{sec:chap3-opencg}

\begin{itemize}[noitemsep]
  \item single physics-agnostic geometries
  \item compatibility modules with OpenMC, OpenMOC
  \item ``glue'' code to map OpenMC distribcells to OpenMOC spatial mesh
\end{itemize}

%%%%%%%%%%%%%%%%%%%%%%%%%%%%%%%%%%%%
\subsection{Local Neighbor Symmetry}
\label{sec:chap3-lns}

\begin{itemize}[noitemsep]
  \item analogy to lattice physics geometric templates based on eng. approx.
\end{itemize}

%%%%%%%%%%%%%%%%%%%%%%%%%%%%%%%%%%%%
\subsection{Region Differentiation}
\label{sec:chap3-region-diff}

\begin{itemize}[noitemsep]
  \item retain CSG's simplicity of repeated primitives for user input
  \item build combinatorial geometries based on arbitrary clustering of cell instances
\end{itemize}