\begin{appendices}

%%%%%%%%%%%%%%%%%%%%%%%%%%%%%%%%%%%%%%%%%%%%%%%%%%%%%%%%%%%%%%%%%%%%%%%%%%%%%%%
\chapter{Energy Group Structures}
\label{app:energy-groups}

The energy group structures are from the CASMO-4 lattice physics code~\cite{edenius1995casmo}.

\renewcommand{\arraystretch}{0.8}% Tighter

\begin{table}[h!]
  \centering
  \footnotesize
  \caption{One group energy boundaries.}
  \label{table:app-1-groups} 
  \vspace{14pt}
  \begin{tabular}{c r r}
    \toprule
    {\bf Group No.} &
    {\bf Lower Bound [MeV]} &
    {\bf Upper Bound [MeV]} \\
    \midrule
1 & 0.0000E+00 & 2.0000E+01 \\
    \bottomrule
   \end{tabular}
\end{table}

\begin{table}[h!]
  \centering
  \footnotesize
  \caption{Two group energy boundaries.}
  \label{table:app-2-groups} 
  \vspace{14pt}
  \begin{tabular}{c r r}
    \toprule
    {\bf Group No.} &
    {\bf Lower Bound [MeV]} &
    {\bf Upper Bound [MeV]} \\
    \midrule
2 & 0.0000E+00 & 6.2500E-07 \\
1 & 6.2500E-07 & 2.0000E+01 \\
  \bottomrule
 \end{tabular}
\end{table}

\begin{table}[h!]
  \centering
  \footnotesize
  \caption{Four group energy boundaries.}
  \label{table:app-4-groups} 
  \vspace{14pt}
  \begin{tabular}{c r r}
    \toprule
    {\bf Group No.} &
    {\bf Lower Bound [MeV]} &
    {\bf Upper Bound [MeV]} \\
    \midrule
4 & 0.0000E+00 & 6.2500E-07 \\
3 & 6.2500E-07 & 5.5300E-03 \\
2 & 5.5300E-03 & 8.2100E-01 \\
1 & 8.2100E-01 & 2.0000E+01 \\
  \bottomrule
 \end{tabular}
\end{table}

\begin{table}[h!]
  \centering
  \footnotesize
  \caption{Eight group energy boundaries.}
  \label{table:app-8-groups} 
  \vspace{14pt}
  \begin{tabular}{c r r}
    \toprule
    {\bf Group No.} &
    {\bf Lower Bound [MeV]} &
    {\bf Upper Bound [MeV]} \\
    \midrule
8 & 0.0000E+00 & 5.8000E-08 \\
7 & 5.8000E-08 & 1.4000E-07 \\
6 & 1.4000E-07 & 2.8000E-07 \\
5 & 2.8000E-07 & 6.2500E-07 \\
4 & 6.2500E-07 & 4.0000E-06 \\
3 & 4.0000E-06 & 5.5300E-03 \\
2 & 5.5300E-03 & 8.2100E-01 \\
1 & 8.2100E-01 & 2.0000E+01 \\
  \bottomrule
 \end{tabular}
\end{table}

\begin{table}[h!]
	\centering
	\footnotesize
	\caption{Eleven group energy boundaries.}
	\label{table:app-11-groups} 
	\vspace{14pt}
	\begin{tabular}{c r r}
		\toprule
		{\bf Group No.} &
		{\bf Lower Bound [MeV]} &
		{\bf Upper Bound [MeV]} \\
		\midrule
		11 & 0.0000E+00 & 6.2500E-07 \\
		10 & 5.8000E-08 & 8.5000E-07 \\
		9 & 1.4000E-07 & 2.8000E-07 \\
		8 & 2.8000E-07 & 6.2500E-07 \\
		7 & 6.2500E-07 & 4.0000E-06 \\
		6 & 4.0000E-06 & 9.8770E-06 \\
		5 & 9.8770E-06 & 1.5968E-05 \\
		4 & 1.5968E-05 & 2.7700E-05 \\
		3 & 2.7700E-05 & 5.5300E-03 \\
		2 & 5.5300E-03 & 8.2100E-01 \\
		1 & 8.2100E-01 & 2.0000E+01 \\
		\bottomrule
	\end{tabular}
\end{table}

\begin{table}[h!]
  \centering
  \footnotesize
  \caption{Sixteen group energy boundaries.}
  \label{table:app-16-groups} 
  \vspace{14pt}
  \begin{tabular}{c r r}
    \toprule
    {\bf Group No.} &
    {\bf Lower Bound [MeV]} &
    {\bf Upper Bound [MeV]} \\
    \midrule
16 & 0.0000E+00 & 3.0000E-08 \\
15 & 3.0000E-08 & 5.8000E-08 \\
14 & 5.8000E-08 & 1.4000E-07 \\
13 & 1.4000E-07 & 2.8000E-07 \\
12 & 2.8000E-07 & 3.5000E-07 \\
11 & 3.5000E-07 & 6.2500E-07 \\
10 & 6.2500E-07 & 8.5000E-07 \\
9 & 8.5000E-07 & 9.7200E-07 \\
8 & 9.7200E-07 & 1.0200E-06 \\
7 & 1.0200E-06 & 1.0970E-06 \\
6 & 1.0970E-06 & 1.1500E-06 \\
5 & 1.1500E-06 & 1.3000E-06 \\
4 & 1.3000E-06 & 4.0000E-06 \\
3 & 4.0000E-06 & 5.5300E-03 \\
2 & 5.5300E-03 & 8.2100E-01 \\
1 & 8.2100E-01 & 2.0000E+01 \\
  \bottomrule
 \end{tabular}
\end{table}

\begin{table}[h!]
  \centering
  \footnotesize
  \caption{Twenty-five group energy boundaries.}
  \label{table:app-25-groups} 
  \vspace{14pt}
  \begin{tabular}{c r r}
    \toprule
    {\bf Group No.} &
    {\bf Lower Bound [MeV]} &
    {\bf Upper Bound [MeV]} \\
    \midrule
25 & 0.0000E+00 & 3.0000E-08 \\
24 & 3.0000E-08 & 5.8000E-08 \\
23 & 5.8000E-08 & 1.4000E-07 \\
22 & 1.4000E-07 & 2.8000E-07 \\
21 & 2.8000E-07 & 3.5000E-07 \\
20 & 3.5000E-07 & 6.2500E-07 \\
19 & 6.2500E-07 & 9.7200E-07 \\
18 & 9.7200E-07 & 1.0200E-06 \\
17 & 1.0200E-06 & 1.0970E-06 \\
16 & 1.0970E-06 & 1.1500E-06 \\
15 & 1.1500E-06 & 1.8550E-06 \\
14 & 1.8550E-06 & 4.0000E-06 \\
13 & 4.0000E-06 & 9.8770E-06 \\
12 & 9.8770E-06 & 1.5968E-05 \\
11 & 1.5968E-05 & 1.4873E-04 \\
10 & 1.4873E-04 & 5.5300E-03 \\
9 & 5.5300E-03 & 9.1180E-03 \\
8 & 9.1180E-03 & 1.1100E-01 \\
7 & 1.1100E-01 & 5.0000E-01 \\
6 & 5.0000E-01 & 8.2100E-01 \\
5 & 8.2100E-01 & 1.3530E+00 \\
4 & 1.3530E+00 & 2.2310E+00 \\
3 & 2.2310E+00 & 3.6790E+00 \\
2 & 3.6790E+00 & 6.0655E+00 \\
1 & 6.0655E+00 & 2.0000E+01 \\
  \bottomrule
 \end{tabular}
\end{table}

\afterpage{
	\renewcommand*{\arraystretch}{0.6}% Tighter
	\footnotesize
	\begin{longtable}[h!]{c r r}
		\caption{Seventy group energy boundaries.} \\
		\label{table:app-70-groups} \\
		\toprule
		{\bf Group No.} &
		{\bf Lower Bound [MeV]} &
		{\bf Upper Bound [MeV]} \\
		\midrule
		70 & 0.0000E+00 & 5.0000E-09 \\
		69 & 5.0000E-09 & 1.0000E-08 \\
		68 & 1.0000E-08 & 1.5000E-08 \\
		67 & 1.5000E-08 & 2.0000E-08 \\
		66 & 2.0000E-08 & 2.5000E-08 \\
		65 & 2.5000E-08 & 3.0000E-08 \\
		64 & 3.0000E-08 & 3.5000E-08 \\
		63 & 3.5000E-08 & 4.2000E-08 \\
		62 & 4.2000E-08 & 5.0000E-08 \\
		61 & 5.0000E-08 & 5.8000E-08 \\
		60 & 5.8000E-08 & 6.7000E-08 \\
		59 & 6.7000E-08 & 8.0000E-08 \\
		58 & 8.0000E-08 & 1.0000E-07 \\
		57 & 1.0000E-07 & 1.4000E-07 \\
		56 & 1.4000E-07 & 1.8000E-07 \\
		55 & 1.8000E-07 & 2.2000E-07 \\
		54 & 2.2000E-07 & 2.5000E-07 \\
		53 & 2.5000E-07 & 2.8000E-07 \\
		52 & 2.8000E-07 & 3.0000E-07 \\
		51 & 3.0000E-07 & 3.2000E-07 \\
		50 & 3.2000E-07 & 3.5000E-07 \\
		49 & 3.5000E-07 & 4.0000E-07 \\
		48 & 4.0000E-07 & 5.0000E-07 \\
		47 & 5.0000E-07 & 6.2500E-07 \\
		46 & 6.2500E-07 & 7.8000E-07 \\
		45 & 7.8000E-07 & 8.5000E-07 \\
		44 & 8.5000E-07 & 9.1000E-07 \\
		43 & 9.1000E-07 & 9.5000E-07 \\
		42 & 9.5000E-07 & 9.7200E-07 \\
		41 & 9.7200E-07 & 9.9600E-07 \\
		40 & 9.9600E-07 & 1.0200E-06 \\
		39 & 1.0200E-06 & 1.0450E-06 \\
		38 & 1.0450E-06 & 1.0710E-06 \\
		37 & 1.0710E-06 & 1.0970E-06 \\
		36 & 1.0970E-06 & 1.1230E-06 \\
		35 & 1.1230E-06 & 1.1500E-06 \\
		34 & 1.1500E-06 & 1.3000E-06 \\
		33 & 1.3000E-06 & 1.5000E-06 \\
		32 & 1.5000E-06 & 1.8550E-06 \\
		31 & 1.8550E-06 & 2.1000E-06 \\
		30 & 2.1000E-06 & 2.6000E-06 \\
		29 & 2.6000E-06 & 3.3000E-06 \\
		28 & 3.3000E-06 & 4.0000E-06 \\
		27 & 4.0000E-06 & 9.8770E-06 \\
		26 & 9.8770E-06 & 1.5968E-05 \\
		25 & 1.5968E-05 & 2.7700E-05 \\
		24 & 2.7700E-05 & 4.8052E-05 \\
		23 & 4.8052E-05 & 7.5501E-05 \\
		22 & 7.5501E-05 & 1.4873E-04 \\
		21 & 1.4873E-04 & 3.6726E-04 \\
		20 & 3.6726E-04 & 9.0690E-04 \\
		19 & 9.0690E-04 & 1.4251E-03 \\
		18 & 1.4251E-03 & 2.2395E-03 \\
		17 & 2.2395E-03 & 3.5191E-03 \\
		16 & 3.5191E-03 & 5.5300E-03 \\
		15 & 5.5300E-03 & 9.1180E-03 \\
		14 & 9.1180E-03 & 1.5030E-02 \\
		13 & 1.5030E-02 & 2.4780E-02 \\
		12 & 2.4780E-02 & 4.0850E-02 \\
		11 & 4.0850E-02 & 6.7340E-02 \\
		10 & 6.7340E-02 & 1.1100E-01 \\
		9 & 1.1100E-01 & 1.8300E-01 \\
		8 & 1.8300E-01 & 3.0250E-01 \\
		7 & 3.0250E-01 & 5.0000E-01 \\
		6 & 5.0000E-01 & 8.2100E-01 \\
		5 & 8.2100E-01 & 1.3530E+00 \\
		4 & 1.3530E+00 & 2.2310E+00 \\
		3 & 2.2310E+00 & 3.6790E+00 \\
		2 & 3.6790E+00 & 6.0655E+00 \\
		1 & 6.0655E+00 & 2.0000E+01 \\
		\bottomrule
	\end{longtable}
}

\chapter{On-the-fly Ray Tracing by $z$-Stack}
\label{app:z-stack}

In Chapter~\ref{chap:ray-tracing}, on-the-fly ray tracing was introduced. In this appendix, the relationships formed for calculating segment crossings of a $z$-stack are more thoroughly explained.

Recall that all tracks in a $z$-stack have the same polar angle $\theta$, project onto the same 2D track, and are separated by a constant axial ray spacing $\delta z$. Therefore, the axial height $z_i$ of the $i^{\textit{th}}$ lowest track (starting from 0) can be given as a function of distance $s$ along the associated 2D track as
\begin{equation}
z_i(s) = z_0(0) + i\delta z + s \cot{\theta}
\label{eq::app-track_projection}
\end{equation}
where $z_0(0)$ is the $z$-coordinate at the intersection of the lowest track with the $z$-axis at the start of the associated 2D track.

For each 2D segment in the 2D track, there is an associated axially extruded region which contains a list of 3D \ac{SR}s in the region. Using the boundaries of each \ac{SR} and Eq.~\ref{eq::app-track_projection}, it is possible to analytically compute the indexes in the $z$-stack of tracks that will cross the \ac{SR}.

To begin, the first track to traverse the \ac{SR} (lowest index $i$) is determined. All tracks that traverse the \ac{SR} must have their highest $z$-height above the lowest boundary of the \ac{SR}, $z_{\textit{min}}$. Specifically,
\begin{equation}
\max_{s_{\textit{start}} \, \leq \, s \, \leq \, s_{\textit{end}}} z_i(s) > z_{\textit{min}}
\end{equation}
where $s_{\textit{start}}$ is the $s$ distance along the 2D track at the start of the 2D segment and $s_{\textit{end}}$ is the $s$ distance along the 2D track at the end of the 2D segment. Inserting the relationship in Eq.~\ref{eq::app-track_projection}, this can be expanded to
\begin{equation}
\max_{s_{\textit{start}} \, \leq \, s \, \leq \, s_{\textit{end}}} z_0(0) + i\delta z + s \cot{\theta} > z_{\textit{min}}.
\end{equation}
Noting the linear relationship of the tracks and the constant axial ray spacing, this can be simplified in terms of the lowest track with height $z_0$ as
\begin{equation}
\max\left(z_0(s_{\textit{start}}), z_0(s_{\textit{end}})\right) + i\delta z > z_{\textit{min}}
\end{equation}
and cast in terms of the index $i$ as
\begin{equation}
i > \frac{z_{\textit{min}} - \max\left(z_0(s_{\textit{start}}), z_0(s_{\textit{end}})\right)}{\delta z}.
\end{equation}
Since $i$ is an integer, the lowest track index $i_{\textit{start}}$ to traverse the region can be given by
\begin{equation}
i_{\textit{start}} = \Bigg\lceil\frac{z_{\textit{min}} - \max\left({z_0(s_{\textit{start}}), z_0(s_{\textit{end}})}\right) }{\delta z}\Bigg\rceil
\end{equation}

Next, the last track to traverse the \ac{SR} (highest index $i$) is determined. All tracks that traverse the \ac{SR} must have their lowest $z$-height below the highest boundary of the \ac{SR}, $z_{\textit{max}}$. Specifically,
\begin{equation}
\min_{s_{\textit{start}} \, \leq \, s \, \leq \, s_{\textit{end}}} z_i(s) < z_{\textit{max}}
\end{equation}
which can be expanded to
\begin{equation}
\min_{s_{\textit{start}} \, \leq \, s \, \leq \, s_{\textit{end}}} z_0(0) + i\delta z + s \cot{\theta} < z_{\textit{max}}.
\end{equation}
Similar to the arguments for the first track index, the last track index follows the criteria
\begin{equation}
i < \frac{z_{\textit{max}} - \min\left(z_0(s_{\textit{start}}), z_0(s_{\textit{end}})\right)}{\delta z}.
\end{equation}
Therefore, the last track to cross the \ac{SR} has index $i_{\textit{end}}$ given by
\begin{equation}
i_{\textit{end}} = \Bigg\lfloor\frac{z_{\textit{max}} - \min\left({z_0(s_{\textit{start}}), z_0(s_{\textit{end}})}\right) }{\delta z}\Bigg\rfloor.
\end{equation}

In addition to analytically calculating the first and last track indexes, it is possible to calculate the indexes of tracks with the same length. Specifically, it is possible to calculate the first and last tracks to cross the entire 2D segment length or the entire 3D source height.

For a track to cross the entire 2D segment length, both its beginning and ending heights must be between the minimum and maximum \ac{SR} boundaries. For a track to cross the entire axial height of the \ac{SR}, its beginning and ending heights must be above and below the minimum and maximum \ac{SR} boundaries, respectively.

Two indexes $i_{\textit{in}}$ and $i_{\textit{out}}$ are calculated. The index $i_{\textit{in}}$ represents the first track to start above the lowest \ac{SR} boundary and $i_{\textit{out}}$ represents the first track to end above the maximum \ac{SR} boundary. All tracks with index between $i_{\textit{in}}$ and $i_{\textit{out}}$ traverse the entire 2D segment length. All tracks with index between $i_{\textit{out}}$ and $i_{\textit{in}}$ will traverse the entire \ac{SR} height. Notice that these are mutually exclusive: if some tracks traverse the entire 2D segment length, none will traverse the entire \ac{SR} height.

Tracks that start above the minimum boundary satisfy
\begin{equation}
\min_{s_{\textit{start}} \, \leq \, s \, \leq \, s_{\textit{end}}} z_i(s) > z_{\textit{min}}
\end{equation}
and tracks that end above the maximum boundary satisfy
\begin{equation}
\max_{s_{\textit{start}} \, \leq \, s \, \leq \, s_{\textit{end}}} z_i(s) > z_{\textit{max}}.
\end{equation}
Therefore, in a similar process to determining the first and last tracks to cross the \ac{SR}, the indexes $i_{\textit{in}}$ and $i_{\textit{out}}$ can be calculated as:
\begin{equation}
i_{\textit{in}} = \Bigg\lceil\frac{z_{\textit{min}} - \min\left({z_0(s_{\textit{start}}), z_0(s_{\textit{end}}})\right) }{\delta z}\Bigg\rceil
\end{equation}
\begin{equation}
i_{\textit{out}} = \Bigg\lceil\frac{z_{\textit{max}} - \max\left({z_0(s_{\textit{start}}), z_0(s_{\textit{end}}})\right) }{\delta z}\Bigg\rceil
\end{equation}

\end{appendices}
