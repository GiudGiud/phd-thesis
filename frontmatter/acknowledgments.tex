\section*{Acknowledgments}

This thesis was based upon work supported by the U.S. Department of Energy Office of Science Advanced Scientific Computing Research's Center for Exascale Simulation of Advanced Reactors (CESAR) under Contract DE-AC02-06CH11357. The author was additionally supported by the Idaho National Laboratory and the National Science Foundation Graduate Research Fellowship Grant No. 1122374. This research made use of the resources of the High Performance Computing Center at Idaho National Laboratory, which is supported by the Office of Nuclear Energy of the U.S. Department of Energy and the Nuclear Science User Facilities under Contract No. DE-AC07-05ID14517.

I would first like to thank my thesis co-advisors Professors Kord Smith and Benoit Forget for empowering me with the intellectual freedom to explore a novel topic for my thesis research. The encouragement they provided during our numerous enlightening discussions helped me surmount the challenging obstacles that I faced with this thesis research. I am grateful for their personal and professional guidance which has instilled in me the values of hard work and integrity, as well as a practical view on how to effect the change I seek to make in the world. I look forward to maintaining a collegial friendship with each of them as I embark upon the next chapter in my life.

I am incredibly thankful for my former and current peers within the Computational Reactor Physics Group (CRPG) with whom I have developed deep friendships that will outlast my time at MIT. I am especially grateful for Samuel Shaner, my roommate, officemate, collaborator and friend with whom I created the OpenMOC code. I am thankful for my productive collaboration with Paul Romano to develop a Python interface, including a module for multi-group cross section generation, for the OpenMC code. I am appreciative of the many fruitful conversations that I shared with Nathan Gibson to diagnose the bias induced by the flux separability approximation in multi-group cross section generation. Adam Nelson was gracious to correspond with me by email to share his experience in generating multi-group cross sections with OpenMC for use in the MPACT code. Finally, my collaborations with many other fellow students, including Nicholas Horelik, Bryan Herman, Jon Walsh, Derek Lax, Sterling Harper, Logan Abel, Matthew Ellis, Geoffrey Gunow, Lulu Li, Derek Gaston and John Tramm made it fun and exciting to contribute to open source code projects as a member of CRPG.

I am grateful for my loving parents, Bill and Gail Boyd, who have tirelessly supported each and every opportunity I have sought out in my educational career. I cannot thank them enough for the sacrifices that they have made throughout their lives to enable me to pursue my dreams. Lastly, I am incredibly thankful for my girlfriend Frances Chiang and her love and patience during the five and a half years of late nights and working weekends that I have spent working on this thesis. Her calm and steady hand helped me remain focused on both our short- and long-term goals, and I am incredibly excited to continue our partnership together in the coming years.

%third paragraph: rest of committee?
%-koroush, emilio

%fifth paragraph: family
%-parents, brother and sister
%-frances