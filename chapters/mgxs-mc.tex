\chapter{MGXS Generation with Monte Carlo}
\label{chap:mgxs-mc}

In the preceding chapter it was observed that many approximations are made in multi-group theory and the generation of multi-group cross sections. Monte Carlo is an approach to replace some of the steps in the standard multi-level framework for \ac{MGXS} generation with a natural and reactor agnostic treatment of energy and spatial self-shielding. This chapter presents a brief overview of \ac{MC} tallies and statistics in Sec.~\ref{sec:chap3-mc-overview}, outlines the necessary computation needed to generate \ac{MGXS} with \ac{MC} in Sec.~\ref{sec:chap3-mgxs-gen}, and discusses past studies which applied \ac{MC} for \ac{MGXS} generation in Sec.~\ref{sec:chap3-lit-review}.

%and Sec.~\ref{sec:chap3-mgxs-gen} outlines the necessary computation needed to generate \ac{MGXS} with \ac{MC}. Sec.~\ref{sec:chap3-lit-review} discusses past work to apply \ac{MC} for \ac{MGXS} generation, while Sec.~\ref{sec:chap3-latent-variables} introduces a new approach based on a latent variable model for spatial self-shielding.


%%%%%%%%%%%%%%%%%%%%%%%%%%%%%%%%%%%%%%%%%%%%%%%%%%%%%%%%%%%%%%%%%%%%%%%%%%%%%%%
\section{Overview of Monte Carlo Methods}
\label{sec:chap3-mc-overview}

Monte Carlo methods have been successfully applied to neutron transport calculations for many decades. A detailed accounting of the physics models and algorithms used in \ac{MC} methods can be found in the manuals for the Serpent~\cite{serpent2013manual}, MCNP~\cite{mcnpx2003manual}, and OpenMC~\cite{openmc2016manual} Monte Carlo particle transport codes. This section presents a few key aspects related to tallies and statistics, and follows directly from the manual for the OpenMC~\cite{openmc2016manual} code which is used throughout this work.

%%%%%%%%%%%%%%%%%%%%%%%%%%%%%%%%
\subsection{Monte Carlo Tallies}
\label{subsec:chap3-mc-tallies}

\ac{MC} simulations sample the particle distribution in order to compute integral quantities of interest called \textit{tallies}. A tally is an integral of a \textit{scoring function} $f$ weighted by the neutron distribution, or flux, across some region of phase space. A general form for tally $\mathcal{T}$ is given by the following integral expression:

\begin{dmath}
\label{eqn:chap3-tallies-general}
\mathcal{T} = \int_{V} \int_{S} \int_{E}  f(\mathbf{r},\mathbf{\Omega},E)\psi(\mathbf{r},\mathbf{\Omega},E)\mathrm{d}E\mathrm{d}\mathbf{\Omega}\mathrm{d}\mathbf{r} 
\end{dmath}

In OpenMC parlance, the integration bounds over space, angle and energy are termed \textit{filters}, while the scoring function $f$ is simply known as a \textit{score}. Various scores may be used to compute volume-integrated fluxes, reaction rates and functional expansions. \ac{MC} does not perform the integration in Eqn.~\ref{eqn:chap3-tallies-general} with the exact flux specified at all points in phase space. Instead, \ac{MC} performs stochastic integration by sampling the particle population across the entirety of phase space to compute a \textit{statistical estimate} $\hat{\mathcal{T}}$ of the true integral $\mathcal{T}$.

There are a number of different techniques to estimate a tally. The first and most general method is known as an \textit{analog estimator}. An analog estimator $\hat{\mathcal{T}}$ increments a tally by the particle weight $w_{i}$ each time an event $i$ occurs from the set of all events of interest $A$ (\textit{e.g.}, fission events). The sum of particle weights is then normalized by the total weight of all particles $W$ to compute the interaction frequency on a per-particle basis within the phase space volume of interest:

\begin{dmath}
\label{eqn:chap3-tallies-analog}
\hat{\mathcal{T}} = \frac{1}{W}\displaystyle\sum\limits_{i \in A} w_{i}
\end{dmath}

Although analog estimators permit general filters and scoring functions, they may suffer from poor tallying efficiency if the size of set $A$ is very small compared to the total number of events in a simulation. A \textit{collision estimator} improves the tallying efficiency by incrementing a tally more frequently than is possible with analog estimators. In particular, a collision estimator increments a tally at each collision $i$ from the set of all collisions $C$ irregardless of the types of collisions that took place. By noting that the total collision rate is given by $R_{t} = \Sigma_{t}\phi$, a collision estimator for the flux $\hat{\phi}$ may be simply defined by dividing the particle weights by the total macroscopic cross section:

\begin{dmath}
\label{eqn:chap3-tallies-collision-flux}
\hat{\phi} = \frac{1}{W}\displaystyle\sum\limits_{i \in C}\frac{w_{i}}{\Sigma_{t}(E_{i})}
\end{dmath}

\noindent The collision estimator depends on the energy of the incoming particle $E_{i}$ in order to scale the particle weight by the energy-dependent total cross section. It follows that the collision estimator for a reaction rate $\hat{\mathcal{R}}_{x}$ is the product of the flux in Eqn.~\ref{eqn:chap3-tallies-collision-flux} and the cross section for the reaction type $x$ of interest:

\begin{dmath}
\label{eqn:chap3-tallies-collision-rxn}
\hat{\mathcal{R}}_{x} = \frac{1}{W}\displaystyle\sum\limits_{i \in C}\frac{w_{i}\Sigma_{x}(E_{i})}{\Sigma_{t}(E_{i})}
\end{dmath}

The collision estimator increases the number of events in $C$ to improve the tallying efficiency with respect to analog tallies. A third method known as \textit{track-length estimators} goes one step further and increments a tally each time a particle trajectory crosses the phase space of interest (\textit{e.g.}, a spatial tally mesh zone) even if a collision did not take place. The track-length estimator makes use of a particle's distance traveled $\Delta\mathbf{r}$ to estimate the flux. A track-length estimator of the flux is therefore:

\begin{dmath}
\label{eqn:chap3-tallies-track-length-flux}
  \hat{\phi} = \frac{1}{W}\displaystyle\sum\limits_{i \in T}w_{i}\Delta\mathbf{r}_{i}
\end{dmath}

\noindent where the set $T$ represents each particle trajectory through the phase space volume of interest. Similarly, a track-length estimate of a reaction rate $x$ can be found by simply multiplying the flux in Eqn.~\ref{eqn:chap3-tallies-track-length-flux} by the cross section:

\begin{dmath}
\label{eqn:chap3-tallies-track-length-rxn}
  \hat{\mathcal{R}}_{x} = \frac{1}{W}\displaystyle\sum\limits_{i \in T}w_{i}\Delta\mathbf{r}_{i}\Sigma_{x}(E_{i})
\end{dmath}

The tallying efficiency for track-length estimators is greatly improved by incrementing a tally for each particle trajectory. As a result, the confidence intervals are generally much tighter for track-length estimators than those for analog and collision estimators. 

Each of the three estimators -- analog, collision and track-length -- may be useful for different scenarios. Although track-length and collision estimators improve statistics over analog estimators, they cannot be employed for all types of filters. For example, track-length tallies may not be used if the scoring function requires information about the outgoing particle since this is not available unless a collision has taken place. As discussed in Sec.~\ref{sec:chap3-mgxs-gen}, a mixture of estimators which tradeoff generality with efficiency must be used to generate \ac{MGXS} from \ac{MC} tallies.


%%%%%%%%%%%%%%%%%%%%%%%%%%%%%%
\subsection{Sample Statistics}
\label{subsec:chap3-mc-stats}

\ac{MC} performs stochastic integration with one of a number of different estimators. In each case, the tally estimator $\hat{\mathcal{T}}$ is computed as a sample mean of of all of the events or particle trajectories simulated. An unbiased estimate of the sample mean is given by:

\begin{equation}
\label{eqn:chap3-sample-mean}
\bar{x} = \frac{1}{N} \displaystyle\sum\limits_{i=1}^{N} x_{i}
\end{equation}

\noindent where each of the $N$ samples is given by a random variable $x_{i}$. For \ac{MC} codes that use batch-based statistics, such as OpenMC, $N$ is the number of batches of particles simulated and $x_{i}$ is the tally estimator for the $i$\textsuperscript{th} batch of particles. 

Each of the random variables $x_{i}$ is sampled from some probability distribution representative of the physics in the simulation. The sampling distribution is normal such that $x_{i} \sim \mathcal{N}(\mu,\sigma^{2})$ where $\mathcal{N}(\mu,\sigma^{2})$ signifies a normal distribution with mean $\mu$ and variance $\sigma^{2}$ if each batch of particles is independent. An unbiased estimate of the variance $\sigma^{2}$ of the normal distribution from which the population $x_{i}$ is drawn may be estimated using Bessel's correction for the sample variance $s^{2}$:

\begin{equation}
\label{eqn:chap3-variance-sample}
s^{2} = \frac{1}{N-1}\displaystyle\sum\limits_{i=1}^{N}\left(x_{i} - \bar{x}\right)^{2}
\end{equation}

\noindent As $N \rightarrow \infty$ the population variance estimator will approach the true variance $\sigma^{2}$ of the underlying distribution. In the case of batch-based statistics, the variance $\sigma^{2}$ will be determined by the number of particles simulated per batch -- the more particle histories simulated per batch, the smaller $\sigma^{2}$ will be, and vice versa.

In general, it is more useful to quantify the uncertainty of a tally estimator than it is to compute the sample variance. The variance of the sample mean is representative of the distribution from which the random variable $\bar{x}$ is drawn and indicates the degree of confidence one may have in a tally estimator. By the Central Limit Theorem, the sample mean $\bar{x}$ will converge to the mean of a normal distribution if the samples $x_{i}$ are uncorrelated. An unbiased estimate of the variance of the sample mean can be derived from the Bienaym\'{e} formula to give:

\begin{equation}
\label{eqn:chap3-variance-mean}
s_{\bar{x}}^{2} = \frac{1}{N-1}\left(\frac{1}{N}\displaystyle\sum\limits_{i=1}^{N}
x_{i}^{2} - \bar{x}^2\right)
\end{equation}

A key observation is that the standard deviation of the sample mean is directly proportional to $\nicefrac{1}{\sqrt{N}}$ if the samples $x_{i}$ are uncorrelated. This necessarily implies that the uncertainties on a tally estimator can be made arbitrarily small given enough simulated particle histories. However, recent studies have shown that tally estimators in eigenvalue calculations are not generally independent and identically distributed realizations due to correlated fission sources between batches~\cite{herman2014correlation,miao2016correlation}. 

\begin{emphbox}
\textbf{Monte Carlo provides statistical estimators for quantities such as energy- and volume-integrated reaction rates and fluxes. Track-length estimators are more statistically efficient than analog and collision estimators, but are not generally applicable for scoring functions dependent on outgoing neutron energy.}
\end{emphbox}


%%%%%%%%%%%%%%%%%%%%%%%%%%%%%%%%%%%%%%%%%%%%%%%%%%%%%%%%%%%%%%%%%%%%%%%%%%%%%%%
\section{MGXS Generation with Monte Carlo}
\label{sec:chap3-mgxs-gen}

This section describes how multi-group cross sections may be computed using stochastic integration. Sec.~\ref{subsec:chap3-tally-types} outlines the types of OpenMC tallies needed to generate \ac{MGXS} -- including the scores, filters and estimators for each tally -- and the arithmetic combinations used to combine different tallies. Sec.~\ref{subsec:chap3-uncertainty-prop} illustrates how the uncertainties of the \ac{MGXS} may be estimated using error propagation theory.


%%%%%%%%%%%%%%%%%%%%%%%%
\subsection{Tally Types Needed for MGXS Generation}
\label{subsec:chap3-tally-types}

The types of \ac{MGXS} needed to solve the neutron transport equation were outlined in Chap.~\ref{chap:mgxs}, including expressions for the transport-corrected total cross section and scattering matrix, and the fission production cross section and emission spectrum. This section outlines the types of tallies needed to compute these \ac{MGXS}. It is important to note that the flux separability approximation (Sec.~\ref{subsec:chap2-angle}) is applied in the tally formulations for each of the group constants.

\subsubsection{Inner Product Notation}
\label{subsec:chap3-tally-types-notation}

The following sections use angle bracket notation $\langle \cdot , \cdot \rangle$ to represent inner products in phase space. This may correspond to integrals over incoming and/or outgoing energy, space, and angle. Using this notation, a tally estimator for reaction rate $x$ is represented as follows: 

\begin{equation}
\label{eqn:chap3-inner-prod-notation}
\langle \Sigma_x, \psi \rangle = \int_{V} \int_{S} \int_{E} \Sigma_{x}(\mathbf{r},E)\psi(\mathbf{r},E,\mathbf{\Omega}) \mathrm{d}E\mathrm{d}\mathbf{\Omega}\mathrm{d}\mathbf{r}
\end{equation}

\noindent This notation is specialized throughout this section with subscripts to indicate the subsets of phase space that are integrated over in the inner product. In particular, subscript $k$ refers to a volume integral over $V_{k}$ for some region of space $k$ for spatial homogenization (Sec.~\ref{subsec:chap2-space}), while subscript $g$ corresponds to an integral over energies with $E \in [E_{g}, E_{g-1}]$ for energy condensation (Sec.~\ref{subsec:chap2-energy}). For example, the microscopic reaction rate for reaction $x$ by nuclide $i$ is denoted as:

\begin{equation}
\label{eqn:chap3-angle-rxn-rate}
\langle \sigma_{x,i}, \psi \rangle_{k,g} = \int_{\mathbf{r} \in V_{k}} \int_{4\pi} \int_{E_{g}}^{E_{g-1}} \sigma_{x,i}(\mathbf{r},E)\psi(\mathbf{r},E,\mathbf{\Omega}) \mathrm{d}E\mathrm{d}\mathbf{\Omega}\mathrm{d}\mathbf{r}
\end{equation}

\noindent The inner product of a function with unity, such as the spatially-homogenized and energy-integrated flux is denoted by:

\begin{equation}
\label{eqn:chap3-angle-flux}
\langle \psi \rangle_{k,g} \equiv \langle \psi, \mathbb{1} \rangle_{k,g} = \int_{\mathbf{r} \in V_{k}} \int_{4\pi} \int_{E_{g}}^{E_{g-1}} \psi(\mathbf{r},E,\mathbf{\Omega}) \mathrm{d}E\mathrm{d}\mathbf{\Omega}\mathrm{d}\mathbf{r}
\end{equation}

Finally, the superscripts $a$ and $t\ell$ are given to those inner products computed with analog and track-length estimators, respectively -- \textit{i.e.}, $\langle \cdot,\cdot \rangle^{a}$ is an analog tally estimator and $\langle \cdot,\cdot \rangle^{t\ell}$ is a track-length tally estimator of the corresponding inner products.


\subsubsection{General Reaction Cross Section}
\label{subsubsec:chap3-gen-xs}

A general spatially-homogenized and energy condensed macroscopic multi-group cross section for reaction $x$, spatial zone $k$ and energy group $g$ can be computed with track-length tally estimators in OpenMC. The \ac{MGXS} is simply the ratio of the group-wise reaction rates $\langle \Sigma_{x}, \psi \rangle_{k,g}^{t\ell}$ and fluxes $\langle \psi \rangle_{k,g}^{t\ell}$:

\begin{equation}
\label{eqn:chap3-general-macro}
\hat{\Sigma}_{x,k,g} = \frac{\langle \Sigma_{x}, \psi \rangle_{k,g}^{t\ell}}{\langle \psi \rangle_{k,g}^{t\ell}}
\end{equation}

\noindent Likewise, a microscopic \ac{MGXS} for nuclide $i$ can be computed as follows:

\begin{equation}
\label{eqn:chap3-general-micro}
\hat{\sigma}_{x,i,k,g} = \frac{\langle \sigma_{x,i}, \psi \rangle_{k,g}^{t\ell}}{\langle \psi \rangle_{k,g}^{t\ell}}
\end{equation}

These estimators are used for reaction types which are only dependent on the incoming energy of a neutron, such as total and radiative capture reactions.


\subsubsection{Total Cross Section}
\label{subsubsec:chap3-tally-types-tot-xs}

The total macroscopic cross section $\Sigma_{t}$ is a special case of Eqn.~\ref{eqn:chap3-general-micro}, with track-length estimators for the total collision rate and flux:

\begin{equation}
\label{eqn:chap3-total-macro}
\hat{\Sigma}_{t,k,g} = \frac{\langle \Sigma_{t}, \psi \rangle_{k,g}^{t\ell}}{\langle \psi \rangle_{k,g}^{t\ell}}
\end{equation}

As discussed in Sec.~\ref{subsec:chap2-transport-corr}, a transport correction is often used to incorporate anisotropic scattering effects into the transport equation with an isotropic scattering kernel. An expression for the in-scatter approximation~\cite{yamamoto2008simplified} to the transport correction can be computed with an OpenMC tally for the first Legendre scattering moment\footnote{It is assumed that scattering multiplicity is included in the scattering moments as discussed in Sec.~\ref{sec:chap2-scatt-prod}.}. The inner product for this tally is given by:

\begin{equation}
\label{eqn:chap3-sigs1}
\langle \Sigma_{s1}, \psi \rangle_{k,g'\rightarrow g} = \int_{\mathbf{r} \in V_{k}} \int_{4\pi} \int_{E_{g}}^{E_{g-1}} \int_{E_{g'}}^{E_{g'-1}} \Sigma_{s1}(\mathbf{r},E'\rightarrow E)\psi(\mathbf{r},E',\mathbf{\Omega}) \mathrm{d}E'\mathrm{d}E\mathrm{d}\mathbf{\Omega}\mathrm{d}\mathbf{r}
\end{equation}

\noindent An analog estimator must be used in OpenMC since the tally includes an integral over the outgoing neutron energy. The spatially-homogenized and energy condensed transport-corrected total cross section given in Eqn.~\ref{eqn:chap2-transport-xs} is computed by summing over all incoming energy groups:

\begin{equation}
\label{eqn:chap3-transport-corr-macro}
\Delta\hat{\Sigma}_{tr,k,g} = \displaystyle\sum\limits_{g'=1}^{G} \langle{\Sigma_{s1}, \psi \rangle_{k,g'\rightarrow g}^{a}}
\end{equation}

\noindent The transport correction is then subtracted from the group-wise total collision rate and normalized by the flux to compute the transport-corrected total cross section:

\begin{equation}
\label{eqn:chap3-sigt-transport-macro}
\hat{\tilde{\Sigma}}_{t,k,g} = \frac{\langle \Sigma_{t}, \psi \rangle_{k,g}^{a} - \Delta\hat{\Sigma}_{tr,k,g}}{\langle \psi \rangle_{k,g}^{a}}
\end{equation}

\noindent Note that since the transport correction must be computed using an analog estimator, the total collision and flux in Eqn.~\ref{eqn:chap3-sigt-transport-macro} must also be computed with analog estimators.


\subsubsection{Scattering Matrix}
\label{subsubsec:chap3-tally-types-scatt-mat}

The isotropic scattering matrix is computed with an inner product of scattering reactions over both incoming and outgoing energies. An analog estimator must be used since the integral is dependent on the neutron's outgoing energy. Similar to the first Legendre moment in Eqn.~\ref{eqn:chap3-sigs1}, the isotropic scattering moment is given by the following expression:

\begin{equation}
\label{eqn:chap3-sigs0}
\langle \Sigma_{s0}, \psi \rangle_{k,g'\rightarrow g} = \int_{\mathbf{r} \in V_{k}} \int_{4\pi} \int_{E_{g}}^{E_{g-1}} \int_{E_{g'}}^{E_{g'-1}} \Sigma_{s0}(\mathbf{r},E'\rightarrow E)\psi(\mathbf{r},E,\mathbf{\Omega}) \mathrm{d}E'\mathrm{d}E\mathrm{d}\mathbf{\Omega}\mathrm{d}\mathbf{r}
\end{equation}

\noindent The isotropic scattering matrix is then:

\begin{equation}
\label{eqn:chap3-scatter-macro}
\hat{\Sigma}_{s,k,g'\rightarrow g} = \frac{\langle \Sigma_{s0}, \psi \rangle_{k,g'\rightarrow g}^{a}}{\langle \psi \rangle_{k,g'}^{a}}
\end{equation}

\noindent The transport correction in Eqn.~\ref{eqn:chap3-transport-corr-macro} can be applied by subtracting it from the diagonal elements in the matrix to compute the transport-corrected scattering matrix:

\begin{equation}
\label{eqn:chap3-scatter-trans-macro}
\hat{\tilde{\Sigma}}_{s,k,g'\rightarrow g} = \frac{\langle \Sigma_{s0}, \psi \rangle_{k,g'\rightarrow g}^{a} - \delta_{g,g'} \Delta\hat{\Sigma}_{tr,k,g}}{\langle \psi \rangle_{k,g'}^{a}}
\end{equation}

%\begin{equation}
%\label{eqn:chap3-scatter-trans-macro}
%\hat{\tilde{\Sigma}}_{s,k,g'\rightarrow g} = \frac{\langle \Sigma_{s0}, \psi \rangle_{k,g'\rightarrow g}^{a} - \delta_{g,g'} \displaystyle\sum\limits_{g''=1}^{G} \langle{\Sigma_{s1}, \psi \rangle_{k,g''\rightarrow g}^{a}}}{\langle \psi \rangle_{k,g'}^{a}}
%\end{equation}


\subsubsection{Fission Production Cross Section}
\label{subsubsec:chap3-tally-types-fiss-prod}

The fission production cross section was condensed in Eqn.~\ref{eqn:chap2-nusifg} to make it independent of the energies of the neutrons emitted from fission. It is therefore straightforward to treat the fission product macroscopic cross section $\nu\Sigma_{f}$ as a special case of Eqn.~\ref{eqn:chap3-general-micro}, with track-length estimators for the total collision rate and flux:

\begin{equation}
\label{eqn:chap3-nu-fiss-macro}
\nu\hat{\Sigma}_{f,k,g} = \frac{\langle \nu\Sigma_{f}, \psi \rangle_{k,g}^{t\ell}}{\langle \psi \rangle_{k,g}^{t\ell}}
\end{equation}


\subsubsection{Fission Energy Spectrum}
\label{subsubsec:chap3-tally-types-chi}

Unlike the fission production cross section, the fission spectrum is dependent on the outgoing neutron energy and must be computed with analog estimators. The fission production matrix from group $g'$ into group $g$ is given by the following inner product:

\begin{equation}
\label{eqn:chap3-nu-fiss-energies}
\langle \nu\Sigma_{f}, \psi \rangle_{k,g'\rightarrow g} = \int_{\mathbf{r} \in V_{k}} \int_{4\pi} \int_{E_{g}}^{E_{g-1}} \int_{E_{g'}}^{E_{g'-1}} \nu\Sigma_{f}(\mathbf{r},E'\rightarrow E)\psi(\mathbf{r},E,\mathbf{\Omega}) \mathrm{d}E'\mathrm{d}E\mathrm{d}\mathbf{\Omega}\mathrm{d}\mathbf{r}
\end{equation}

\noindent The fission spectrum in Eqn.~\ref{eqn:chap2-nusifg} can then be computed from this tally by summing over incoming and outgoing energy groups:

\begin{equation}
\label{eqn:chap3-chi}
\hat{\chi}_{k,g} = \frac{\displaystyle\sum\limits_{g'=1}^{G} \langle \nu\Sigma_{f}, \psi \rangle_{k,g'\rightarrow g}^{a}}{\displaystyle\sum\limits_{g=1}^{G} \displaystyle\sum\limits_{g'=1}^{G} \langle \nu\Sigma_{f}, \psi \rangle_{k,g'\rightarrow g}^{a}}
\end{equation}

\noindent This expression for the fission spectrum will result in a normalized discrete probability distribution for the energy of neutrons emitted from fission.


\subsubsection{Summary}
\label{subsubsec:chap3-tally-types-summary}

The tallies needed to generate \ac{MGXS} libraries were outlined in detail in the preceding sections, and are summarized in Table~\ref{table:chap3-tally-types}. The scores and filters correspond to the notation used by the OpenMC code to describe the scoring function and integration bounds used in Eqn.~\ref{eqn:chap3-tallies-general}. 
The energy group structure for energy condensation is specified by \texttt{energy} and/or \texttt{energyout} filters in the table. The regions for spatial homogenization are specified by \texttt{material} or \texttt{cell} filters, although this could potentially include \texttt{universe}, \texttt{distribcell} and \texttt{mesh} filters as well.

%\renewcommand{\arraystretch}{1.5}

\begin{table}[h!]
  \centering
  \caption[Tally types for \ac{MGXS} generation]{The types of tallies used in \ac{MGXS} generation with OpenMC.}
  \scriptsize
  \label{table:chap3-tally-types}
  \vspace{6pt}
  \begin{tabular}{ m{1.3cm} m{1cm} m{2cm} m{2.5cm} m{2.5cm} m{1.5cm} }
  \toprule
  {\bf Name} &
  {\bf Symbol} &
  {\bf Tally} &
  {\bf Score} &
  {\bf Filters} &
  {\bf Estimator} \\

  \specialrule{.2em}{.1em}{.1em}

  \multirow{2}{*}[-0.7em]{\bf General} & \multirow{2}{*}[-0.7em]{$\hat{\Sigma}_{x,k,g}$} & $\langle \Sigma_{x}, \psi \rangle_{k,g}$ & reaction $x$ & \parbox{2cm}{\texttt{material}/\texttt{cell} \texttt{energy}} & \texttt{track-length} \\
  \cline{3-6}
  & & $\langle \psi \rangle_{k,g}$ & {\texttt{flux}} & \parbox{2cm}{\texttt{material}/\texttt{cell} \texttt{energy}} & \texttt{track-length} \\

  \specialrule{.2em}{.1em}{.1em}

  \multirow{2}{*}[-0.7em]{\bf Total} & \multirow{2}{*}[-0.7em]{$\hat{\Sigma}_{t,k,g}$} & $\langle \Sigma_{t}, \psi \rangle_{k,g}$ & \texttt{total} & \parbox{2cm}{\texttt{material}/\texttt{cell} \texttt{energy}} & \texttt{track-length} \\
  \cline{3-6}
  & & $\langle \psi \rangle_{k,g}$ & \texttt{flux} & \parbox{2cm}{\texttt{material}/\texttt{cell} \texttt{energy}} & \texttt{track-length} \\

  \specialrule{.2em}{.1em}{.1em}

  \multirow{3}{*}[-1em]{\parbox{1.5cm}{\bf Transport-Corrected Total}} & \multirow{3}{*}[-1em]{$\hat{\tilde{\Sigma}}_{t,k,g}$} & $\langle \Sigma_{t}, \psi \rangle_{k,g}$ & \texttt{total} & \parbox{2cm}{\texttt{material}/\texttt{cell} \texttt{energy}} & \texttt{analog} \\
  \cline{3-6}
  & & $\langle \Sigma_{s1}, \psi \rangle_{k,g'\rightarrow g}$ & \texttt{nu-scatter-1} & \parbox{2cm}{\texttt{material}/\texttt{cell} \texttt{energyout}} & \texttt{analog} \\
  \cline{3-6}
  & & $\langle \psi \rangle_{k,g}$ & \texttt{flux} & \parbox{2cm}{ \texttt{material}/\texttt{cell} \texttt{energy}} & \texttt{analog} \\

  \specialrule{.2em}{.1em}{.1em}

  \multirow{2}{*}[-0.5em]{\parbox{1.5cm}{\bf Scattering Matrix}} & \multirow{2}{*}[-0.5em]{$\hat{\Sigma}_{s,k,g'\rightarrow g}$} & $\langle \Sigma_{s0}, \psi \rangle_{k,g'\rightarrow g}$ & \texttt{nu-scatter-0} & \parbox{2cm}{\texttt{material}/\texttt{cell} \texttt{energy} \texttt{energyout}} & \texttt{analog} \\
  \cline{3-6}
  & & $\langle \psi \rangle_{k,g}$ & \texttt{flux} & \parbox{2cm}{\texttt{material}/\texttt{cell} \texttt{energy}} & \texttt{analog} \\

  \specialrule{.2em}{.2em}{.2em}

  \multirow{3}{*}[-1em]{\parbox{1.5cm}{\bf Transport-Corrected Scattering Matrix}} & \multirow{3}{*}[-1em]{$\hat{\tilde{\Sigma}}_{s,k,g'\rightarrow g}$} & $\langle \Sigma_{s0}, \psi \rangle_{k,g'\rightarrow g}$ & \texttt{nu-scatter-0} & \parbox{2cm}{\texttt{material}/\texttt{cell} \texttt{energy} \texttt{energyout}} & \texttt{analog} \\
  \cline{3-6}
  & & $\langle \Sigma_{s1}, \psi \rangle_{k,g'\rightarrow g}$ & \texttt{nu-scatter-1} & \parbox{2cm}{\texttt{material}/\texttt{cell} \texttt{energyout}} & \texttt{analog} \\
  \cline{3-6}
  & & $\langle \psi \rangle_{k,g}$ & \texttt{flux} & \parbox{2cm}{\texttt{material}/\texttt{cell} \texttt{energy}} & \texttt{analog} \\

  \specialrule{.2em}{.1em}{.1em}

  \multirow{2}{*}[-0.5em]{\parbox{1.5cm}{\bf Fission \hspace{1cm} Production}} & \multirow{2}{*}[-0.5em]{$\nu\hat{\Sigma}_{f,k,g}$} & $\langle \nu\Sigma_{f}, \psi \rangle_{k,g}$ & \texttt{nu-fission} & \parbox{2cm}{\texttt{material}/\texttt{cell} \texttt{energy}} & \texttt{track-length} \\
  \cline{3-6}
  & & $\langle \psi \rangle_{k,g}$ & \texttt{flux} & \parbox{2cm}{\texttt{material}/\texttt{cell} \texttt{energy}} & \texttt{track-length} \\

  \specialrule{.2em}{.1em}{.1em}
  
  \parbox{1.5cm}{\bf Fission Spectrum} & $\hat{\chi}_{k,g}$ & $\langle \nu\Sigma_{f}, \psi \rangle_{k,g'\rightarrow g}$ & \texttt{nu-fission} & \parbox{2cm}{\texttt{material}/\texttt{cell} \texttt{energy} \texttt{energyout}} & \texttt{analog} \\
  \midrule

\end{tabular}
\end{table}


%%%%%%%%%%%%%%%%%%%%%%%%%%%%%%%%%%%%
\subsection{Uncertainty Propagation}
\label{subsec:chap3-uncertainty-prop}

As discussed in the preceding sections, \ac{MGXS} may be computed using arithmetic combinations of tally estimators for reaction rates and fluxes. Each tally estimator is a random variable with an associated uncertainty estimated by the variance of the sample mean in Eqn.~\ref{eqn:chap3-variance-mean}. As a result, each multi-group cross section computed for a spatial zone and energy group is itself a random variable from a distribution with some unknown variance. It is therefore useful to estimate the uncertainty of \ac{MGXS} computed from \ac{MC} tallies in order to quantify whether the \ac{MGXS} are known with enough precision for accurate multi-group calculations. 

Estimates of the variance may be deduced from standard error propagation theory. Such analysis is widely discussed in the literature~\cite{bevington2003data}. A few key equations necessary to estimate the variance for \ac{MGXS} are reproduced here. The arithmetic combinations of interest for \ac{MGXS} generation include addition, subtraction, multiplication and division. 

Consider two random variables $X$ and $Y$, generated from distributions with variances $\sigma_{X}^2$ and $\sigma_{Y}^2$ which are arithmetically combined into a new random variable $Z$ with variance $\sigma_{Z}^2$. The random variables $X$ and $Y$ may correspond to tallies for reaction rates and the flux, while $Z$ could correspond to a \ac{MGXS}. The following expressions can be derived for the variance $\sigma_{Z}^{2}$ for binary combinations of $X$ and $Y$:

\vspace{-0.4in}

\begin{align*}
Z &= X + Y & \sigma_{Z}^{2} &= \sigma_{X}^{2} + \sigma_{Y}^{2} + 2\sigma_{XY} \numberthis \label{eqn:chap3-add} \\
Z &= X - Y & \sigma_{Z}^{2} &= \sigma_{X}^{2} + \sigma_{Y}^{2} - 2\sigma_{XY} \numberthis \label{eqn:chap3-sub} \\
Z &= XY & \sigma_{Z}^{2} &\approx Z^{2}\left[\left(\frac{\sigma_{X}}{X}\right)^{2} + \left(\frac{\sigma_{Y}}{Y}\right)^{2} + 2\frac{\sigma_{XY}}{Z}\right] \numberthis \label{eqn:chap3-mult} \\
Z &= \frac{X}{Y} & \sigma_{Z}^{2} &\approx Z^{2}\left[\left(\frac{\sigma_{X}}{X}\right)^{2} + \left(\frac{\sigma_{Y}}{Y}\right)^{2} - 2\frac{\sigma_{XY}}{Z}\right] \numberthis \label{eqn:chap3-div} \\
\end{align*}

\vspace{-0.4in}

\noindent These expressions are given in terms of the covariance $\sigma_{XY}$ of $X$ and $Y$:

\vspace{-0.1in}

\begin{equation}
\label{eqn:chap3-covariance}
\sigma_{XY} = \mathbb{E}[(X - \mathbb{E}[X])(Y - \mathbb{E}[Y])]
\end{equation}

\noindent where $\mathbb{E}[\cdot]$ is the expectation operator. The covariance is not generally computable using the standard formulation for a tally estimator in a Monte Carlo simulation. Although it would be possible to estimate the covariance using ensemble statistics\footnote{The covariance could be estimated from the results of an ensemble of independent \ac{MC} simulations.}, this is not often feasible. Instead, the covariance terms in Eqns.~\Crefrange{eqn:chap3-add}{eqn:chap3-div} are typically neglected. In general, the random variables for reaction rates and fluxes in the same volume of phase space are highly correlated, and neglecting the covariance leads to a poor approximation for the variance of \ac{MGXS}. However, it should be noted that division is the primary operation needed to combine tallies to compute \ac{MGXS}. Since the reaction rates and flux tallies must be positively correlated, the covariance term in Eqn.~\ref{eqn:chap3-div} reduces the estimate of the covariance. It therefore follows that a conservative estimate of the variance for \ac{MGXS} is obtained by neglecting the covariance.

\begin{emphbox}
\textbf{A mixture of analog and track-length \ac{MC} tallies for reaction rates and fluxes are used to generate spatially-homogenized and energy condensed \ac{MGXS}. Error propagation theory is used to estimate the \ac{MGXS} uncertainties.}
\end{emphbox}


%%%%%%%%%%%%%%%%%%%%%%%%%%%%%%%%%%%%%%%%%%%%%%%%%%%%%%%%%%%%%%%%%%%%%%%%%%%%%%%
\section{A Literature Review of MGXS Generation with MC}
\label{sec:chap3-lit-review}

The last two decades have seen growing interest in Monte Carlo as a means to generate \ac{MGXS} libraries. This section presents a brief overview of the literature which documents these efforts. As discussed in Sec.~\ref{subsec:chap3-lit-review-diffusion}, most of the work to date has been directed at generating homogenized few-group constants for coarse mesh diffusion-based codes. Sec.~\ref{subsec:chap3-lit-review-transport} reviews a few recent theses which develop \ac{MC}-based methods to generate \ac{MGXS} for fine-mesh transport-based simulations, which is the motivation for this thesis.


\subsection{MGXS for Coarse Mesh Diffusion Calculations}
\label{subsec:chap3-lit-review-diffusion}

Most \ac{MC}-based \ac{MGXS} generation schemes to date focus on generating few-group constants for coarse mesh diffusion codes\footnote{In this context, \textit{coarse mesh} refers to the use of one or a few homogenized mesh cells per assembly.}. These schemes aim to improve the accuracy of standard diffusion codes for analysis of atypical core configurations for which the simplifications made by multi-level deterministic \ac{MGXS} generation methods are not necessarily applicable. These efforts replace the separate resonance self-shielding and deterministic lattice physics calculation steps in multi-level approaches (see Sec.~\ref{subsec:chap2-mgxs-lib-std-approach}) with fully-detailed \ac{MC} calculations of each assembly to compute the few-group constants needed by whole core diffusion codes. The widely used Serpent code discussed in Sec.~\ref{subsec:chap3-lit-review-diffusion-serpent} has led this trend over the last decade, and a few authors have applied the MCNP and McCARD codes in a similar fashion as will be highlighted in Secs.~\ref{subsec:chap3-lit-review-diffusion-mcnp} and \ref{subsec:chap3-lit-review-diffusion-mccard}. The latter subsections summarize studies which used the MC21, MVP-BURN, RCP01 and VIM codes to generate \ac{MGXS} for coarse mesh diffusion calculations.

\subsubsection{Serpent}
\label{subsec:chap3-lit-review-diffusion-serpent}

Serpent is a continuous energy Monte Carlo code developed by the VTT Technical Research Centre of Finland, and is one of the most widely used \ac{MC} particle transport codes in the world~\cite{serpent2013manual}. Serpent was initially created as part of Lepp{\"a}nen's thesis~\cite{leppanen2007serpent} to generate few-group constants for nodal diffusion codes. More recently, Serpent has developed into a general purpose reactor physics burnup code with features including isotopic depletion, on-the-fly Doppler broadening and support for CAD geometries.

Serpent is designed to be a drop-in replacement for deterministic resonance self-shielding and lattice physics calculation codes and can generate homogenized few-group constants for coarse mesh sub-assembly geometries. Serpent is uniquely designed for coarse mesh \ac{MGXS} generation since it uses the Woodcock-Delta method~\cite{woodcock1965techniques} to greatly reduce the computational expense of tracking particles in geometries with complicated surface crossings. Unlike deterministic multi-level approaches, Serpent uses \ac{MC} to precisely model self-shielding effects in complicated geometries. In addition, Serpent simplifies the validation of downstream diffusion codes which use the \ac{MGXS} it generates since it is also capable of computing full-core reference solutions.

One of the challenges for lattice physics calculations is the appropriate treatment of net current between assemblies. In order to address this, Serpent employs a two-step energy condensation scheme including an infinite lattice calculation followed by a B$_{1}$ leakage correction~\cite{fridman2011serpent}. The scheme first performs an infinite lattice calculation for each unique assembly and tallies \ac{MGXS} in the WIMS 69-group structure. The \ac{MGXS} are used to form the B$_{1}$ equations for the homogenized system which are solved for the critical flux spectrum accounting for inter-assembly leakage. The critical flux is finally used to collapse the 69-group \ac{MGXS} into few-group constants. Although the B$_{1}$ equations make assumptions that are not always true -- such as an energy-independent buckling -- the approach has been demonstrated to largely resolve 20\% errors in 2-group diffusion coefficients compared to those collapsed with the infinite flux for simple \ac{PWR} benchmarks~\cite{fridman2011serpent}. However, the B$_{1}$ leakage correction does have some known deficiencies, including its inability to treat non-fissile regions such as homogenized reflectors or transmutation cross sections in burnup calculations~\cite{leppanen2016overview}.

Much of the research focus for Serpent has revolved around the need to accurately compute diffusion coefficients. Unlike the total, scattering, and fission production \ac{MGXS} discussed in Chap.~\ref{chap:mgxs}, diffusion coefficients do not have a continuous energy counterpart in transport theory. Serpent uses an approximate method to compute the Selengut-Goertzel diffusion coefficient from the inverse of the transport cross section\footnote{This is an approximate method since the energy collapse of the transport cross section is not equivalent to the energy collapse of its inverse, which is what is needed to collapse the diffusion coefficient.}. In particular, Serpent tallies a 69-group \ac{MGXS} transport cross section, and condenses its inverse with the B$_{1}$ leakage corrected spectra to compute diffusion coefficients. In addition, since it is not straightforward to compute current-weighted tallies in \ac{MC}, Serpent makes a heuristic approximation and uses a scalar flux-weighted scattering moment to compute the transport cross section (see Sec.~\ref{subsec:chap2-transport-corr}). More recently, the Serpent team has investigated a novel current-weighted scheme for directional diffusion coefficients~\cite{dorval2015diff}, and introduced Liu's novel Cumulative Migration Method (CMM)~\cite{liuphysor2016} to mitigate shortcomings in the current approach~\cite{leppanen2016overview}.

Although Serpent is the most popular tool for \ac{MGXS} generation, its use thus far has been specifically focused on coarse mesh diffusion applications. To the first author's knowledge, there have not yet been any published works which use Serpent to generate \ac{MGXS} for fine-mesh transport calculations.

%In addition, as discussed in Sec.~\ref{subsec:chap2-transport-corr}, it is not is not a straighforward exercise to compute the transport cross section from Monte Carlo since it requires a first order current-weighted scattering moment matrix. Instead, Serpent makes a heuristic approximation and uses a scalar flux-weighted scattering moment to compute the transport cross section. 

\subsubsection{MCNP}
\label{subsec:chap3-lit-review-diffusion-mcnp}

The widely used MCNP code~\cite{mcnpx2003manual} developed by Los Alamos National Laboratory has also been employed to generate few-group \ac{MGXS} for coarse mesh diffusion methods. Perhaps the most comprehensive study can be found in Pounders' thesis~\cite{pounders2006stochastically} which, like Serpent, aimed to improve the accuracy of standard diffusion codes by generating improved diffusion coefficients with Monte Carlo. In addition to the reactor problems considered in Pounders' thesis, MCNP has also been used to generate diffusion coefficients for analysis of spent fuel storage lattices~\cite{ilas2003monte}.

Pounders considered several approaches to compute diffusion coefficients with Monte Carlo and implemented each in MCNP. The simplest method computed the diffusion coefficient as the inverse of the transport cross section. This formulation is most similar to that used in Serpent, but did not include a fine-to-few-group collapse with the B$_{1}$ leakage corrected spectra. Pounders also considered the flux-limited diffusion coefficient~\cite{pomraning1984flux} widely used in radiation hydrodynamics codes. In addition, Pounders developed an approach to use directional neutron currents to appropriately weight the first order scattering moment needed to compute the transport cross section. Lastly, Pounders' most novel contribution, termed the ``stochastic diffusion coefficient,'' attempted to embed higher order anisotropies in the diffusion coefficient by combining direction-dependent diffusion coefficients with volume-integrated flux gradient tallies.

Pounders' results demonstrated that the stochastic diffusion coefficient produced the best results since it mitigated the approximations imposed by P$1$ theory. However, the analysis was limited to simple 1D slab and 2D pin cell problems since the implementation of stochastic diffusion coefficients for arbitrary 3D reactor geometries would be challenging. Most interestingly, Pounders' thesis concluded that the error resulting from diffusion theory's assumption of a linearly anisotropic flux dominated the error induced by each of the diffusion coefficient formulations. Pounders' thesis has served as the basis for the development of new methods to generate \ac{MGXS} with MCNP, including new diffusion coefficient formulations~\cite{pounders2009diffusion} and an approach to represent the Legendre expansion of the scattering kernel with equiprobable cosine bins~\cite{pounders2015history}.

A few recent studies have evaluated new leakage models for multi-group diffusion coefficient generation with MCNP. Yun and Cho evaluated a hybrid approach to compute diffusion coefficients with MCNP using an albedo-corrected leakage spectrum~\cite{cho2009generation,yun2010monte}. This approach iterated between a Monte Carlo lattice and deterministic full-core calculations to converge nodal diffusion parameters to preserve surface currents to match a criterion defined by any arbitrary leakage model. Cho evaluated the method for a \ac{PWR} fuel assembly and demonstrated a slight improvement in the solution compared to a diffusion calculation with \ac{MGXS} weighted by an infinite spectrum.

Yamamoto~\cite{yamamoto2012buckling} developed a method to model neutron leakage with a correction term in the transport equation which is equivalent to the B$_{1}$ leakage correction method. The correction term was introduced in MCNP as a complex-valued particle weight in order to generate anisotropic diffusion coefficients~\cite{yamamoto2012diff}. Few-group diffusion coefficients for pin cell and assembly benchmarks exhibited good agreement with those generated by a reference deterministic code, but no criticality calculations were performed to validate eigenvalues and power distributions computed by a diffusion code.

\subsubsection{McCARD}
\label{subsec:chap3-lit-review-diffusion-mccard}

The McCARD Monte Carlo transport code developed by Seoul National University has been utilized in a number of studies to generate few-group constants for full-core diffusion analysis~\cite{shim2008generation, park2010assembly, park2012generation}. The over-arching objective in these studies was to develop and evaluate a method to model the leakage spectrum in \ac{MC} lattice physics calculations. An approach was independently developed which used the solution to the homogeneous B$_{1}$ equations which is equivalent to the methodology used in the Serpent code~\cite{fridman2011serpent}. In summary, McCARD tallied fine group \ac{MGXS} and used the critical spectrum solved from the B$_{1}$ equations to collapse the fine group \ac{MGXS} into few-group constants. Like Serpent, McCARD uses the flux rather than the current to weight the first order scattering moment tallies to approximately compute fine group transport cross sections. The studies published in the literature validated the \ac{MGXS} generated with McCARD with a reference deterministic lattice code. In addition, good agreement was demonstrated between reference \ac{MC} and deterministic diffusion solutions for the eigenvalues and power distributions for 3D \ac{PWR} and gas-cooled, TRistructural-ISOtopic (TRISO) fueled Very High Temperature Reactor (VHTR) benchmark configurations. Finally, the B$_{1}$ leakage corrected spectra was shown to significantly improve the eigenvalue estimates and assembly-wise power distributions with respect to \ac{MGXS} computed from an infinite lattice spectrum.

\subsubsection{MC21}
\label{subsec:chap3-lit-review-diffusion-mc21}

Herman utilized the MC21 code developed by Knolls Atomic Power Laboratory to evaluate two approximations used in Monte Carlo calculations of diffusion coefficients~\cite{herman2013improved}. His work was motivated by approximations made by the multi-step energy condensation methodology used to compute diffusion coefficients in Serpent with a B$_{1}$ leakage corrected spectra. First, Herman considered whether to energy condense the fine group transport cross sections or diffusion coefficients when computing few-group diffusion coefficients. Second, he presented an approach to use a pre-tabulated form of the energy-dependent scattering cosine to correct the traditional form of the diffusion coefficient. His results demonstrated a significant reduction from more than 3\% to nearly 0.25\% in the reconstructed pin powers for a \ac{PWR} assembly. Herman's corrected diffusion coefficient was implemented in the OpenMC code~\cite{romano2013openmc} for use in \ac{CMFD} acceleration~\cite{herman2014monte}, and has inspired the development of the Cumulative Migration Method (CMM) to calculate diffusion coefficients with Monte Carlo~\cite{liuphysor2016}.

\subsubsection{MVP-BURN}
\label{subsec:chap3-lit-review-diffusion-mvp-burn}

Tohjoh used the MVP-BURN Monte Carlo code~\cite{okumura2000validation} to generate few-group constants for full core \ac{BWR} analysis~\cite{tohjoh2005application}. The objective of the study was to evaluate \ac{MC} as an approach to generate 3-group constants for advanced \ac{BWR} assemblies with novel geometries and reactivity controls. Although MVP-BURN could be used to tally reaction rates and fluxes to compute standard \ac{MGXS} (\textit{e.g.}, total, fission), it was not modified to directly generate diffusion coefficients. Unlike past work with Serpent and MCNP, the diffusion coefficient was computed with MVP-BURN by collapsing the total and scattering cross sections in energy, rather than the transport cross section or the diffusion coefficient itself. In particular, the diffusion coefficient was computed from tallied total and scattering cross sections and a simple heuristic for the average scattering cosine. Furthermore, the code system was unable to produce scattering matrices. Instead, up-scattering was assumed to be negligible and the downscattering cross sections were computed directly from the removal and capture \ac{MGXS}. The group constants were validated with those produced from a deterministic lattice code, and a full-core nodal diffusion burnup calculation was performed to validate the eigenvalues and power distributions with reference solutions.

\subsubsection{RCP01}
\label{subsec:chap3-lit-review-diffusion-rcp01}

Gast authored an early study which analyzed a variety of deterministic formulations of the diffusion coefficient to determine the most relevant candidate(s) for Monte Carlo~\cite{gast1981procedure}. Gast concluded that most formulations were not practically realizable with \ac{MC} due to their reliance on spatially integrated flux gradients and current densities. Instead, he deduced that the Selengut-Goertzel diffusion coefficient based on the transport cross section was best suited for computation with \ac{MC}, and implemented it in the RCP01 \ac{MC} code~\cite{ondis2000rcp01} developed by Bettis Atomic Power Laboratory. The implementation applied an empirical correction factor to the final computed diffusion coefficient to address the approximation made by using flux-weighted rather than current-weighted tallies. The first author could not find any subsequent analyses in the publicly available literature which validate the implementation for coarse mesh diffusion calculations.

\subsubsection{VIM}
\label{subsec:chap3-lit-review-diffusion-vim}

The VIM continuous energy Monte Carlo code is developed by Argonne National Laboratory~\cite{blomquist2002status} for neutron and photon transport calculations. VIM is capable of generating isotopic macroscopic or microscopic \ac{MGXS}, including group-to-group scattering matrices, for deterministic applications. The code is capable of producing diffusion coefficients derived from tallied multi-group total and scattering cross sections. A data processing tool may be used to convert the computed \ac{MGXS} into the ISOTXS file format accepted by ANL's deterministic diffusion and transport codes. The first author could not find any analyses in the publicly available literature which validate the \ac{MGXS} generated by VIM for coarse mesh diffusion calculations.

\subsection{MGXS for Fine-Mesh Transport Calculations}
\label{subsec:chap3-lit-review-transport}

\subsubsection{MCNP}
\label{subsec:chap3-lit-review-transport-mcnp}

Redmond~\cite{redmond1997multigroup} performed one of the earliest known in-depth analyses of \ac{MGXS} generation with \ac{MC} methods. Redmond studied two methods to compute group-to-group scattering moment matrices with the MCNP code~\cite{mcnpx2003manual}. The first approach -- known as the \textit{direct method} -- is equivalent to the analog estimator for computing scattering matrices presented in Sec.~\ref{subsubsec:chap3-tally-types-scatt-mat}. The second approach -- termed the \textit{explicit method} -- imposes a finite sampling of possible outgoing energies and angles for each scattering and fission event. The explicit method may thereby improve the sampling statistics for the scattering cross section and the fission spectrum. To the first author's knowledge, the explicit method is not implemented in any presently available production code nor are there results beyond Redmond's work which evaluate the sampling efficiency of the method.

More recently, Van der Marck~\cite{van2006homogenized} applied Redmond's methods in MCNP to generate \ac{MGXS} to model the Petten High Flux Reactor (HFR) in the Netherlands. Van der Marck was concerned with the slow computational performance of computing \ac{MGXS} with MCNP as a result of using track-length tallies. Although track-length tallies are an efficient statistical estimator, they require scoring to each tally every time a neutron travels between material zones, which can be exceedingly slow in MCNP. Instead, Van der Marck created a downstream data processing tool called ELNINJO to analyze MCNP binary files with collision data to compute collision estimators for \ac{MGXS} generation. The ELNINJO code was developed to generate either \ac{MGXS} for diffusion or transport theory codes, and was validated for a few simple 1D and 2D test cases.

Hoogenboom employed a similar downstream data processing approach in a tool to parse MCNP's PTRAC event file to compute multi-group scattering matrices~\cite{hoogenboom2007generation}. The tool permitted the computation of multi-group scattering matrices within arbitrary geometric regions with no changes necessary to the MCNP source code or input files. The tool was used to generate 2-group constants in a simple 1D slab geometry which were then validated with respect to those generated with the deterministic SCALE code system~\cite{bucholz1982scale}. There are no known published results to validate the efficacy of the \ac{MGXS} generated from the tool in a downstream fixed source or eigenvalue calculation code.

Most recently, Yoshioka evaluated a methodology for \ac{MGXS} generation with MCNP for deterministic diffusion or transport methods~\cite{yoshioka2010multigroup, yoshioka2011multi}. Yoshioka developed a \textit{weight-to-flux ratio} method to compute scattering matrices as an extension to an earlier study by Tohjoh with the MVP-BURN code~\cite{tohjoh2005application}. The weight-to-flux ratio scheme was derived as a simple tally scheme for 3-group down-scatter cross sections and neglected up-scattering and self-scattering. The \ac{MGXS} generated with MCNP were validated with respect to those computed using a reference deterministic lattice code. In addition, steady-state, burnup and transient calculations were performed to validate eigenvalues, reactivity margins, Minimum Critical Power Ratio (MCPR), and power responses for a variety of \ac{BWR} benchmark configurations. Interestingly, the \ac{MGXS} generated by MCNP led to a negative bias of 300--600 \ac{pcm} in deterministic transport calculations for a \ac{BWR} fuel pin cell model. Although the source of this bias was not identified, the results presented in Chap.~\ref{chap:biases} indicate that it may have been due in part to the flux separability approximation (see Sec.~\ref{subsec:chap2-angle}).

\subsubsection{TRIPOLI-4}
\label{subsec:chap3-lit-review-transport-tripoli}

Cai~\cite{cai2014condensation} investigated the use of the TRIPOLI-4 code to generate \ac{MGXS} with continuous energy Monte Carlo simulations. Unlike other published works, Cai validated the \ac{MGXS} using the multi-group Monte Carlo solver in TRIPOLI-4 rather than a deterministic multi-group solver. Her thesis primarily focused on methods to enforce neutron balance between reference continuous energy and multi-group Monte Carlo calculations. Cai developed a method termed the ``In-Group Scattering Correction'' (IGSC) as a means to mitigate the flux separability approximation discussed in Sec.~\ref{subsec:chap2-angle}. The IGSC technique employed volumetric neutron current-weighted moments of the total cross section as correction terms to the diagonal entries in the scattering moment matrices. 

%Cai evaluated two approximations to the neutron current with their corresponding \ac{MC} estimators for a variety of homogenized 2D and 3D fast reactor benchmarks. Although the IGSC method appeared to improve the consistency between continuous energy and multi-group \ac{MC} results for some cases, it led to a several thousand \ac{pcm} eigenvalue bias for a few of the benchmarks. Furthermore, Cai's results demonstrated the of approximating volumetric currents within a heterogeneous geometry with strongly varying material properties. Finally, Cai's methodology 

Cai explored IGSC with a \ac{MC} estimator of Todorova's current approximation for a variety of homogenized 2D and 3D fast reactor benchmarks. Although the IGSC method improved the consistency between continuous energy and multi-group \ac{MC} results for some cases, it led to a several thousand \ac{pcm} eigenvalue bias for a few of the benchmarks. Furthermore, Cai's results demonstrated the deficiency of using Todorova's current approximation in a heterogeneous geometry with strongly varying material properties. Cai developed and evaluated an alternative approximation to the current, termed the ``Direction-X'' current, to eliminate the approximations made in Todorova's current -- namely, that the gradient of the flux is similar to the flux spectrum itself. Although the ``Direction-X'' current greatly improved the results with respect to Todorova's current, its implementation and analysis was limited to 1D slab geometries.

\subsubsection{OpenMC}
\label{subsec:chap3-lit-review-transport-openmc}

Nelson~\cite{nelson2014improved} identified the convergence rate of scattering moments to be a key bottleneck to computing \ac{MGXS} libraries, and developed an approach to mitigate this with deterministic tallies of the energy and angle for outgoing neutrons in collisions. Nelson's methodology permits track-length estimators for tallies which depend on the outgoing neutron energy, and scores to all outgoing energy groups for each particle trajectory. This is advantageous since it greatly improves the tallying efficiency and convergence rate of scattering moment matrices and the fission energy spectrum. However, Nelson's methodology requires a computationally expensive pre-processing step to transform each nuclide's nuclear data for use in an \ac{MC} simulation. Furthermore, the pre-processing step is specific to the energy group structure used in an \ac{MGXS} library, and the computational expense and memory footprint grow as $\mathcal{O}(NG^{2})$ with the number of groups $G$ and nuclides $N$ in a simulation. Although Nelson implemented the scheme in a developmental version of OpenMC, it was not utilized by the first author in this thesis work.

\subsection{Concluding Remarks}
\label{subsec:chap3-lit-review-conclusion}

A large body of recent work has aimed to replace resonance scattering and lattice physics codes to instead compute \ac{MGXS} with Monte Carlo for coarse mesh diffusion codes. These studies have primarily explored various formulations to tally diffusion coefficients, and various models to account for leakage in assembly homogenized few-group constants. In comparison, relatively less consideration has been given to \ac{MGXS} generation for fine-mesh transport calculations. These efforts have focused on improving the statistical efficiency of \ac{MC} tallies for scattering moment matrices and the fission spectrum, and alternatives to the flux separability approximation for more accurate total cross sections. 

The work to date has demonstrated the promise for \ac{MC} to replace many of the complicated steps and approximations made in the traditional multi-level framework for \ac{MGXS} generation. However, past studies have typically generated \ac{MGXS} for subsets of a reactor geometry -- such as individuals fuel assemblies -- for subsequent use in multi-group full-core analysis. This thesis aims to build upon the progress made in this area by investigating the opportunity to directly use full-core continuous energy \ac{MC} simulations to compute \ac{MGXS} for full-core deterministic multi-group calculations. The subsequent chapters of this thesis develop and evaluate a novel methodology for statistically efficient spatial homogenization which accounts for self-shielding effects in \ac{MGXS} tallied on a fine spatial mesh.

%the intrinsic bias in \ac{MGXS} generated with \ac{MC} for transport calculations of \ac{PWRs}. The latter chapters in this thesis develop a new methodology for spatial homogenization which accounts for spatial self-shielding effects in \ac{MGXS} tallied on a fine spatial mesh.

%-this is more relevant to this thesis since fine-mesh transport calculations
%-utilized multi-group monte carlo or deterministic method of characteristics
%-differentiate myself from the rest: the work to date has validated MGXS generated for small benchmarks, including fuel pins, single assemblies, and assembly homogenized geometries

\vfill
\begin{highlightsbox}[frametitle=Highlights]
\begin{itemize}
  \item \ac{MGXS} are computed from a mixture of analog and track-length Monte Carlo reaction rate and flux tally estimators. Error propagation theory is used to estimate the variance of \ac{MGXS} computed from \ac{MC} tallies.
  \item \ac{MC} is increasingly used to generate few-group constants for coarse mesh diffusion calculations with tools such as the Serpent \ac{MC} code.
  \item \ac{MC}-based \ac{MGXS} generation methods to date rely on a multi-level framework similar to that employed in conventional deterministic methods. These approaches generate \ac{MGXS} for sub-assembly geometries with infinite boundary conditions to generate few-group constants for full-core analysis.
  \item Multi-level methods suffer from approximations used to treat neutron leakage and spatial heterogeneities such as neutron reflectors.  
  \item Less attention has been directed to \ac{MC}-based \ac{MGXS} generation for fine-mesh transport calculations -- the focus of this thesis.
\end{itemize}
\end{highlightsbox}
\vfill