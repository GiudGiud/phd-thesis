\chapter{Benchmark Models and Reference Results}
\label{chap:benchmarks}

%%%%%%%%%%%%%%%%%%%%%%%%%%%%%%%%%%%%%%%%%%%%%%%%%%%%%%%%%%%%%%%%%%%%%%%%%%%%%%%
\section{Motivation}
\label{sec:chap7-motivate}

first paragraph: part iv -- clustering of mgxs
-tie back to part iii where we looked at intra-pin effects
  -want to match CE MC as closely as possible with deterministic MG theory
  -but in this part we are looking at larger geometries
  -spatial self-shielding effects on a macro scale (e.g., inter-pin, inter-assm)
-heterogeneous geometries needed to explore this
  -this chapter presents heterogeneous geoetry benchmark specs and reference results
  -benchmarks and reference results used throughout the subsequent chapters in part iv

second paragraph: heterogeneous benchmarks
-constructed different benchmarks from \ac{BEAVRS}
-each was intended to change one major variable to explore spatial self-shielding effects
  -understand spatial self-shielding effects in pin-wise MGXS
-single assembly
  -fuel enrichment, \ac{BP}s, 
-multiple assembly colorsets
  -inter-assembly effects 
  -water reflector
-full \ac{BEAVRS} core
  -baffle, barrel, vessel

third paragraph: metrics of interest
-source convergence with shannon entropy
  -used to determine t
-converged eigenvalues
-Pin-wise fission rates
-Pin-wise U-238 capture rates

fourth paragraph: outline chapter
-Sec.~\ref{sec:chap7-benchmarks} - outlines benchmark specifications
  -taken from \ac{BEAVRS}
  -geometric parameters
  -isotopics
-Sec.~\ref{sec:chap7-ref-results}
  -details reference results computed with OpenMC
  -source convergence with shannon entropy
  -converged eigenvalues
  -Pin-wise fission rates
  -Pin-wise U-238 capture rates


%%%%%%%%%%%%%%%%%%%%%%%%%%%%%%%%%%%%%%%%%%%%%%%%%%%%%%%%%%%%%%%%%%%%%%%%%%%%%%%
\section{Benchmark Configurations}
\label{sec:chap7-benchmarks}

first paragraph: refer to the \ac{BEAVRS} model~\cite{horelik2013beavrs}
-all heterogeneous geometries were subsets of the \ac{BEAVRS} geometry
-\ac{BEAVRS} consists of fuel assemblies with four different enrichments and various
-sliced at core mid-plane (axial level)
-2D rather than 3D
  -used reflective z-planes in openmc model
  -sliced to encompass the spacer grid
    -designed model to stress the geometric modeling capabilities of OpenMC/MOC/CG
    -grid spacers main axial heterogeneities that induce spatial self-shielding in \ac{LWR}s
    -stress geometric modeling capabilities since framework developed to eventually support 3D \ac{MOC}
-HZP

second paragraph: outline section
-Sec.~\ref{subsec:chap7-materials} materials and isotopic compositions
-Sec.~\ref{subsec:chap7-pin-cells} pin cell types (fuel pins, instr tubes, \ac{CRGT}s, \ac{BP}s
-Sec.~\ref{subsec:chap7-fuel-assms} fuel assembly types (varying enrichments, with and without \ac{BP}s)
-Sec.~\ref{subsec:chap7-2x2-colorsets} mixtures of assemblies (with and without water reflector)
-Sec.~\ref{subsec:chap7-full-core} full \ac{BEAVRS} core

%%%%%%%%%%%%%%%%%%%%%%%%%%%%
\subsection{Materials}
\label{subsec:chap7-materials}

first paragraph: types of materials
-see Table 1 in BEAVRS specifications
-isotopic compositions tabulated in App.~\ref{app:beavrs-isotopes}.
-1.6\% and 3.1\% enriched UO$_2$ fuel
-borated water (975 ppm boron)
-stainless steel (SS304)
-borosilicate glass
-zircaloy 4

%%%%%%%%%%%%%%%%%%%%%%%%%%%%
\subsection{Pin Cells}
\label{subsec:chap7-pin-cells}

first paragraph: pin cell figures
-walk through Fig.~\ref{fig:chap7-pin-cells}
-fuel pin, \ac{CRGT}, instrument tube, \ac{BP} above dashpot
-grid spacer thickness
-pin pitch
-blue for 3.1\% enriched pins

\begin{figure}[h!]
\centering
\begin{subfigure}{.5\textwidth}
  \centering
  \includegraphics[width=0.9\linewidth]{figures/benchmarks/fuel-pin-16}
  \caption{}
  \label{fig:chap7-pin-1.6}
\end{subfigure}%
\begin{subfigure}{.5\textwidth}
  \centering
  \includegraphics[width=0.9\linewidth]{figures/benchmarks/guide-tube}
  \caption{}
  \label{fig:chap7-pin-3.1}
\end{subfigure}
\begin{subfigure}{.5\textwidth}
  \centering
  \includegraphics[width=0.9\linewidth]{figures/benchmarks/instr-tube}
  \caption{}
  \label{fig:chap7-guide-tube}
\end{subfigure}%
\begin{subfigure}{.5\textwidth}
  \centering
  \includegraphics[width=0.9\linewidth]{figures/benchmarks/burn-abs}
  \caption{}
  \label{fig:chap7-instr-tube}
\end{subfigure}%
\caption[BEAVRS pin cell geometries]{1.6\% enriched fuel pin (a), control rod guide tube (b), instrument tube (c) and burnable poison (d). Light blue is borated water, red is UO$_2$ fuel, gray is zircaloy, brown is helium, white is air, green is borosolicate glass, and black is stainless steel.}
\label{fig:chap7-pin-cells}
\end{figure}

second paragraph: geometric specifications
-walk through Tab.~\ref{table:chap7-pin-cell-radii} of pin cell radii
-pin cell pitch - everything beyond the outermost radius in Tab.~\ref{table:chap7-pin-cell-radii} is borated water and the zircaloy grid spacer

\begin{table}[h!]
  \centering
  \caption[BEAVRS pin cell radii]{Pin cell radii for the \ac{BEAVRS} model.}
  \small
  \label{table:chap7-pin-cell-radii} 
  \vspace{6pt}
  \begin{tabular}{l c}
  \toprule
  \rowcolor{lightgray}
  \multicolumn{1}{c}{\bf Material} &
  \multicolumn{1}{c}{\bf Radius [cm]} \\
  \midrule
  \multicolumn{2}{c}{\bf Fuel Pin} \\
  \midrule
  Fuel &  0.39218 \\
  Helium & 0.40005 \\
  Zircaloy & 0.45720 \\
  \midrule
  \multicolumn{2}{c}{\bf Empty Guide Tube\footnotemark} \\
  \midrule
  Borated Water & 0.56134 \\
  Zircaloy & 0.60198 \\
  \midrule
  \multicolumn{2}{c}{\bf Instrument Tube} \\
  \midrule
  Air & 0.43688 \\
  Zircaloy & 0.48387 \\
  Borated Water & 0.56134 \\
  Zircaloy & 0.60198 \\
  \midrule
  \multicolumn{2}{c}{\bf Burnable Poison\footnotemark} \\
  \midrule
  Air & 0.21400 \\
  Stainless Steel & 0.23051 \\
  Air & 0.24130 \\
  Borosilicate Glass & 0.42672 \\
  Air & 0.43688 \\
  Stainless Steel & 0.48387 \\
  Borated Water & 0.56134 \\
  Zircaloy & 0.60198 \\
  \bottomrule
\end{tabular}
\end{table}

\addtocounter{footnote}{-2}
\stepcounter{footnote}\footnotetext{The control rod guide tube geometry above the dashpot.}
\stepcounter{footnote}\footnotetext{The burnable poison geometry above the dashpot.}


%%%%%%%%%%%%%%%%%%%%%%%%%%%%
\subsection{Fuel Assemblies}
\label{subsec:chap7-fuel-assms}

first paragraph: different assembly types
-three fuel assemblies taken from \ac{BEAVRS} full core benchmark
  -modeled in isolation w/ reflective BCs
-width of each assembly (assembly pitch)
-stainless steel grid sleeve around each fuel assembly was not modeled
-water between assemblies was not modeled

second paragraph: types of fuel assemblies
-increasing degrees of spatial heterogeneities
-Fig.~\ref{fig:chap7-assm-16} - 1.6\% enriched with \ac{CRGT}s and a central instrument tube
-Fig.~\ref{fig:chap7-assm-31} - 3.1\% enriched with same configuration to consider the impact of fuel enrichment
-Fig.~\ref{fig:chap7-assm-31-20BPs} 3.1\% enriched with mixture of \ac{CRGT}s and \ac{BP}s to measure impact
-all kinds of \ac{BP} configurations could be analyzed, but only a few were chosen for simplicity

\begin{figure}[h!]
  \centering
  \includegraphics[width=0.65\linewidth]{figures/benchmarks/assembly-16}
\caption[BEAVRS 1.6\% enriched assembly]{Whoa}
\label{fig:chap7-assm-16}
\end{figure}

\begin{figure}[h!]
  \centering
  \includegraphics[width=0.65\linewidth]{figures/benchmarks/assembly-31}
\caption[BEAVRS 3.1\% enriched assembly]{Whoa}
\label{fig:chap7-assm-31}
\end{figure}

\begin{figure}[h!]
  \centering
  \includegraphics[width=0.65\linewidth]{figures/benchmarks/assembly-31-20BPs}
\caption[BEAVRS 3.1\% enriched assembly with 20 \ac{BP}s]{Whoa}
\label{fig:chap7-assm-31-20BPs}
\end{figure}

%%%%%%%%%%%%%%%%%%%%%%%%%%%%%%%%%%%%%%
\section{2$\times$2 Assembly Colorsets}
\label{subsec:chap7-2x2-colorsets}

first paragraph: two different colorsets
-comprised of the fuel assemblies discussed earlier
-three fuel assemblies taken from \ac{BEAVRS} full core benchmark
  -modeled in isolation w/ reflective BCs
-width of each assembly (assembly pitch)
-stainless steel grid sleeve around each fuel assembly was not modeled
-water between assemblies was not modeled

second paragraph: types of fuel assemblies
-increasing degrees of spatial heterogeneities
-Fig.~\ref{fig:chap7-2x2} - 1.6\% and 3.1\% enriched fuel assemblies with \ac{BP}s
  -introduces inter-assembly spatial heterogeneities
  -periodics BCs
  -fuel pins with different enrichments next to each other
-Fig.~\ref{fig:chap7-reflector} - same 2$\times$2 but with reflector
  -introduces water reflector
  -reflective BCs on left, top
  -vacuum BCs on right and bottom
  -added moderation for pins near reflector
  -leakage out of the geometry
  -tilts power distribution across the colorset
  -add baffle to benchmark

\begin{figure}[h!]
  \centering
  \includegraphics[width=0.65\linewidth]{figures/benchmarks/2x2}
\caption[A 2$\times$2 colorset of BEAVRS assemblies]{Whoa}
\label{fig:chap7-2x2}
\end{figure}

\begin{figure}[h!]
  \centering
  \includegraphics[width=0.65\linewidth]{figures/benchmarks/reflector}
\caption[A reflected 2$\times$2 colorset of BEAVRS assemblies]{Whoa}
\label{fig:chap7-reflector}
\end{figure}

%%%%%%%%%%%%%%%%%%%%%%%%%%%%%
\subsection{BEAVRS Full Core}
\label{subsec:chap7-full-core}

first paragraph: full \ac{BEAVRS} core in all its glory
-cycle 1 loading pattern
-includes baffle, barrel, core
-includes inter-assembly 
-\# of assemblies, \# of each type of pin
--stainless steel grid sleeve around each fuel assembly included
-water between assemblies included
  -these radial heterogeneities will induce further spatial self-shielding effects!

\begin{figure}[h!]
  \centering
  \includegraphics[width=0.9\linewidth]{figures/benchmarks/full-core}
\caption[The \ac{BEAVRS} core at the axial mid-plane.]{A radial view of the \ac{BEAVRS} core at the axial mid-plane.}
\label{fig:chap7-full-core}
\end{figure}


%%%%%%%%%%%%%%%%%%%%%%%%%%%%%%%%%%%%%%%%%%%%%%%%%%%%%%%%%%%%%%%%%%%%%%%%%%%%%%%
\section{Reference Results}
\label{sec:chap7-ref-results}

-mention iso-in-lab
-NNDC cross sections
  -not looking for truly accurate results
  -generating reference results for comparison with multi-group \ac{MOC}
-metrics of interest:
  -eigenvalue
  -energy-integrated pin-wise fission rates using an OpenMC rectilinear mesh tally
  -energy-integrated pin-wise U-238 capture rates using an OpenMC rectilinear mesh tally
 
runtime parameters:
-1000 total batches
  -200 inactive batches
  -800 active batches
-5 nodes of 24 Intel Xeon cores each
-2 MPI processes per node
-12 OpenMP threads per MPI process

%%%%%%%%%%%%%%%%%%%%%%%%%%%%%%%
\subsection{Source Convergence}
\label{subsec:chap7-src-converge}

first paragraph: shannon entropy
-reference for shannon entropy to indicate source stationarity (Herman's thesis??)
-pin-wise entropy mesh, used OpenMC's internally calculated shannon entropy
-need to determine \# batches to reach reasonable source stationarity
-used to fix the number of inactive batches before tallying 

second paragraph:
-Fig.~\ref{fig:chap7-entropy} - convergence of entropy for each of the benchmarks
-normalized entropy to the final value on the last batch
-stationary entropy is different for each geometry
-normalization makes it easy to compare entropies for each benchmark
-trivial convergence for individual fuel assemblies
-takes $\sim$10 batches for the 2$\times$2 with reflector to converge
-takes $\sim$200 batches for the full core \ac{BEAVRS} to converge
-will use 100 inactive batches for all fuel assembly and 2$\times$ colorsets in subsequent simulations
-will use 200 inactive batches for all full core \ac{BEAVRS} simulations in subsequent simulations

\begin{figure}[h!]
  \centering
  \includegraphics[width=0.9\linewidth]{figures/benchmarks/entropy/entropy-all}
\caption[Shannon entropy source convergence for BEAVRS geometries]{Shannon entropy source convergence for BEAVRS geometries.}
\label{fig:chap7-entropy}
\end{figure}


%%%%%%%%%%%%%%%%%%%%%%%%
\subsection{Eigenvalues}
\label{subsec:chap7-eigenvalues}

first paragraph: eigenvalues
-Tab.~\ref{table:chap7-ref-eigenvalues} - converged eigenvalues for each benchmark
-total runtime in CPU-hours for each case
  -used same \# batches, \# particle histories / batch
  -variation due to varying complexity of tracking particles in different geometries
-trends
  -greater enrichment (1.6\% to 3.1\%) increases fission-to-absorption ratio and increases eigenvalue
  -\ac{BP}s increases absorption and reduces eigenvalue
  -reflector reduces eigenvalue due to absorption and leakage
-final converged eigenvalues serve as the reference for each benchmark throughout the remainder of thesis

\begin{table}[h!]
  \centering
  \caption[Reference $k^{OpenMC}_{eff}$ for heterogeneous benchmarks]{Reference $k^{OpenMC}_{eff}$ for heterogeneous benchmarks.}
  \small
  \label{table:chap7-ref-eigenvalues}
  \vspace{6pt}
  \begin{tabular}{l c c}
  \toprule
  \textbf{Benchmark} & $\mathbf{k^{OpenMC}_{eff}}$ & \textbf{Runtime [CPU-hours]} \\
  \midrule
  1.6\% Enriched Assembly (no \ac{BP}s) & 1.00987 $\pm$ 0.00003 & 331 \\
  3.1\% Enriched Assembly (no \ac{BP}s) & 1.22344 $\pm$ 0.00003 & 326 \\
  3.1\% Enriched Assembly (20 \ac{BP}s) & 1.04530 $\pm$ 0.00003 & 331 \\
  2$\times$2 Colorset & 1.03196 $\pm$ 0.00003 & 345 \\
  2$\times$2 Colorset w/ Reflector & 0.95462 $\pm$ 0.00003 & 337 \\
  \ac{BEAVRS} Full Core & 1.02497 $\pm$ 0.00003 & 387 \\
  \bottomrule
\end{tabular}
\end{table}

%%%%%%%%%%%%%%%%%%%%%%%%%%%%%%%%%%%%
\subsection{Fission Rate Spatial Distributions}
\label{subsec:chap7-pin-powers}

first paragraph: fission rates
-how they were computed 
  -rectilinear tally mesh in OpenMC
  -energy-integrated
  -volume-integrated across each pin
  -fission from all nuclides
-relative error or uncertainty computed how???

second paragraph: spatial distributions
-Fig.~\Crefrange{fig:chap7-fiss-rates-1.6-assm}{fig:chap7-fiss-rates-full-core} - all benchmark geometries
-trends:
  -adjacency to \ac{CRGT}s provides added moderation and increases fission rates
  -adjacency to \ac{BP}s reduces neutron population and reduces fission rates
  -pins in 3.1\% assm in 2$\times$2 have $\sim$35\% higher fission rates than those in 1.6\% assm
    -more peaking / variation due to \ac{BP}s
  -adjacency to reflector reduces fission rates since leakage reduces neutrons
  -full core has complex distribution of fission rates

third paragraph: convergence rate
-Fig.~\ref{fig:chap7-fiss-rates-conv} has convergence of fission rate uncertainties
-well below 1\% uncertainty of fission rates in each pin for 2D \ac{BEAVRS} benchmark
-smoother convergence for mean than for max

\begin{figure}[h!]
\centering
\begin{subfigure}{0.5\textwidth}
  \centering
  \includegraphics[width=\linewidth]{figures/benchmarks/fission-rates/fiss-mean-fuel-16}
  \caption{}
  \label{fig:chap7-fiss-rate-mean-1.6-assm}
\end{subfigure}%
\begin{subfigure}{0.5\textwidth}
  \centering
  \includegraphics[width=\linewidth]{figures/benchmarks/fission-rates/fiss-rel-err-fuel-16}
  \caption{}
  \label{fig:chap7-fiss-rate-rel-err-1.6-assm}
\end{subfigure}%
\caption[Fission rates for a 1.6\% enriched assembly]{Fission rates for a 1.6\% enriched assembly.}
\label{fig:chap7-fiss-rates-1.6-assm}
\end{figure}

\begin{figure}[h!]
\centering
\begin{subfigure}{0.5\textwidth}
  \centering
  \includegraphics[width=\linewidth]{figures/benchmarks/fission-rates/fiss-mean-fuel-31}
  \caption{}
  \label{fig:chap7-fiss-rate-mean-3.1-assm}
\end{subfigure}%
\begin{subfigure}{0.5\textwidth}
  \centering
  \includegraphics[width=\linewidth]{figures/benchmarks/fission-rates/fiss-rel-err-fuel-31}
  \caption{}
  \label{fig:chap7-fiss-rate-rel-err-3.1-assm}
\end{subfigure}%
\caption[Fission rates for a 3.1\% enriched assembly]{Fission rates for a 3.1\% enriched assembly.}
\label{fig:chap7-fiss-rates-3.1-assm}
\end{figure}

\begin{figure}[h!]
\centering
\begin{subfigure}{0.5\textwidth}
  \centering
  \includegraphics[width=\linewidth]{figures/benchmarks/fission-rates/fiss-mean-fuel-31-20BAs}
  \caption{}
  \label{fig:chap7-fiss-rate-mean-3.1-20BAs-assm}
\end{subfigure}%
\begin{subfigure}{0.5\textwidth}
  \centering
  \includegraphics[width=\linewidth]{figures/benchmarks/fission-rates/fiss-rel-err-fuel-31-20BAs}
  \caption{}
  \label{fig:chap7-fiss-rate-rel-err-3.1-20BAs-assm}
\end{subfigure}%
\caption[Fission rates for a 3.1\% enriched assembly with 20 BPs]{Fission rates for a 3.1\% enriched assembly with 20 \ac{BP}s.}
\label{fig:chap7-fiss-rates-3.1-assm-20BAs}
\end{figure}

\begin{figure}[h!]
\centering
\begin{subfigure}{0.5\textwidth}
  \centering
  \includegraphics[width=\linewidth]{figures/benchmarks/fission-rates/fiss-mean-2x2}
  \caption{}
  \label{fig:chap7-fiss-rate-mean-2x2}
\end{subfigure}%
\begin{subfigure}{0.5\textwidth}
  \centering
  \includegraphics[width=\linewidth]{figures/benchmarks/fission-rates/fiss-rel-err-2x2}
  \caption{}
  \label{fig:chap7-fiss-rate-rel-err-2x2}
\end{subfigure}%
\caption[Fission rates for a 2$\times$2 colorset]{Fission rates for a 2$\times$2 colorset.}
\label{fig:chap7-fiss-rates-2x2}
\end{figure}

\begin{figure}[h!]
\centering
\begin{subfigure}{0.5\textwidth}
  \centering
  \includegraphics[width=\linewidth]{figures/benchmarks/fission-rates/fiss-mean-reflector}
  \caption{}
  \label{fig:chap7-fiss-rate-conv}
\end{subfigure}%
\begin{subfigure}{0.5\textwidth}
  \centering
  \includegraphics[width=\linewidth]{figures/benchmarks/fission-rates/fiss-rel-err-reflector}
  \caption{}
  \label{fig:chap7-fiss-rate-conv}
\end{subfigure}%
\caption[Fission rates for a 2$\times$2 colorset with a reflector]{Fission rates for a 2$\times$2 colorset with a reflector.}
\label{fig:chap7-fiss-rates-2x2}
\end{figure}

\begin{figure}[h!]
\centering
\begin{subfigure}{0.5\textwidth}
  \centering
  \includegraphics[width=\linewidth]{figures/benchmarks/fission-rates/fiss-mean-full-core}
  \caption{}
  \label{fig:chap7-fiss-rate-mean-full-core}
\end{subfigure}%
\begin{subfigure}{0.5\textwidth}
  \centering
  \includegraphics[width=\linewidth]{figures/benchmarks/fission-rates/fiss-rel-err-full-core}
  \caption{}
  \label{fig:chap7-fiss-rate-rel-err-full-core}
\end{subfigure}%
\caption[Fission rates for the full 2D BEAVRS core]{Fission rates for the full 2D \ac{BEAVRS} core.}
\label{fig:chap7-fiss-rates-full-core}
\end{figure}

\begin{figure}[h!]
\centering
\begin{subfigure}{\textwidth}
  \centering
  \includegraphics[width=0.8\linewidth]{figures/benchmarks/fission-rates/fiss-conv-max-assms}
  \caption{}
  \label{fig:chap7-fiss-rate-max-conv-max}
\end{subfigure}
\begin{subfigure}{\textwidth}
  \centering
  \includegraphics[width=0.8\linewidth]{figures/benchmarks/fission-rates/fiss-conv-mean-assms}
  \caption{}
  \label{fig:chap7-fiss-rate-max-conv-mean}
\end{subfigure}
\caption[Fission rate error convergence for BEAVRS geometries]{Fission rate relative error convergence for \ac{BEAVRS} geometries.}
\label{fig:chap7-fiss-rates-conv}
\end{figure}

%%%%%%%%%%%%%%%%%%%%%%%%%%%%%%%%%%%%%%%%%%%%%
\subsection{U-238 Capture Rate Spatial Distributions}
\label{subsec:chap7-capture-rates}

first paragraph: U-238 capture rates
-how they were computed 
  -rectilinear tally mesh in OpenMC
  -energy-integrated
  -volume-integrated across each pin
  -only for U-238
-relative error or uncertainty computed how???

second paragraph: spatial distributions
-Fig.~\Crefrange{fig:chap7-capt-rates-1.6-assm}{fig:chap7-capt-rates-full-core} - all benchmark geometries
-trends - similar to fission rates since U-238 capture dominated by low-lying resonances
  -adjacency to \ac{CRGT}s provides added moderation and increases capture rates
  -adjacency to \ac{BP}s reduces neutron population and reduces capture rates
  -pins in 3.1\% assm in 2$\times$2 have $\sim$35\% higher capture rates than those in 1.6\% assm
    -more peaking / variation due to \ac{BP}s
  -adjacency to reflector reduces capture rates since leakage reduces neutrons
  -full core has complex distribution of capture rates

third paragraph: convergence rate
-Fig.~\ref{fig:chap7-capt-rates-conv} has convergence of capture rate uncertainties
-well below 1\% uncertainty of capture rates in each pin for 2D \ac{BEAVRS} benchmark
-smoother convergence for mean than for max

\begin{figure}[h!]
\centering
\begin{subfigure}{0.5\textwidth}
  \centering
  \includegraphics[width=\linewidth]{figures/benchmarks/capture-rates/capt-mean-fuel-16}
  \caption{}
  \label{fig:chap7-capt-rate-mean-1.6-assm}
\end{subfigure}%
\begin{subfigure}{0.5\textwidth}
  \centering
  \includegraphics[width=\linewidth]{figures/benchmarks/capture-rates/capt-rel-err-fuel-16}
  \caption{}
  \label{fig:chap7-capt-rate-rel-err-1.6-assm}
\end{subfigure}%
\caption[U-238 capture rates for a 1.6\% enriched assembly]{U-238 capture rates for a 1.6\% enriched assembly.}
\label{fig:chap7-capt-rates-1.6-assm}
\end{figure}

\begin{figure}[h!]
\centering
\begin{subfigure}{0.5\textwidth}
  \centering
  \includegraphics[width=\linewidth]{figures/benchmarks/capture-rates/capt-mean-fuel-31}
  \caption{}
  \label{fig:chap7-capt-rate-mean-3.1-assm}
\end{subfigure}%
\begin{subfigure}{0.5\textwidth}
  \centering
  \includegraphics[width=\linewidth]{figures/benchmarks/capture-rates/capt-rel-err-fuel-31}
  \caption{}
  \label{fig:chap7-capt-rate-rel-err-3.1-assm}
\end{subfigure}%
\caption[U-238 capture rates for a 3.1\% enriched assembly]{U-238 capture rates for a 3.1\% enriched assembly.}
\label{fig:chap7-capt-rates-3.1-assm}
\end{figure}

\begin{figure}[h!]
\centering
\begin{subfigure}{0.5\textwidth}
  \centering
  \includegraphics[width=\linewidth]{figures/benchmarks/capture-rates/capt-mean-fuel-31-20BAs}
  \caption{}
  \label{fig:chap7-capt-rate-mean-3.1-20BAs-assm}
\end{subfigure}%
\begin{subfigure}{0.5\textwidth}
  \centering
  \includegraphics[width=\linewidth]{figures/benchmarks/capture-rates/capt-rel-err-fuel-31-20BAs}
  \caption{}
  \label{fig:chap7-capt-rate-rel-err-3.1-20BAs-assm}
\end{subfigure}%
\caption[U-238 capture rates for a 3.1\% enriched assembly with 20 BPs]{U-238 capture rates for a 3.1\% enriched assembly with 20 \ac{BP}s.}
\label{fig:chap7-capt-rates-3.1-assm-20BAs}
\end{figure}

\begin{figure}[h!]
\centering
\begin{subfigure}{0.5\textwidth}
  \centering
  \includegraphics[width=\linewidth]{figures/benchmarks/capture-rates/capt-mean-2x2}
  \caption{}
  \label{fig:chap7-capt-rate-mean-2x2}
\end{subfigure}%
\begin{subfigure}{0.5\textwidth}
  \centering
  \includegraphics[width=\linewidth]{figures/benchmarks/capture-rates/capt-rel-err-2x2}
  \caption{}
  \label{fig:chap7-capt-rate-rel-err-2x2}
\end{subfigure}%
\caption[U-238 capture rates for a 2$\times$2 colorset]{U-238 capture rates for a 2$\times$2 colorset.}
\label{fig:chap7-capt-rates-2x2}
\end{figure}

\begin{figure}[h!]
\centering
\begin{subfigure}{0.5\textwidth}
  \centering
  \includegraphics[width=\linewidth]{figures/benchmarks/capture-rates/capt-mean-reflector}
  \caption{}
  \label{fig:chap7-capt-rate-mean-reflector}
\end{subfigure}%
\begin{subfigure}{0.5\textwidth}
  \centering
  \includegraphics[width=\linewidth]{figures/benchmarks/capture-rates/capt-rel-err-reflector}
  \caption{}
  \label{fig:chap7-capt-rate-rel-err-reflector}
\end{subfigure}%
\caption[U-238 capture rates for a 2$\times$2 colorset with a reflector]{U-238 capture rates for a 2$\times$2 colorset with a reflector.}
\label{fig:chap7-capt-rates-2x2}
\end{figure}

\begin{figure}[h!]
\centering
\begin{subfigure}{0.5\textwidth}
  \centering
  \includegraphics[width=\linewidth]{figures/benchmarks/capture-rates/capt-mean-full-core}
  \caption{}
  \label{fig:chap7-capt-rate-mean-full-core}
\end{subfigure}%
\begin{subfigure}{0.5\textwidth}
  \centering
  \includegraphics[width=\linewidth]{figures/benchmarks/capture-rates/capt-rel-err-full-core}
  \caption{}
  \label{fig:chap7-capt-rate-rel-err-full-core}
\end{subfigure}%
\caption[U-238 capture rates for the full 2D BEAVRS core]{U-238 capture rates for the full 2D \ac{BEAVRS} core.}
\label{fig:chap7-capt-rates-full-coe}
\end{figure}

\begin{figure}[h!]
\centering
\begin{subfigure}{\textwidth}
  \centering
  \includegraphics[width=0.8\linewidth]{figures/benchmarks/capture-rates/capt-conv-max-assms}
  \caption{}
  \label{fig:chap7-capt-rate-max-conv}
\end{subfigure}
\begin{subfigure}{\textwidth}
  \centering
  \includegraphics[width=0.8\linewidth]{figures/benchmarks/capture-rates/capt-conv-mean-assms}
  \caption{}
  \label{fig:chap7-capt-rate-max-conv}
\end{subfigure}
\caption[U-238 capture rate error convergence for BEAVRS geometries]{U-238 capture rate relative error convergence for \ac{BEAVRS} fuel geometries.}
\label{fig:chap7-capt-rates-conv}
\end{figure}


-summary box!!!!