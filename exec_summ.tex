\documentclass[12pt,twoside]{mitthesis-exec}

%%%%%%%%%%%%%%%%%%%%%%%%%%%%%%%%%%%%%%%%%%%%%%%%%%%%%%%%%%%%%%%%%%%%%%%%%%%%%%%%
% PREAMBLE

\usepackage[bitstream-charter]{mathdesign} % Use BT Charter font
\usepackage[T1]{fontenc}                   % Use T1 encoding instead of OT1
\usepackage[utf8]{inputenc}                % Use UTF8 input encoding
\usepackage{microtype}                     % Improve typography
\usepackage{amsmath}                       % AMS Math extensions
\usepackage{booktabs}                      % Improve table spacing
\usepackage{graphicx}                      % Extended graphics capabilities
\usepackage{tocbibind}                     % Include listings in TOC
\usepackage[printonlyused]{acronym} % withpage: for showing page of use
\usepackage{listings}                      % Source code listings
\usepackage{caption}
\usepackage{subcaption}
\usepackage[rgb,table]{xcolor}
\usepackage{url}
\usepackage{soul}
\usepackage{array}
\usepackage{pdfpages}
\usepackage{mathtools}
\usepackage{setspace}
\usepackage{pbox}
\usepackage{tikz}
\usetikzlibrary{calc,shapes,decorations.pathreplacing,positioning}
\usepackage{pgfplots}
\usepackage{siunitx}
\usepackage{multirow}
\usepackage{breqn}

% Specialties for tables
\usepackage{array}
\newcolumntype{L}[1]{>{\raggedright\let\newline\\\arraybackslash\hspace{0pt}}m{#1}}
\newcolumntype{C}[1]{>{\centering\let\newline\\\arraybackslash\hspace{0pt}}m{#1}}
\newcolumntype{R}[1]{>{\raggedleft\let\newline\\\arraybackslash\hspace{0pt}}m{#1}}

\usepackage[breaklinks=true]{hyperref}
\hypersetup{colorlinks=true, linkcolor=black, citecolor=black, urlcolor=black,
  pdftitle={Reactor Agnostic Multi-Group Cross Section Generation for Fine Mesh Deterministic Neutron Transport Simulations},
  pdfauthor={William Robert Dawson Boyd}
}
\pagestyle{plain}

%\usepackage{floatrow}
%\floatsetup[table]{style=plaintop}
%\floatsetup[widefigure]{margins=hangleft}

% Highlights and emphasis boxes from Bryans thesis
\usepackage[framemethod=tikz]{mdframed}
\definecolor{mitred}{rgb}{0.698,0.0314,0.216}
\definecolor{mitgray}{rgb}{0.690,0.694,0.710}
\definecolor{canyellow}{rgb}{0.933, 0.965, 0.424}
\newmdenv[nobreak=false, skipabove=2ex, skipbelow=2ex, innerlinewidth=3pt, innerlinecolor=black, backgroundcolor=mitgray!75, roundcorner=10pt, frametitlerule=true, frametitlerulewidth=2.5pt, frametitlefont=\color{white}\Large\bfseries, frametitlealignment=\centering, frametitlebackgroundcolor=mitred, frametitleaboveskip=2ex, frametitlebelowskip=2ex, innertopmargin=3ex, innerbottommargin=2ex]{highlightsbox}
\newmdenv[nobreak=false, skipabove=2ex, skipbelow=2ex, innerlinewidth=2pt, innerlinecolor=black, backgroundcolor=white, roundcorner=10pt]{emphbox}

% Appendix
\usepackage[toc,page]{appendix}

% Don't reset footnote counter between chapters
\usepackage{chngcntr}
\counterwithout{footnote}{chapter}

% Algorithm constructs
\usepackage[chapter]{algorithm} % Provides algorithm environment
\usepackage{algorithmicx}       % Provides algorithmic block
\usepackage{algpseudocode}      % Option of algorithmicx package
\renewcommand{\thealgorithm}{\thechapter-\arabic{algorithm}}
\newcommand\Algphase[1]{%
\vspace*{-.7\baselineskip}\Statex\hspace*{\dimexpr-\algorithmicindent-2pt\relax}\rule{\columnwidth}{0.4pt}%
\Statex\hspace*{-\algorithmicindent}{#1}%
\vspace*{-.7\baselineskip}\Statex\hspace*{\dimexpr-\algorithmicindent-2pt\relax}\rule{\columnwidth}{0.4pt}%
}
\newcommand{\algrule}[1][.4pt]{\par\vskip.5\baselineskip\hrule height #1\par\vskip.5\baselineskip}

% Configure captions
\captionsetup{labelfont=bf, labelsep=colon}
\captionsetup[algorithm]{labelfont=bf, labelsep=colon}

% Use Latin Modern for typewriter fonts
\renewcommand{\ttdefault}{lmtt}

% Add \unit macro
\newcommand{\unit}[1]{\ensuremath{\, \mathrm{#1}}}

\definecolor{gray}{rgb}{0.4,0.4,0.4}
\definecolor{darkblue}{rgb}{0.0,0.0,0.6}
\definecolor{cyan}{rgb}{0.0,0.6,0.6}
\lstset{
  basicstyle=\footnotesize\ttfamily,
  columns=fullflexible,
  showstringspaces=false,
  commentstyle=\color{gray}\upshape,
  frame=single,
  xleftmargin=0.55in
}

\lstdefinelanguage{XML}
{
  morestring=[b]",
  morestring=[s]{>}{<},
  morecomment=[s]{<?}{?>},
  morecomment=[s]{<!--}{-->},
  stringstyle=\color{black},
  identifierstyle=\color{darkblue},
  keywordstyle=\color{cyan},
  morekeywords={}
}

\setcounter{secnumdepth}{4}
\setcounter{tocdepth}{3}

\renewcommand{\contentsname}{Table of Contents}
\renewcommand{\bibname}{References}

\makeatletter \renewcommand\thealgorithm{\arabic{algorithm}} \@addtoreset{algorithm}{chapter} \makeatother

\begin{document}

%%%%%%%%%%%%%%%%%%%%%%%%%%%%%%%%%%%%%%%%%%%%%%%%%%%%%%%%%%%%%%%%%%%%%%%%%%%%%%%%
% TITLE PAGE

\title{EXECUTIVE SUMMARY \\~\\ Reactor Agnostic Multi-Group Cross Section Generation for Fine Mesh Deterministic Neutron Transport Simulations}

\author{William Robert Dawson Boyd III}
\prevdegrees{B.S., Georgia Institute of Technology (2010) \\
             M.S., Massachusetts Institute of Technology (2014)}
\department{Department of Nuclear Science and Engineering}
\degree{Doctor of Philosophy in Nuclear Science and Engineering}

\degreemonth{February}
\degreeyear{2017}
\thesisdate{November 4, 2016}

\supervisor{Benoit Forget}{Associate Professor of Nuclear Science and Engineering}
\reader{Kord S. Smith}{KEPCO Professor of the Practice of Nuclear Science and Engineering}
\chairman{Emilio Bagglietto}{Associate Professor of Nuclear Science and Engineering}


%%%%%%%%%%%%%%%%%%%%%%%%%%%%%%%%%%%%%%%%%%%%%%%%%%%%%%%%%%%%%%%%%%%%%%%%%%%%%%%%
%%%%%%%%%%%%%%%%%%%%%%%%%%%%%%%%%%%%%%%%%%%%%%%%%%%%%%%%%%%%%%%%%%%%%%%%%%%%%%%%
% ABSTRACT

\setcounter{savepage}{\thepage}

\begin{abstractpage}

A key challenge for whole-core transport methods is accurate reactor agnostic multi-group cross section (MGXS) generation. Monte Carlo (MC) presents the most accurate method for reactor agnostic multi-group cross section generation since it does not require the use of any local approximations to the flux. This accuracy comes at the computational expense of converging group constant tallies to acceptably low uncertainties. This work develops methods that use MC to generate the fine-spatial mesh MGXS that are needed by high-fidelity whole-core transport codes. The novel techniques developed by this thesis employ engineering-based and statistical clustering algorithms to accelerate the convergence rate and reduce the data storage requirements for MGXS tallied on fine, heterogeneous spatial meshes in Monte Carlo.

-need to discuss angular-dependent MGXS, SPH
  -simple 1D slab and 2D fuel pin cell geometries
-need stronger conclusions
-need sentence summarizing future work
-mention the flux separability approximation, angular-dependent MGXS
-define all acronyms
-simplify discussion of the benchmarks
  -just mention increasing heterogeities, metrics for verification
-mention simulation triad - openmc, openmoc, opencg

%MC methods have increasingly been used to generate few group constants for coarse mesh diffusion, most notably by the Serpent MC code.

%The nuclear reactor physics community has long strived for deterministic neutron transport-based tools for whole-core reactor analysis. A key challenge for whole-core transport methods is accurate reactor agnostic multi-group cross section (MGXS) generation. Monte Carlo (MC) presents the most accurate method for reactor agnostic multi-group cross section generation since it does not require the use of any local approximations to the flux. This accuracy comes at the computational expense of converging group constant tallies to acceptably low uncertainties. MC methods have increasingly been used to generate few group constants for coarse mesh diffusion, most notably by the Serpent MC code. This work develops methods that use MC to generate the fine-spatial mesh MGXS that are needed by high-fidelity whole-core transport codes. The novel techniques developed by this thesis employ engineering-based and statistical clustering algorithms to accelerate the convergence rate and reduce the data storage requirements for MGXS tallied on fine, heterogeneous spatial meshes in Monte Carlo.

%No systematic methodology of approximations has been developed which is broadly applicable to all reactor types.

-need to mention replacement of multi-level scheme
-two themes:
1) quantify and diagnose approximation error
2) replace multi-level scheme with single full-core MC calculations
    -engineering-based and statistical clustering-based methods
      -simultaneously model spatial self-shielding effects while accelerating MC tally convergence to a prescribed accuracy 
    -model pin-wise spatial self-shielding effects in MGXS generation
      -mention clustering of pin-wise MGXS from spatial self-shielding
      -Local Neighbor Symmetry (LNS) algorithm to model spatial self-shielding effects with ``geometric templates''
      -
    -iMGXS data processing pipeline, uses unsupervised machine learning to identify clustering trends
  -must mention that global reactivity is shown to be very nearly preserved for all clustering schemes 
  -pin-wise fission rates are only marginally impacted by clustering of U-235 fission
  -pin-wise U-238 capture rates are most impacted by clustering predictions
    -this is important to predict Pu-239 buildup in burnup calculations, etc.

This thesis is organized along two main themes. First, the efficacy of MGXS generation with MC for fine-mesh transport calculations is rigorously assessed. Some of the approximations made by MC-based MGXS generation are quantified, including the angular, energy- and spatial-dependence of condensed MGXS. The second theme develops a novel methodology called inferential MGXS (\textit{i}MGXS) which directly models all energy and spatial self-shielding effects to generate accurate MGXS with a single whole-core MC calculation. The \textit{i}MGXS scheme is a data processing pipeline that uses unsupervised machine learning algorithms to infer the clustering of MGXS from ``noisy'' MC tally data with limited human supervision.

The \textit{i}MGXS scheme is evaluated for six PWR benchmarks with heterogeneities with varying spatial self-shielding effects, including control rod guide tubes and water reflectors. The \textit{i}MGXS scheme was shown to enable highly accurate predictions of U-238 capture rates, reducing the error by a factor of four with respect to schemes which neglect to model MGXS clustering. In addition, the scheme requires at least an order of magnitude fewer MC particle histories to converge MGXS for accurate multi-group deterministic calculations than a reference MC calculation. These results demonstrate the promise for \textit{i}MGXS as a means to efficiently generate MGXS with reactor agnostic MC calculations of the complete heterogeneous geometry with little human oversight. The \textit{i}MGXS scheme may be valuable for future reactor physics analyses of advanced LWR core designs and next generation reactors with spatial heterogeneities that are poorly modeled by the engineering approximations in today's methods for MGXS generation.

%for which engineering approximations for MGXS generation poorly model spatial self-shielding effects of aribtrary core heterogeneities.

%which do not easily lend themselves to engineering approximations for generating MGXS for neutron transport calculations.

%Nevertheless, it points to the potential for \textit{i}MGXS to harness \ac{MC} to efficiently generate accurate \ac{MGXS} for deterministic transport codes.

%The inference of MGXS clustering with the \textit{i}MGXS scheme enables deterministic reactor physics simulations to produce accurate results from MGXS generated by MC faster than would be possible with a direct calculation with MC.

\end{abstractpage}


\singlespacing 

%%%%%%%%%%%%%%%%%%%%%%%%%%%%%%%%%%%%%%%%%%%%%%%%%%%%%%%%%%%%%%%%%%%%%%%%%%%%%%%%
\section*{Introduction}

%%%%%%%%%%%%%%%%%%%%%%%%%%%%%%%%%%%%%%%
\subsection*{Background and Motivation}

-brief literature review\\
  -use MC to generate coarse mesh constants (Serpent)~\cite{serpent2013manual} \\
  -relatively little work to use MC for fine mesh transport~\cite{redmond1997multigroup,cai2014condensation,nelson2014improved} \\

\begin{emphbox}
\textbf{This thesis is motivated by the desire to obtain Monte Carlo-quality solutions with computationally efficient deterministic neutron transport methods.}
\end{emphbox}

\begin{figure}[h!]
\centering
\includegraphics[width=0.6\linewidth]{figures/intro/u235-ce-mg-xs}
\caption[U-235 continuous energy and multi-group fission cross section]{U-235 continuous energy and 16-group fission cross section.}
\label{fig:u235-sigf}
\end{figure}

\begin{figure}[h!]
\centering
\includegraphics[width=0.9\linewidth]{figures/mgxs/mgxs-overlay}
\caption[Energy and spatial variation in MGXS]{The calculation of multi-group cross sections requires knowledge of the spatial and energy variation of the continuous energy cross section and flux. The color-coded position $\mathbf{r}$ and energy $E$ variables correspond to the figures with the matching colored outlines. The microscopic cross section $\sigma_{x,i}$ depends on reactor configuration (a) and neutron energy (b). The flux $\phi$ varies with position (c) and energy (d).}
\label{fig:mgxs-overlay}
\end{figure}

\begin{figure}[h!]
\centering
\includegraphics[width=0.9\linewidth]{figures/intro/multi-step-flow-chart}
\caption[Multi-level approach to reactor analysis]{Current multi-level framework for reactor analysis.}
\label{fig:multi-level-flow-chart}
\end{figure}

\begin{emphbox}
\textbf{This thesis develops and evaluates MC-based methods to generate MGXS for fine mesh deterministic neutron transport codes.}
\end{emphbox}

\begin{emphbox}
\textbf{This thesis uses statistical clustering algorithms to accelerate whole-core MC calculations
which simultaneously model all energy and spatial self-shielding effects for fine mesh MGXS generation in a single step.}
\end{emphbox}

\clearpage

%%%%%%%%%%%%%%%%%%%%%%%%
\subsection*{Objectives}

The subject matter of this thesis is organized along two main themes:

\begin{itemize}
\item \textbf{\textit{Approximation Error}} -- Quantify and diagnose approximation error in MGXS generated from MC methods for simple heterogeneous benchmark problems.
\item \textbf{\textit{Statistical Clustering}} -- Develop statistical clustering methods to accelerate the convergence of MGXS on heterogeneous MC tally meshes.
\end{itemize}

\clearpage

%%%%%%%%%%%%%%%%%%%%%%%%%%%%%%%%%%%%%%%%%%%%%%%%%%%%%%%%%%%%%%%%%%%%%%%%%%%%%%%%
\section*{Software Infrastructure: A Simulation Triad}

-OpenMC~\cite{romano2013openmc} generates MGXS \\
-OpenMOC~\cite{boyd2014openmoc} uses MGXS in deterministic multi-group transport calculations \\
-OpenCG~\cite{boyd2015opencg} enables processing and transfer of tally data  \\

\begin{figure}[h!]
  \centering
  \includegraphics[width=\linewidth]{figures/workflow/triad/simulation-triad}
\caption[A simulation triad of OpenMC, OpenMOC and OpenCG]{A simulation triad consisting of the OpenMC, OpenMOC and OpenCG codes ``glued'' together with Python formed the foundation for this thesis research.}
\label{fig:simulation-triad}
\end{figure}

\clearpage

%%%%%%%%%%%%%%%%%%%%%%%%%%%%%%%%%%%%%%%%%%%%%%%%%%%%%%%%%%%%%%%%%%%%%%%%%%%%%%%%
\section*{Angular-Dependent MGXS and SPH Factors}

%%%%%%%%%%%%%%%%%%%%%%%%%%%%%%%%%%%%%%%%%%%%%
\subsection*{Flux Separability Approximation}

\begin{dmath}
\label{eqn:sigt-flux-separable}
\Sigma_{t,g}(\mathbf{r}) = \frac{\int\displaylimits_{E_{g}}^{E_{g-1}} \Sigma_{t}(\mathbf{r},E)\psi(\mathbf{r},\mathbf{\Omega},\mathrm{d}E)}{\psi_{g}(\mathbf{r},\mathbf{\Omega})} \approx \frac{\int\displaylimits_{E_{g}}^{E_{g-1}} \Sigma_{t}(\mathbf{r},E)\phi(\mathbf{r},E)}{\phi_{g}(\mathbf{r})}
\end{dmath}

\clearpage

%%%%%%%%%%%%%%%%%%%%%%%%%%%%%%%%%%%%%%%%%%
\subsection*{Energy-Dependent Flux Errors}

\begin{figure}[h!]
\centering
\includegraphics[width=\linewidth]{figures/sph/pin-cell/rel-err-inner-outer}
\caption[Flux relative error by energy group with SPH]{The energy-dependent relative error of the 70-group OpenMOC scalar flux with respect to the OpenMC flux in a fuel pin for the innermost, outermost and all FSRs.}
\label{fig:rel-err-energy}
\end{figure}

\clearpage

%%%%%%%%%%%%%%%%%%%%%%%%%%%%%%%%%%%%
\subsection*{Angular-Dependent MGXS}

\begin{figure}[h]
\begin{subfigure}{.5\textwidth}
  \centering
  \includegraphics[width=\linewidth]{figures/sph/batman-1}
  \caption{}
  \label{fig:batman-plots-a}
\end{subfigure}
\begin{subfigure}{.5\textwidth}
  \centering
  \includegraphics[width=\linewidth]{figures/sph/batman-2}
  \caption{}
  \label{fig:batman-plots-b}
\end{subfigure}
\caption[Angular-dependent capture MGXS]{Angular-dependent capture MGXS for the 6.67 eV resonance group as a function of azimuthal angle for two different FSRs. The radial axis is given in units of barns and the azimuthal axis in units of degrees. \textit{Image courtesy of N. Gibson~\cite{gibson2016thesis}.}}
\label{fig:batman-plots}
\end{figure}

\clearpage

%%%%%%%%%%%%%%%%%%%%%%%%%%%%%%%%%%%%%%%%%%%%%%%%%%
\subsection*{SuPerHomog\'{e}n\'{e}isation Factors}

-SPH factors were first proposed by H\'{e}bert~\cite{hebert1993consistent} to preserve reaction rates during energy condensation and spatial homogenization \\
-SPH factor algorithm requires knowledge of a reference source that is used in a multi-group fixed source solver to derive multiplicative factors that adjust the total MGXS to force neutron balance \\

\begin{dmath}
\label{eqn:sph-transport-eqn-iterate}
\mathbf{\Omega} \cdot \nabla \psi_{g}^{(n)}(\mathbf{r},\mathbf{\Omega}) + \mu_{k,g}^{(n-1)}\Sigma_{t,k,g}\psi_{g}^{(n)}(\mathbf{r},\mathbf{\Omega}) = Q_{k,g}(\mathbf{\Omega})
\end{dmath}

\begin{dmath}
\label{eqn:sph-update-sigt}
\Sigma_{t,k,g}^{(n)} = \mu_{k,g}^{(n-1)}\Sigma_{t,k,g}^{(0)}
\end{dmath}

\begin{algorithm}[h]
\caption{SPH Factor Algorithm}
\label{alg:sph}
\begin{algorithmic}[1]
%  \State Initialize MGXS from MC tallies
%  \State Compute neutron source from MC flux and MGXS
  \State Initialize SPH factors to unity
  \While{SPH factors are not converged}
    \State Update MGXS with SPH factors \Comment{Eqn.~\ref{eqn:sph-update-sigt}}
    \State Solve fixed source transport problem \Comment{Eqn.~\ref{eqn:sph-transport-eqn-iterate}}
    \State Compute new SPH factors
  \EndWhile
  \State Compute final MGXS with SPH factors \Comment{Eqn.~\ref{eqn:sph-update-sigt}}
\end{algorithmic}
\end{algorithm}

\begin{table}[h!]
  \centering
  \caption[Eigenvalues with and without SPH factors for a fuel pin]{The impact of SPH factors on the eigenvalue bias $\Delta\rho$ with varying energy group structures for a fuel pin.}  
  \label{table:sph-keff}
  \vspace{6pt}
  \begin{tabular}{c S[table-format=6.1] S[table-format=6.1]}
  \toprule
  & \multicolumn{2}{c}{{\bf $\boldsymbol{\Delta\rho}$ [pcm]}} \\
  \cline{2-3}
  \multirow{-2}{*}{{\bf \# Groups}} &
  \multicolumn{1}{c}{{\bf Without SPH}} &
  \multicolumn{1}{c}{{\bf With SPH}} \\
  \midrule
1 & 66 & -14 \\
2 & 34 & -6\\
4 & -57 & 1 \\
8 & -102 & 2 \\
16 & -111 & 4 \\
25 & -182 & -1 \\
40 & -202 & 2 \\
70 & -211 & -3 \\
  \bottomrule
\end{tabular}
\end{table}

\clearpage

%%%%%%%%%%%%%%%%%%%%%%%%%%%%%%%%%%%%%%%%%
\section*{Spatial Homogenization Schemes}

%%%%%%%%%%%%%%%%%%%%%%%%%%%%%%%%%%%%%%%%%%%%%%%%%%%
\subsection*{Track Density-Weighted Homogenization}

\begin{equation}
\label{eqn:imgxs-set}
\mathbb{S}_{m} = \left\{1 \le k \le K: S(k) = m\right\}
\end{equation}

LNS homogenization computes a single set of MGXS for the fuel pin instances in each set $k \in \mathbb{S}_{m}$ classified by the LNS algorithm. This is equivalent to a specialization of Eqn.~\ref{eqn:imgxs-micro} with track density-weighted averages of the reaction rates and flux tallies in each pin instance:

\begin{equation}
\label{eqn:imgxs-micro}
\hat{\sigma}_{x,i,m,g} = \frac{\displaystyle\sum\limits_{k=1}^{K}\mathbb{1}_{\mathbb{S}_{m}}(k) \langle \sigma_{x,i}, \psi \rangle_{k,g}^{t\ell}}{\displaystyle\sum\limits_{k=1}^{K}\mathbb{1}_{\mathbb{S}_{m}}(k) \langle \psi \rangle_{k,g}^{t\ell}}
\end{equation}

%%%%%%%%%%%%%%%%%%%%%%%%%%%%%%%%%
\subsection*{Null Homogenization}

%%%%%%%%%%%%%%%%%%%%%%%%%%%%%%%%%%%%%%%
\subsection*{Degenerate Homogenization}

%%%%%%%%%%%%%%%%%%%%%%%%%%%%%%%%
\subsection*{LNS Homogenization}

%%%%%%%%%%%%%%%%%%%%%%%%%%%%%%%%%%%%%%%%%%%
\subsection*{\textit{i}MGXS Homogenization}

\begin{figure}[h!]
\centering
\includegraphics[width=0.9\linewidth]{figures/unsupervised/features/engineering/flow-chart}
\vspace{2mm}
\caption[\textit{i}MGXS flow chart]{The \textit{i}MGXS data processing pipeline.}
\label{fig:imgxs-flow-chart}
\end{figure}

\begin{figure}[h!]
\centering
\begin{subfigure}{0.45\textwidth}
  \centering
  \includegraphics[width=0.9\linewidth]{figures/unsupervised/features/assm-16/geometry}
  \caption{}
  \label{fig:capt-mean-std-geom}
\end{subfigure}%
\begin{subfigure}{0.45\textwidth}
  \centering
  \includegraphics[width=0.9\linewidth]{figures/unsupervised/features/assm-16/u238-capt/mean-std/mgxs}
  \caption{}
  \label{fig:capt-mean-std-mgxs}
\end{subfigure}
\begin{subfigure}{0.45\textwidth}
  \centering
  \includegraphics[width=0.9\linewidth]{figures/unsupervised/features/assm-16/u238-capt/mean-std/geometry-3}
  \caption{}
  \label{fig:capt-mean-std-geom-3}
\end{subfigure}%
\begin{subfigure}{0.45\textwidth}
  \centering
  \includegraphics[width=0.9\linewidth]{figures/unsupervised/features/assm-16/u238-capt/mean-std/mgxs-3}
  \caption{}
  \label{fig:capt-mean-std-mgxs-3}
\end{subfigure}
\caption[Clustering of U-238 capture MGXS standard deviations]{Scatter plots of the pin-wise U-238 capture (group 1 of 2) MGXS means ($x$) and standard deviations ($y$) for a 1.6\% enriched fuel assembly.}
\label{fig:capt-mean-std}
\end{figure}

\clearpage

%%%%%%%%%%%%%%%%%%%%%%%%%%%
\section*{Model Validation}

%%%%%%%%%%%%%%%%%%%%%%%%%%%%%%%%%%%%%%%%%%
\subsection*{Heterogeneous PWR Benchmarks}

-construction of six heterogeneous benchmarks from BEAVRS~\cite{horelik2013beavrs} \\

\begin{figure}[h!]
\centering
\begin{subfigure}{0.47\textwidth}
  \centering
  \includegraphics[width=0.9\linewidth]{figures/benchmarks/2x2}
  \caption{}
  \label{fig:2x2}
\end{subfigure}%
\begin{subfigure}{0.47\textwidth}
  \centering
  \includegraphics[width=0.9\linewidth]{figures/benchmarks/reflector}
  \caption{}
  \label{fig:reflector}
\end{subfigure}
\caption[A reflected 2$\times$2 colorset]{A 2$\times$2 colorset of BEAVRS assemblies with periodic boundary conditions (a), and the same colorset surrounded by a water reflector with reflective (left and top) and vacuum (right and bottom) boundary conditions (b).}
\label{fig:benchmark-geometries}
\end{figure}

%%%%%%%%%%%%%%%%%%%%%%%%%%%%%%%%%%%%%%%%%%
\subsection*{Validation Metrics}

-eigenvalues \\
-fission rates \\
-U-238 capture rates \\
-only present U-238 capture rates here \\

\clearpage

%%%%%%%%%%%%%%%%%%%%%%%%%%%
\section*{Results}

%%%%%%%%%%%%%%%%%%%%%%%%%%%%%%%%%%%%%%%%%%%%%%%%%
\subsection*{\textit{i}MGXS Clustered Geometries}

\begin{figure}[h!]
\centering
\begin{subfigure}{0.47\textwidth}
  \centering
  \includegraphics[width=0.7\linewidth]{figures/quantification/homogenization/2x2-degenerate-materials}
  \caption{}
  \label{fig:2x2-degenerate}
\end{subfigure}%
\begin{subfigure}{0.47\textwidth}
  \centering
  \includegraphics[width=0.7\linewidth]{figures/quantification/homogenization/2x2-degenerate-materials}
  \caption{}
  \label{fig:reflector-degenerate}
\end{subfigure}
\begin{subfigure}{0.47\textwidth}
  \centering
  \includegraphics[width=0.7\linewidth]{figures/patterns/lns/2x2/materials}
  \caption{}
  \label{fig:2x2-lns}
\end{subfigure}%
\begin{subfigure}{0.47\textwidth}
  \centering
  \includegraphics[width=0.7\linewidth]{figures/patterns/lns/reflector/materials}
  \caption{}
  \label{fig:reflector-lns}
\end{subfigure}
\begin{subfigure}{0.47\textwidth}
  \centering
  \includegraphics[width=0.7\linewidth]{figures/unsupervised/geometries/with-features/8-clusters/combined/2x2}
  \caption{}
  \label{fig:2x2-8-clusters}
\end{subfigure}%
\begin{subfigure}{0.47\textwidth}
  \centering
  \includegraphics[width=0.7\linewidth]{figures/unsupervised/geometries/with-features/8-clusters/combined/reflector}
  \caption{}
  \label{fig:reflector-8-clusters}
\end{subfigure}
\caption[Materials for the 2$\times$2 colorsets]{The materials for the periodic 2$\times$2 colorset with degenerate, LNS \textit{i}MGXS spatial homogenization are illustrated in (a), (c) and (e), while those for the colorset with a water reflector are depicted in (b), (d) and (f). The materials for the \textit{i}MGXS scheme represent 8 GMM clusters.}
\label{fig:colorset-geometries}
\end{figure}

\clearpage

%%%%%%%%%%%%%%%%%%%%%%%%%%%%%%%%%%%%%%%%%%%%%%%%
\subsection*{Pin-Wise U-238 Capture Rate Errors}

\begin{figure}[h!]
\centering
\includegraphics[width=0.8\linewidth]{figures/results/spatial/reflector/capt-err}
\vspace{2mm}
\caption[U-238 capture rate errors for the 2$\times$2 colorset with reflector]{U-238 capture rate percentage relative errors for the 2$\times$2 colorset with a water reflector with null, degenerate, LNS and \textit{i}MGXS spatial homogenization with 2, 8 and 16 GMM clusters.}
\label{fig:refl-capt-err}
\end{figure}

\clearpage

%%%%%%%%%%%%%%%%%%%%%%%%%%%%%%%%%%%%%%%%%%%%%%%%%%%%
\subsection*{Comparing Pin-Wise U-238 Capture Rates}

\begin{figure}[h!]
\centering
\begin{subfigure}{0.9\textwidth}
  \centering
\includegraphics[width=0.55\linewidth]{figures/results/compare/2x2/compare-capt}
  \caption{}
  \label{fig:refl-capt-rates-comp-refl}
\end{subfigure}
\par\bigskip
\begin{subfigure}{0.9\textwidth}
  \centering
\includegraphics[width=0.55\linewidth]{figures/results/compare/reflector/compare-capt}
  \caption{}
  \label{fig:refl-capt-rates-comp-2x2}
\end{subfigure}
\caption[U-238 capture rate comparison for the colorset]{A comparison of U-238 capture rate spatial distributions for \textit{i}MGXS with GMM clustering and null spatial homogenization schemes for the periodic 2$\times$2 colorset (a) and colorset with a water reflector (b).}
\label{fig:refl-capt-rates-comp}
\end{figure}

\clearpage

%%%%%%%%%%%%%%%%%%%%%%%%%%%%%%%%%%%%%%%%%%%%%%%%
\subsection*{Convergence Rates of MOC Solutions}

\begin{figure}[h!]
\centering
\begin{subfigure}{\textwidth}
  \centering
  \includegraphics[width=0.8\linewidth]{figures/results/convergence/2x2/keff-bias-evo}
  \caption{}
  \label{fig:2x2-eigenvalue-converge}
\end{subfigure}
\begin{subfigure}{\textwidth}
  \centering
  \includegraphics[width=0.8\linewidth]{figures/results/convergence/2x2/mean-capt-err-evo}
  \caption{}
  \label{fig:2x2-capture-converge-mean}
\end{subfigure}
\vspace{2mm}
\caption[Fission rate covergence for the periodic 2$\times$2 colorset]{Convergence of the eigenvalue bias (a) and mean absolute U-238 capture rate percent relative errors (b) for the periodic 2$\times$2 colorset.}
\label{fig:2x2-capture-converge}
\end{figure}

\clearpage

%%%%%%%%%%%%%%%%%%%%%%%%%%%%%%%%%%%%%%%%%%%%%%%%%%%%%%%%%
\subsection*{Runtimes for Spatial Homogenization Schemes}

\begin{table}[ht!]
  \centering
  \caption[Runtimes]{Runtimes.}
  \small
  \label{table:imgxs-runtimes}
  \vspace{6pt}
  \begin{tabular}{l l R{2.5cm} S[table-format=3.2] S[table-format=3.2] S[table-format=3.2]}
  \toprule
  & & & \multicolumn{3}{c}{\bf Runtime [core-hours]} \\
  \cline{4-6}
  \multirow{-2}{*}{\bf Benchmark} &
  \multirow{-2}{*}{\bf Scheme} &
  \multirow{-2}{*}{\bf \# Particles} &
  \multicolumn{1}{c}{\bf OpenMC} &
  \multicolumn{1}{c}{\bf OpenMOC} &
  \multicolumn{1}{c}{\bf Total} \\
  \midrule
\multirow{4}{*}{\parbox{2.5cm}{1.6\% Assm}} & Reference & 550,000,000 & 57.2 & & 57.2 \\
& Null & 100,000 & 0.02 & 0.36 & 0.38 \\
& Degenerate & 115,000,000 & 27.7 & 0.40 & 28.1 \\
& \textit{i}MGXS & 4,000,000 & 0.96 & 0.38 & 1.34 \\
  \midrule
\multirow{4}{*}{\parbox{2.5cm}{3.1\% Assm}} & Reference & 550,000,000 & 50.7 & & 50.7 \\
& Null & 100,000 & 0.02 & 0.40 & 0.42 \\
& Degenerate & 115,000,000 & 25.3 & 0.40 & 25.7 \\
& \textit{i}MGXS & 4,000,000 & 0.88 & 0.37 & 1.25 \\
  \midrule
\multirow{4}{*}{\parbox{2.5cm}{3.1\% Assm w/ 20 BPs}} & Reference & 550,000,000 & 50.1 & & 50.1 \\
& Null & 100,000 & 0.02 & 0.41 & 0.43 \\
& Degenerate & 115,000,000 & 27.7 & 0.41 & 28.1 \\
& \textit{i}MGXS & 4,000,000 & 0.96 & 0.42 & 1.38 \\
  \midrule
\multirow{4}{*}{\parbox{2.5cm}{2$\times$2 Colorset}} & Reference & 8,755,000 & 81.7 & & 81.7 \\
& Null & 100,000 & 0.04 & 2.00 & 2.03 \\
& Degenerate & 700,000,000 & 248 & 2.29 & 250 \\
& \textit{i}MGXS & 10,000,000 & 3.54 & 1.96 & 5.50 \\
  \midrule
\multirow{4}{*}{\parbox{2.5cm}{2$\times$2 Colorset w/ Reflector}} & Reference & & & & \\
& Null & & & 5.07 & \\
& Degenerate & & & 5.36 & \\
& \textit{i}MGXS & & & 4.90 & \\
  \midrule
\multirow{4}{*}{\parbox{2.5cm}{BEAVRS Quarter Core}} & Reference & & & & \\
& Null & & & 419 & \\
& Degenerate & & & 426 & \\
& \textit{i}MGXS & & & 423 & \\
  \bottomrule
\end{tabular}
\end{table}

\clearpage

%%%%%%%%%%%%%%%%%%%%%%%%%%%%%%%%%%%%%%%%%%%%%%%%%%%%%%%%%%%%%%%%%%%%%%%%%%%%%%%%
\section*{Conclusions}

%%%%%%%%%%%%%%%%%%%%%%%%%%%%%%%%%%%%%%%%
\subsection*{Contributions to the Field}

-tightly coupled framework for MC and MOC for MGXS \\
-useful for both MGXS generation as well as validation of MOC \\
-identified fundamental issue with flux separability approx. and 

%%%%%%%%%%%%%%%%%%%%%%%%%
\subsection*{Future Work}

-speed up OpenMOC full core: \\
  -find a way use quarter assembly CMFD mesh \\
  -linear source to reduce number of spatial zones \\
  -vectorize transport solver over energy groups \\

\break

evaluate \textit{i}MGXS scheme:
-add anisotropic scattering to OpenMOC to enable solution of the ``correct'' problem
-systematic study of featues -- ones actually matter? \\
-systematic study to understand impact of dimensionality reduction \\
-systematic study to understand impact of clustering algorithms \\
-more research into model selection schemes -- none of them robustly works for my case studies! \\
-evaluate \textit{i}MGXS with clustering ``on-the-fly'' with noisy MC tally data \\
-can \textit{i}MGXS reduce the number of necessary energy groups?? \\

\break

reach goals:\\
-multi-physics applications: moderator density, fuel temperature, burnup, etc. as features \\
-machine learning to optimize energy group structures \\


%%%%%%%%%%%%%%%%%%%%%%%%%%%%%%%%%%%%%%%%%%%%%%%%%%%%%%%%%%%%%%%%%%%%%%%%%%%%%%%%
%%%%%%%%%%%%%%%%%%%%%%%%%%%%%%%%%%%%%%%%%%%%%%%%%%%%%%%%%%%%%%%%%%%%%%%%%%%%%%%%
% BIBLIOGRAPHY

\begin{singlespace}
\bibliographystyle{ans}
\bibliography{references}
\end{singlespace}

\end{document}
