\chapter{MGXS Generation with Monte Carlo}
\label{chap:mgxs-mc}

In the preceding chapter it was observed that many approximations are made in multi-group theory and the generation of multi-group cross sections. Monte Carlo is an approach to replace some of the steps in the standard multi-level framework for \ac{MGXS} generation with a natural and reactor agnostic treatment of energy and spatial self-shielding. This chapter presents a brief overview of \ac{MC} tallies and statistics in Sec.~\ref{sec:chap3-mc-overview}, and Sec.~\ref{sec:chap3-mgxs-gen} outlines the necessary computation needed to generate \ac{MGXS} with \ac{MC}. Sec.~\ref{sec:chap3-lit-review} discusses past work to apply \ac{MC} for \ac{MGXS} generation, while Sec.~\ref{sec:chap3-latent-variables} introduces a new approach based on a latent variable model for spatial self-shielding.


%%%%%%%%%%%%%%%%%%%%%%%%%%%%%%%%%%%%%%%%%%%%%%%%%%%%%%%%%%%%%%%%%%%%%%%%%%%%%%%
\section{Overview of Monte Carlo Methods}
\label{sec:chap3-mc-overview}

Monte Carlo methods have been successfully applied to neutron transport calculations for many decades. A detailed accounting of the physics models and algorithms used in \ac{MC} methods can be found in the manuals for the MCNP~\cite{mcnpx2003manual}, Serpent~\cite{serpent2013manual} and OpenMC~\cite{openmc2016manual} Monte Carlo particle transport codes. This section presents a few key aspects related to tallies and statistics, and follows directly from the manual for the OpenMC~\cite{openmc2016manual} code which is used throughout this work.

%%%%%%%%%%%%%%%%%%%%%%%%%%%%%%%%
\subsection{Monte Carlo Tallies}
\label{subsec:chap3-mc-tallies}

\ac{MC} simulations sample the particle distribution in order to compute integral quantities of interest called \textit{tallies}. A tally is an integral of a \textit{scoring function} $f$ weighted by the neutron distribution, or flux, across some region of phase space. A general form for tally $\mathcal{T}$ is given by the following integral expression:

\begin{dmath}
\label{eqn:chap3-tallies-general}
\mathcal{T} = \int_{V} \int_{S} \int_{E}  f(\mathbf{r},\mathbf{\Omega},E)\psi(\mathbf{r},\mathbf{\Omega},E)\mathrm{d}E\mathrm{d}\mathbf{\Omega}\mathrm{d}\mathbf{r} 
\end{dmath}

In OpenMC parlance, the integration bounds over space, angle and energy are termed \textit{filters}, while the scoring function $f$ is simply known as a \textit{score}. Various scores may be used to compute volume-integrated fluxes, reaction rates, functional expansions and more. \ac{MC} does not perform the integration in Eqn.~\ref{eqn:chap3-tallies-general} with the exact flux specified at all points in phase space. Instead, \ac{MC} performs stochastic integration by sampling the particle population across the entirety of phase space to compute a \textit{statistical estimate} $\hat{\mathcal{T}}$ of the true integral $\mathcal{T}$.

There are a number of different techniques to estimate a tally. The first and most general method is known as an \textit{analog estimator}. An analog estimator $\hat{\mathcal{T}}$ increments a tally by the particle weight $w_{i}$ each time an event $i$ occurs from the set of all events of interest $A$ (\textit{e.g.}, fission events). The sum of particle weights is then normalized by the total weight of all particles $W$ to compute the interaction frequency on a per-particle basis within the phase space volume of interest:

\begin{dmath}
\label{eqn:chap3-tallies-analog}
\hat{\mathcal{T}} = \frac{1}{W}\displaystyle\sum\limits_{i \in A} w_{i}
\end{dmath}

Although analog estimators permit general filters and scoring functions, they may suffer from poor tallying efficiency if the size of set $A$ is very small compared to the total number of events in a simulation. A \textit{collision estimator} improves the tallying efficiency by incrementing a tally more frequently than is possible with analog estimators. In particular, a collision estimator increments a tally at each collision $i$ from the set of all collisions $C$ irregardless of the types of collisions that took place. By noting that the total collision rate is given by $R_{t} = \Sigma_{t}\phi$, a collision estimator for the flux $\hat{\phi}$ may be simply defined by dividing the particle weights by the total macroscopic cross section:

\begin{dmath}
\label{eqn:chap3-tallies-collision-flux}
\hat{\phi} = \frac{1}{W}\displaystyle\sum\limits_{i \in C}\frac{w_{i}}{\Sigma_{t}(E_{i})}
\end{dmath}

\noindent The collision estimator depends on the energy of the incoming particle $E_{i}$ in order to scale the particle weight by the energy-dependent total cross section. It follows that the collision estimator for a reaction rate $\hat{\mathcal{R}}_{x}$ is the product of the flux in Eqn.~\ref{eqn:chap3-tallies-collision-flux} and the cross section for the reaction type $x$ of interest:

\begin{dmath}
\label{eqn:chap3-tallies-collision-rxn}
\hat{\mathcal{R}}_{x} = \frac{1}{W}\displaystyle\sum\limits_{i \in C}\frac{w_{i}\Sigma_{x}(E_{i})}{\Sigma_{t}(E_{i})}
\end{dmath}

The collision estimator increases the number of events in $C$ to improve the tallying efficiency with respect to analog tallies. A third known as \textit{track-length estimators} goes one step further and increments a tally each time a particle trajectory crosses the phase space of interest (\textit{e.g.}, a spatial tally mesh zone) even if a collision did not take place. The track-length estimator makes use of a particle's distance travelled $\Delta\mathbf{r}$ to estimate the flux. A track-length estimator of the flux is therefore:

\begin{dmath}
\label{eqn:chap3-tallies-tracklength-flux}
  \hat{\phi} = \frac{1}{W}\displaystyle\sum\limits_{i \in T}w_{i}\Delta\mathbf{r}_{i}
\end{dmath}

\noindent where the set $T$ represents each particle trajectory through the phase space volume of interest. Similarly, a track-length estimate of a reaction rate $x$ can be found from simply multiplying the flux in Eqn.~\ref{eqn:chap3-tallies-tracklength-flux} by the cross section:

\begin{dmath}
\label{eqn:chap3-tallies-tracklength-rxn}
  \hat{\mathcal{R}}_{x} = \frac{1}{W}\displaystyle\sum\limits_{i \in T}w_{i}\Delta\mathbf{r}_{i}\Sigma_{x}(E_{i})
\end{dmath}

The tallying efficiency for track-length estimators is greatly improved by incrementing a tally for each particle trajectory. As a result, the confidence intervals are generally much tighter for track-length estimators than those for analog and collision estimators. 

Each of the three estimators -- analog, collision and track-length -- may be useful for different scenarios. Although track-length and collision estimators improve statistics over analog estimates, they cannot be employed for all types of filters. For example, track-length tallies may not be used if the scoring function requires information about the outgoing particle since this is not available unless a collision took place. As discussed in Sec.~\ref{sec:chap3-mgxs-gen}, a mixture of estimators which tradeoff generality with efficiency must be used to generate \ac{MGXS} from \ac{MC} tallies.


%%%%%%%%%%%%%%%%%%%%%%%%%%%%%%
\subsection{Sample Statistics}
\label{subsec:chap3-mc-stats}

\begin{itemize}[noitemsep]
  \item should the statistics be presented earlier?
  \item batch statistics
  \item sample mean (Eqn. (11))
  \item variance of the population (Eqn. (14)
  \item variance of the mean (Eqn. (17))
  \item root mean square convergence rate - mention correlation
\end{itemize}


\begin{dmath}
\label{eqn:chap3-sample-mean}
\bar{x} = \frac{1}{N} \displaystyle\sum\limits_{i=1}^{N} x_{i}
\end{dmath}

unbiased estimate of the sample variance:

\begin{dmath}
\label{eqn:chap3-variance-sample}
s^{2} = \frac{1}{N-1}\displaystyle\sum\limits_{i=1}^{N}\left(x_{i} - \bar{x}\right)^{2}
\end{dmath}

unbiased estimate of the variance of the mean:

\begin{dmath}
\label{eqn:chap3-variance-mean}
s_{\bar{x}}^{2} = \frac{1}{N-1}\left(\frac{1}{N}\displaystyle\sum\limits_{i=1}^{N}
x_{i}^{2} - \bar{x}^2\right)
\end{dmath}

The standard deviation will decrease in proportion to $s_{\bar{x}} \propto \nicefrac{1}{\sqrt{N}}$

\begin{dmath}
\label{eqn:chap3-conv-rate}
s_{\bar{x}} = \frac{\sigma}{\sqrt{N}} \approx \frac{s}{\sqrt{N}}
\end{dmath}


%%%%%%%%%%%%%%%%%%%%%%%%%%%%%%%%%%%%%%%%%%%%%%%%%%%%%%%%%%%%%%%%%%%%%%%%%%%%%%%
\section{\ac{MGXS} Generation with Monte Carlo}
\label{sec:chap3-mgxs-gen}

DIY \ac{MGXS} generation with \ac{MC} \\
Stochastic approx. to integrals in Section~\ref{sec:chap2-background} \\


%%%%%%%%%%%%%%%%%%%%%%%%
\subsection{Tally Types Needed for \ac{MGXS} Generation}
\label{subsec:chap3-tally-types}

\begin{itemize}[noitemsep]
  \item recall flux separability approx.
\end{itemize}

\subsubsection{General Form}
\label{subsubsec:chap3-gen-xs}

\subsubsection{Total and Transport Cross Sections}
\label{subsubsec:chap3-tot-xs}

\subsubsection{Scattering Matrices}
\label{subsubsec:chap3-scatt-mat}

\begin{itemize}[noitemsep]
  \item refer to adam/redmond's thesis for more details on scattering moments
\end{itemize}

\subsubsection{Fission Production Cross Sections}
\label{subsubsec:chap3-fiss-prod}

\subsubsection{Chi Fission Spectrum}
\label{subsubsec:chap3-chi}


%%%%%%%%%%%%%%%%%%%%%%%%%%%%%%%%%%%%
\subsection{Uncertainty Propagation}
\label{subsec:chap3-uncertainty-prop}

\begin{itemize}[noitemsep]
  \item uncertainty propagation eqns
  \item need a reference
  \item https://en.wikipedia.org/wiki/Propagation_of_uncertainty
\end{itemize}

define covariance\\
two random variables $X$ and $Y$ with standard deviations $\sigma_{X}$ and $\sigma_{Y}$\\

\begin{equation}
Z = aX \qquad,\qquad \sigma_{Z}^{2} = a^{2}\sigma_{X}^{2}
\end{equation}

\begin{equation}
Z = X + Y \qquad,\qquad \sigma_{Z}^{2} = \sigma_{X}^{2} + \sigma_{Y}^{2} + 2\sigma_{XY}
\end{equation}

\begin{equation}
Z = X - Y \qquad,\qquad \sigma_{Z}^{2} = \sigma_{X}^{2} + \sigma_{Y}^{2} - 2\sigma_{XY}
\end{equation}

\begin{equation}
Z = XY \qquad,\qquad \sigma_{Z}^{2} \approx Z^{2}\left[\left(\frac{\sigma_{X}}{X}\right)^{2} + \left(\frac{\sigma_{Y}}{Y}\right)^{2} + 2\frac{\sigma_{XY}}{Z}\right] \approx Z^{2}\left[\left(\frac{\sigma_{X}}{X}\right)^{2} + \left(\frac{\sigma_{Y}}{Y}\right)^{2}\right]
\end{equation}

\begin{equation}
Z = \frac{X}{Y} \qquad,\qquad \sigma_{Z}^{2} \approx Z^{2}\left[\left(\frac{\sigma_{X}}{X}\right)^{2} + \left(\frac{\sigma_{Y}}{Y}\right)^{2} - 2\frac{\sigma_{XY}}{Z}\right] \approx Z^{2}\left[\left(\frac{\sigma_{X}}{X}\right)^{2} + \left(\frac{\sigma_{Y}}{Y}\right)^{2}\right]
\end{equation}


%%%%%%%%%%%%%%%%%%%%%%%%%%%%%%%%%%%%%%%%%%%%%%%%%%%%%%%%%%%%%%%%%%%%%%%%%%%%%%%
\section{Review of Previous Efforts}
\label{sec:chap3-lit-review}

\begin{itemize}[noitemsep]
  \item \ac{MGXS} for coarse mesh diffusion
  \begin{itemize}[noitemsep]
    \item Serpent
    \item Redmond
    \item Pounders
  \end{itemize}
  \item \ac{MGXS} for fine mesh transport
  \begin{itemize}[noitemsep]
    \item Nelson with NDPP
  \end{itemize}
\end{itemize}


%%%%%%%%%%%%%%%%%%%%%%%%%%%%%%%%%%%%%%%%%%%%%%%%%%%%%%%%%%%%%%%%%%%%%%%%%%%%%%%
\section{A New Approach: A Latent Variable Model}
\label{sec:chap3-latent-variables}

\begin{itemize}[noitemsep]
  \item tally ``noise'' model
  \item spatial self-shielding treatment is natural in \ac{MC}
  \item challenge is slow convergence rate
  \item motivate clustering
\end{itemize}