\section*{Acknowledgments}

This thesis was supported in part by the Center for Exascale Simulation of Advanced Reactors (CESAR), a co-design center under the U.S. Department of Energy Contract No. DE-AC02-06CH11357. The author was also a recipient of the DOE Office of Nuclear Energy's Nuclear Energy University Programs Fellowship.  The research was partially funded from the DOE Office of Nuclear Energy's Nuclear Energy University Programs (contract number: DE-NE0008578). This research made use of the resources of the High Performance Computing Center at Idaho National Laboratory, which is supported by the Office of Nuclear Energy of the U.S. Department of Energy and the Nuclear Science User Facilities under Contract No. DE-AC07-05ID14517. Additionally, this research used resources of the Argonne Leadership Computing Facility, which is a DOE Office of Science User Facility supported under Contract DE-AC02-06CH11357.

I would like to thank my research advisors Kord Smith and Benoit Forget for their incredible support and expanding my knowledge of both nuclear science and computational science. The flexibility they gave me allowed me to pursue my academic interests in computing, greatly expanding both my knowledge and practical skills. Their guidance was critical during my time at MIT, pushing me to conquer larger goals. They were patient, supportive, and helpful when confronting difficult challenges throughout the research. I am incredibly grateful for their support and leadership and hope to maintain my collegial friendship with them as I embark on new challenges.

I am incredibly grateful for the support and guidance of my peers within the Computational Reactor Physics Group (CRPG). I am especially grateful to Sam Shaner and Will Boyd for their guidance in developing OpenMOC. Both were original developers of the OpenMOC code for 2D simulations and have helped me learn good software habits when developing OpenMOC. Sam Shaner also helped lay the groundwork for the 3D MOC solver presented in this thesis. Additionally, his work on track laydown algorithms was particularly helpful in implementing an efficient solver. Without his support, developing the 3D MOC solver have been tremendously difficult. In addition to his help in developing OpenMOC, Will Boyd's thesis on multi-group cross-section generation was directly used in this thesis for determining the input multi-group cross-sections for the BEAVRS benchmark. Both Sam Shaner and Will Boyd were extremely helpful when problems arose during the development of the 3D MOC solver in this thesis.

I would also like to thank my family: my parents Nancy and Sandy and my sister Genna. They have all been extremely supportive of my academic career and have been there for me every step of the way. From early grade school studies through graduate studies, they have always been encouraging. I would also like to thank my girlfriend Ankita for being there for me during the last difficult years of my graduate studies. She was incredibly patient during my long nights and weekends at the office, finishing my graduate work. She helped me remain focused on my work during the past years, allowing me to achieve my academic goals.