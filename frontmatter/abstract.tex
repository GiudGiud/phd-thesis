\begin{abstractpage}

The nuclear reactor physics community has long strived for deterministic neutron transport-based tools for full-core reactor analysis. A key challenge for deterministic methods is accurate reactor agnostic multi-group cross section (MGXS) generation. No systematic methodology of approximations has been developed which is broadly applicable to all reactor types. Monte Carlo (MC) presents the most accurate method for reactor agnostic multi-group cross section generation since it does not require the use of any local approximations to the flux. This improvement in accuracy comes at the computational expense of converging group constant tallies to acceptably low uncertainties. MC methods have increasingly been used to generate few group constants for coarse mesh diffusion, most notably by the Serpent MC code. This work investigates the use of MC methods to generate MGXS for high-fidelity whole-core transport codes.

This thesis is organized along two main themes. First, the efficacy of MGXS generation with MC for fine mesh transport calculations is rigorously assessed. Some of the approximations made by MC-based MGXS generation are quantified, including the angular, energy- and spatial-dependence of condensed MGXS. The second theme develops a novel methodology called inferential MGXS (\textit{i}MGXS) which capture local and global spatial self-shielding effects in MGXS for whole-core calculations. The \textit{i}MGXS scheme applies statistical clustering algorithms to accelerate the convergence rate of MGXS tallied on high-fidelity spatial meshes in Monte Carlo.

-talk about six heterogeneous benchmarks
-a few key results - errors, convergence
-potential for the future
 A series of case studies will be presented which compare the accuracy and convergence rate of the scheme with traditional MC-based MGXS generation.

\end{abstractpage}
