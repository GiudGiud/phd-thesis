\chapter{MGXS Generation with Monte Carlo}
\label{chap:mgxs-mc}

\begin{itemize}[noitemsep]
  \item mitigate all of the pain and effort of coming up with approximations!!!
  \item \ac{MC} is \emph{reactor agnostic}
  \item spatial self-shielding is natural
  \item look at prospectus for paragraph
\end{itemize}


%%%%%%%%%%%%%%%%%%%%%%%%%%%%%%%%%%%%%%%%%%%%%%%%%%%%%%%%%%%%%%%%%%%%%%%%%%%%%%%
\section{Overview of Monte Carlo Methods}
\label{sec:chap3-mc-overview}

Everything can be found in the MCNP~\cite{mcnpx2003manual}, Serpent~\cite{serpent2013manual} and OpenMC~\cite{openmc2016manual} manuals. This section follows the OpenMC manual~\cite{openmc2016manual}.

%%%%%%%%%%%%%%%%%%%%%%%%%%%%%%%%
\subsection{Monte Carlo Tallies}
\label{subsec:chap3-mc-tallies}

\begin{itemize}[noitemsep]
  \item scores/score function, filters
  \item analog vs. collision vs. tracklength tallies (Eqns. (2), (4), (9))
  \item use tilde above transport / scatter xs, and hat above estimates
\end{itemize}

\begin{dmath}
\label{eqn:chap3-tallies-general}
T = \int_{V} \int_{S} \int_{E}  f(\mathbf{r},\mathbf{\Omega},E)\psi(\mathbf{r},\mathbf{\Omega},E)\mathrm{d}E\mathrm{d}\mathbf{\Omega}\mathrm{d}\mathbf{r} 
\end{dmath}

In OpenMC parlance the integration bounds over space, angle and energy are known as \textit{filters}, while the response function $f$ is termed the \textit{scoring function} or more simply \textit{score}.

Use $i$ here? It is used to index nuclides\\
Need to define $W$ and $w_{i}$\\

Analog tallies:

\begin{dmath}
\label{eqn:chap3-tallies-analog}
\hat{R}_{x} = \frac{1}{W}\displaystyle\sum\limits_{i \in A} w_{i}
\end{dmath}

Collision tallies - recall that the reaction rates are defined by $R_{x} = \Sigma_{x}\phi$

\begin{dmath}
\label{eqn:chap3-tallies-collision-flux}
\hat{\phi} = \frac{1}{W}\displaystyle\sum\limits_{i \in C}\frac{w_{i}}{\Sigma_{t}(E_{i})}
\end{dmath}

\begin{dmath}
\label{eqn:chap3-tallies-collision-rxn}
\hat{R}_{x} = \frac{1}{W}\displaystyle\sum\limits_{i \in C}\frac{w_{i}\Sigma_{x}(E_{i})}{\Sigma_{t}(E_{i})}
\end{dmath}

Tracklength tallies - need to define $T$ as set of all particle trajectories within volume of interest

\begin{dmath}
\label{eqn:chap3-tallies-tracklength-flux}
  \hat{\phi} = \frac{1}{W}\displaystyle\sum\limits_{i \in T}w_{i}\Delta\mathbf{r}_{i}
\end{dmath}

\begin{dmath}
\label{eqn:chap3-tallies-tracklength-rxn}
  \hat{R}_{x} = \frac{1}{W}\displaystyle\sum\limits_{i \in T}w_{i}\Delta\mathbf{r}_{i}\Sigma_{x}(E_{i})
\end{dmath}


%%%%%%%%%%%%%%%%%%%%%%%%%%%%%%
\subsection{Sample Statistics}
\label{subsec:chap3-mc-stats}

\begin{itemize}[noitemsep]
  \item batch statistics
  \item sample mean (Eqn. (11))
  \item variance of the population (Eqn. (14)
  \item variance of the mean (Eqn. (17))
  \item root mean square convergence rate - mention correlation
\end{itemize}


\begin{dmath}
\label{eqn:chap3-sample-mean}
\bar{x} = \frac{1}{N} \displaystyle\sum\limits_{i=1}^{N} x_{i}
\end{dmath}

unbiased estimate of the sample variance:

\begin{dmath}
\label{eqn:chap3-variance-sample}
s^{2} = \frac{1}{N-1}\displaystyle\sum\limits_{i=1}^{N}\left(x_{i} - \bar{x}\right)^{2}
\end{dmath}

unbiased estimate of the variance of the mean:

\begin{dmath}
\label{eqn:chap3-variance-mean}
s_{\bar{x}}^{2} = \frac{1}{N-1}\left(\frac{1}{N}\displaystyle\sum\limits_{i=1}^{N}
x_{i}^{2} - \bar{x}^2\right)
\end{dmath}

The standard deviation will decrease in proportion to $s_{\bar{x}} \propto \nicefrac{1}{\sqrt{N}}$

\begin{dmath}
\label{eqn:chap3-conv-rate}
s_{\bar{x}} = \frac{\sigma}{\sqrt{N}} \approx \frac{s}{\sqrt{N}}
\end{dmath}


%%%%%%%%%%%%%%%%%%%%%%%%%%%%%%%%%%%%%%%%%%%%%%%%%%%%%%%%%%%%%%%%%%%%%%%%%%%%%%%
\section{\ac{MGXS} Generation with Monte Carlo}
\label{sec:chap3-mgxs-gen}

DIY \ac{MGXS} generation with \ac{MC} \\
Stochastic approx. to integrals in Section~\ref{sec:chap2-background} \\


%%%%%%%%%%%%%%%%%%%%%%%%
\subsection{Tally Types Needed for \ac{MGXS} Generation}
\label{subsec:chap3-tally-types}

\begin{itemize}[noitemsep]
  \item recall flux separability approx.
\end{itemize}

\subsubsection{General Form}
\label{subsubsec:chap3-gen-xs}

\subsubsection{Total and Transport Cross Sections}
\label{subsubsec:chap3-tot-xs}

\subsubsection{Scattering Matrices}
\label{subsubsec:chap3-scatt-mat}

\begin{itemize}[noitemsep]
  \item refer to adam/redmond's thesis for more details on scattering moments
\end{itemize}

\subsubsection{Fission Production Cross Sections}
\label{subsubsec:chap3-fiss-prod}

\subsubsection{Chi Fission Spectrum}
\label{subsubsec:chap3-chi}


%%%%%%%%%%%%%%%%%%%%%%%%%%%%%%%%%%%%
\subsection{Uncertainty Propagation}
\label{subsec:chap3-uncertainty-prop}

\begin{itemize}[noitemsep]
  \item uncertainty propagation eqns
  \item need a reference
  \item https://en.wikipedia.org/wiki/Propagation_of_uncertainty
\end{itemize}

define covariance\\
two random variables $X$ and $Y$ with standard deviations $\sigma_{X}$ and $\sigma_{Y}$\\

\begin{equation}
Z = aX \qquad,\qquad \sigma_{Z}^{2} = a^{2}\sigma_{X}^{2}
\end{equation}

\begin{equation}
Z = X + Y \qquad,\qquad \sigma_{Z}^{2} = \sigma_{X}^{2} + \sigma_{Y}^{2} + 2\sigma_{XY}
\end{equation}

\begin{equation}
Z = X - Y \qquad,\qquad \sigma_{Z}^{2} = \sigma_{X}^{2} + \sigma_{Y}^{2} - 2\sigma_{XY}
\end{equation}

\begin{equation}
Z = XY \qquad,\qquad \sigma_{Z}^{2} \approx Z^{2}\left[\left(\frac{\sigma_{X}}{X}\right)^{2} + \left(\frac{\sigma_{Y}}{Y}\right)^{2} + 2\frac{\sigma_{XY}}{Z}\right] \approx Z^{2}\left[\left(\frac{\sigma_{X}}{X}\right)^{2} + \left(\frac{\sigma_{Y}}{Y}\right)^{2}\right]
\end{equation}

\begin{equation}
Z = \frac{X}{Y} \qquad,\qquad \sigma_{Z}^{2} \approx Z^{2}\left[\left(\frac{\sigma_{X}}{X}\right)^{2} + \left(\frac{\sigma_{Y}}{Y}\right)^{2} - 2\frac{\sigma_{XY}}{Z}\right] \approx Z^{2}\left[\left(\frac{\sigma_{X}}{X}\right)^{2} + \left(\frac{\sigma_{Y}}{Y}\right)^{2}\right]
\end{equation}


%%%%%%%%%%%%%%%%%%%%%%%%%%%%%%%%%%%%%%%%%%%%%%%%%%%%%%%%%%%%%%%%%%%%%%%%%%%%%%%
\section{Review of Previous Efforts}
\label{sec:chap3-lit-review}

\begin{itemize}[noitemsep]
  \item \ac{MGXS} for coarse mesh diffusion
  \begin{itemize}[noitemsep]
    \item Serpent
    \item Redmond
    \item Pounders
  \end{itemize}
  \item \ac{MGXS} for fine mesh transport
  \begin{itemize}[noitemsep]
    \item Nelson with NDPP
  \end{itemize}
\end{itemize}


%%%%%%%%%%%%%%%%%%%%%%%%%%%%%%%%%%%%%%%%%%%%%%%%%%%%%%%%%%%%%%%%%%%%%%%%%%%%%%%
\section{A New Approach: A Latent Variable Model}
\label{sec:chap3-latent-variables}

\begin{itemize}[noitemsep]
  \item tally ``noise'' model
  \item spatial self-shielding treatment is natural in \ac{MC}
  \item challenge is slow convergence rate
  \item motivate clustering
\end{itemize}