\begin{abstractpage}

The development of high fidelity multi-group neutron transport-based simulation tools for full core Light Water Reactor (LWR) analysis has been a long-standing goal of the reactor physics community. While direct transport simulations have previously been far too computationally expensive, advances in computer hardware have allowed large scale simulations to become feasible. Therefore, many have focused on developing full core neutron transport solvers that do not incorporate the approximations and assumptions of traditional nodal diffusion solvers. 

Due to the computational expense of direct full core 3D deterministic neutron transport methods, many have focused on 2D/1D methods which solve 3D problems as a coupled system of radial and axial transport problems. However, the coupling of radial and axial problems also introduces approximations. Instead, the work in this thesis focuses on explicitly solving the 3D deterministic neutron transport equations with the Method of Characteristics (MOC).

MOC has been widely used for 2D lattice physics calculations due to its ability to accurately and efficiently simulate reactor physics problems with explicit geometric detail. The work in this thesis strives to overcome the significant computational cost of solving the 3D MOC equations by implementing efficient track generation, axially extruded ray tracing, Coarse Mesh Finite Difference (CMFD) acceleration, linear track-based source approximations, and scalable domain decomposition. Transport-corrected cross-sections are used to account for anisotropic without needing to store angular-dependent sources.

Additionally, significant attention has been given to complications that arise in full core simulations with transport-corrected cross-sections. The convergence behavior of transport methods is analyzed, leading to a new strategy for stabilizing the source iteration scheme for neutron transport simulations. The methods are incorporated into the OpenMOC reactor physics code and simulation results are presented for the full core BEAVRS LWR benchmark. Parameter refinement studies and comparisons with reference OpenMC Monte Carlo solutions show that converged full core 3D MOC simulations are feasible on modern supercomputers for the first time.

%However, 3D full core LWR simulations present significant challenges due to greatly increased computational cost.

%The Method of Characteristics (MOC) has seen wide interest in reactor physics because of its accuracy and efficiency in computing lattice physics problems. While most of its use has been in solving 2D problems, there has been recent interest in extending MOC to 3D in order to more accurately calculate 3D power distributions in LWRs. While the method is naturally extensible to 3D, it presents significant computational difficulties. Methods will be presented which mitigate the computational difficulties of 3D MOC by using domain decomposition, efficient track generation, axially extruded ray tracing, CMFD acceleration, and a linear source approximation. Significant attention will be given to complications that arise in full core simulations. 3D MOC results will be presented for the full core simulation of the BEAVRS benchmark, showing that 3D MOC can be a viable tool for anal

\end{abstractpage}