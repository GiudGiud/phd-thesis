\begin{appendices}
\chapter{OpenMC Particle Data Structure}
\label{app:particle_struct}

Since particles must be communicated across domain boundaries for a
domain-decomposed problem, the amount of information that must travel with
particles can have important implications given the bandwidth of node
interconnects. Every attempt should be made to minimize the amount
of sent over the network, and data should be packed in a way that aligns in
contiguous memory for best performance.

Listing~\ref{lst:particle} shows the full OpenMC data structure for particles.
Many of the members of this structure facilitate particle tracking local to a
domain and thus don't need to be communicated.  Listing~\ref{lst:particle_buff}
shows the particle buffer data structure for sending particles over the network,
which mirrors the full particle data structure.  However, it is stripped down to
the base essentials needed to maintain random number reproducibility as
discussed in Section~\ref{sec:rng_repro}. To that end several additional members
are included here, notably the random number seeds for tracking, tallying, and
cross section probability table sampling. The ordering of fields and the
sequence Fortran keyword are important for efficient storage and data transfer
with custom MPI datatypes derived from the structure.

With integers, longs, and doubles requiring four, eight, and
eight bytes each, respectively, we can see from Listing~\ref{lst:particle_buff}
that OpenMC will need to send 220 bytes per particle across the network,
assuming two random number streams (one for tracking, and one for tallies).

\singlespacing
\begin{lstlisting}[language=Fortran, frame=none, keepspaces=true, frame=single, linewidth=6in, xleftmargin=0.0in,
caption={Full OpenMC particle data structure. \label{lst:particle}}]

type Particle
  ! Basic data
  integer(8) :: id            ! Unique ID
  integer    :: type          ! Particle type (n, p, e, etc)

  ! Particle coordinates
  type(LocalCoord), pointer :: coord0 => null() ! coordinates on universe 0
  type(LocalCoord), pointer :: coord  => null() ! coordinates on lowest univ

  ! Other physical data
  real(8)    :: wgt           ! particle weight
  real(8)    :: E             ! energy
  real(8)    :: mu            ! angle of scatter
  logical    :: alive         ! is particle alive?

  ! Pre-collision physical data
  real(8)    :: last_xyz(3)   ! previous coordinates
  real(8)    :: last_uvw(3)   ! previous direction coordinates
  real(8)    :: last_wgt      ! pre-collision particle weight
  real(8)    :: last_E        ! pre-collision energy
  real(8)    :: absorb_wgt    ! weight absorbed for survival biasing

  ! What event last took place
  logical    :: fission       ! did the particle cause implicit fission
  integer    :: event         ! scatter, absorption
  integer    :: event_nuclide ! index in nuclides array
  integer    :: event_MT      ! reaction MT

  ! Post-collision physical data
  integer    :: n_bank        ! number of fission sites banked
  real(8)    :: wgt_bank      ! weight of fission sites banked

  ! Indices for various arrays
  integer    :: surface       ! index for surface particle is on
  integer    :: cell_born     ! index for cell particle was born in
  integer    :: material      ! index for current material
  integer    :: last_material ! index for last material
  
  ! Statistical data
  integer    :: n_collision   ! # of collisions

  ! Track output
  logical    :: write_track = .false.

  ! Distributed Mapping Info
  integer, allocatable :: mapping(:)  ! records the current sum of all maps
  integer              :: inst        ! records the current material instance
  integer              :: last_inst   ! records the previous material instance

  ! Was this particle just created?
  logical    :: new_particle = .true.

  ! Data needed to restart a particle stored in a bank after changing domains
  real(8)    :: stored_xyz(3)
  real(8)    :: stored_uvw(3)
  real(8)    :: stored_distance ! sampled distance to go after changing domain
  integer(8) :: prn_seed(N_STREAMS) ! the  next random number seed 
  integer(8) :: xs_seed(N_STREAMS)  ! the previously seed used for xs gen

  ! Domain information
  integer    :: dd_meshbin    ! DD meshbin the particle is to be run in next
  integer    :: outscatter_destination ! Which domain to transmit particle to

end type Particle

\end{lstlisting}
\onehalfspacing

\singlespacing
\begin{lstlisting}[language=Fortran, frame=none, keepspaces=true, frame=single, linewidth=6in, xleftmargin=0.0in,
caption={OpenMC particle buffer data structure. \label{lst:particle_buff}}]

type ParticleBuffer

  sequence

  integer(8) :: id
  integer(8) :: prn_seed(N_STREAMS)
  integer(8) :: xs_seed(N_STREAMS)
  real(8)    :: wgt
  real(8)    :: E
  real(8)    :: mu
  real(8)    :: last_wgt
  real(8)    :: last_E
  real(8)    :: absorb_wgt
  real(8)    :: wgt_bank
  real(8)    :: stored_distance
  real(8)    :: last_xyz(3)
  real(8)    :: stored_xyz(3)
  real(8)    :: stored_uvw(3)
  integer    :: type
  integer    :: event
  integer    :: event_nuclide
  integer    :: event_MT
  integer    :: n_bank
  integer    :: surface
  integer    :: cell_born
  integer    :: n_collision
  integer    :: material
  integer    :: last_material
  integer    :: last_inst

end type ParticleBuffer

\end{lstlisting}
\onehalfspacing

\chapter{Depletion Nuclides}
\label{app:nuclides}

The set of 206 nuclides used for depletion tally runs was taken from an ORIGEN-S
verification and validation report \cite{nuclides206}.

\begin{table}[h!]
  \begin{tabular}{|r|r|r|r|r|r|r|r|r|}
    \toprule
    Ag-107 & Ce-143 & Gd-156 & Lu-176 & Pd-108 & Ru-104 & Sn-118 & Th-232 & Zr-93 \\
    Ag-109 & Ce-144 & Gd-157 & Mo-100 & Pd-110 & Ru-105 & Sn-119 & U-233 & Zr-94 \\
    Ag-111 & Cm-244 & Gd-158 & Mo-94 & Pm-147 & Ru-106 & Sn-120 & U-234 & Zr-95 \\
    Am-241 & Co-59 & Gd-160 & Mo-95 & Pm-148 & Ru-99 & Sn-122 & U-235 & Zr-96 \\
    Am-243 & Cs-133 & Ge-72 & Mo-96 & Pm-149 & Sb-121 & Sn-123 & U-236 & \\
    As-75 & Cs-134 & Ge-73 & Mo-97 & Pm-151 & Sb-123 & Sn-124 & U-238 & \\
    Au-197 & Cs-135 & Ge-74 & Mo-98 & Pr-141 & Sb-124 & Sn-125 & W-182 & \\
    Ba-134 & Cs-136 & Ge-76 & Mo-99 & Pr-142 & Sb-125 & Sn-126 & W-183 & \\
    Ba-135 & Cs-137 & Ho-165 & Nb-93 & Pr-143 & Sb-126 & Sr-86 & W-184 & \\
    Ba-136 & Dy-160 & I-127 & Nb-94 & Pu-238 & Se-76 & Sr-87 & W-186 & \\
    Ba-137 & Dy-161 & I-129 & Nb-95 & Pu-239 & Se-77 & Sr-88 & Xe-128 & \\
    Ba-138 & Dy-162 & I-130 & Nd-142 & Pu-240 & Se-78 & Sr-89 & Xe-129 & \\
    Ba-140 & Dy-163 & I-131 & Nd-143 & Pu-241 & Se-80 & Sr-90 & Xe-130 & \\
    Br-79 & Dy-164 & I-135 & Nd-144 & Pu-242 & Se-82 & Ta-181 & Xe-131 & \\
    Br-81 & Er-166 & In-113 & Nd-145 & Rb-85 & Sm-147 & Tb-159 & Xe-132 & \\
    Cd-108 & Er-167 & In-115 & Nd-146 & Rb-86 & Sm-148 & Tb-160 & Xe-133 & \\
    Cd-110 & Eu-151 & Kr-80 & Nd-147 & Rb-87 & Sm-149 & Tc-99 & Xe-134 & \\
    Cd-111 & Eu-152 & Kr-82 & Nd-148 & Re-185 & Sm-150 & Te-122 & Xe-135 & \\
    Cd-112 & Eu-153 & Kr-83 & Nd-150 & Re-187 & Sm-151 & Te-123 & Xe-136 & \\
    Cd-113 & Eu-154 & Kr-84 & Np-237 & Rh-103 & Sm-152 & Te-124 & Y-89 & \\
    Cd-114 & Eu-155 & Kr-85 & Pa-233 & Rh-105 & Sm-153 & Te-125 & Y-90 & \\
    Cd-116 & Eu-156 & Kr-86 & Pd-104 & Ru-100 & Sm-154 & Te-126 & Y-91 & \\
    Ce-140 & Eu-157 & La-139 & Pd-105 & Ru-101 & Sn-115 & Te-128 & Zr-90 & \\
    Ce-141 & Gd-154 & La-140 & Pd-106 & Ru-102 & Sn-116 & Te-130 & Zr-91 & \\
    Ce-142 & Gd-155 & Lu-175 & Pd-107 & Ru-103 & Sn-117 & Te-132 & Zr-92 & \\
    \bottomrule
  \end{tabular}
\end{table}

\end{appendices}
