\chapter{Multi-Group Transport Theory}
\label{chap:mgxs}

%The continuous energy neutron transport equation describes the distribution of neutrons as a function of time, position, energy and direction of motion. The transport equation equates the rate of change of the population of neutrons in some closed system of interest to the difference between the production rate and loss rate of neutrons within that system. 

The steady-state continuous energy transport equation models the distribution of neutrons within a six-dimensional phase space with respect to a neutron's position $\mathbf{r}$, energy $E$ and direction of travel $\mathbf{\Omega}$\footnote{Vector-valued quantities are expressed in boldface font.}. The transport equation balances the rate of change of the population of neutrons to the difference between the production and loss rates of neutrons within some closed system. The transport equation is integro-differential in the neutron angular flux $\psi(\mathbf{r},\mathbf{\Omega},E)$:

\begin{dmath}
\label{eqn:chap1-eqn1}
\mathbf{\Omega} \cdot \nabla \psi(\mathbf{r},\mathbf{\Omega},E) + \Sigma_{t}(\mathbf{r},E)\psi(\mathbf{r},\mathbf{\Omega},E) = \;\;\; \int_{0}^{\infty}\int_{4\pi} \Sigma_{s}(\mathbf{r},{\mathbf{\Omega'}\rightarrow\mathbf{\Omega}},{E'\rightarrow E}) \psi(\mathbf{r},\mathbf{\Omega'},E') \mathrm{d}\mathbf{\Omega'} \mathrm{d}E' + Q(\mathbf{r},\mathbf{\Omega},E)
\end{dmath}

The angular flux $\psi(\mathbf{r},\mathbf{\Omega},E)$ is the density distribution of neutrons in phase space multiplied by the neutron velocity. On the left hand side of the equation, the first term represents the streaming of neutrons within space and the second term is the total neutron collision rate determined by the total scattering cross section $\Sigma_{t}$. On the right hand side, the first term models the scattering of neutrons at energy $E'$ and direction $\Omega'$ into some new energy $E$ and direction $\Omega$. The final term represents a generic source $Q$ of neutrons. In the case of critical systems, such as nuclear reactors, $Q$ is a source of fission neutrons:

%, such that $\psi(\mathbf{r},\mathbf{\Omega},E)dVd\mathbf{\Omega}dE$ is the total distance travelled by neutrons per unit time in volume $dV$ centered at $\mathbf{r}$ with energies $dE$ about $E$ and directions of motion $d\mathbf{\Omega}$ about $\mathbf{\Omega}$

\begin{dmath}
\label{eqn:chap1-source}
Q(\mathbf{r},\mathbf{\Omega},E) = \frac{1}{4\pi k_{eff}}\int_{0}^{\infty}\int_{4\pi} \nu\Sigma_{f}(\mathbf{r},E')\psi(\mathbf{r},\mathbf{\Omega'},{E'\rightarrow E}) \mathrm{d}\mathbf{\Omega'} \mathrm{d}E'
\end{dmath}

The fission source is 

The total cross section $\Sigma_{t}(\mathbf{r},E)$ represents t


*we want to prepare the reader to understand the approximations introduced next
-explain the total xs, scattering xs
-explain the fission source here

In this equation, the neutron angular flux $\psi(\mathbf{r},\mathbf{\Omega},E)$ is defined in terms of the total, scattering and fission cross sections. 

second paragraph:
-define terms within the equation
-mention phase space volume (nelson thesis)

first paragraph:
-used for nuclear reactor physics calculations
-compute reaction rates

Each of the variables in use is defined in \autoref{tab:chap1-variables}. This is a balance equation between neutrons lost to transport, lost to absorption, produced or lost from scattering and those produced from fission. It should be noted that this equation assumes isotropic emission from fission.

\begin{table}[hbt]
  \caption{Variables in the Boltzmann equation.}
  \label{tab:chap1-variables}
  \begin{center}
    \begin{tabular}{ l l }
    \toprule
    Variable & Description \\
    \midrule
    $k_{eff}$ & Effective neutron multiplication factor \\
    $\Sigma^T$ & Neutron total cross section \\
    $\Sigma^S$ & Neutron scattering cross section \\
    $\Sigma^F$ & Neutron fission cross section \\
    $\chi$ & Energy spectrum for fission neutrons \\
    $\nu$ & Average number of neutrons emitted per fission \\
    \bottomrule
  \end{tabular}
  \end{center}
\end{table}

The first step is to simplify this equation by defining those quantities on the right hand side as the total neutron source $Q(\mathbf{r},\mathbf{\Omega},E)$:

\begin{dmath}
\label{eqn:chap1-source}
Q(\mathbf{r},\mathbf{\Omega},E) = \int_{0}^{\infty} \mathrm{d}E' \int_{4\pi} \mathrm{d}\mathbf{\Omega'}\Sigma^S(\mathbf{r},{\mathbf{\Omega'}\rightarrow\mathbf{\Omega}},{E'\rightarrow E}) \Psi(\mathbf{r},\mathbf{\Omega'},E') + \frac{\chi(\mathbf{r},E)}{4\pi k_{eff}} \int_{0}^{\infty} \mathrm{d}E' \int_{4\pi} \mathrm{d}\mathbf{\Omega'} \nu\Sigma^F(\mathbf{r},E')\Psi(\mathbf{r},\mathbf{\Omega'},E')
\end{dmath}

The transport equation can now be more concisely written as follows:

\begin{equation}
\label{eqn:chap1-transport-eqn-src}
\mathbf{\Omega} \cdot \nabla \Psi(\mathbf{r},\mathbf{\Omega},E) + \Sigma^T(\mathbf{r},E)\Psi(\mathbf{r},\mathbf{\Omega},E) = Q(\mathbf{r},\mathbf{\Omega},E)
\end{equation}



%%%%%%%%%%%%%%%%%%%%%%%%%%%%%%%%%%%%%%%%%%%%%%%%%%%%%%%%%%%%%%%%%%%%%%%%%%%%%%%
\section{Background}
\label{sec:chap2-background}





-discuss each approximation in sequence, building approximations upon one another similar to \ac{MOC} derivation in M.S. thesis??

\begin{itemize}[noitemsep]
  \item continuous energy neutron transport equation
  \item \ac{MGXS} equations
  \begin{itemize}[noitemsep]
    \item generic reaction
    \item scattering matrix
    \item chi fission spectrum
  \end{itemize}
  \item equation and figures mapped to variables
  \item tally ``noise'' model
  \item uncertainty propagation equations
  \item mention that this intro is agnostic to the method used to solve the transport eqn.
   \item define subscripts for $\Sigma$ and $R$ - $a$, $f$, $\gamma$, $t$, $s$
   \item define neutron transport equation with generic source $Q$
   \item then define source specifically for criticality problem
   \item energy indices are defined in order for decreasing energy
   \item in constant-in-angle section, mention Nate Eqn. 1.13 as one possible approach for deterministic methods for the collapse. In MC we use the angular flux in the collapse though.
   \item cite boltzmann transport equation (see adam's thesis pg. 1) 
   \item tie back to experimental data and its uncertainties - MC is only as accurate as measured reaction data (maybe in the intro chap. 1)
   \item mention recent work to use MC for MGXS (chap. 3 of nelson's thesis)
   \item define angular flux - neutron angular density multiplied by neutron speed
   \item phase space defn from adam's thesis pg 5 - as tot dist traveled
   \item talk about (n,xn) reactions
   \item keep fission as matrix early on
   \item figure 2.2 from adam's thesis to describe typical approach
\end{itemize}


%%%%%%%%%%%%%%%%%%%%%%%%%%%
\subsection{Approximations}
\label{subsec:chap2-approx}

\begin{itemize}
  \item motivate bias studies in chapter 4
\end{itemize}

Equation~\ref{eqn:chap1-transport-eqn-src} is defined with $\Psi$, $Q$ and $\Sigma^T$ as continuous functions of energy. The first approximation to numerically solve this equation is to discretize the energy domain into distinct \textit{energy groups} $g \in G = \{1, 2, ..., G\}$ where group $g$ spans the continuous range of energies from $E_{g}$ to $E_{g-1}$. This is otherwise known as the \textit{multi-group approximation}. The multi-group form of the Boltzmann equation is presented below:

\begin{equation}
\label{eqn:chap2-transport-eqn-mg}
\mathbf{\Omega} \cdot \nabla \Psi_g(\mathbf{r},\mathbf{\mathbf{\Omega}}) + \Sigma^T_{g}(\mathbf{r})\Psi_g(s,\mathbf{\Omega}) = Q_g(\mathbf{r},\mathbf{\Omega})
\end{equation}

Likewise, the multi-group form of the neutron source (Equation~\ref{eqn:chap1-source}) is given by:

\begin{dmath}
\label{eqn:chap2-source-mg}
Q_g(\mathbf{r},\mathbf{\Omega}) = \displaystyle\sum\limits_{g'=1}^G \int_{4\pi} \mathrm{d}\mathbf{\Omega'}\Sigma_{g'\rightarrow g}^S(\mathbf{r},{\mathbf{\Omega'}\rightarrow\mathbf{\Omega}}) \Psi_{g'}(\mathbf{r},\mathbf{\Omega'}) + \\ \frac{\chi_{g}(\mathbf{r})}{4\pi k_{eff}} \displaystyle\sum\limits_{g'=1}^G \nu\Sigma_{g'}^F(\mathbf{r}) \int_{4\pi} \mathrm{d}\mathbf{\Omega'} \Psi_{g'}(\mathbf{r},\mathbf{\Omega'})
\end{dmath}

\begin{equation}
\label{eqn:moc-theory-condensed-total-xs}
\Sigma_{g}^T(\mathbf{r}) = \frac{\int_{E_{g}}^{E_{g-1}}\mathrm{d}E'\Sigma^T(\mathbf{r},E')\Psi(\mathbf{r},\mathbf{\Omega},E')}{\int_{E_{g}}^{E_{g-1}}\mathrm{d}E'\Psi(\mathbf{r},\mathbf{\Omega},E')}
\end{equation}

\begin{equation}
\label{eqn:moc-theory-condensed-fission-xs}
\Sigma_{g}^F(\mathbf{r}) = \frac{\int_{E_{g}}^{E_{g-1}}\mathrm{d}E'\Sigma^F(\mathbf{r},E')\Psi(\mathbf{r},\mathbf{\Omega},E')}{\int_{E_{g}}^{E_{g-}1}\mathrm{d}E'\Psi(\mathbf{r},\mathbf{\Omega},E')}
\end{equation}

\begin{equation}
\label{eqn:moc-theory-condensed-scatter-xs}
\Sigma_{g'\rightarrow g}^S(\mathbf{r},\mathbf{\Omega'}\rightarrow \mathbf{\Omega}) = \frac{\int_{E_{g'}}^{E_{g'-1}}\mathrm{d}E'\int_{E_{g}}^{E_{g-1}}\mathrm{d}E''\Sigma^S(\mathbf{r},\mathbf{\Omega'}\rightarrow \mathbf{\Omega},E'\rightarrow E'')\Psi(\mathbf{r},\mathbf{\Omega'},E')}{\int_{E_{g'}}^{E_{g'-1}}\mathrm{d}E'\Psi(\mathbf{r},\mathbf{\Omega'},E')}
\end{equation}

\begin{equation}
\label{eqn:moc-theory-condensed-chi}
\chi_{g'\rightarrow g}(\mathbf{r}) = \frac{\int_{E_{g'}}^{E_{g'-1}}\mathrm{d}E'\int_{E_{g}}^{E_{g-1}}\mathrm{d}E''\chi(\mathbf{r},E'\rightarrow E'')\nu\Sigma^F(\mathbf{r},\mathbf{\Omega},E')\Psi(\mathbf{r},\mathbf{\Omega'},E')}{\int_{E_{g'}}^{E_{g'-1}}\mathrm{d}E'\nu\Sigma^F(\mathbf{r},\mathbf{\Omega},E')\Psi(\mathbf{r},\mathbf{\Omega'},E')}
\end{equation}

Although \autoref{eqn:moc-theory-condensed-chi} assumes a dependence of $\chi$ on both the energy of the neutron causing fission $g'$ and the fission emission energy group $g$, the former is typically summed over to simplify the multi-group $\chi$ to the following approximation:

\begin{equation}
\label{eqn:moc-theory-condensed-chi-group}
\chi_{g}(\mathbf{r}) = \displaystyle\sum\limits_{g=1}^{G}\chi_{g'\rightarrow g}(\mathbf{r})
\end{equation}


%%%%%%%%%%%%%%%%%%%%%%%%%%%%%%%%%%%%%
\subsubsection{Energy Discretization}
\label{subsubsec:chap2-energy-discrete}

%%%%%%%%%%%%%%%%%%%%%%%%%%%%%%%%%%%%%%
\subsubsection{Spatial Discretization}
\label{subsubsec:chap2-space-discrete}

In addition to the flat source approximation, it is assumed that the material properties are constant across each FSR. The area-averaged cross sections for FSR $i \in \{1, 2, ..., I\}$ with area $A_{i}$ are defined as:

\begin{equation}
\label{eqn:chap2-area-avg-total-xs}
\Sigma_{i,g}^{T} = \frac{\int_{\mathbf{r}\in A_{i}}\mathrm{d}\mathbf{r}\Sigma_{g}^T(\mathbf{r})\Phi_{g}(\mathbf{r})}{\int_{\mathbf{r}\in A_{i}}\mathrm{d}\mathbf{r}\Phi_{g}(\mathbf{r})}
\end{equation}

\begin{equation}
\label{eqn:chap2-area-avg-fission-xs}
\Sigma_{i,g}^{F} = \frac{\int_{\mathbf{r}\in A_{i}}\mathrm{d}\mathbf{r}\Sigma_{g}^F(\mathbf{r})\Phi_{g}(\mathbf{r})}{\int_{\mathbf{r}\in A_{i}}\mathrm{d}\mathbf{r}\Phi_{g}(\mathbf{r})}
\end{equation}

\begin{equation}
\label{eqn:chap2-area-avg-scatter-xs}
\Sigma_{i,g'\rightarrow g}^{S} = \frac{\int_{\mathbf{r}\in A_{i}}\mathrm{d}\mathbf{r}\Sigma_{g'\rightarrow g}^S(\mathbf{r})\Phi_{g'}(\mathbf{r})}{\int_{\mathbf{r}\in A_{i}}\mathrm{d}\mathbf{r}\Phi_{g'}(\mathbf{r})}
\end{equation}

\begin{equation}
\label{eqn:chap2-area-avg-chi}
\chi_{i,g} = \frac{\int_{\mathbf{r}\in A_{i}}\mathrm{d}\mathbf{r}\chi_{g}(\mathbf{r})}{\int_{\mathbf{r}\in A_{i}}\mathrm{d}\mathbf{r}}
\end{equation}


%%%%%%%%%%%%%%%%%%%%%%%%%%%%%%%%%%%%
\subsubsection{Isotropic Scattering}
\label{subsubsec:chap2-iso-scatter}

\begin{itemize}
  \item mention ``iso-in-lab'' feature in OpenMC
\end{itemize}

%%%%%%%%%%%%%%%%%%%%%%%%%%%%%%%%%
\subsubsection{Scalar Flux Weighting}
\label{subsubsec:chap2-const-in-angle}

\begin{itemize}[noitemsep]
  \item connect to Nate's ``batman'' plot
\end{itemize}


%%%%%%%%%%%%%%%%%%%%%%%%%%%%%%%%%%%%%%%%%%%%%%%%%%%%%%%%%%%%%%%%%%%%%%%%%%%%%%%
\section{MGXS Generation with Monte Carlo}
\label{sec:chap2-mgxs-mc}

\begin{itemize}[noitemsep]
  \item \ac{MC} is \emph{reactor agnostic}
  \item stochastic approx. to integrals in Section~\ref{sec:chap2-background}
\end{itemize}

%%%%%%%%%%%%%%%%%%%%%%
\subsection{Past Work}
\label{subsec:chap2-past-work}

\begin{itemize}[noitemsep]
  \item \ac{MGXS} for coarse mesh diffusion
  \begin{itemize}[noitemsep]
    \item Redmond
    \item Pounders
    \item Serpent
  \end{itemize}
  \item \ac{MGXS} for fine mesh transport
  \begin{itemize}[noitemsep]
    \item Nelson with NDPP
  \end{itemize}
\end{itemize}

%%%%%%%%%%%%%%%%%%%%%%%%%%%%%%%%%%%%%%%%%%%%%
\subsection{Spatial Self-Shielding Treatment}
\label{subsec:chap2-spatial-shield}

\begin{itemize}[noitemsep]
  \item spatial self-shielding treatment is natural in \ac{MC}
  \item challenge is slow convergence rate
  \item motivate clustering
\end{itemize}