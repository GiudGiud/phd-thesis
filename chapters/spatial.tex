\chapter{Structural Patterns in Pin-Wise MGXS}
\label{chap:spatial}

The preceding chapter quantified the benefit of using degenerate spatial homogenization to best predict the spatial distribution of reaction rates -- most notably, pin-wise U-238 capture rates -- with high-fidelity multi-group transport methods. However, it was also noted that the degenerate scheme requires far more Monte Carlo particle histories to converge \ac{MGXS} tallies than are necessary for the simpler null or infinite schemes. In addition, orders of magnitude more memory is needed to store \ac{MGXS} libraries produced from degenerate homogenization. This observations motivate the need for a more sophisticated approach to pin-wise spatial homogenization which can simultaneously achieve nearly the accuracy of degenerate homogenization and the convergence rate of null homogenization.

This chapter seeks to accomplish this by leveraging the finding from Chap.~\ref{chap:quantify} that pins with similar neighboring heterogeneities generally have similar reaction rate errors. For example, null homogenization led to a structured spatial distribution of errors with systematically similar errors in pins facially adjacent to one or two \acp{CRGT}, \acp{BP}, along assemby-assembly or assembly-reflector interfaces, and so on. Since degenerate homogenization largely erased this structural error distribution, it follows that pins with similar errors likely experience similar spatial self-shielding effects due to neighboring heterogeneities. As a result, this and the following chapters develop the hypothesis that \textbf{pins with similar neighboring heterogeneities have similar microscopic \ac{MGXS}}. If pins with similar microscopic \ac{MGXS} can be identified, the \ac{MGXS} tallied in these pin instances may be \textit{averaged} to compute an estimate which is nearly as accurate as the \ac{MGXS} from degenerate homogenization, and nearly as converged as the \ac{MGXS} from null homogenization. 

This chapter investigates this hypothesis by analyzing the pin-wise \ac{MGXS} tallied with OpenMC to identify structural patterns -- namely, clustering -- for pins with similar neighbors. Furthermore, this chapter develops and quantifies a new spatial homogenization technique which analyzes a core geometry to predict which fuel pin instances have similar microscopic \ac{MGXS} due to neighboring heterogeneities. This technique applies OpenCG's Local Neighbor Symmetry algorithm to generate a ``geometric template'' of fuel pins and averages the \ac{MGXS} for pins across a core geometry with the same \ac{LNS} identifiers. The impact of using \ac{LNS} homogenization is compared to the null and degenerate schemes with respect to both general predictive accuracy as well as convergence rate. The results presented for \ac{LNS} homogenization underscore the promise for an approach which combines the benefits of the null and degenerate schemes. However, the results also highlight some notable shortcomings to \ac{LNS} which motivates the need for an unsupervised approach to \ac{MGXS} clustering, as developed in the following chapter.

This chapter begins by analyzing structural patterns in visualizations of pin-wise \ac{MGXS} in Sec.~\ref{sec:chap9-clustering}. This includes a case study of the population variance of pin-wise \ac{MGXS} in Sec.~\ref{subsec:chap9-pop-var}, and an analysis of the distributions of U-235 fission and U-238 capture \ac{MGXS} for each of the six heterogeneous benchmarks with histograms and quantile-quantile plots in Secs.~\ref{subsec:chap9-histograms} and ~\ref{subsec:chap9-qq-plots}, respectively. A new spatial homogenization scheme based upon OpenCG's \ac{LNS} algorithm is introduced in Sec.~\ref{sec:chap9-lns-homogenize}. The OpenMOC eigenvalues and pin-wise fission and U-238 capture rates with \ac{LNS} spatial homogenization are presented in Sec.~\ref{subsec:chap9-lns-results}. Finally, the convergence rate for \ac{MGXS} generated with the null, degenerate and \ac{LNS} schemes are compared in Sec.~\ref{sec:chap9-convergence}.


%%%%%%%%%%%%%%%%%%%%%%%%%%%%%%%%%%%%%%%%%%%%%%%%%%%%%%%%%%%%%%%%%%%%%%%%%%%%%%%
\section{Clustering of Pin-Wise MGXS}
\label{sec:chap9-clustering}

first paragraph: postulate existence of clusters
-use infinite lattice example as ``thought experiment''
  -the \ac{MGXS} estimates for each pin from \ac{MC} tallies for an inf. lattice will be samples from a normal distribution
-spatial self-shielding experienced by different fuel pins induces clustering in pin-wise MGXS
  -i.e., differential moderation from \acp{CRGT} perturbs/softens flux in nearby pins
  -this affects the \ac{MGXS} in certain energy groups for certain nuclides
-explain why clustering only occurs for micro xs rather than macro
  -only way to compare differential impact of self-shielding on different pins is with micros
  -the macros will necessarily ``cluster'' according to nuclide density
  
second paragraph: intro to visualizations
-can be informative to illustrate \ac{MGXS} clusters with an exploration of the data through visualizations
-introduce infinite lattice geometries w/o \acp{CRGT} and w/o \acp{BP}
-section looks at different types of visualizations to understand
-note that although we identified U-238 capture \ac{MGXS} as being more impacted by pin-wise spatial homogenization in the last chapter, we present visualizations of U-235 fission here
  -other reaction rates may also be used to identify clustering in \ac{MGXS}
  -even if the clusters in those \ac{MGXS} don't make as much of an impact for their respective reaction types (e.g., fission vs. capture)

third paragraph: outline
-Sec.~\ref{subsec:chap9-pop-var} - population variance of \ac{MGXS} in each benchmark
-Sec.~\ref{subsec:chap9-histograms} - histograms plots of pin-wise \ac{MGXS}
  -for both U-238 capture and U-235 fission
  -although 
-Sec.~\ref{subsec:chap9-qq-plots} - quantile-quantile plots of pin-wise \ac{MGXS}

%%%%%%%%%%%%%%%%%%%%%%%%%%%%%%%%%%%%%%%%%%%%%%
\subsection{Pin-Wise MGXS Population Variance}
\label{subsec:chap9-pop-var}

first paragraph: heterogeneities induce a dispersion of pin-wise \ac{MGXS}
-hypothesize that 
  -evidence from degenerate scheme in preceding chapter backs this up
-Tab.~\ref{table:chap9-pop-var-mgxs} shows the population variance of pin-wise \ac{MGXS}
  -each sample in the population represents a single fuel pin
    -used OpenMC distributed cell tallies
  -for each benchmark, for both 1.6\% and 3.1\% enriched fuel pins
  	-note that there are 2.4\% enriched fuel pins in the \ac{BEAVRS} quarter core model
  -for group 1/2 U-238 capture \ac{MGXS}
  -for group 2/2 U-235 fission \ac{MGXS}

second paragraph: identify trends
-following sections seek to characterize structure in the dispersion of \ac{MGXS}
-inf lattice vs. not
  -dispersion / variance is over two orders of magnitude larger with the presence of heterogeneities
    -for same \ac{MC} track density and uncertainties for each pin's \ac{MGXS}
    -for both fission and capture 
-by fission vs. capture rxns
-by GTs
  -presence of GTS increases dispersion by two orders of magnitude
-by BPs
  -adding BPs actually slightly reduces dispersion of both capture \ac{MGXS}
  -increases dispersion by nearly 3$\times$ for fission \ac{MGXS}
-by enrichment
  -dispersion is greater for 3.1\% enr. pins - esp. for fission - over 2$\times$ larger!! 
    -e.g., 6.78 vs. 16.8 barns for 1.6\% and 3.1\% enr. pins in the \ac{BEAVRS} quarter core)
    -fission \ac{MGXS} is more sensitive to self-shielding
    -BUT fission rxn rates were not improved with degenerate homogenization
      -perhaps fission is still useful to identify clusters to improve U-238 capture predictions
-inter-assembly
  -increases capture dispersion by 10--20\%
  -increases fission dispersion by 5$\times$ (wrt assms w/o BPs) 
-by reflector
  -increases capture dispersion 5--15\%
  -increases fission dispersion by 60\% and over 100\% for 1.6\% and 3.1\% enr. fuel 
-by BEAVRS
  -dispersion decreases for BEAVRS wrt colorset w/ reflector by 1/2 -- 1/3 
  -dispersion increases slightly for U-238 capture in 3.1\% enr. fuel

%-plot the convergence of null MGXS by batch along with max for distribcell MGXS
%-plot the pop. var. convergence of the distribcell MGXS and show that they don't go to zero

\begin{table}[h!]
  \centering
  \caption[Population variance for pin-wise MGXS]{The population variance for pin-wise fission and U-238 capture \ac{MGXS}.}
  \small
  \label{table:chap9-pop-var-mgxs}
  \vspace{6pt}
  \begin{tabular}{C{2.5cm} l C{2.5cm} C{2.5cm}}
  \toprule
  \rowcolor{lightgray}
  \multicolumn{1}{C{2.5cm}}{\textbf{Fuel Enrichment}} & \multicolumn{1}{c}{\textbf{Benchmark}} & \boldmath$\mathrm{Var}\left[\sigma_{c}^{U-238}\right]$ \textbf{[barns]} & \boldmath$\mathrm{Var}\left[\sigma_{f}\right]$ \textbf{[barns]} \\
  \toprule
\multirow{5}{*}{1.6\%} & Assm. (no \acp{CRGT}, no \acp{BP}) & 6.29E-07 & 5.06E--03 \\
& Assm. (no \acp{BP}) & 1.61E-04 & 1.81E+00 \\
& 2$\times$2 Colorset & 1.89E-04 & 1.18E+01 \\
& 2$\times$2 Colorset w/ Reflector & 2.03E-04 & 1.93E+01 \\
& \ac{BEAVRS} Quarter Core & 1.64E-04 & 6.78E+00 \\
\midrule
\multirow{6}{*}{3.1\%} & Assm. (no \acp{CRGT}, no \acp{BP}) & 5.83E-07 & 6.12E--03 \\
& Assm. (no \acp{BP}) & 1.53E-04 & 4.22E+00 \\
& Assm. (20 \acp{BP}) & 1.37E-04 & 1.11E+01 \\
& 2$\times$2 Colorset & 1.53E-04 & 2.15E+01 \\
& 2$\times$2 Colorset w/ Reflector & 1.79E-04 & 4.78E+01 \\
& \ac{BEAVRS} Quarter Core & 2.34E-04 & 1.68E+01 \\
\bottomrule
\end{tabular}
\end{table}

%%%%%%%%%%%%%%%%%%%%%%%%%%%%%%%%%%%%%%%%
\subsection{Histograms of Pin-Wise MGXS}
\label{subsec:chap9-histograms}

first paragraph: histograms and rug plots
-describe what a histrogram is?
-describe what a rug plot is
-construct plots for each benchmark
-construct plots of the U-238 capture (group 1/2) and U-235 fission (group 2/2) \ac{MGXS} in each pin
  -the ``samples'' in the plots each correspond to a single fuel pin instance
    -i.e., each green mark in the rug plots represents a single fuel pin
-could clearly have looked at other reactions, groups, nuclides
  -for brevity, selected just a few which largely govern reactor performance to present here
  -could have selected data from more groups, but it is ``noisier''
    -the trends are not as easy to identify as they are for 2-group data

%%%%%%%%%%%%%%%%%%%%%%%%%%%%%%%
\subsubsection{U-238 Capture MGXS}
\label{subsubsec:chap9-histograms-capt}

first paragraph: intro U-238 capture \ac{MGXS} histograms
-Fig.~\ref{fig:chap9-hist-1.6-capt} and ~\ref{fig:chap9-hist-3.1-capt}
  -U-238 capture \ac{MGXS} for 1.6\% and 3.1\% enriched fuel pins
  -in each of the heterogeneous benchmarks, respectively
-recall these are micro xs
-recall the inf. lattice case

second paragraph: trends w/ enrichment:
-generally 1.6\% enr. is 0.01-0.02 barns larger
  -micros for inf. lattice are $\sim$ 0.01 barns larger for 1.6\% than 3.1\% enr.
  -micros for lattice w/ \acp{CRGT} are $\sim$ 0.01 barns larger for 1.6\% than 3.1\% enr.
  -micros for 2$\times$2 colorsets are $\sim$0.01-0.02 barns larger for 1.6\% than 3.1\% enr.
  -micros for full core vary over wider range for 3.1\% enr.
    -extend to over 0.9 barns for a few pins
    -while 0.87 is the highest of any 1.6\% enr. pin in the full core
    -the micros for the 3.1\% enriched pins are typically a few hundredths of a barn larger 

third paragraph: analyze trends
-trends w/ \acp{CRGT}
  -addition of \acp{CRGT} induces four clearly discernible clusters
  -separated by $\sim$0.1-0.15 barns for both 1.6\% and 3.1\% enriched pins
  -potentially may be sub-clusters, especially b/w smallest two clusters
    -in range of 0.8-0.81 barns and 0.79-0.8 barns for 1.6\%/3.1\% enriched pins, respectively
  -four cases in order of increasing degree of flux softening due to differential moderation:
    -not adjacent to \ac{CRGT}
    -corner adjacent to \ac{CRGT}
    -facially adjacent to \ac{CRGT}
    -facially and corner adjacent to two separate \acp{CRGT}

-trends w/ \acp{BP}
  -addition of \acp{BP} for 3.1\% enr. assm ``smears'' clusters
  -seems to shift \ac{MGXS} down by $\sim$0.005 barns wrt case w/o \acp{BP}
  -still seem to be four distinct clusters, but more liklely 6+ clusters
  
-trends w/ inter-assembly interface (2$\times$2 colorset w/o reflector)
  -``smearing'' of clusters
  -important to recognize that there are now 264 times 2 = 528 samples
    -expect it to be more difficult to discern clusters from rug plot with more samples
  -still seem to be four distinct clusters for 1.6\% enriched fuel pins
    -largest cluster is $\sim$0.01 barn higher at 0.85 barns
  -more difficult to discern clusters for 3.1\% enr. pins, but appears to be at least 3
    -all \ac{MGXS} shifted down by $\sim$0.0075 barns 

-trends w/ reflector
  -recall which assm is where in the configuration
    -3.1\% w/ 20 \acp{BP} are adjacent to reflector
    -1.6\% w/o \acp{BP} are on the interior and corner adjacent to reflector
  -still 3 strong clusters for 1.6\% enr. pins
    -more ``smeared'', especially on the low side where
    -a few pins have \ac{MGXS} as low as $\sim$0.76 barns
  -similar trends for the 3.1\% enr. pins
    -pins with smallest \ac{MGXS} shifted down by $\sim$0.025 barns
  
-trends with full core
  -recall how many pins of each type are in the full core
    -4236 3.1\% enriched
    -4332 1.6\% enriched
    -4260 2.4\% enriched
  -1.6\% enr. seems to have two peaks at 0.815 and 0.83 barns
  -3.1\% enr. seems to have two peaks at 0.785 and 0.805 barns
    -with a much smaller shoulder centered 0.83 barns
    -a few pins with much higher \ac{MGXS} b/w 0.86 and 0.9 barns
  -\ac{MGXS} are not generally dispersed any more than they were for the colorsets
  

SUMMARY BOX

-think about shifts in \ac{MGXS} in terms of percentages

\begin{figure}[h!]
\centering
\begin{subfigure}{0.5\textwidth}
  \centering
  \includegraphics[width=\linewidth]{figures/patterns/assm-1.6-inf/hist-kde-rug/assm-16-inf-capt-1}
  \caption{}
  \label{fig:chap9-hist-assm-1.6-inf-capt}
\end{subfigure}%
\begin{subfigure}{0.5\textwidth}
  \centering
  \includegraphics[width=\linewidth]{figures/patterns/assm-1.6/hist-kde-rug/assm-16-capt-1}
  \caption{}
  \label{fig:chap9-hist-assm-1.6-capt}
\end{subfigure}
\begin{subfigure}{0.5\textwidth}
  \centering
  \includegraphics[width=\linewidth]{figures/patterns/2x2/hist-kde-rug/16-enr-capt-1}
  \caption{}
  \label{fig:chap9-hist-2x2-1.6-capt}
\end{subfigure}%
\begin{subfigure}{0.5\textwidth}
  \centering
  \includegraphics[width=\linewidth]{figures/patterns/reflector/hist-kde-rug/16-enr-capt-1}  \caption{}
  \label{fig:chap9-hist-reflector-1.6-capt}
\end{subfigure}
\begin{subfigure}{0.5\textwidth}
  \centering
  \includegraphics[width=\linewidth]{figures/patterns/full-core/hist-kde-rug/16-enr-capt-1} \caption{}
  \label{fig:chap9-hist-full-core-1.6-capt}
\end{subfigure}
\caption[Histogram of U-238 capture MGXS for 1.6\% enriched fuel]{Histograms of U-238 capture \ac{MGXS} (group 1 of 2) for 1.6\% enriched fuel.}
\label{fig:chap9-hist-1.6-capt}
\end{figure}

\begin{figure}[h!]
\centering
\begin{subfigure}{0.5\textwidth}
  \centering
  \includegraphics[width=\linewidth]{figures/patterns/assm-3.1-inf/hist-kde-rug/assm-31-inf-capt-1}
  \caption{}
  \label{fig:chap9-hist-assm-3.1-inf-capt}
\end{subfigure}%
\begin{subfigure}{0.5\textwidth}
  \centering
  \includegraphics[width=\linewidth]{figures/patterns/assm-3.1/hist-kde-rug/assm-31-capt-1}
  \caption{}
  \label{fig:chap9-hist-assm-3.1-capt}
\end{subfigure}
\begin{subfigure}{0.5\textwidth}
  \centering
  \includegraphics[width=\linewidth]{figures/patterns/assm-3.1-20BPs/hist-kde-rug/assm-31-20BPs-capt-1}
  \caption{}
  \label{fig:chap9-hist-assm-3.1-20BPs-capt}
\end{subfigure}%
\begin{subfigure}{0.5\textwidth}
  \centering
  \includegraphics[width=\linewidth]{figures/patterns/2x2/hist-kde-rug/31-enr-capt-1}
  \caption{}
  \label{fig:chap9-hist-2x2-3.1-capt}
\end{subfigure}
\begin{subfigure}{0.5\textwidth}
  \centering
  \includegraphics[width=\linewidth]{figures/patterns/reflector/hist-kde-rug/31-enr-capt-1}  \caption{}
  \label{fig:chap9-hist-reflector-3.1-capt}
\end{subfigure}%
\begin{subfigure}{0.5\textwidth}
  \centering
  \includegraphics[width=\linewidth]{figures/patterns/full-core/hist-kde-rug/31-enr-capt-1} \caption{}
  \label{fig:chap9-hist-full-core-3.1-capt}
\end{subfigure}
\caption[Histogram of U-238 capture MGXS for 3.1\% enriched fuel]{Histograms of U-238 capture \ac{MGXS} (group 2 of 2) for 3.1\% enriched fuel.}
\label{fig:chap9-hist-3.1-capt}
\end{figure}

%%%%%%%%%%%%%%%%%%%%%%%%%%%%%%%%%%
\subsubsection{U-235 Fission MGXS}
\label{subsubsec:chap9-histograms-fiss}

first paragraph: intro U-235 fission \ac{MGXS} histograms
-Fig.~\ref{fig:chap9-hist-1.6-fiss} and ~\ref{fig:chap9-hist-3.1-fiss}
  -U-235 fission \ac{MGXS} for 1.6\% and 3.1\% enriched fuel pins
  -in each of the heterogeneous benchmarks, respectively
-recall these are micro xs
-recall the inf. lattice case
-micro U-235 \ac{MGXS} for fission is over 300$\times$ larger than U-238 capture \ac{MGXS}
  -BUT the nuclide density for U-238 is over 30--60$\times$ larger than U-235 for 1.6\% and 3.1\% enriched fuel, respectively
  -hence, clustering for micro U-238 capture \ac{MGXS} is important even though the micro is much smaller
  -an error of 1\% in the \ac{MGXS} for either is problematic

second paragraph: trends w/ enrichment:
-generally 1.6\% enr. is 0.01-0.02 barns larger
  -micros for inf. lattice are $\sim$ 18 barns larger for 1.6\% than 3.1\% enr.
  -micros for lattice w/ \acp{CRGT} are $\sim$ 18 barns larger for 1.6\% than 3.1\% enr.
  -micros for 2$\times$2 colorsets are up to 22 barns larger for 1.6\% than 3.1\% enr.
  -micros for full core vary over wider range for 3.1\% enr.
    -extend up to 305 barns for a few 1.6\% enr. pins, but only 287.5 barns for 3.1\% enr. pins
    -do not extend over as wide of a range as the 2$\times$2 colorset w/ reflector for either enr.

third paragraph: analyze trends
-trends w/ \acp{CRGT}
  -histograms indicate roughly 3 clusters for each enrichment
  -upper bound \ac{MGXS} are shifted up by more than 5 barns with \acp{CRGT} (nearly 2\%)
  -rug plots tell a more complicated story
    -the data is more ``smeared'' across the range than for U-238 capture
    -clusters appear to be more distinct for 3.1\% enrichment - coincidence???
      -seem to be about 8 clusters, tougher to tell for 1.6\% enrichment
      -regardless, seems to be more sensitive to other self-shielding / differential moderation effects than U-238 capture
        -BUT the total range of MGXS varies by only 1.3\% as compared to 5\% for U-238 capture

-trends w/ \acp{BP}
  -shifts lower bound down by 9 barns (3\%)
  -shifts upper bound down by 4 barns (1.3\%)
  -appear to be 3 strong clusters centered at 272, 276 and 300 barns (w/ sub-clusters)
    -adjacent clusters differ by $\sim$1.3\%
  -addition of \acp{BP} makes clusters more distinct (unlike for U-238)
  
-trends w/ inter-assembly interface (2$\times$2 colorset w/o reflector)
  -seems to be 3 and 4 strong clusters for 1.6\% and 3.1\% enriched pins (based on histograms)
    -centered at 294, 298 and 301 barns for 1.6\% enr
    -centered at 271, 275, 278 and 283 barns for 3.1\% enr
    -adjacent clusters differ by 1--2\%
  -rug plots indicate more clusters than this
  -clusters more distinguishable than they were for single assembly  
  -clusters are more distinguishable than they were for U-238 capture \ac{MGXS}
  -lower bound for 1.6\% enr. fuel is shifted down by $\sim$6 barns (2\%)
  -upper bound for 3.1\% enr. fuel is shifted up by $\sim$3 barns (1\%)

-trends w/ reflector
  -recall which assm is where in the configuration
    -3.1\% w/ 20 \acp{BP} are adjacent to reflector
    -1.6\% w/o \acp{BP} are on the interior and corner adjacent to reflector
  -structure is very different for the two enrichments
    -dispersion of a few pins with largest \ac{MGXS} of 15 barns greater than case w/o reflector (5\%)
  -roughly six clusters for 1.6\% enr. centered at 294, 298, 301, 304, 306, and 310 barns
    -largest cluster at 301 barns
    -certainly more sub-clusters
    -one pin with 315 barn \ac{MGXS} (cornermost pin with most added moderation from reflector)
  -rougly four clusters for 3.1\% enr. centered at 276, 284, 290 296 barns
    -largest cluster at 276 barrns
    -certainly more sub-clusters
  
-trends with full core
  -recall how many pins of each type are in the full core
    -4236 3.1\% enriched
    -4332 1.6\% enriched
    -4260 2.4\% enriched
  -1.6\% enr. seems to have two clearly defined peaks 297 and 302 barns
    -peaks are more clearly discernible than for 3.1\% enr.
  -3.1\% enr. seems to have three most clearly defined peaks at 278, 283 and 284 barns
  -both have \ac{MGXS} smeared across 15 barns (5\%) which is less than the 20--25 barns for the 2$\times$2 colorsets w/ a reflector and more similar to the range seen for the colorset w/o a reflector
  -range of \ac{MGXS} is are not generally dispersed any more than they were for the colorsets 

SUMMARY BOX

-think about shifts in \ac{MGXS} in terms of percentages

\begin{figure}[h!]
\centering
\begin{subfigure}{0.5\textwidth}
  \centering
  \includegraphics[width=\linewidth]{figures/patterns/assm-1.6-inf/hist-kde-rug/assm-16-inf-fiss-2}
  \caption{}
  \label{fig:chap9-hist-assm-1.6-inf-fiss}
\end{subfigure}%
\begin{subfigure}{0.5\textwidth}
  \centering
  \includegraphics[width=\linewidth]{figures/patterns/assm-1.6/hist-kde-rug/assm-16-fiss-2}
  \caption{}
  \label{fig:chap9-hist-assm-1.6-fiss}
\end{subfigure}
\begin{subfigure}{0.5\textwidth}
  \centering
  \includegraphics[width=\linewidth]{figures/patterns/2x2/hist-kde-rug/16-enr-fiss-2}
  \caption{}
  \label{fig:chap9-hist-2x2-1.6-fiss}
\end{subfigure}%
\begin{subfigure}{0.5\textwidth}
  \centering
  \includegraphics[width=\linewidth]{figures/patterns/reflector/hist-kde-rug/16-enr-fiss-2}  \caption{}
  \label{fig:chap9-hist-reflector-1.6-fiss}
\end{subfigure}
\begin{subfigure}{0.5\textwidth}
  \centering
  \includegraphics[width=\linewidth]{figures/patterns/full-core/hist-kde-rug/16-enr-fiss-2} \caption{}
  \label{fig:chap9-hist-full-core-1.6-fiss}
\end{subfigure}
\caption[Histogram of U-235 fission MGXS for 1.6\% enriched fuel]{Histograms of U-235 fission \ac{MGXS} (group 1 of 2) for 1.6\% enriched fuel.}
\label{fig:chap9-hist-1.6-fiss}
\end{figure}

\begin{figure}[h!]
\centering
\begin{subfigure}{0.5\textwidth}
  \centering
  \includegraphics[width=\linewidth]{figures/patterns/assm-3.1-inf/hist-kde-rug/assm-31-inf-fiss-2}
  \caption{}
  \label{fig:chap9-hist-assm-3.1-inf-fiss}
\end{subfigure}%
\begin{subfigure}{0.5\textwidth}
  \centering
  \includegraphics[width=\linewidth]{figures/patterns/assm-3.1/hist-kde-rug/assm-31-fiss-2}
  \caption{}
  \label{fig:chap9-hist-assm-3.1-fiss}
\end{subfigure}
\begin{subfigure}{0.5\textwidth}
  \centering
  \includegraphics[width=\linewidth]{figures/patterns/assm-3.1-20BPs/hist-kde-rug/assm-31-20BPs-fiss-2}
  \caption{}
  \label{fig:chap9-hist-assm-3.1-20BPs-fiss}
\end{subfigure}%
\begin{subfigure}{0.5\textwidth}
  \centering
  \includegraphics[width=\linewidth]{figures/patterns/2x2/hist-kde-rug/31-enr-fiss-2}
  \caption{}
  \label{fig:chap9-hist-2x2-3.1-fiss}
\end{subfigure}
\begin{subfigure}{0.5\textwidth}
  \centering
  \includegraphics[width=\linewidth]{figures/patterns/reflector/hist-kde-rug/31-enr-fiss-2}  \caption{}
  \label{fig:chap9-hist-reflector-3.1-fiss}
\end{subfigure}%
\begin{subfigure}{0.5\textwidth}
  \centering
  \includegraphics[width=\linewidth]{figures/patterns/full-core/hist-kde-rug/31-enr-fiss-2} \caption{}
  \label{fig:chap9-hist-full-core-3.1-fiss}
\end{subfigure}
\caption[Histogram of U-235 fission MGXS 3.1\% enriched fuel]{Histograms of U-235 fission \ac{MGXS} (group 2 of 2) for 3.1\% enriched fuel.}
\label{fig:chap9-hist-3.1-fiss}
\end{figure}


%%%%%%%%%%%%%%%%%%%%%%%%%%%%%%%%%%%%%%%%%%%%%%%%%%%%%
\subsection{Quantile-Quantile Plots of Pin-Wise MGXS}
\label{subsec:chap9-qq-plots}

first paragraph: qq plots
-describe what a qq plot is
-construct plots for each benchmark
-construct plots of the U-238 capture (group 1/2) and U-235 fission (group 2/2) \ac{MGXS} in each pin
  -the ``samples'' in the plots each correspond to a single fuel pin instance
    -i.e., each blue data point represents a single fuel pin
-could clearly have looked at other reactions, groups, nuclides
  -for brevity, selected just a few which largely govern reactor performance to present here
  -could have selected data from more groups, but it is ``noisier''
    -the trends are not as easy to identify as they are for 2-group data

second paragraph: normality tests
-explain what a p-value is
-reference shannon wilkes test or whatever in scipy


%%%%%%%%%%%%%%%%%%%%%%%%%%%%%%%%%%
\subsubsection{U-238 Capture MGXS}
\label{subsubsec:chap9-qq-plots-capt}

first paragraph: intro U-238 capture \ac{MGXS} qq plots
-Fig.~\ref{fig:chap9-qq-1.6-capt} and ~\ref{fig:chap9-qq-3.1-capt}
  -U-238 capture \ac{MGXS} for 1.6\% and 3.1\% enriched fuel pins
  -in each of the heterogeneous benchmarks, respectively
-recall the inf. lattice case

-second paragraph:
  -the qq plot shows that data for the inf. lattices very nearly normal (p-values well above 0.05, closer to 0.7)
  -clear deviation from normality with addition of \acp{CRGT}
  -structure is more challenging to understand for more complex benchmarks
    -addition of \acp{BP} for 3.1\% enr. assembly makes clusters/shoulders ``sharper'' steps
    -more data points (e.g., fuel pins) for each benchmark clouds the structure
	  -generally with more data points the trends are ``smoother''
	    -does NOT necessarily indicate better approx. to normality however!!
	    -``slow'' and ``smooth'' trends can still deviate widely from that expected for normal samples, e.g., 3.1\% enr. full core pin
  -p-value is very clearly smallest for the full core (1e-22 and 1e-39 for 1.6\% and 3.1\% enr fuel)

-mention that tails above and below on lower/upper range indicate that tails of distribution are not as wide as would be expected for normal samples -- e.g., a ``tighter'' or ``narrower'' distribution
-3.1\% enr pins in full core are tighter on bottom edge, but wider on upper edge
  -indicative of those few ``outlier'' pins with \ac{MGXS} near 0.9 barns in Fig.~\ref{fig:chap9-hist-3.1-capt}
  
SUMMARY BOX

-add ``Assm.'' to titles of plots for individual fuel assemblies
-add group 1/2 to captions

\begin{figure}[h!]
\centering
\begin{subfigure}{0.5\textwidth}
  \centering
  \includegraphics[width=\linewidth]{figures/patterns/assm-1.6-inf/quantile/assm-16-inf-capt-1}
  \caption{}
  \label{fig:chap9-qq-assm-1.6-inf-capt}
\end{subfigure}%
\begin{subfigure}{0.5\textwidth}
  \centering
  \includegraphics[width=\linewidth]{figures/patterns/assm-1.6/quantile/assm-16-capt-1}
  \caption{}
  \label{fig:chap9-qq-assm-1.6-capt}
\end{subfigure}
\begin{subfigure}{0.5\textwidth}
  \centering
  \includegraphics[width=\linewidth]{figures/patterns/2x2/quantile/16-enr-capt-1}
  \caption{}
  \label{fig:chap9-qq-2x2-1.6-capt}
\end{subfigure}%
\begin{subfigure}{0.5\textwidth}
  \centering
  \includegraphics[width=\linewidth]{figures/patterns/reflector/quantile/16-enr-capt-1}  \caption{}
  \label{fig:chap9-qq-reflector-1.6-capt}
\end{subfigure}
\begin{subfigure}{0.5\textwidth}
  \centering
  \includegraphics[width=\linewidth]{figures/patterns/full-core/quantile/16-enr-capt-1} \caption{}
  \label{fig:chap9-qq-full-core-1.6-capt}
\end{subfigure}
\caption[Q-Q plots of U-238 capture MGXS for 1.6\% enriched fuel]{Q-Q plots of U-238 capture \ac{MGXS} for 1.6\% enriched fuel.}
\label{fig:chap9-qq-1.6-capt}
\end{figure}

\begin{figure}[h!]
\centering
\begin{subfigure}{0.5\textwidth}
  \centering
  \includegraphics[width=\linewidth]{figures/patterns/assm-3.1-inf/quantile/assm-31-inf-capt-1}
  \caption{}
  \label{fig:chap9-qq-assm-3.1-inf-capt}
\end{subfigure}%
\begin{subfigure}{0.5\textwidth}
  \centering
  \includegraphics[width=\linewidth]{figures/patterns/assm-3.1/quantile/assm-31-capt-1}
  \caption{}
  \label{fig:chap9-qq-assm-3.1-capt}
\end{subfigure}
\begin{subfigure}{0.5\textwidth}
  \centering
  \includegraphics[width=\linewidth]{figures/patterns/assm-3.1-20BPs/quantile/assm-31-20BPs-capt-1}
  \caption{}
  \label{fig:chap9-qq-assm-3.1-20BPs-capt}
\end{subfigure}%
\begin{subfigure}{0.5\textwidth}
  \centering
  \includegraphics[width=\linewidth]{figures/patterns/2x2/quantile/31-enr-capt-1}
  \caption{}
  \label{fig:chap9-qq-2x2-3.1-capt}
\end{subfigure}
\begin{subfigure}{0.5\textwidth}
  \centering
  \includegraphics[width=\linewidth]{figures/patterns/reflector/quantile/31-enr-capt-1}  \caption{}
  \label{fig:chap9-qq-reflector-3.1-capt}
\end{subfigure}%
\begin{subfigure}{0.5\textwidth}
  \centering
  \includegraphics[width=\linewidth]{figures/patterns/full-core/quantile/31-enr-capt-1} \caption{}
  \label{fig:chap9-qq-full-core-3.1-capt}
\end{subfigure}
\caption[Q-Q plots of U-238 capture MGXS for 3.1\% enriched fuel]{Q-Q plots of U-238 capture \ac{MGXS} for 3.1\% enriched fuel.}
\label{fig:chap9-qq-3.1-capt}
\end{figure}

%%%%%%%%%%%%%%%%%%%%%%%%%%%%%%%%%%
\subsubsection{U-235 Fission MGXS}
\label{subsubsec:chap9-qq-plots-fiss}

first paragraph: intro U-235 fission \ac{MGXS} qq plots
-Fig.~\ref{fig:chap9-qq-1.6-fiss} and ~\ref{fig:chap9-qq-3.1-fiss}
  -U-235 fission \ac{MGXS} for 1.6\% and 3.1\% enriched fuel pins
  -in each of the heterogeneous benchmarks, respectively
-recall the inf. lattice case

-second paragraph:
  -the qq plot shows that data for the inf. lattices very nearly normal (p-values well above 0.05, closer to 0.35)
  -clear deviation from normality with addition of \acp{CRGT}
    -more complicated and less distinct structure than was seen for U-238 capture
  -crazy piecewise step-like structure with addition of \acp{BP}
    -much more fragmented than for U-238 capture
  -colorsets indicate sharper steps than for U-238 capture which ``smoothed'' out with more complexity
  -p-value is very clearly smallest for the full core (1e-22 and 1e-39 for 1.6\% and 3.1\% enr fuel)

-mention that tails above and below on lower/upper range indicate that tails of distribution are not as wide as would be expected for normal samples -- e.g., a ``tighter'' or ``narrower'' distribution
-3.1\% enr pins in full core are tighter on bottom edge, but wider on upper edge
  -indicative of those few ``outlier'' pins with \ac{MGXS} near 0.9 barns in Fig.~\ref{fig:chap9-hist-3.1-capt}

SUMMARY BOX

-what are key differences with U-238 capture plots?
-add ``Assm.'' to titles of plots for individual fuel assemblies
-add group 1/2 to captions

\begin{figure}[h!]
\centering
\begin{subfigure}{0.5\textwidth}
  \centering
  \includegraphics[width=\linewidth]{figures/patterns/assm-1.6-inf/quantile/assm-16-inf-fiss-2}
  \caption{}
  \label{fig:chap9-qq-assm-1.6-inf-fiss}
\end{subfigure}%
\begin{subfigure}{0.5\textwidth}
  \centering
  \includegraphics[width=\linewidth]{figures/patterns/assm-1.6/quantile/assm-16-fiss-2}
  \caption{}
  \label{fig:chap9-qq-assm-1.6-fiss}
\end{subfigure}
\begin{subfigure}{0.5\textwidth}
  \centering
  \includegraphics[width=\linewidth]{figures/patterns/2x2/quantile/16-enr-fiss-2}
  \caption{}
  \label{fig:chap9-qq-2x2-1.6-fiss}
\end{subfigure}%
\begin{subfigure}{0.5\textwidth}
  \centering
  \includegraphics[width=\linewidth]{figures/patterns/reflector/quantile/16-enr-fiss-2}  \caption{}
  \label{fig:chap9-qq-reflector-1.6-fiss}
\end{subfigure}
\begin{subfigure}{0.5\textwidth}
  \centering
  \includegraphics[width=\linewidth]{figures/patterns/full-core/quantile/16-enr-fiss-2} \caption{}
  \label{fig:chap9-qq-full-core-1.6-fiss}
\end{subfigure}
\caption[Q-Q plots of U-235 fission MGXS for 1.6\% enriched fuel]{Q-Q plots of U-235 fission \ac{MGXS} for 1.6\% enriched fuel.}
\label{fig:chap9-qq-1.6-fiss}
\end{figure}

\begin{figure}[h!]
\centering
\begin{subfigure}{0.5\textwidth}
  \centering
  \includegraphics[width=\linewidth]{figures/patterns/assm-3.1-inf/quantile/assm-31-inf-fiss-2}
  \caption{}
  \label{fig:chap9-qq-assm-3.1-inf-fiss}
\end{subfigure}%
\begin{subfigure}{0.5\textwidth}
  \centering
  \includegraphics[width=\linewidth]{figures/patterns/assm-3.1/quantile/assm-31-fiss-2}
  \caption{}
  \label{fig:chap9-qq-assm-3.1-fiss}
\end{subfigure}
\begin{subfigure}{0.5\textwidth}
  \centering
  \includegraphics[width=\linewidth]{figures/patterns/assm-3.1-20BPs/quantile/assm-31-20BPs-fiss-2}
  \caption{}
  \label{fig:chap9-qq-assm-3.1-20BPs-fiss}
\end{subfigure}%
\begin{subfigure}{0.5\textwidth}
  \centering
  \includegraphics[width=\linewidth]{figures/patterns/2x2/quantile/31-enr-fiss-2}
  \caption{}
  \label{fig:chap9-qq-2x2-3.1-fiss}
\end{subfigure}
\begin{subfigure}{0.5\textwidth}
  \centering
  \includegraphics[width=\linewidth]{figures/patterns/reflector/quantile/31-enr-fiss-2}  \caption{}
  \label{fig:chap9-qq-reflector-3.1-fiss}
\end{subfigure}%
\begin{subfigure}{0.5\textwidth}
  \centering
  \includegraphics[width=\linewidth]{figures/patterns/full-core/quantile/31-enr-fiss-2} \caption{}
  \label{fig:chap9-qq-full-core-3.1-fiss}
\end{subfigure}
\caption[Q-Q plots of U-235 fission MGXS 3.1\% enriched fuel]{Q-Q plots of U-235 fission \ac{MGXS} for 3.1\% enriched fuel.}
\label{fig:chap9-qq-3.1-fiss}
\end{figure}


%%%%%%%%%%%%%%%%%%%%%%%%%%%%%%%%%%%%%%%%%%%%%%%%%%%%%%%%%%%%%%%%%%%%%%%%%%%%%%%
\section{LNS Spatial Homogenization}
\label{sec:chap9-lns-homogenize}

first paragraph: 
-recall geometric templates used in traditional lattice physics codes
  -CASMO~\cite{rhodes2006casmo}
-recall OpenCG's \ac{LNS} algorithm from Sec.~\ref{sec:chap4-lns}
  -used as a proxy for geometric templates
-this section applies it as a deterministic atttempt to ``cluster'' like \ac{MGXS}
-aim is to predict U-238 capture rates better than is possible with null/infinite homogenization
  -approach degenerate homogenization's accuracy
  -accelerate tally convergence rate by averaging \ac{MGXS} across many pin instances
    -refer to last section of this chapter

second paragraph: describe \ac{LNS} spatial homogenization
-tally \ac{MGXS} for each fuel pin instance across the core
  -as is done for degenerate homogenization
-use OpenCG \ac{LNS} algorithm to assign an \ac{LNS} identifier to each unique pin instance in \ac{CG}
  -pins with like neighboring pins will receive the same \ac{LNS} identifier
-average \ac{MGXS} for each ``family'' of pin instances with like \ac{LNS} identifiers
  -mention what this means - sum up flux and reaction rate tallies
    -effectively an average weighted by track density in each pin
    -equivalent to defining flux/rxn rate tallies specifically for those fuel pin intances
    -perhaps use an equation for this???
-\ac{MGXS} for all other materials are averaged across the complete heterogeneous geometry
  -take wording from last chapter for this

third paragraph: point out materials for the six benchmarks
-Tab.~\ref{table:chap9-num-materials-lns} 
  -number of materials for each of the six benchmarks with \ac{LNS} homogenization
-Fig.~\ref{fig:chap9-lns-materials}
  -materials configurations for individual assembly and 2$\times$2 colorset benchmarks with \ac{LNS} spatial homogenization
-Fig.~\ref{fig:chap9-lns-materials-beavrs}
  -materials configuration for 2D quarter core \ac{BEAVRS} model with \ac{LNS} spatial homogenization

fourth paragraph: segue to next chapter
-objective is to capture clustering effects in MGXS
-LNS isn't adaptable
  -point out pins along assembly-assembly and assembly-reflector interfaces treated the same
  -point out large number of materials in full core - may be more than necessary
  -algorithm must be customized to specific types of core geometries
-following section will highlight the results with \ac{LNS} spatial homogenization

\begin{table}[h!]
  \centering
  \caption[Number of materials for LNS spatial homogenization]{Number of materials modeled with unique \ac{MGXS} in each heterogeneous benchmark for \ac{LNS} spatial homogenization.}
  \small
  \label{table:chap9-num-materials-lns}
  \vspace{6pt}
  \begin{tabular}{l r r r}
  \toprule
  \rowcolor{lightgray}
  & \multicolumn{3}{c}{\cellcolor{lightgray} \bf \# Materials} \\
  \multirow{-2}{*}{\cellcolor{lightgray} \bf Benchmark} &
  \multicolumn{1}{c}{\cellcolor{lightgray} \bf Null/Infinite} &
  \multicolumn{1}{c}{\cellcolor{lightgray} \bf \ac{LNS}} &
  \multicolumn{1}{c}{\cellcolor{lightgray} \bf Degenerate} \\
  \midrule
1.6\% Assm & 5 & 13 & 268 \\
  \midrule
3.1\% Assm & 5 & 13 & 268 \\
  \midrule
3.1\% Assm w/ 20 BPs & 7 & 17 & 270  \\
  \midrule
2$\times$2 Colorset & 8 & 25 & 1,062 \\
  \midrule
2$\times$2 Colorset w/ Reflector & 8 & 35 & 1,062 \\
  \midrule
\ac{BEAVRS} Quarter Core & 10 & 497 & 13,000 \\ % LNS = ceil(193 / 4) * 10 + 7
  \bottomrule
\end{tabular}
\end{table}

\begin{figure}[h!]
\centering
\begin{subfigure}{.47\textwidth}
  \centering
  \includegraphics[width=0.9\linewidth]{figures/patterns/lns/assm-31/materials}
  \caption{}
  \label{fig:chap9-assm-31-lns-materials}
\end{subfigure}%
\begin{subfigure}{.47\textwidth}
  \centering
  \includegraphics[width=0.9\linewidth]{figures/patterns/lns/assm-31-20BPs/materials}
  \caption{}
  \label{fig:chap9-31-20BPs-lns-materials}
\end{subfigure}
\begin{subfigure}{.47\textwidth}
  \centering
  \includegraphics[width=0.9\linewidth]{figures/patterns/lns/2x2/materials}
  \caption{}
  \label{fig:chap9-2x2-lns-materials}
\end{subfigure}%
\begin{subfigure}{.47\textwidth}
  \centering
  \includegraphics[width=0.9\linewidth]{figures/patterns/lns/reflector/materials}
  \caption{}
  \label{fig:chap9-reflector-lns-materials}
\end{subfigure}
\caption[Depiction of LNS spatially homogenized materials]{OpenMOC materials with \ac{LNS} spatial homogenization for a 1.6/3.1\% enriched fuel assembly (a), a 3.1\% enriched assembly with 20 \acp{BP} (b), a 2$\times$2 colorset without (c) and with (d) a reflector. Each uniquely colored material represents a unique set of \ac{MGXS}.}
\label{fig:chap9-lns-materials}
\end{figure}

\begin{figure}[h!]
\centering
\includegraphics[width=\linewidth]{figures/patterns/lns/full-core/materials}
\caption{}
\caption[Depiction of LNS spatially homogenized materials for quarter core BEAVRS]{OpenMOC materials with \ac{LNS} spatial homogenization for the 2D quarter core \ac{BEAVRS} model. Each uniquely colored material represents a unique set of \ac{MGXS}.}
\label{fig:chap9-lns-materials-beavrs}
\end{figure}


%%%%%%%%%%%%%%%%%%%%%%%%%%%%%%%%%%%%%%%%%%%%%%%%%%%%%%%%%%%%%%%%%%%%%%%%%%%%%%%%
\section{Multi-Group Results with LNS}
\label{subsec:chap9-lns-results}

first paragraph: introduce / outline results
-ran simulations with 2, 8 and 70 groups as in Chap.~\ref{chap:quantify} for each of the six benchmarks
-used \ac{LNS} spatial homogenization in each one with same OpenMOC simulation parameters
  -128 azimuthal angles, 0.05 cm track spacing
-Sec.~\ref{subsec:chap9-lns-eigenvalues} - eigenvalues
-Sec.~\ref{subsec:chap9-lns-fiss-rates} - pin-wise energy-integrated fission rates
-Sec.~\ref{subsec:chap9-lns-capt-rates} - pin-wise energy-integrated capture rates

%%%%%%%%%%%%%%%%%%%%%%%%
\subsection{Eigenvalues}
\label{subsec:chap9-lns-eigenvalues}

first paragraph: eigenvalues
-as was observed when comparing null to degenerate homogenization in Chap.~\ref{chap:quantify}, spatial homogenization cannot be expected to systematically improve the eigenvalue wrt OpenMC
-but these results are presented here to ensure that \ac{LNS} homogenization was properly implemented
  -and results simply reinforce this notion
-Tab.~\ref{table:chap9-lns-eigenvalues} tabulates eigenvalues for 2, 8, and 70 groups
-eigenvalues closely match null and degenerate homogenization from Chap.~\ref{chap:quantify}
  -match to within less than 10 \ac{pcm} for nearly all cases
  -due to reaction rate preservation

\begin{table}[ht!]
  \centering
  \caption[OpenMOC eigenvalue bias with LNS homogenization]{OpenMOC eigenvalue bias $\Delta\rho$ for heterogeneous benchmarks with \ac{LNS} homogenization and varying energy group structures.}
  \small
  \label{table:chap9-lns-eigenvalues}
  \vspace{6pt}
  \begin{tabular}{l R{2.5cm} R{2.5cm} R{2.5cm}}
  \toprule
  \rowcolor{lightgray}
  & \multicolumn{3}{S[table-format=6.1]}{\cellcolor{lightgray} {$\bm{\Delta\rho}$ \textbf{[pcm]}}} \\
  \multirow{-2}{*}{\cellcolor{lightgray} \bf Benchmark} &
  \multicolumn{1}{r}{{\cellcolor{lightgray} \bf 2-Group}} &
  \multicolumn{1}{r}{{\cellcolor{lightgray} \bf 8-Group}} &
  \multicolumn{1}{r}{{\cellcolor{lightgray} \bf 70-Group}} \\
  \midrule
1.6\% Assm & 56 & -79 & -168 \\
3.1\% Assm & 97 & -80 & -202 \\
3.1\% Assm w/ 20 BPs & -157 & -161 & -247 \\
2$\times$2 Colorset & 10 & -94 & -195 \\
2$\times$2 Colorset w/ Reflector & 1792 & 476 & -143 \\
BEAVRS Full Core & 2225 & 448 & -81 \\
  \bottomrule
\end{tabular}
\end{table}

%%%%%%%%%%%%%%%%%%%%%%%%
\subsection{Fission Rates}
\label{subsec:chap9-lns-fiss-rates}

first paragraph: overview of error trends
-Tab.~\ref{table:chap9-lns-fiss-rates} - max and mean fission rate errors
  -compare to Tabs.~\ref{table:chap8-openmoc-max-fiss-rates} and~\ref{table:chap8-openmoc-mean-fiss-rates}
  -recall that degenerate homogenization provided little improvement in fission rates
-max rates - observations
  -assm 1.6 - very nearly the same as degenerate
  -assm 3.1 - similar to degenerate (better than null)
  -assm 3.1 w/ BPs - slightly better than degenerate for 8 and 70 groups
  -2x2 - similar to degenerate, but they all perform the same for 70 groups
  -2x2 w/ reflector - significantly worse with 2 groups, and over 0.12\% worse than degenerate w/ 70 groups
  -largest deviation with degenerate is for 2 groups
-mean rates - observations
  -assm 1.6 - slightly better than degenerate for 8 and 70 groups (0.02\%)
  -assm 3.1 - slightly better than degenerate for 8 and 70 groups (0.02\%)
  -assm 3.1 w/ BPs - slightly better than degenerate for 70 groups (0.02\%)
  -2x2 - slightly better than degenerate for 8 and 70 groups (0.02\%)
  -2x2 w/ reflector - worse than degenerate, but better than null
-GENERAL TAKEWAWAY: generally better than null but approaches accuracy of degenerate case
  -except for reflector and full core
-depictions of spatial distributions of errors not shown here since it was shown in Chap.~\ref{chap:quantify} that spatial homogenization has little impact on fission rates

-SUMMARY BOX

\begin{table}[ht!]
  \centering
  \caption[OpenMOC fission rate errors with LNS homogenization]{OpenMOC fission rate percent relative errors for heterogeneous benchmarks with \ac{LNS} spatial homogenization and varying energy group structures.}
  \small
  \label{table:chap9-lns-fiss-rates}
  \vspace{6pt}
  \begin{tabular}{l l R{2.5cm} R{2.5cm} R{2.5cm}}
  \toprule
  \rowcolor{lightgray}
  & & \multicolumn{3}{c}{\cellcolor{lightgray} \textbf{Error [\%]}} \\
  \multirow{-2}{*}{\cellcolor{lightgray} \bf Benchmark} &
  \multirow{-2}{*}{\cellcolor{lightgray} \bf Metric} &
  \multicolumn{1}{r}{{\cellcolor{lightgray} \bf 2-Group}} &
  \multicolumn{1}{r}{{\cellcolor{lightgray} \bf 8-Group}} &
  \multicolumn{1}{r}{{\cellcolor{lightgray} \bf 70-Group}} \\
  \midrule
\multirow{2}{*}{\parbox{2.2cm}{1.6\% Assm}} & Max & 2.131 & 0.764 & 0.313 \\
& Mean & 0.714 & 0.239 & 0.083 \\
\midrule
\multirow{2}{*}{\parbox{2.2cm}{3.1\% Assm}} & Max & 2.476 & 0.919 & 0.429 \\
& Mean & 0.832 & 0.287 & 0.091 \\
\midrule
\multirow{2}{*}{\parbox{2.2cm}{3.1\% Assm w/ 20 BPs}} & Max & -2.009 & -0.715 & 0.369 \\
& Mean & 0.719 & 0.216 & 0.095 \\
\midrule
\multirow{2}{*}{\parbox{2.2cm}{2$\times$2 Colorset}} & Max & -5.517 & -1.490 & 0.584 \\
& Mean & 2.940 & 0.701 & 0.138 \\
\midrule
\multirow{2}{*}{\parbox{2.2cm}{2$\times$2 Colorset w/ Reflector}} & Max & -15.714 & -2.894 & 0.887 \\
& Mean & 5.170 & 1.076 & 0.178 \\
\midrule
\multirow{2}{*}{\parbox{2.2cm}{BEAVRS Full Core}} & Max & -86.166 & -35.548 & 12.330 \\
& Mean & 40.190 & 11.465 & 1.454 \\
\bottomrule
\end{tabular}
\end{table}

%%%%%%%%%%%%%%%%%%%%%%%%%%%%%%%%%%%%%%%%%%%%%
\subsection{U-238 Capture Rate Distributions}
\label{subsec:chap9-lns-capt-rates}

first paragraph:
-Tab.~\ref{table:chap9-lns-capt-rates} - max and mean U-238 capture rate errors
  -compare to Tabs.~\ref{table:chap8-openmoc-max-capt-rates} and~\ref{table:chap8-openmoc-mean-capt-rates}
  -recall that degenerate homogenization significantly reduced U-238 capture rate errors 
-max rates - observations
  -assm 1.6 - better than degenerate w/ 70 groups (0.04\%), but worse with 2 and 8 groups
  -assm 3.1 - better than degenerate w/ 70 groups (0.07\%), but worse with 2 and 8 groups
  -assm 3.1 w/ BPs - worse than degenerate w/ 70 groups (0.06\%)
  -2x2 - significantly better than degenerate w/ 70 groups (0.16\%)
  -2x2 w/ reflector - significantly worse than degenerate (and null) w/ 70 groups (0.8\%)
    -and with opposite sign since the errors are under-predicted near reflector
  -generally worse with 2 and 8 groups
-mean rates - observations
  -assm 1.6 - better than degenerate w/ 70 groups (0.01\%)
  -assm 3.1 - better than degenerate w/ 70 groups (0.01\%)
  -assm 3.1 w/ BPs - better than degenerate w/ 70 groups (0.01\%)
  -2x2 - better than degenerate w/ 70 groups (0.04\%)
  -2x2 w/ reflector - worse than degenerate w/ 70 groups (0.07\%)
    -though still much better than null/infinite
-GENERAL TAKEWAWAY: generally better than null but approaches accuracy of degenerate case
  -even slightly beats accuracy of degenerate case in most cases
  -except for reflector and full core

second paragraph: spatial distributions
-Figs.~\Crefrange{fig:chap9-assm-1.6-lns-capt-err}{fig:chap9-full-core-capt-err-degenerate}
  -U-238 capture rate error distributions with respect to OpenMC reference results
  -akin to Figs.~\Crefrange{fig:chap8-assm-1.6-capt-err}{fig:chap8-full-core-capt-err}
  -compare null, \ac{LNS} and degenerate homogenization for 8 and 70 groups
  -shows the error distributions for degenerate and \ac{LNS} spatial homogenization are very nearly the same for the individual fuel assemblies and 2$\times$2 colorset
  -error distributions deviate for the 2$\times$2 colorset with a reflector

third paragraph: observations with reflector?? w/ 70 groups
  -NULL
    -largest errors along assembly-assembly / assembly-reflector interfaces, and pins near \acp{CRGT}
  -DEGENERATE
    -errors randomly distributed across pins
  -LNS
    -largest errors along assembly-reflector interface, and lesser extent assembly-reflector interface
    -as shown in table, largest errors are larger than those for null case
    -reason is that LNS averages \ac{MGXS} pins along both interfaces rather than differentiating b/w them
    -this is actually a worse approx. than averaging across ALL pins as is done by null case
    -highlights sensitivity to how \ac{MGXS} averaging is performed
      -if averaged in the wrong way, errors may actually increase in some pins
      -worthwhile to note that errors will generally improve in those pins with the largest reaction rates since the averaging is done as a weighted average

-SUMMARY BOX

-put magnitude plots in appendices

\begin{table}[ht!]
  \centering
  \caption[OpenMOC U-238 capture rate errors with LNS homogenization]{OpenMOC U-238 capture rate percent relative errors for heterogeneous benchmarks with \ac{LNS} spatial homogenization and varying energy group structures.}
  \small
  \label{table:chap9-lns-capture-rates}
  \vspace{6pt}
  \begin{tabular}{l l R{2.5cm} R{2.5cm} R{2.5cm}}
  \toprule
  \rowcolor{lightgray}
  & & \multicolumn{3}{c}{\cellcolor{lightgray} \textbf{Error [\%]}} \\
  \multirow{-2}{*}{\cellcolor{lightgray} \bf Benchmark} &
  \multirow{-2}{*}{\cellcolor{lightgray} \bf Metric} &
  \multicolumn{1}{r}{{\cellcolor{lightgray} \bf 2-Group}} &
  \multicolumn{1}{r}{{\cellcolor{lightgray} \bf 8-Group}} &
  \multicolumn{1}{r}{{\cellcolor{lightgray} \bf 70-Group}} \\
  \midrule
\multirow{2}{*}{\parbox{2.2cm}{1.6\% Assm}} & Max & 1.095 & 0.441 & 0.358 \\
& Mean & 0.387 & 0.097 & 0.090 \\
\midrule
\multirow{2}{*}{\parbox{2.2cm}{3.1\% Assm}} & Max & 1.128 & 0.477 & 0.331 \\
& Mean & 0.362 & 0.115 & 0.102 \\
\midrule
\multirow{2}{*}{\parbox{2.2cm}{3.1\% Assm w/ 20 BPs}} & Max & 2.085 & 0.676 & 0.454 \\
& Mean & 0.525 & 0.163 & 0.104 \\
\midrule
\multirow{2}{*}{\parbox{2.2cm}{2$\times$2 Colorset}} & Max & 2.939 & -0.956 & 0.643 \\
& Mean & 1.514 & 0.173 & 0.151 \\
\midrule
\multirow{2}{*}{\parbox{2.2cm}{2$\times$2 Colorset w/ Reflector}} & Max & 10.373 & 2.981 & -1.893 \\
& Mean & 3.501 & 0.636 & 0.280 \\
\midrule
\multirow{2}{*}{\parbox{2.2cm}{BEAVRS Quarter Core}} & Max & -85.623 & -35.604 & -7.765 \\
& Mean & 39.911 & 11.013 & 1.375 \\
\bottomrule
\end{tabular}
\end{table}


\begin{figure}[h!]
\centering
\includegraphics[width=\linewidth]{figures/patterns/lns/assm-16/capt-err}
\vspace{2mm}
\caption[U-238 capture rate errors for a 1.6\% enriched assembly]{U-238 capture rate percent relative errors errors for a 1.6\% enriched assembly with \ac{LNS} spatial homogenization.}
\label{fig:chap9-assm-1.6-lns-capt-err}
\end{figure}

\clearpage

\begin{figure}[h!]
\centering
\includegraphics[width=\linewidth]{figures/patterns/lns/assm-31/capt-err}
\vspace{2mm}
\caption[U-238 capture rate errors for a 3.1\% enriched assembly]{U-238 capture rate percent relative errors errors for a 3.1\% enriched assembly with \ac{LNS} spatial homogenization.}
\label{fig:chap9-assm-3.1-lns-capt-err}
\end{figure}

\clearpage

\begin{figure}[h!]
\centering
\includegraphics[width=\linewidth]{figures/patterns/lns/assm-31-20BPs/capt-err}
\vspace{2mm}
\caption[U-238 capture rate errors for a 3.1\% enriched assembly with 20 BPs]{U-238 capture rate percent relative errors errors for a 3.1\% enriched assembly with 20 \acp{BP} with \ac{LNS} spatial homogenization.}
\label{fig:chap9-assm-3.1-20BPs-lns-capt-err}
\end{figure}

\clearpage

\begin{figure}[h!]
\centering
\includegraphics[width=\linewidth]{figures/patterns/lns/2x2/capt-err}
\vspace{2mm}
\caption[U-238 capture rate errors for a 2$\times$2 colorset]{U-238 capture rate percent relative errors errors for a 2$\times$2 colorset with \ac{LNS} spatial homogenization.}
\label{fig:chap9-2x2-lns-capt-err}
\end{figure}

\clearpage

\begin{figure}[h!]
\centering
\includegraphics[width=\linewidth]{figures/patterns/lns/reflector/capt-err}
\vspace{2mm}
\caption[U-238 capture rate errors for a 2$\times$2 colorset with a reflector]{U-238 capture percent relative errors rate errors for a 2$\times$2 colorset with \ac{LNS} spatial homogenization.}
\label{fig:chap9-reflector-lns-capt-err}
\end{figure}

\clearpage

\begin{figure}[h!]
\centering
\includegraphics[width=\linewidth]{figures/patterns/lns/full-core/capt-err-null}
\vspace{2mm}
\caption[U-238 capture rate errors for \ac{BEAVRS} with null homogenization]{U-238 capture rate absolute errors for the 2D quarter core \ac{BEAVRS} model with null spatial homogenization.}
\label{fig:chap9-full-core-capt-err-null}
\end{figure}

\clearpage

\begin{figure}[h!]
\centering
\includegraphics[width=\linewidth]{figures/patterns/lns/full-core/capt-err-lns}
\vspace{2mm}
\caption[U-238 capture rate absolute errors for \ac{BEAVRS} with LNS homogenization]{U-238 capture rate absolute errors for the 2D quarter core \ac{BEAVRS} model with \ac{LNS} spatial homogenization.}
\label{fig:chap9-full-core-capt-err-lns}
\end{figure}

\clearpage

\begin{figure}[h!]
\centering
\includegraphics[width=\linewidth]{figures/patterns/lns/full-core/capt-err-degenerate}
\vspace{2mm}
\caption[U-238 capture rate absolute errors for \ac{BEAVRS} with degenerate homogenization]{U-238 capture rate absolute errors for the 2D quarter core \ac{BEAVRS} model with degenerate spatial homogenization.}
\label{fig:chap9-full-core-capt-err-degenerate}
\end{figure}

%\begin{figure}[h!]
%\centering
%\begin{subfigure}{.5\textwidth}
%  \centering
%  \includegraphics[width=0.9\linewidth]{figures/patterns/lns/full-core/capt-err-null}
%  \caption{}
%  \label{fig:chap9-full-core-null}
%\end{subfigure}%
%\begin{subfigure}{.5\textwidth}
%  \centering
%  \includegraphics[width=0.9\linewidth]{figures/patterns/lns/full-core/capt-err-null-magnitude}
%  \caption{}
%  \label{fig:chap9-full-core-null-magnitude}
%\end{subfigure}
%\begin{subfigure}{.5\textwidth}
%  \centering
%  \includegraphics[width=0.9\linewidth]{figures/patterns/lns/full-core/capt-err-lns}
%  \caption{}
%  \label{fig:chap9-full-core-lns}
%\end{subfigure}%
%\begin{subfigure}{.5\textwidth}
%  \centering
%  \includegraphics[width=0.9\linewidth]{figures/patterns/lns/full-core/capt-err-lns-magnitude}
%  \caption{}
%  \label{fig:chap9-full-core-lns-magnitude}
%\end{subfigure}
%\begin{subfigure}{.5\textwidth}
%  \centering
%  \includegraphics[width=0.9\linewidth]{figures/patterns/lns/full-core/capt-err-degenerate}
%  \caption{}
%  \label{fig:chap7-pin-1.6}
%\end{subfigure}%
%\begin{subfigure}{.5\textwidth}
%  \centering
%  \includegraphics[width=0.9\linewidth]{figures/patterns/lns/full-core/capt-err-degenerate-magnitude}
%  \caption{}
%  \label{fig:chap7-pin-crgt}
%\end{subfigure}
%\caption[U-238 capture rate errors for the \ac{BEAVRS} quarter core model]{U-238 capture percent relative errors rate errors for the \ac{BEAVRS} quarter core model with \ac{LNS} spatial homogenization.}\label{fig:chap9-full-core-lns-capt-err}
%\end{figure}

%\clearpage


%%%%%%%%%%%%%%%%%%%%%%%%%%%%%%%%%%%%%%%%%%%%%%%%%%%%%%%%%%%%%%%%%%%%%%%%%%%%%%%
\section{MGXS Convergence Rate Analysis}
\label{sec:chap9-convergence}

first paragraph: 
-Fig.~\ref{fig:chap9-converge}
  -batch-wise percentage change for \ac{MGXS} with null, degenerate and \ac{LNS} spatial homogenization
  -explain what ``batch-wise change'' means
-\ac{MGXS} for null changes the least from batch to batch
  -track density in each tally volume (in this case, all pins taken together) is greatest
-\ac{MGXS} for degenerate changes the most from batch to batch
  -track density in each tally volume (in this case, each unique fuel pin instance) is least
-\ac{MGXS} for \ac{LNS} changes less than degenerate but more than null
  -track density is in between since \ac{MGXS} are averaged across groups of like fuel pin instance

second paragraph: key takeaways
-convergence rate (slope) is same for all schemes
-averaging across pins shifts convergence curves to right
-the size/magnitude of the leftward shift depends on the number of pins being averaged across
-hence, the more pins we can average across, the faster we can compute \ac{MGXS} with \ac{MC}

third paragraph: segue into next chapter
-goal should be to best identify groups of pins with like \ac{MGXS}
-\ac{LNS} was used as one scheme
  -demonstrated promise for simple benchmarks
  -illustrated shortcomings for benchmarks with complicated heterogeneties
    -assembly-assembly and assembly-reflector interfaces
  -algorithm could be tweaked to pick up on these effects
    -customizations highly reactor dependent - NOT reactor agnostic
-would be nice to identify adequate pin groupings directly from structural patterns in the \ac{MGXS} data
-will do this in the following chapter

\begin{figure}[h!]
\centering
\begin{subfigure}{.87\textwidth}
  \centering
  \includegraphics[width=\linewidth]{figures/patterns/convergence/assm-16/assm-16-capt}
  \caption{}
  \label{fig:chap9-assm-16-converge}
\end{subfigure}
\begin{subfigure}{.87\textwidth}
  \centering
  \includegraphics[width=\linewidth]{figures/patterns/convergence/full-core/16-enr-capt}
  \caption{}
  \label{fig:chap9-reflector-converge}
\end{subfigure}
\caption[Convergence of pin-wise U-238 capture MGXS]{The batch-wise change of pin-wise U-238 capture \ac{MGXS} (group 27 of 70) for 1.6\% enriched fuel pins in a single assembly (a) and the quarter core \ac{BEAVRS} model (b). The convergence of both the maximum and mean absolute percent change are shown for the degenerate and \ac{LNS} homogenization schemes.}
\label{fig:chap9-converge}
\end{figure}


SUMMARY BOXES!!!
